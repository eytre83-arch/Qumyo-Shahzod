  \item $\mathrm{H}_{2} \mathrm{O}$ molekulasidagi $\sigma^{-}$"sigma" va $\Pi^{-}$"pi" bog'lar sonini toping.\\
A) $2 ; 0$\\
B) $2 ; 1$\\
C) $3 ; 1$\\
D) $2 ; 2$\\
  \item CaO molekulasidagi o- "sigma" va $\Pi^{-}$ "pi" bog'lar sonini toping.\\
A) $1 ; 0$\\
B) $1 ; 1$\\
C) $3 ; 1$\\
D) $1 ; 2$
  \item $\mathrm{Al}_{2} \mathrm{O}_{3}$ molekulasidagi $\sigma^{-}$"sigma" va $\Pi^{-}$ "pi" bog'lar sonini toping.\\
A) $2 ; 4$\\
B) $2 ; 2$\\
C) $3 ; 2$\\
D) $4 ; 2$
  \item CO2 molekulasidagi o- "sigma" va п- "pi" bog'lar sonini toping.\\
A) $2: 0$\\
B) $2: 1$\\
C) $3: 1$\\
D) $2: 2$
  \item $\mathrm{P}_{2} \mathrm{O}_{5}$ molekulasidagi $\sigma$ "sigma" va $\mathrm{rr}^{-}$ "pi" bog'lar sonini toping.\\
A) $4: 6$\\
B) $6: 4$\\
C) $3: 2$\\
D) $2: 3$
  \item $\mathrm{SO}_{s}$ molekulasidagi $\sigma^{-}$"sigma" va $\pi^{-}$"pi" boglar sonini toping.\\
A) $2: 2$\\
B) $4 ; 2$\\
C) $3 ; 3$\\
D) $5 ; 2$
  \item $\mathrm{Cl}_{2} \mathrm{O}_{7}$ molekulasidagi $\sigma^{-}$"sigma" va $\Pi^{-}$ "pi" bog'lar sonini toping.\\
A) $8: 6$\\
B) $6 ; 8$\\
C) $3 ; 5$\\
D) $5 ; 2$
  \item $\mathrm{PtO}_{4}$ molekulasidagi $\sigma^{-}$"sigma" va $\Pi^{-}$ "pi" bog'lar sonini toping.\\
A) $2 ; 7$\\
B) $2 ; 5$\\
C) $4 ; 4$\\
D) $2 ; 6$
  \item $\mathrm{H}_{2} \mathrm{O}_{2}$ molekulasidagi $\sigma^{-}$"sigma" va $\pi^{-}$ "pi" bog'lar sonini toping.\\
A) $2 ; 1$\\
B) $2: 2$\\
C) $3 ; 1$\\
D) $3 ; 0$
  \item $\mathrm{Na}_{2} \mathrm{O}_{2}$ molekulasidagi $\sigma$ "sigma" va $\pi^{-}$ "pi" bog'lar sonini toping.\\
A) $2: 1$\\
B) $2 ; 2$\\
C) $3 ; 1$\\
D) $3 ; 0$
  \item $\mathrm{Ca}(\mathrm{OH})_{2}$ molekulasidagi $o^{-}$"sigma" va n- "pi" bog'lar sonini toping.\\
A) $2 ; 1$\\
B) $2 ; 2$\\
C) $3 ; 1$\\
D) $4 ; 0$\\
  \item NaOH molekulasidagi o- "sigma" va $\Pi^{-}$ "pi" bog'lar sonini toping.\\
A) $2 ; 0$\\
B) $2 ; 1$\\
C) $3 ; 0$\\
D) $4 ; 0$
  \item $\mathrm{Al}(\mathrm{OH})_{3}$ molekulasidagi $\sigma$ - "sigma" va п- "pi" bog'lar sonini toping.\\
A) $6: 0$\\
B) $2 ; 0$\\
C) $8: 0$\\
D) $4 ; 0$
14. $\mathrm{Pb}(\mathrm{OH})_{4}$ molekulasidagi $\sigma$ "sigma" va п' "pi" bog'lar sonini toping.\\
A) $6: 0$\\
B) $2: 0$\\
C) $8 ; 0$\\
D) $4 ; 0$\\
15. $\mathrm{Mg}(\mathrm{OH})_{2}$ molekulasidagi $\sigma$ - "sigma" va n- "pi" bog'lar sonini toping.\\
A) $6: 0$\\
B) $2: 0$\\
C) $8: 0$\\
D) $4 ; 0$\\
16. $\mathrm{Cr}(\mathrm{OH})_{2}$ molekulasidagi $\sigma$ " "sigma" va $\pi^{-}$"pi" bog'lar sonini toping.\\
A) $6 ; 0$\\
B) $2: 0$\\
C) $8: 0$\\
D) $4 ; 0$\\
17. $\mathrm{Cr}(\mathrm{OH})_{3}$ molekulasidagi $\sigma$ " "sigma" va $\pi^{-}$"pi" bog'lar sonini toping.\\
A) $6: 0$\\
B) $2 ; 0$\\
C) $8 ; 0$\\
D) $4 ; 0$\\
18. KOH molekulasidagi $\sigma$ - "sigma" va $\Pi^{-}$ "pi" bog'lar sonini toping.\\
A) $6: 0$\\
B) $2 ; 0$\\
C) $8 ; 0$\\
D) $4 ; 0$
19. $\mathrm{Be}(\mathrm{OH})_{2}$ molekulasidagi $\sigma$ " "sigma" va п- "pi" bog'lar sonini toping.\\
A) $6 ; 0$\\
B) 2;0\\
C) $8 ; 0$\\
D) $4 ; 0$\\
20. $\mathrm{Fe}(\mathrm{OH})_{3}$ molekulasidagi $\sigma$ - "sigma" va $п^{-}$"pi" bog'lar sonini toping.\\
A) $6 ; 0$\\
B) $2 ; 0$\\
C) $8 ; 0$\\
D) $4 ; 0$
  \item $\mathrm{H}_{2} \mathrm{SO}_{4}$ molekulasidagi $\sigma$ - "sigma" va $\Pi^{-}$ "pi" bog'lar sonini toping.\\
A) $6 ; 2$\\
B) $2 ; 7$\\
C) $8 ; 4$\\
D) $4 ; 2$\\
  \item $\mathrm{HClO}_{4}$ molekulasidagi $\sigma^{-}$"sigma" va $\Pi^{-}$ "pi" bog'lar sonini toping.\\
A) $6 ; 2$\\
B) $2 ; 4$\\
C) $5 ; 3$\\
D) $4 ; 2$
  \item $\mathrm{H}_{2} \mathrm{~S}_{2} \mathrm{O}_{3}$ molekulasidagi $\sigma^{-}$"sigma" va $\Pi^{-}$ "pi" bog'lar sonini toping.\\
A) $6 ; 2$\\
B) $2 ; 6$\\
C) $8 ; 2$\\
D) $4 ; 2$
  \item $\mathrm{HClO}_{3}$ molekulasidagi $\sigma$ - "sigma" va $\Pi^{-}$ "pi" bog'lar sonini toping.\\
A) $4 ; 1$\\
B) $2 ; 4$\\
C) $8 ; 2$\\
D) $4 ; 2$
  \item $\mathrm{H}_{3} \mathrm{PO}_{4}$ molekulasidagi $\sigma^{-}$"sigma" va $\Pi^{-}$ "pi" bog'lar sonini toping.\\
A) $6 ; 2$\\
B) $4 ; 3$\\
C) $7 ; 1$\\
D) $4 ; 2$
  \item $\mathrm{HPO}_{3}$ molekulasidagi $\sigma$ - "sigma" va $\Pi^{-}$ "pi" bog'lar sonini toping.\\
A) $6 ; 2$\\
B) $4 ; 1$\\
C) $4 ; 0$\\
D) $4 ; 2$
  \item $\mathrm{H}_{2} \mathrm{Cr}_{2} \mathrm{O}_{7}$ molekulasidagi $\sigma$ "sigma" va $п^{-}$"pi" bog'lar sonini toping.\\
A) $10 ; 4$\\
B) $2 ; 0$\\
C) $8 ; 0$\\
D) $4 ; 0$
  \item $\mathrm{H}_{4} \mathrm{P}_{2} \mathrm{O}_{7}$ molekulasidagi $\sigma$ - "sigma" va $\pi^{-}$"pi" bog'lar sonini toping.\\
A) $6 ; 8$\\
B) $2 ; 2$\\
C) $12 ; 2$\\
D) $4 ; 4$
  \item $\mathrm{H}_{2} \mathrm{SO}_{3}$ molekulasidagi $\sigma$ "sigma" va $\pi^{-}$ "pi" bog'lar sonini toping.\\
A) $5 ; 1$\\
B) $2 ; 3$\\
C) $4 ; 2$\\
D) $4 ; 1$
  \item $\mathrm{H}_{2} \mathrm{~S}$ molekulasidagi $\sigma^{-}$"sigma" va $\pi^{-}$ "pi" bog'lar sonini toping.\\
A) $6: 0$\\
B) $2 ; 0$\\
C) $8 ; 0$\\
D) $4 ; 0$
  \item $\mathrm{Na}_{2} \mathrm{SO}_{4}$ molekulasidagi $\sigma^{-}$"sigma" va $\pi^{-}$"pi" bog'lar sonini toping.\\
A) $6 ; 2$\\
B) $2 ; 6$\\
C) $8 ; 1$\\
D) $4 ; 2$\\
  \item $\mathrm{K}_{3} \mathrm{PO}_{4}$ molekulasidagi $\sigma^{-}$"sigma" va $\Pi^{-}$ "pi" bog'lar sonini toping.\\
A) $6 ; 2$\\
B) $4 ; 3$\\
C) $7 ; 1$\\
D) $4 ; 2$
  \item $\mathrm{Al}_{2}\left(\mathrm{SO}_{4}\right)_{3}$ molekulasidagi $\sigma^{-}$"sigma" va $\pi^{-}$"pi" bog'lar sonini toping.\\
A) $16: 4$\\
B) $12 ; 4$\\
C) $18 ; 6$\\
D) $14 ; 2$
  \item $\mathrm{KMnO}_{4}$ molekulasidagi $\sigma^{-}$"sigma" va $\pi^{-}$"pi" bog'lar sonini toping.\\
A) $6: 2$\\
B) $5 ; 3$\\
C) $6 ; 1$\\
D) $5 ; 1$
  \item $\mathrm{Cr}\left(\mathrm{ClO}_{4}\right)_{3}$ molekulasidagi $\sigma$ - "sigma" va $\pi^{-}$"pi" bog'lar sonini toping.\\
A) $15 ; 9$\\
B) $12 ; 5$\\
C) $18 ; 7$\\
D) $14 ; 3$
  \item $\mathrm{Zn}\left(\mathrm{PO}_{3}\right)_{2}$ molekulasidagi $\sigma$ "sigma" va $п^{-}$"pi" bog'lar sonini toping.\\
A) $6: 3$\\
B) $7 ; 3$\\
C) $8 ; 4$\\
D) $4 ; 2$
  \item $\mathrm{BaCrO}_{4}$ molekulasidagi $\sigma^{-}$"sigma" va $\pi^{-}$"pi" bog'lar sonini toping.\\
A) $6: 2$\\
B) $5 ; 3$\\
C) $7 ; 0$\\
D) $4 ; 1$
  \item $\mathrm{CdSiO}_{3}$ molekulasidagi $\sigma$ - "sigma" va $\pi^{-}$"pi" bog'lar sonini toping.\\
А) $6: 4$\\
B) $10 ; 2$\\
C) $5 ; 1$\\
D) $8 ; 2$
  \item $\mathrm{Li}_{2} \mathrm{SO}_{3}$ molekulasidagi $\sigma^{-}$"sigma" va $\Pi^{-}$ "pi" bog'lar sonini toping.\\
A) $6 ; 2$\\
B) $5 ; 3$\\
C) $6 ; 1$\\
D) $5 ; 1$
  \item $\mathrm{K}_{4} \mathrm{P}_{2} \mathrm{O}_{7}$ molekulasidagi $\sigma$ - "sigma" va $\Pi^{-}$ "pi" bog'lar sonini toping.\\
A) $6 ; 1$\\
B) $12 ; 2$\\
C) $8 ; 3$\\
D) $4: 1$
  \item Quyidagi moddalardan qaysi birining tarkibida ion bog' bor?\\
A) $\mathrm{H}_{2} \mathrm{O}$\\
B) $NaCl$\\
C) $\mathrm{CH}_{4}$\\
D) $\mathrm{HNO}_{3}$
  \item Quyidagi moddalardan qaysi birining tarkibida ion bog' bor?\\
A) $\mathrm{N}_{2} \mathrm{O}$\\
B) $HCl$\\
C) $\mathrm{SiH}_{4}$\\
D) $\mathrm{LiNO}_{3}$
  \item Quyidagi moddalardan qaysi birining tarkibida ion bog' bor?\\
A) $\mathrm{Na}_{2} \mathrm{O}$\\
B) Cu\\
C) Olmos\\
D) $\mathrm{HNO}_{3}$
  \item Quyidagi moddalardan qaysi birining tarkibida ion bog' bor?\\
A) $\mathrm{Cl}_{2} \mathrm{O}$\\
B) $\mathrm{NCl}_{3}$\\
C) RbH\\
D) $\mathrm{HNO}_{2}$
  \item Quyidagi moddalardan qaysi birining tarkibida ion bog' bor?\\
A) $\mathrm{C}_{2} \mathrm{H}_{2}$\\
B) $\mathrm{NF}_{3}$\\
C) $\mathrm{SiH}_{4}$\\
D) $\mathrm{MgSO}_{4}$
  \item Quyidagi moddalardan qaysi birining tarkibida ion bog' bor?\\
A) $\mathrm{S}_{8}$\\
B) $\mathrm{ClF}_{3}$\\
C) $\mathrm{AlF}_{3}$\\
D) $\mathrm{HNO}_{3}$
  \item Quyidagi moddalardan qaysi birining tarkibida ion bog' bor?\\
A) $\mathrm{H}_{2}$\\
B) KCl\\
C) $\mathrm{CH}_{4}$\\
D) $\mathrm{NH}_{3}$
  \item Quyidagi moddalardan qaysi birining tarkibida ion bog' bor?\\
A) $\mathrm{K}_{2} \mathrm{O}$\\
B) $\mathrm{SCl}_{6}$\\
C) $\mathrm{SiH}_{4}$\\
D) $\mathrm{HN}_{3}$
  \item Quyidagi moddalardan qaysi birining tarkibida ion bog' bor?\\
A) $\mathrm{H}_{2} \mathrm{O}_{2}$\\
B) LiH\\
C) $\mathrm{C}_{2} \mathrm{H}_{4}$\\
D) HF
  \item Quyidagi moddalardan qaysi birining tarkibida ion bog' bor?\\
A) $\mathrm{H}_{2} \mathrm{O}_{2}$\\
B) $\mathrm{NCl}_{3}$\\
C) $\mathrm{SiH}_{4}$\\
D) $\mathrm{FrNO}_{3}$
  \item Quyidagi moddalardan qaysi birining tarkibida donor-akseptor bog' bor?\\
A) $\mathrm{H}_{2} \mathrm{O}_{2}$\\
B) $\mathrm{NCl}_{3}$\\
C) $\mathrm{SiH}_{4}$\\
D) $\mathrm{FrNO}_{3}$
  \item Quyidagi moddalardan qaysi birining tarkibida donor-akseptor bog' bor?\\
A) $\mathrm{H}_{2} \mathrm{O}$\\
B) CO\\
C) $\mathrm{SiH}_{4}$\\
D) $\mathrm{CO}_{2}$
  \item Quyidagi moddalardan qaysi birining tarkibida donor-akseptor bog' bor?\\
A) $\mathrm{H}_{2}$\\
B) $\mathrm{N}_{2}$\\
C) $\mathrm{NH}_{4} \mathrm{Cl}$\\
D) Cu
  \item Quyidagi moddalardan qaysi birining tarkibida donor akseptor bog' bor?\\
A) HCl\\
B) $\mathrm{O}_{3}$\\
C) $\mathrm{CH}_{4}$\\
D) FrCl
  \item Quyidagi moddalardan qaysi birining tarkibida donor-akseptor bog' bor?\\
A) HF\\
B) $\mathrm{K}_{3}\left[\mathrm{Fe}(\mathrm{CN})_{6}\right]$\\
C) $\mathrm{PbH}_{4}$\\
D) $\mathrm{SeO}_{3}$
  \item Quyidagi moddalardan qaysi birining tarkibida donor-akseptor bog' bor?\\
A) $\mathrm{H}_{2} \mathrm{SO}_{4}$\\
B) $\mathrm{NH}_{3}$\\
C) $\mathrm{N}_{2} \mathrm{O}$\\
D) $\mathrm{CrCl}_{3}$
  \item Quyidagi moddalardan qaysi birining tarkibida donor-akseptor bog' bor?\\
A) $\mathrm{N}_{2} \mathrm{O}_{5}$\\
B) $\mathrm{NF}_{3}$\\
C) $\mathrm{GeH}_{4}$\\
D) $\mathrm{Fr}_{3} \mathrm{~N}$
  \item Quyidagi moddalardan qaysi birining tarkibida donor-akseptor bog' bor?\\
A) $\mathrm{H}_{2} \mathrm{~S}$\\
B) $\mathrm{K}_{4}\left[\mathrm{Fe}(\mathrm{CN})_{6}\right]$\\
C) $\mathrm{CaH}_{2}$\\
D) $\mathrm{CuCl}_{2}$
  \item Quyidagi moddalardan qaysi birining tarkibida donor-akseptor bog' bor?\\
A) $\mathrm{SO}_{3}$\\
B) $\mathrm{NCl}_{3}$\\
C) $\mathrm{MgF}_{2}$\\
D) $\mathrm{Al}\left(\mathrm{NO}_{3}\right)_{3}$
  \item Quyidagi moddalardan qaysi birining tarkibida donor-akseptor bog' bor?\\
A) HF\\
B) $\mathrm{NCl}_{3}$\\
C) $\mathrm{CO}_{2}$\\
D) $\mathrm{NO}_{2}$
  \item Quyidagi moddalardan qaysi birining tarkibida qutbsiz kovalent bog' bor?\\
A) $\mathrm{H}_{2}$\\
B) $\mathrm{NCl}_{3}$\\
C) ${ }^{\prime} \mathrm{CO}_{2}$\\
D) $\mathrm{NO}_{2}$
  \item Quyidagi moddalardan qaysi birining tarkibida qutbsiz kovalent bog' bor?\\
A) $\mathrm{H}_{2} \mathrm{O}$\\
B) olmos\\
C) dolomit\\
D) osh tuzi
  \item Quyidagi moddalardan qaysi birining tarkibida qutbsiz kovalent bog' bor?\\
A) HF\\
B) $\mathrm{S}_{8}$\\
C) $\mathrm{NO}_{2}$\\
D) $\mathrm{PbO}_{2}$
  \item Quyidagi moddalardan qaysi birining tarkibida qutbsiz kovalent bog' bor?\\
A) $\mathrm{I}_{2}$\\
B) $\mathrm{NCl}_{3}$\\
C) Cu\\
D) $\mathrm{Cl}_{2} \mathrm{O}_{7}$
  \item Quyidagi moddalardan qaysi birining tarkibida qutbsiz kovalent bog' bor?\\
A) $\mathrm{H}_{2} \mathrm{O}$\\
B) Cr\\
C) $\mathrm{O}_{3}$\\
D) $\mathrm{CH}_{4}$
  \item Quyidagi moddalardan qaysi birining tarkibida qutbsiz kovalent bog' bor?\\
A) $\mathrm{H}_{2} \mathrm{SO}_{4}$\\
B) $\mathrm{P}_{4}$\\
C) $\mathrm{NO}_{2}$\\
D) Fe
  \item Quyidagi moddalardan qaysi birining tarkibida qutbsiz kovalent bog' bor?\\
A) HCl\\
B) $\mathrm{CuCl}_{2}$\\
C) $\mathrm{H}_{2} \mathrm{~S}_{2} \mathrm{O}_{3}$\\
D) $\mathrm{AlF}_{3}$
  \item Quyidagi moddalardan qaysi birining tarkibida qutbsiz kovalent bog' bor?\\
A) $\mathrm{H}_{2} \mathrm{SO}_{4}$\\
B) $\mathrm{PCl}_{3}$\\
C) $\mathrm{C}_{60}$\\
D) $\mathrm{NCl}_{3}$
  \item Quyidagi moddalardan qaysi birining tarkibida qutbsiz kovalent bog' bor?\\
A) osh tuzi\\
B) grafit\\
C) $\mathrm{CO}_{2}$\\
D) $\mathrm{NO}_{2}$
  \item Quyidagi moddalardan qaysi birining tarkibida qutbli kovalent bog' bor?\\
A) osh tuzi\\
B) grafit\\
C) $\mathrm{CO}_{2}$\\
D) $\mathrm{N}_{2}$
  \item Quyidagi moddalardan qaysi birining tarkibida qutbli kovalent bog' bor?\\
A) $\mathrm{Na}_{2} \mathrm{SO}_{4}$\\
B) $\mathrm{CaH}_{2}$\\
C) $\mathrm{MgCl}_{2}$\\
D) Ne
  \item Quyidagi moddalardan qaysi birining tarkibida qutbli kovalent bog' bor?\\
A) $\mathrm{SO}_{3}$\\
B) olmos\\
C) $\mathrm{C}_{60}$\\
D) Cr
  \item Quyidagi moddalardan qaysi birining tarkibida qutbli kovalent bog' bor?\\
A) $\mathrm{CrF}_{3}$\\
B) $\mathrm{ZnCl}_{2}$\\
C) $\mathrm{SO}_{2}$\\
D) $\mathrm{N}_{2}$
  \item Quyidagi moddalardan qaysi birining tarkibida qutbli kovalent bog' bor?\\
A) Cu\\
B) He\\
C) $\mathrm{CaCl}_{2}$\\
D) $\mathrm{MgSO}_{4}$
  \item Quyidagi moddalardan qaysi birining tarkibida qutbli kovalent bog' bor?\\
A) $\mathrm{SeO}_{3}$\\
B) Hg\\
C) $\mathrm{S}_{\mathrm{n}}$\\
D) $\mathrm{N}_{2}$
  \item Quyidagi moddalardan qaysi birining tarkibida qutbli kovalent bog' bor?\\
A) osh tuzi\\
B) olmos\\
C) CO\\
D) $\mathrm{Br}_{2}$
  \item Quyidagi moddalardan qaysi birining tarkibida qutbli kovalent bog' bor?\\
A) $\mathrm{AlF}_{3}$\\
B) $\mathrm{O}_{3}$\\
C) $\mathrm{NO}_{2}$\\
D) Hg
  \item Quyidagi moddalardan qaysi birining tarkibida qutbli kovalent bog' bor?\\
A) $\mathrm{Mg}_{3} \mathrm{~N}_{2}$\\
B) $\mathrm{Hg}\left(\mathrm{NO}_{3}\right)_{2}$\\
C) $\mathrm{CdF}_{2}$\\
D) $\mathrm{N}_{2}$
  \item Quyidagi moddalardan qaysi birining tarkibida qutbli kovalent bog' bor?\\
A) Al\\
B) $\mathrm{HNO}_{3}$\\
C) C\\
D) $\mathrm{N}_{2}$
  \item Quyidagi moddalar orasidan molekulyar kristall panjarali moddani aniqlang.\\
A) $\mathrm{CH}_{4}$\\
B) olmos\\
C) NaI\\
D) Mg
  \item Quyidagi moddalar orasidan molekulyar kristall panjarali moddani aniqlang.\\
A) SiC\\
B) $\mathrm{H}_{2} \mathrm{O}$\\
C) NaCl\\
D) karbin
  \item Quyidagi moddalar orasidan molekulyar kristall panjarali moddani aniqlang.\\
A) Fe\\
B) Si\\
C) KCl\\
D) fullerin
  \item Quyidagi moddalar orasidan molekulyar kristall panjarali moddani aniqlang.\\
A) $\mathrm{NH}_{3}$\\
B) grafit\\
C) $\mathrm{MgCO}_{3}$\\
D) Cr
  \item Quyidagi moddalar orasidan molekulyar kristall panjarali moddani aniqlang.\\
A) $\mathrm{SiO}_{2}$\\
B) $\mathrm{Cu}_{3} \mathrm{Zn}_{5}$\\
C) $\mathrm{H}_{2}$\\
D) $\mathrm{MgCl}_{2}$
  \item Quyidagi moddalar orasidan molekulyar kristall panjarali moddani aniqlang.\\
A) qizil fosfor\\
B) He\\
C) $\mathrm{FeCl}_{2}$\\
D) cho'yan
  \item Quyidagi moddalar orasidan molekulyar kristall panjarali moddani aniqlang.\\
A) $\mathrm{C}_{2} \mathrm{H}_{2}$\\
B) qora fosfor\\
C) $\mathrm{CrF}_{2}$\\
D) Cu
  \item Quyidagi moddalar orasidan molekulyar kristall panjarali moddani aniqlang.\\
A) K\\
B) olmos\\
C) NaI\\
D) $\mathrm{Cl}_{2}$
  \item Quyidagi moddalar orasidan molekulyar kristall panjarali moddani aniqlang.\\
A) $\mathrm{SO}_{3}$\\
B) SiC\\
C) FrI\\
D) Mg
  \item Quyidagi moddalar orasidan molekulyar kristall panjarali moddani aniqlang.\\
A) $\mathrm{C}_{2} \mathrm{H}_{4}$\\
B) Al\\
C) NaH\\
D) Si
  \item Quyidagi moddalar orasidan ion kristall panjarali moddani aniqlang.\\
A) $\mathrm{CH}_{4}$\\
B) olmos\\
C) NaI\\
D) Mg
12. Quyidagi moddalar orasidan ion kristall panjarali moddani aniqlang.\\
A) SiC\\
B) $\mathrm{H}_{2} \mathrm{O}$\\
C) NaCl\\
D) karbin\\
13. Quyidagi moddalar orasidan ion kristall panjarali moddani aniqlang.\\
A) Fe\\
B) Si\\
C) KCl\\
D) fullerin\\
14. Quyidagi moddalar orasidan ion kristall panjarali moddani aniqlang.\\
A) $\mathrm{NH}_{3}$\\
B) grafit\\
C) $\mathrm{MgCO}_{3}$\\
D) Cr\\
15. Quyidagi moddalar orasidan ion kristall panjarali moddani aniqlang.\\
A) $\mathrm{SiO}_{2}$\\
B) $\mathrm{Cu}_{3} \mathrm{Zn}_{5}$\\
C) $\mathrm{H}_{2}$\\
D) $\mathrm{MgCl}_{2}$\\
16. Quyidagi moddalar orasidan ion kristall panjarali moddani aniqlang.\\
A) qizil fosfor\\
B) He\\
C) $\mathrm{FeCl}_{2}$\\
D) cho'yan\\
17. Quyidagi moddalar orasidan ion kristall panjarali moddani aniqlang.\\
A) $\mathrm{C}_{2} \mathrm{H}_{2}$\\
B) qora fosfor\\
C) $\mathrm{CrF}_{2}$\\
D) Cu\\
18. Quyidagi moddalar orasidan ion kristall panjarali moddani aniqlang.\\
A) K\\
B) olmos\\
C) NaI\\
D) $\mathrm{Cl}_{2}$\\
19. Quyidagi moddalar orasidan ion kristall panjarali moddani aniqlang.\\
A) $\mathrm{SO}_{3}$\\
B) SiC\\
C) FrI\\
D) Mg\\
20. Quyidagi moddalar orasidan ion kristall panjarali moddaní aniqlang.\\
A) $\mathrm{C}_{2} \mathrm{H}_{4}$\\
B) Al\\
C) NaH\\
D) Si\\
21. Quyidagi moddalar orasidan atom kristall panjarali moddani aniqlang. 
A) $\mathrm{CH}_{4}$\\
B) olmos\\
C) NaI\\
D) Mg\\
22. Quyidagi moddalar orasidan atom kristall panjarali moddani aniqlang.\\
A) SiC\\
B) $\mathrm{H}_{2} \mathrm{O}$\\
C) NaCl\\
D) Ti\\
23. Quyidagi moddalar orasidan atom kristall panjarali moddani aniqlang.\\
A) Fe\\
B) Si\\
C) KCl\\
D) fullerin\\
24. Quyidagi moddalar orasidan atom kristall panjarali moddani aniqlang.\\
A) $\mathrm{NH}_{3}$\\
B) grafit\\
C) $\mathrm{MgCO}_{3}$\\
D) Cr\\
25. Quyidagi moddalar orasidan atom kristall panjarali moddani aniqlang.\\
A) $\mathrm{SiO}_{2}$\\
B) $\mathrm{Cu}_{3} \mathrm{Zn}_{5}$\\
C) $\mathrm{H}_{2}$\\
D) $\mathrm{MgCl}_{2}$\\
26. Quyidagi moddalar orasidan atom kristall panjarali moddani aniqlang.\\
A) qizil fosfor\\
B) He\\
C) $\mathrm{FeCl}_{2}$\\
D) cho'yan\\
27. Quyidagi moddalar orasidan atom kristall panjarali moddani aniqlang.\\
A) $\mathrm{C}_{2} \mathrm{H}_{2}$\\
B) qora fosfor\\
C) $\mathrm{CrF}_{2}$\\
D) Cu\\
28. Quyidagi moddalar orasidan atom kristall panjarali moddani aniqlang.\\
A) K\\
B) olmos\\
C) NaI\\
D) $\mathrm{Cl}_{2}$\\
29. Quyidagi moddalar orasidan atom kristall panjarali moddani aniqlang.\\
A) $\mathrm{SO}_{3}$\\
B) SiC\\
C) FrI\\
D) Mg\\
30. Quyidagi moddalar orasidan atom kristall panjarali moddani aniqlang.\\
A) $\mathrm{C}_{2} \mathrm{H}_{4}$\\
B) Al\\
C) NaH\\
D) Si
  \item Quyidagi moddalar orasidan metall kristall panjarali moddani aniqlang.\\
A) $\mathrm{CH}_{4}$\\
B) olmos\\
C) NaI\\
D) Mg
32. Quyidagi moddalar orasidan metall kristall panjarali moddani aniqlang.\\
A) SiC\\
B) $\mathrm{H}_{2}$\\
C) NaCl\\
D) Au
S5. Querdagi moddalar orasidan metall kristall panjarali moddani aninqlang.\\
A) $\mathrm{SiO}_{4}$\\
B) $\mathrm{Cu}_{3} \mathrm{Zn} \mathrm{n}_{4}$\\
C) $\mathrm{H}_{2}$\\
D) $\mathrm{MgCl}_{2}$\\
36. Quydagi moddalar orasidan metall kristall panjarali moddani aniqlang.\\
A) gizil fosfor\\
B) He\\
C) $\mathrm{FeCl}_{2}$\\
D) choran\\
37. Quyidagi moddalar orasidan metall kristall panjarali moddani aniqlang.\\
A) $\mathrm{C}_{2} \mathrm{H}_{3}$\\
B) qora fostor\\
C) $\mathrm{CrF}_{2}$\\
D) Cu\\
38. Quyidagi moddalar orasidan metall kristall panjarali moddani aniqlang.\\
A) K\\
B) olnos\\
C) NaI\\
D) $\mathrm{Cl}_{2}$\\
39. Quyidagi moddalar orasidan metall kristall panjarali moddani aniqlang.\\
A) $\mathrm{SO}_{3}$\\
B) SiC\\
C) FrI\\
D) jez qotishmasi\\
40. Quyidagi moddalar orasidan metall kristall panjarali moddani aniqlang.\\
A) $\mathrm{C}_{2} \mathrm{H}_{4}$\\
B) Al\\
C) NaH\\
D) Si
  \item Quyidagi moddalar orasidan markaziy atomi sp gibridlangan moddani toping.\\
A) $\mathrm{C}_{2} \mathrm{H}_{2}$\\
B) $\mathrm{SO}_{3}$\\
C) $\mathrm{CH}_{4}$\\
D) $\mathrm{PCl}_{5}$
44. Quyidagi moddalar orasidan markaziy atomi sp gibridlangan moddani toping.\\
A) Bolly\\
B) $\mathrm{SCl}_{1}$\\
C) $\mathrm{NH}_{8}$\\
D) $\mathrm{PCl}_{6}$\\
45. Quyidagi moddalau orasidan mankaziy atomi ap gibridlangan moddani toping.\\
A) $\mathrm{SiO}_{2}$\\
B) $\mathrm{HNO}_{3}$\\
C) $\mathrm{CO}_{4}$\\
D) $\mathrm{PH}_{3}$\\
46. Quyidagi moddalar orasidan markaziy atomi sp gibridlangan moddani toping.\\
A) $\mathrm{HClO}_{3}$\\
B) HCN\\
C) $\mathrm{CCl}_{4}$\\
D) $\mathrm{Ki}\left[\mathrm{Fe}(\mathrm{CN})_{\mathrm{d}}\right]$\\
47. Quyidagi moddalar orasidan markaziy atomi sp gibridlangan moddani toping.\\
A) $\mathrm{N}_{2} \mathrm{H}_{4}$\\
B) $\mathrm{BCl}_{3}$\\
C) $\mathrm{HClO}_{4}$\\
D) KCN\\
48. Quyidagi moddalar orasidan markaziy atomi sp gibridlangan moddani toping.\\
A) $\mathrm{N}_{2}$\\
B) $\mathrm{SO}_{3}$\\
C) H 2 Se\\
D) $\mathrm{PCl}_{6}$\\
49. Quyidagi moddalar orasidan markaziy atomi sp gibridlangan moddani toping.\\
A) $\mathrm{Ag}_{2} \mathrm{C}_{2}$\\
B) $\mathrm{HMnO}_{4}$\\
C) $\mathrm{NCl}_{3}$\\
D) $\mathrm{PF}_{5}$\\
50. Quyidagi moddalar orasidan markaziy atomi sp gibridlangan moddani toping.\\
A) $\mathrm{C}_{2} \mathrm{~N}_{2}$\\
B) $\mathrm{SO}_{3}$\\
C) $\mathrm{H}_{2} \mathrm{SO}_{4}$\\
D) $\mathrm{PCl}_{0}$
  \item Quyidagi moddalar orasidan markaziy atomi $\mathrm{sp}^{2}$ gibridlangan moddani toping.\\
A) $\mathrm{C}_{2} \mathrm{H}_{2}$\\
B) $\mathrm{SO}_{3}$\\
C) $\mathrm{CH}_{4}$\\
D) $\mathrm{PCl}_{5}$\\
  \item Quyidagi moddalar orasidan markaziy atomi $\mathrm{sp}^{2}$ gibridlangan moddani toping.\\
A) $\mathrm{H}_{2} \mathrm{O}$\\
B) $\mathrm{Fe}(\mathrm{CO})_{5}$\\
C) $\mathrm{BeCl}_{2}$\\
D) $\mathrm{SO}_{2}$
  \item Quyidagi moddalar orasidan markaziy atomi $\mathrm{sp}^{2}$ gibridlangan moddani toping.\\
A) $\mathrm{BF}_{:}$\\
B) karbin\\
C) $\mathrm{SiH}_{4}$\\
D) $\mathrm{AnF}_{\mathrm{N}} \mathrm{F}_{\mathrm{s}}$
  \item Quyidagi moddalar ornaidan markaziy atomi spy gibridlangan moddani toping.\\
A) $\mathrm{BCF}_{2}$\\
B) $\mathrm{SCl}_{6}$\\
C) $\mathrm{NH}_{:}$\\
D) $\mathrm{H}_{y} \mathrm{CO}_{:}$
  \item Quyidagi moddalar orasidan markaziy atomi spa gibridlangan moddani toping.\\
A) $\mathrm{SiO}_{2}$\\
B) BBr\\
C) $\mathrm{CO}_{2}$\\
D) $\mathrm{PH}_{:}$
  \item Quyidagi moddalar orasidan markaziy ntomi spn gibridlangan moddani toping.\\
A) $\mathrm{HClO}_{8}$\\
B) HCN\\
C) $\mathrm{C}_{2} \mathrm{H}_{4}$\\
D) $\mathrm{K}_{\mathrm{a}}\left[\mathrm{Fe}(\mathrm{CN})_{0}\right]$
  \item Quyidagi moddalar orasidan markaziy atomi sp" gibridlangan moddani toping.\\
A) $\mathrm{N}_{3} \mathrm{H}_{4}$\\
B) $\mathrm{BCl}_{3}$\\
C) $\mathrm{HClO}_{4}$\\
D) KCN
  \item Quyidagi moddalar orasidan markaziy atomi $\mathrm{sp}^{2}$ gibridlangan moddani toping.\\
A) $\mathrm{N}_{2}$\\
B) $\mathrm{SO}_{3}$\\
C) H 2 Se\\
D) $\mathrm{PCl}_{0}$
  \item Quyidagi moddalar orasidan markaziy atomi sp2 gibridlangan moddani toping.\\
A) $\mathrm{Ag}_{2} \mathrm{C}_{2}$\\
B) $\mathrm{K}_{2} \mathrm{CO}_{3}$\\
C) $\mathrm{NCl}_{3}$\\
D) $\mathrm{PF}_{5}$
  \item Quyidagi moddalar orasidan markaziy atomi $\mathrm{sp}^{2}$ gibridlangan moddani toping.\\
A) $\mathrm{C}_{2} \mathrm{~N}_{2}$\\
B) $\mathrm{SO}_{4}$\\
C) $\mathrm{H}_{2} \mathrm{SO}_{4}$\\
D) $\mathrm{PCl}_{5}$
  \item Quyidagi moddalar orasidan markaziy atomi $\mathrm{sp}^{3}$ gibridlangan moddani toping.\\
A) $\mathrm{C}_{2} \mathrm{H}_{2}$\\
B) $\mathrm{SO}_{3}$\\
C) $\mathrm{CH}_{4}$\\
D) $\mathrm{PCl}_{5}$\\
  \item Quyidagi moddalar orasidan markaziy atomi $\mathrm{sp}^{3}$ gibridlangan moddani toping.\\
A) $\mathrm{H}_{2} \mathrm{O}$\\
B) $\mathrm{Fe}(\mathrm{CO})_{0}$\\
C) $\mathrm{BeCl}_{2}$\\
D) $\mathrm{SO}_{2}$
  \item Quyidagi moddalar orasidan markaziy atomi $\mathrm{sp}^{3}$ gibridlangan moddani toping.\\
A) $\mathrm{BF}_{3}$\\
B) karbin\\
C) $\mathrm{SiH}_{4}$\\
D) $\mathrm{AsF}_{5}$
  \item Quyidagi moddalar orasidan markaziy atomi $\mathrm{sp}^{4}$ gibridlangan moddani toping.\\
A) $\mathrm{BeF}_{2}$\\
B) $\mathrm{SCl}_{6}$\\
C) $\mathrm{NH}_{3}$\\
D) $\mathrm{H}_{2} \mathrm{CO}_{3}$
  \item Quyidagi moddalar orasidan markaziy ntomi spi gibridlangan moddani toping.\\
B) $\mathrm{BBr}_{s}$\\
C) $\mathrm{C}^{2} \mathrm{O}_{2}$\\
D) $\mathrm{N}_{2}$\\
N) $\mathrm{SiO}_{2}$
  \item Quyidagi moddalar oranidan markaziy atomi mp ${ }^{3}$ gibridlangan moddani toping.\\
A) $\mathrm{HClO}_{3}$\\
B) HCN\\
C) $\mathrm{C}_{2} \mathrm{H}_{4}$\\
D) $\mathrm{K}_{a}\left[\mathrm{Fu}(\mathrm{CN})_{a}\right]$
  \item Quyidagi moddalar orasidan markaziy atomi $\mathrm{sp}^{3}$ gibridlangan moddani toping.\\
A) $\mathrm{N}_{2} \mathrm{H}_{4}$\\
B) $\mathrm{BCl}_{3}$\\
C) $\mathrm{BoF}_{2}$\\
D) KCN
  \item Quyidagi moddalar orasidan markaziy atomi $\mathrm{sp}^{3}$ gibridlangan moddani toping.\\
A) $\mathrm{N}_{2}$\\
B) $\mathrm{SO}_{3}$\\
C) H 2 Se\\
D) $\mathrm{PCl}_{6}$
  \item Quyidagi moddalar orasidan markaziy atomi $\mathrm{sp}^{3}$ gibridlangan moddani toping.\\
A) $\mathrm{Ag}_{2} \mathrm{C}_{2}$\\
B) $\mathrm{K}_{2} \mathrm{CO}_{3}$\\
C) $\mathrm{NCl}_{3}$\\
D) $\mathrm{PF}_{5}$
  \item Quyidagi moddalar orasidan markaziy atomi $\mathrm{sp}^{3}$ gibridlangan moddani toping.\\
A) $\mathrm{C}_{2} \mathrm{~N}_{2}$\\
B) $\mathrm{SO}_{3}$\\
C) $\mathrm{H}_{2} \mathrm{SO}_{4}$\\
D) $\mathrm{PCl}_{5}$
  \item Quyidagi moddalar orasidan molekulasining fazoviy tuzulishi chiziqli bo'lganini toping.\\
A) $\mathrm{H}_{2} \mathrm{O}$\\
B) $\mathrm{Fe}(\mathrm{CO})_{5}$\\
C) $\mathrm{BeCl}_{2}$\\
D) $\mathrm{SO}_{2}$
  \item Quyidagi moddalar orasidan molekulasining fazoviy tuzulishi chiziqli bo'lganini toping.\\
A) $\mathrm{C}_{2} \mathrm{H}_{2}$\\
B) $\mathrm{SO}_{3}$\\
C) $\mathrm{CH}_{4}$\\
D) $\mathrm{PCl}_{5}$
  \item Quyidagi moddalar orasidan molekulasining fazoviy tuzulishi chiziqli bo'lganini toping.\\
A) $\mathrm{H}_{2} \mathrm{O}$\\
B) $\mathrm{Fe}(\mathrm{CO})_{5}$\\
C) $\mathrm{BeF}_{2}$\\
D) $\mathrm{SO}_{2}$
  \item Quyidagi moddalar orasidan molekulasining fazoviy tuzulishi chiziqli bo'lganini toping.\\
A) $\mathrm{HNO}_{3}$\\
B) $\mathrm{CO}_{2}$\\
C) $\mathrm{SiO}_{2}$\\
D) $\mathrm{PH}_{3}$
  \item Quyidagi moddalar orasidan molekulasining fazoviy tuzulishi chiziqli bo'lganini toping.\\
A) $\mathrm{HClO}_{3}$\\
B) HCN\\
C) $\mathrm{CCl}_{4}$\\
D) $\mathrm{K}_{3}\left[\mathrm{Fe}(\mathrm{CN})_{6}\right]$
  \item Quyidagi moddalar orasidan molekulasining fazoviy tuzulishi chiziqli bo'lganini toping.\\
A) $\mathrm{N}_{2} \mathrm{H}_{4}$\\
B) $\mathrm{BCl}_{3}$\\
C) $\mathrm{HClO}_{4}$\\
D) KCN
  \item Quyidagi moddalar orasidan molekulasining fazoviy tuzulishi chiziqli bo'lganini toping.\\
A) $\mathrm{Ag}_{2} \mathrm{C}_{2}$\\
B) $\mathrm{HMnO}_{4}$\\
C) $\mathrm{NCl}_{3}$\\
D) $\mathrm{PF}_{5}$
  \item Quyidagi moddalar orasidan molekulasining fazoviy tuzulishi chiziqli bo'lganini toping.\\
A) $\mathrm{C}_{2} \mathrm{~N}_{2}$\\
B) $\mathrm{SO}_{3}$\\
C) $\mathrm{H}_{2} \mathrm{SO}_{4}$\\
D) $\mathrm{PCl}_{5}$
  \item Quyidagi moddalar orasidan molekulasining fazoviy tuzulishi chiziqli bo'lganini toping.\\
A) $\mathrm{H}_{2} \mathrm{O}$\\
B) $\mathrm{Fe}(\mathrm{CO})_{5}$\\
C) $\mathrm{BeBr}_{2}$\\
D) $\mathrm{SO}_{2}$
  \item Quyidagi moddalar orasidan molekulasining fazoviy tuzulishi chiziqli bo'lganini toping.\\
A) $\mathrm{H}_{2} \mathrm{O}$\\
B) $\mathrm{NH}_{3}$\\
C) $\mathrm{BeI}_{2}$\\
D) $\mathrm{SO}_{2}$
  \item Quyidagi moddalar orasidan molekulasining fazoviy tuzulishi uchburchak bo'lganini toping.\\
A) $\mathrm{H}_{2} \mathrm{O}$\\
B) $\mathrm{Fe}(\mathrm{CO})_{5}$\\
C) $\mathrm{BeCl}_{2}$\\
D) $\mathrm{SO}_{3}$\\
  \item Quyidagi moddalar orasidan molekulasining fazoviy tuzulishi uchburchak bo'lganini toping.\\
A) $\mathrm{C}_{2} \mathrm{H}_{2}$\\
B) $\mathrm{BF}_{3}$\\
C) $\mathrm{CH}_{4}$\\
D) $\mathrm{PCl}_{5}$
  \item Quyidagi moddalar orasidan molekulasining fazoviy tuzulishi uchburchak bo'lganini toping.\\
A) $\mathrm{H}_{2} \mathrm{O}$\\
B) $\mathrm{Fe}(\mathrm{CO})_{5}$\\
C) $\mathrm{BeF}_{2}$\\
D) $\mathrm{BCl}_{3}$
  \item Quyidagi moddalar orasidan molekulasining fazoviy tuzulishi uchburchak bo'lganini toping.\\
A) $\mathrm{HNO}_{3}$\\
B) $\mathrm{SiO}_{2}$\\
C) $\mathrm{K}_{2} \mathrm{CO}_{3}$\\
D) $\mathrm{PH}_{3}$
  \item Quyidagi moddalar orasidan molekulasining fazoviy tuzulishi uchburchak bo'lganini toping.\\
A) $\mathrm{HClO}_{3}$\\
B) HCN\\
C) $\mathrm{AlCl}_{3}$\\
D) $\mathrm{K}_{3}\left[\mathrm{Fe}(\mathrm{CN})_{6}\right]$
  \item Quyidagi moddalar orasidan molekulasining fazoviy tuzulishi uchburchak bo'lganini toping.\\
A) $\mathrm{N}_{2} \mathrm{H}_{4}$\\
B) $\mathrm{BI}_{3}$\\
C) $\mathrm{HClO}_{4}$\\
D) KCN
  \item Quyidagi moddalar orasidan molekulasining fazoviy tuzulishi uchburchak bo'lganini toping.\\
A) $\mathrm{AlF}_{3}$\\
B) $\mathrm{HMnO}_{4}$\\
C) $\mathrm{NCl}_{3}$\\
D) $\mathrm{PF}_{5}$
  \item Quyidagi moddalar orasidan molekulasining fazoviy tuzulishi uchburchak bo'lganini toping.\\
A) $\mathrm{C}_{2} \mathrm{~N}_{2}$\\
B) $\mathrm{SO}_{3}$\\
C) $\mathrm{H}_{2} \mathrm{SO}_{4}$\\
D) $\mathrm{PCl}_{5}$
  \item Quyidagi moddalar orasidan molekulasining fazoviy tuzulishi uchburchak bo'lganini toping.\\
A) $\mathrm{BBr}_{3}$\\
B) $\mathrm{Fe}(\mathrm{CO})_{5}$\\
C) $\mathrm{BeBr}_{2}$\\
D) $\mathrm{SO}_{2}$
  \item Quyidagi moddalar orasidan molekulasining fazoviy tuzulishi uchburchak bo'lganini toping.\\
A) $\mathrm{H}_{2} \mathrm{O}$\\
B) $\mathrm{NH}_{3}$\\
C) $\mathrm{BI}_{3}$\\
D) $\mathrm{SO}_{2}$
  \item Quyidagi moddalar orasidan molekulasining fazoviy tuzulishi trigonal piramida bo'lganini toping.\\
A) $\mathrm{NH}_{3}$\\
B) $\mathrm{Fe}(\mathrm{CO})_{5}$\\
C) $\mathrm{BeCl}_{2}$\\
D) $\mathrm{SO}_{3}$\\
92. Quyidagi moddalar orasidan molekulasining fazoviy turulishi trigonal piramida bo'lganini toping.\\
A) $\mathrm{C}_{2} \mathrm{H}_{2}$\\
B) $\mathrm{BF}_{3}$\\
C) $\mathrm{CH}_{4}$\\
D) $\mathrm{PH}_{3}$\\
93. Quyidagi moddalar orasidan molekulasining fazoviy tuzulishi trigonal piramida bo'lganini toping.\\
A) $\mathrm{AsH}_{3}$\\
B) $\mathrm{Fe}(\mathrm{CO})_{0}$\\
C) $\mathrm{BeF}_{2}$\\
D) $\mathrm{BH}_{\mathrm{s}}$\\
94. Quyidagi moddalar orasidan molekulasining fazoviy tuzulishi trigonal piramida bo'lganini toping.\\
A) $\mathrm{HNO}_{3}$\\
B) $\mathrm{SiO}_{2}$\\
C) $\mathrm{K}_{3} \mathrm{CO}_{3}$\\
D) $\mathrm{PCl}_{3}$\\
95. Quyidagi moddalar orasidan molekulasining fazoviy tuzulishi trigonal piramida bo'lganini toping.\\
A) $\mathrm{HClO}_{9}$\\
B) $\mathrm{BF}_{3}$\\
C) $\mathrm{AlCl}_{3}$\\
D) $\mathrm{AsCl}_{3}$\\
96. Quyidagi moddalar orasidan molekulasining fazoviy tuzulishi trigonal piramida bo'lganini toping.\\
A) $\mathrm{NF}_{3}$\\
B) $\mathrm{Bl}_{3}$\\
C) $\mathrm{HClO}_{4}$\\
D) KCN\\
97. Quyidagi moddalar orasidan molekulasining fazoviy tuzulishi trigonal piramida bo'lganini toping.\\
A) $\mathrm{AlF}_{3}$\\
B) $\mathrm{HMnO}_{4}$\\
C) $\mathrm{NCl}_{3}$\\
D) $\mathrm{PF}_{5}$\\
98. Quyidagi moddalar orasidan molekulasining fazoviy tuzulishi trigonal piramida bo'lganini toping.\\
A) $\mathrm{PF}_{3}$\\
B) $\mathrm{SO}_{3}$\\
C) $\mathrm{H}_{2} \mathrm{SO}_{4}$\\
D) $\mathrm{PCl}_{5}$\\
99. Quyidagi moddalar orasidan molekulasining fazoviy tuzulishi trigonal piramida bo'lganini toping.\\
A) $\mathrm{BBr}_{3}$\\
B) $\mathrm{Fe}(\mathrm{CO})_{5}$\\
C) $\mathrm{BeBr}_{2}$\\
D) $\mathrm{H}_{3} \mathrm{O}^{+}$\\
100. Quyidagi moddalar orasidan molekulasining fazoviy tuzulishi trigonal piramida bo'lganini toping.\\
A) $\mathrm{H}_{2} \mathrm{O}$\\
B) $\mathrm{NH}_{3}$\\
C) $\mathrm{BI}_{3}$\\
D) $\mathrm{SO}_{2}$
  \item Quyidagi moddalar orasidan molokulasining fnxoviy tuzulishi totracodr bo'lganini toping.\\
A) $\mathrm{C}_{2} \mathrm{H}_{2}$\\
B) $\mathrm{BF}_{y}$\\
C) $\mathrm{SiH}_{4}$\\
D) $\mathrm{PH}_{0}$
  \item Quyidagi moddalar orasidan molokulasining fazoviy tuzulishi totraodr bo'lganini toping.\\
A) $\mathrm{AsH}_{5}$\\
B) $\mathrm{F}_{\mathrm{O}}(\mathrm{CO})_{6}$\\
C) $\mathrm{BoF}_{2}$\\
D) $\mathrm{NH}_{4}{ }^{+}$
  \item Quyidagi moddalar orasidan molekulasining fazoviy tuzulishi totraedr bo'lganini toping.\\
A) $\mathrm{HNO}_{3}$\\
B) $\mathrm{SO}_{4} \cdot{ }^{-2}$\\
C) $\mathrm{K}_{2} \mathrm{CO}_{3}$\\
D) $\mathrm{PCl}_{3}$
  \item Quyidagi moddalar orasidan molekulasining fazoviy tuzulishi totraodr bo'lganini toping.\\
A) $\mathrm{NF}_{3}$\\
B) $\mathrm{BI}_{3}$\\
C) $\mathrm{SiO}_{2}$\\
D) KCN
  \item Quyidagi moddalar orasidan molekulasining fazoviy tuzulishi tetraodr bo'lganini toping.\\
A) $\mathrm{AlF}_{9}$\\
B) $\mathrm{CCl}_{4}$\\
C) $\mathrm{NCl}_{3}$\\
D) $\mathrm{PF}_{6}$
  \item Quyidagi moddalar orasidan molekulasining fazoviy tuzulishi totraodr bo'lganini toping.\\
A) $\mathrm{PF}_{3}$\\
B) $\mathrm{SiF}_{4}$\\
C) $\mathrm{H}_{2} \mathrm{SO}_{4}$\\
D) $\mathrm{PCl}_{6}$
109. Quyidagi moddalar orasidan molekulasining fazoviy tuzulishi totraedr bo'lganini toping.\\
A) $\mathrm{CBr}_{4}$\\
B) $\mathrm{Fe}(\mathrm{CO})_{5}$\\
C) $\mathrm{BeBr}_{2}$\\
D) $\mathrm{H}_{3} \mathrm{O}^{+}$\\
110. Quyidagi moddalar orasidan molekulasining fazoviy tuzulishi tetraedr bo'lganini toping.\\
A) $\mathrm{H}_{2} \mathrm{O}$\\
B) $\mathrm{NH}_{3}$\\
C) $\mathrm{SiI}_{4}$\\
D) $\mathrm{SO}_{2}$
  \item Tarkibida $50 \mathrm{~g} \mathrm{Cu}, 150 \mathrm{~g} \mathrm{Zn}$ saqlagan qotishmadagi Zn ning massa ulushini \% da hisoblang.\\
A) 75\\
B) 50\\
C) 25\\
D) 60
  \item Tarkibida $60 \mathrm{~g} \mathrm{Cr}, 40 \mathrm{~g} \mathrm{Zn}$ saqlagan qotishmadagi Cr ning massa ulushini \% da hisoblang.\\
A) 75\\
B) 50\\
C) 25\\
D) 60
  \item Tarkibida $400 \mathrm{~g} \mathrm{Cu}, 100 \mathrm{~g}$ Fe saqlagan qotishmadagi Fe ning massa ulushini \% da hisoblang.\\
A) 25\\
B) 50\\
C) 20\\
D) 30
  \item Tarkibida $30 \mathrm{~g} \mathrm{Cr}, 70 \mathrm{~g}$ Fe saqlagan qotishmadagi Cr ning massa ulushini \% da hisoblang.\\
A) 25\\
B) 50\\
C) 20\\
D) 30
  \item Tarkibida $25 \mathrm{~g} \mathrm{Cu}, 75 \mathrm{~g} \mathrm{Zn}$ saqlagan qotishmadagi Zn ning massa ulushini \% da hisoblang.\\
A) 75\\
B) 50\\
C) 25\\
D) 60
  \item Tarkibida $20 \mathrm{~g} \mathrm{Au}, 180 \mathrm{~g} \mathrm{Cu}$ saqlagan qotishmadagi Au ning massa ulushini \% da hisoblang.\\
A) 40\\
B) 10\\
C) 20\\
D) 30
  \item Tarkibida $60 \mathrm{~g} \mathrm{Al}, 140 \mathrm{~g} \mathrm{Zn}$ saqlagan qotishmadagi Zn ning massa ulushini \% da hisoblang.\\
A) 70\\
B) 60\\
C) 50\\
D) 40
  \item Tarkibida $90 \mathrm{~g} \mathrm{Ni}, 10 \mathrm{~g} \mathrm{Cd}$ saqlagan qotishmadagi Cd ning massa ulushini \% da hisoblang.\\
A) 40\\
B) 10\\
C) 20\\
D) 30
  \item Tarkibida $80 \mathrm{~g} \mathrm{Cr}, 120 \mathrm{~g}$ Ni saqlagan qotishmadagi Ni ning massa ulushini \% da hisoblang.\\
A) 75\\
B) 50\\
C) 25\\
D) 60
  \item Tarkibida $91 \mathrm{~g} \mathrm{Ni}, 9 \mathrm{~g} \mathrm{Zn}$ saqlagan qotishmadagi Zn ning massa ulushini \% da hisoblang.\\
A) 11\\
B) 8\\
C) 10\\
D) 9
  \item Qotishma tarkibida metallarning mol nisbati quyidagicha berilgan $\mathrm{Cu}: \mathrm{Zn}: \mathrm{Fe}$ mos ravishda 1:2:1 bo'lsa aralashmadagi Zn ning massa ulushuni hisoblang.\\
A) 52\\
B) 48\\
C) 45\\
D) 65
  \item Qotishma tarkibida metallarning mol nisbati quyidagicha berilgan $\mathrm{Cu}: \mathrm{Ca}: \mathrm{Fe}$ mos ravishda 1:2:1 bo'lsa aralashmadagi Fe ning massa ulushuni hisoblang.\\
A) 32\\
B) 28\\
C) 35\\
D) 55
  \item Qotishma tarkibida metallarning mol nisbati quyidagicha berilgan Ti:Ca:Cr mos ravishda $1: 2: 1$ bo'lsa aralashmadagi Ca ning massa ulushuni hisoblang.\\
A) 52,2\\
B) 48,8\\
C) 45,5\\
D) 44,4
  \item Qotishma tarkibida metallarning mol nisbati quyidagicha berilgan $\mathrm{Cd}: \mathrm{Cu}: \mathrm{Ca}$ mos ravishda 1:2:2 bo'lsa aralashmadagi Ca ning massa ulushuni hisoblang.\\
A) 30\\
B) 25\\
C) 40\\
D) 28
  \item Qotishma tarkibida metallarning mol nisbati quyidagicha berilgan $\mathrm{Ag}: \mathrm{Ca}: \mathrm{Fe}$ mos ravishda $1: 2: 2$ bo'lsa aralashmadagi Ag ning massa ulushuni hisoblang.\\
A) 43\\
B) 48\\
C) 44\\
D) 36
  \item Qotishma tarkibida metallarning mol nisbati quyidagicha berilgan $\mathrm{Al}: \mathrm{Fe}: \mathrm{Ca}$ mos ravishda $2: 1: 2,25$ bo'lsa aralashmadagi Fe ning massa ulushuni hisoblang.\\
A) 22\\
B) 28\\
C) 25\\
D) 26
  \item Qotishma tarkibida metallarning mol nisbati quyidagicha berilgan $\mathrm{Ti}: \mathrm{Cr}: \mathrm{Ca}$ mos ravishda $1: 1: 2,5$ bo'lsa aralashmadagi Ti ning massa ulushuni hisoblang.\\
A) 24\\
B) 28\\
C) 44\\
D) 46
  \item Qotishma tarkibida metallarning mol nisbati quyidagicha berilgan $\mathrm{Cu}: \mathrm{Zn}:$ Fe mos ravishda 1:2:1 bo'lsa aralashmadagi Cu ning massa ulushuni hisoblang.\\
A) 52,2\\
B) 14,8\\
C) 24,5\\
D) 25,6
  \item Qotishma tarkibida metallarning mol nisbati quyidagicha berilgan $\mathrm{Al}: \mathrm{Zn}^{:} \mathrm{Ca}$ mos ravishda 2:2:0,4 bo'lsa aralashmadagi Zn ning massa ulushuni hisoblang.\\
A) 52\\
B) 48\\
C) 45\\
D) 65
  \item Qotishma tarkibida metallarning mol nisbati quyidagicha berilgan $\mathrm{Al}: \mathrm{Zn}: \mathrm{Ca}$ mos ravishda 2:2:0,4 bo'lsa aralashmadagi Al ning massa ulushuni hisoblang.\\
A) 27\\
B) 24\\
C) 54\\
D) 47
  \item Qotishma tarkibida metallarning massa ulushlari nisbati quyidagicha berilgan $\mathrm{Al}: \mathrm{Zn}$ mos ravishda 54:65 nisbatda bo'lsa qotishma tarkibini ko'rsating.\\
A) $\mathrm{Al}_{2} \mathrm{Zn}$\\
B) $\mathrm{AlZn}_{2}$\\
C) $\mathrm{Al}_{3} \mathrm{Zn}$\\
D) $\mathrm{AlZn}_{3}$
22. Qotishma tarkibida metallarning massa ulushlari nisbati quyidagicha berilgan Cu :Fe mos ravishda 128:168 nisbatda bo'lsa qotishma tarkibini ko'rsating.\\
A) $\mathrm{CuFe}_{3}$\\
B) $\mathrm{Cu}_{2} \mathrm{Fe}$\\
C) $\mathrm{Cu}_{2} \mathrm{Fe}_{3}$\\
D) $\mathrm{CuFe}_{2}$\\
23. Qotishma tarkibida metallarning massa ulushlari nisbati quyidagicha berilgan Cr:Zn mos ravishda 52:130 nisbatda bo'lsa qotishma tarkibini ko'rsating.\\
A) $\mathrm{Cr}_{2} \mathrm{Zn}$\\
B) $\mathrm{CrZn}_{2}$\\
C) $\mathrm{Cr}_{3} \mathrm{Zn}$\\
D) $\mathrm{CrZn}_{3}$\\
24. Qotishma tarkibida metallarning massa ulushlari nisbati quyidagicha berilgan Fe:Zn mos ravishda 112:195 nisbatda bo'lsa qotishma tarkibini ko'rsating.\\
A) $\mathrm{Fe}_{2} \mathrm{Zn}$\\
B) $\mathrm{FeZn}_{2}$\\
C) $\mathrm{Fe}_{3} \mathrm{Zn}$\\
D) $\mathrm{Fe}_{2} \mathrm{Zn}_{3}$\\
25. Qotishma tarkibida metallarning massa ulushlari nisbati quyidagicha berilgan Al:Fe mos ravishda 54:112 nisbatda bo'lsa qotishma tarkibini ko'rsating.\\
A) $AlFe$\\
B) $\mathrm{AlFe}_{2}$\\
C) $\mathrm{Al}_{3} \mathrm{Fe}$\\
D) $\mathrm{AlFe}_{3}$\\
26. Qotishma tarkibida metallarning massa ulushlari nisbati quyidagicha berilgan Ti:Cr mos ravishda 96:52 nisbatda bo'lsa qotishma tarkibini ko'rsating.\\
A) $\mathrm{Ti}_{2} \mathrm{Cr}_{3}$\\
B) $TiCr$\\
C) $\mathrm{Ti}_{2} \mathrm{Cr}_{4}$\\
D) $\mathrm{Ti}_{2} \mathrm{Cr}$\\
27. Qotishma tarkibida metallarning massa ulushlari nisbati quyidagicha berilgan $\mathrm{Ca}: \mathrm{Cr}$ mos ravishda 40:104 nisbatda bo'lsa qotishma tarkibini ko'rsating.\\
A) $\mathrm{Ca}_{2} \mathrm{Cr}$\\
B) $\mathrm{Ca}_{3} \mathrm{Cr}_{2}$\\
C) $\mathrm{CaCr}_{2}$\\
D) $\mathrm{Ca}_{2} \mathrm{Cr}_{3}$\\
28. Qotishma tarkibida metallarning massa ulushlari nisbati quyidagicha berilgan Fe:Mg mos ravishda 112:24 nisbatda bo'lsa qotishma tarkibini ko'rsating.\\
A) $\mathrm{Fe}_{2} \mathrm{Mg}$\\
B) $FeMg$\\
C) $\mathrm{Fe}_{2} \mathrm{Mg}_{3}$\\
D) $\mathrm{Fe}_{s} \mathrm{Mg}_{2}$\\
29. Qotishma tarkibida metallarning massa ulushlari nisbati quyidagicha berilgan $\mathrm{Al}: \mathrm{Cu}$ mos ravishda $54: 128$ nisbatda bo'lsa qotishma tarkibini ko'rsating.\\
A) $\mathrm{Al}_{2} \mathrm{Cu}$\\
B) $AlCu$\\
C) $\mathrm{Al}_{2} \mathrm{Cu}_{8}$\\
D) $\mathrm{Al}_{2} \mathrm{Cu}_{4}$\\
30. Qotishma tarkibida metallarning massa ulushlari nisbati quyidagicha berilgan Al:Ti mos ravishda 27:48 nisbatda bo'lsa qotishma tarkibini ko'rsating.\\
A) $\mathrm{Al}_{2} \mathrm{Ti}$\\
B) $\mathrm{AlTi}_{2}$\\
C) $\mathrm{Al}_{3} \mathrm{Ti}$\\
D) $AlTi$
  \item Tarkibi massa jihatidan $60 \%$ Cu va qolgan qismi Zn dan iborat 120 g qotishmaga qancha g Zn qo'shsak $\mathrm{Cu}_{3} \mathrm{Zn}_{5}$ tarkibli qotishmaga aylanadi?\\
A) 74\\
B) 48\\
C) 65\\
D) 44
  \item Tarkibi massa jihatidan $56 \%$ Fe va qolgan qismi Sr dan iborat 100 g qotishmaga quncha g Sr qo'shsak $\mathrm{Fo}_{2} \mathrm{Sr}_{4}$ tarkibli qotishmaga aylanadi?\\
A) 84\\
B) 88\\
C) 85\\
D) 44
  \item Tarkibi massa jihatidan $65 \% \mathrm{Zn}$ val qolgan qismi Fe dan iborat 160 g qotishmaga qancha g Fo qo'shsak FoaZny tarkibli qotishmagn aylanadi?\\
A) 74,3\\
B) 48,4\\
C) 65,2\\
D) 78,4
  \item Tarkibi massa jihatidan $52 \% \mathrm{Cr}$ va qolgan qismi Cu dan iborat 120 g\\
qotishmaga qancha g Cu qo`shsak $\mathrm{Cr}_{2} \mathrm{Cu}_{5}$ tarkibli qotishmagn nylanadi?\\
A) 134,4\\
B) 122.4\\
C) 65,6\\
D) 44.6
  \item Tarkibi massa jihatidan $56 \% \mathrm{Cd}$ va qolgan qismi Fe dan iborat 200 g qotishmaga qancha g Fe qo'shsak $\mathrm{Fe}_{3} \mathrm{Cd}$ tarkibli qotishmaga aylanadi?\\
A) 120\\
B) 40\\
C) 80\\
D) 56
  \item Tarkibi massa jihatidan $24 \% \mathrm{Mg}$ va qolgan qismi Cr dan iborat 150 g qotishmaga qancha g Cr qo'shsak $\mathrm{MgaCr}_{3}$ tarkibli qotishmaga aylanadi?\\
A) 16\\
B) 24\\
C) 36\\
D) 44
  \item Tarkibi massa jihatidan $56 \% \mathrm{Cd}$ va qolgan qismi Zn dan iborat 300 g qotishmaga qancha g Zn qo'shsak $\mathrm{Cd}_{3} \mathrm{Zn}_{\mathrm{s}}$ tarkibli qotishmaga aylanadi?\\
A) 74\\
B) 128\\
C) 115\\
D) 44
  \item Tarkibi massa jihatidan $64 \% \mathrm{Cu}$ va qolgan qismi Zn dan iborat 300 g qotishmaga qancha g Zn qo'shsak Cus $\mathrm{Zn}_{\Delta}$ tarkibli qotishmaga aylanadi?\\
A) 274\\
B) 248\\
C) 265\\
D) 217
  \item Tarkibi massa jihatidan $85,6 \% \mathrm{Ag}$ va qolgan qismi Fe dan iborat 100 g qotishmaga quncha g Fe qo`shsak AgyFes tarkibli qotishmaga aylanadi?\\
A) 96.6\\
B) 48.4\\
C) 65.4\\
D) $5 \cdot 4,4$
  \item Tarkibi massa jihatidau $54 \% \mathrm{Al}$ va qolgan qismi Zn dan iborat 150 g qotishmaga quacha g Zn qo'shsak AlsZns taxkibli qotishmaga aylanadi?\\
A) 27.1\\
B) 2.18\\
C) 265\\
D) 256
  \item Fo, $\mathrm{Cr}, \mathrm{Ca}$ dan iborat 92 gr qotishmada Featomlari soni Cr atomlaridan 2 marta ko'p Ca atomlari soni osa Cr atomlaridan 2 martaga kam bo'lsa, qotishma tarkibida nocha ge Ca bor?\\
A) 40\\
B) 30\\
C) 20\\
D) 10
42. Fe, Cr, Ca dan iborat 92 gr qotishmada Fe atomlari soni Cr atomlaridan 2 marta ko'p Ca atomlari soni esa Cr atomlaridan 2 martaga kam bo'lsa, qotishma tarkibida necha g Fe bor?\\
A) 56\\
B) 22.4\\
C) 28\\
D) 14\\
43. $\mathrm{Fe}, \mathrm{Cr}, \mathrm{Ca}$ dan iborat 92 gr qotishmada Fe atomlari soni Cr atomlaridan 2 marta ko'p Ca atomlari soni esa Cr atomlaridan 2 martaga kam bo'lsa, qotishma tarkibida necha g Cr bor?\\
A) 52\\
B) 26\\
C) 13\\
D) 28\\
44. Fe, Cr. Ca dan iborat 138 gr qotishmada Fe atomlari soni Cr atomlaridan 2 marta ko'p Ca atomlari soni esa Cr atomlaridan 2 martaga kam bo'lsa, qotishma tarkibida necha g Fe bor?\\
A) 28\\
B) 56\\
C) 112\\
D) 84\\
45. Fe, Cr, Ca dan iborat 138 gr qotishmada Fe atomlari soni Cr atomlaridan 2 marta ko'p Ca atomlari soni esa Cr atomlaridan 2 martaga kam bo'lsa, qotishma tarkibida necha g Ca bor?\\
A) 15\\
B) 20\\
C) 25\\
D) 30\\
46. $\mathrm{Fe}, \mathrm{Cr}, \mathrm{Ca}$ dan iborat 138 gr qotishmada Fe atomlari soni Cr atomlaridan 2 marta ko'p Ca atomlari soni esa Cr atomlaridan 2 martaga kam bo'lsa, qotishma tarkibida necha g Cr bor?\\
A) 52\\
B) 26\\
C) 39\\
D) 104\\
47. Fe, Cr, Ca dan iborat 184 gr qotishmada Fe atomlari soni Cr atomlaridan 2 marta ko'p Ca atomlari soni esa Cr atomlaridan 2 martaga kam bo'lsa, qotishma tarkibida necha g Cr bor?\\
A) 52\\
B) 26\\
C) 13\\
D) 28\\
48. $\mathrm{Fe}, \mathrm{Cr}, \mathrm{Ca}$ dan iborat 184 gr qotishmada Fe atomlari soni Cr atomlaridan 2 marta ko'p Ca atomlari soni esa Cr atomlaridan 2 martaga kam bo'lsa, qotishma tarkibida necha g Ca bor?\\
A) 40\\
B) 60\\
C) 20\\
D) 80\\
49. Fe, Cr, Ca dan iborat 184 gr qotishmada Fe atomlari soni Cr\\
atomlaridan 2 marta ko'p Ca atomlari soni esa Cr atomlaridan 2 martaga kam bo'lsa, qotishma tarkibida necha g Fe bor?\\
A) 56\\
B) 112\\
C) 14\\
D) 28\\
50. Fe, Cr, Ca dan iborat 230 gr qotishmada Fe atomlari soni Cr atomlaridan 2 marta ko'p Ca atomlari soni esa Cr atomlaridan 2 martaga kam bo'lsa, qotishma tarkibida necha g Cr bor?\\
A) 52\\
B) 130\\
C) 13\\
D) 65
  \item 130 g suvga 70 g eruvchan tuz eritildi. Hosil bo'lgan eritmaning foiz konsentratsiyasini hisoblang.\\
A) 35\\
B) 40\\
C) 60\\
D) 50\\
  \item 80 g suvga 120 g eruvchan tuz eritildi. Hosil bo'lgan eritmaning foiz konsentratsiyasini hisoblang.\\
A) 35\\
B) 40\\
C) 60\\
D) 50
  \item 120 g suvga 80 g eruvchan tuz eritildi. Hosil bo'lgan eritmaning foiz konsentratsiyasini hisoblang.\\
A) 35\\
B) 40\\
C) 60\\
D) 50
  \item 50 g suvga 50 g eruvchan tuz eritildi. Hosil bo'lgan eritmaning foiz konsentratsiyasini hisoblang.\\
A) 35\\
B) 40\\
C) 60\\
D) 50
  \item 86 g suvga 14 g eruvchan tuz eritildi. Hosil bo'lgan eritmaning foiz konsentratsiyasini hisoblang.\\
A) 22\\
B) 14\\
C) 16\\
D) 28
  \item 150 g suvga 50 g eruvchan tuz eritildi. Hosil bo'lgan eritmaning foiz konsentratsiyasini hisoblang.\\
A) 35\\
B) 40\\
C) 25\\
D) 50
  \item 90 g suvga 110 g eruvchan tuz eritildi. Hosil bo'lgan eritmaning foiz konsentratsiyasini hisoblang.\\
A) 55\\
B) 45\\
C) 60\\
D) 50
  \item 190 g suvga 10 g eruvchan tuz eritildi. Hosil bo'lgan eritmaning foiz konsentratsiyasini hisoblang.\\
A) 15\\
B) 6\\
C) 10\\
D) 5
  \item 110 g suvga 90 g eruvchan tuz eritildi. Hosil bo'lgan eritmaning foiz konsentratsiyasini hisoblang.\\
A) 35\\
B) 45\\
C) 65\\
D) 55
  \item 180 g suvga 20 g eruvchan tuz eritildi. Hosil bo'lgan eritmaning foiz konsentratsiyasini hisoblang.\\
A) 35\\
B) 20\\
C) 10\\
D) 30
  \item 107 g suvga $1,5 \mathrm{~mol} \mathrm{Na}_{2} \mathrm{O}$ eritildi.
Hosil bo'lgan eritmaning foiz\\
konsentratsiyasini hisoblang.\\
A) 35\\
B) 40\\
C) 60\\
D) 50\\
  \item 60 g suvga $0,5 \mathrm{~mol} \mathrm{SO}_{3}$ eritildi. Hosil bo'lgan eritmaning foiz konsentratsiyasini hisoblang.\\
A) 49\\
B) 45\\
C) 60\\
D) 50
  \item 53 g suvga $0,5 \mathrm{~mol} \mathrm{~K}_{2} \mathrm{O}$ eritildi. Hosil bo'lgan eritmaning foiz konsentratsiyasini hisoblang.\\
A) 40\\
B) 53\\
C) 47\\
D) 56
  \item 136 g suvga $1 \mathrm{~mol} \mathrm{SO}_{2}$ eritildi. Hosil bo'lgan eritmaning foiz konsentratsiyasini hisoblang.\\
A) 41\\
B) 32\\
C) 68\\
D) 52
  \item 255 g suvga $1,5 \mathrm{~mol} \mathrm{Li}_{2} \mathrm{O}$ eritildi. Hosil bo'lgan eritmaning foiz konsentratsiyasini hisoblang.\\
A) 24\\
B) 22,5\\
C) 65\\
D) 45
  \item 58 g suvga $1 \mathrm{~mol} \mathrm{P}_{2} \mathrm{O}_{5}$ eritildi. Hosil bo'lgan eritmaning foiz konsentratsiyasini hisoblang.\\
A) 35\\
B) 49\\
C) 71\\
D) 98
  \item 169 g suvga $0,5 \mathrm{~mol} \mathrm{Na}_{2} \mathrm{O}$ eritildi.
Hosil bo'lgan eritmaning foiz konsentratsiyasini hisoblang.\\
A) 15,5\\
B) 20\\
C) 16\\
D) 25\\
18. 150 g suvga $0,5 \mathrm{~mol} \mathrm{CrO}_{3}$ eritildi. Hosil bo'lgan eritmaning foiz konsentratsiyasini hisoblang.\\
A) 35\\
B) 29,5\\
C) 59\\
D) 25\\
19. 92 g suvga $1 \mathrm{~mol} \mathrm{~N}_{2} \mathrm{O}_{5}$ eritildi. Hosil bo'lgan eritmaning foiz konsentratsiyasini hisoblang.\\
A) 54\\
B) 45\\
C) 63\\
D) 50\\
20. 76 g suvga $2 \mathrm{~mol} \mathrm{Na}_{2} \mathrm{O}$ eritildi. Hosil bo'lgan eritmaning foiz konsentratsiyasini hisoblang.\\
A) 80\\
B) 40\\
C) 62\\
D) 50
  \item 127 g suvga 44,8 litr n .sh da o'lchangan HCl eritildi. Eritmaning foiz konsentratsiyasini hisoblang.\\
A) 36,5\\
B) 40\\
C) 73\\
D) 50
22. $328,5 \mathrm{~g}$ suvga 22,4 litr metallda o'lchangan HCl eritildi. Eritmaning foiz konsentratsiyasini hisoblang.\\
A) 40\\
B) 30\\
C) 10\\
D) 20\\
23. 324 g suvga 22,4 litr metallda o'lchangan HBr eritildi. Eritmaning foiz konsentratsiyasini hisoblang.\\
A) 40\\
B) 30\\
C) 10\\
D) 20\\
24. $94,5 \mathrm{~g}$ suvga 11,2 litr metallda o'lchangan HBr eritildi. Eritmaning foiz konsentratsiyasini hisoblang.\\
A) 40\\
B) 30\\
C) 10\\
D) 20\\
25. 128 g suvga 22,4 litr metallda o'lchangan HI eritildi. Eritmaning foiz konsentratsiyasini hisoblang.\\
A) 36,5\\
B) 40\\
C) 73\\
D) 50\\
26. 72 g suvga 22,4 litr metallda o'lchangan HI eritildi. Eritmaning foiz konsentratsiyasini hisoblang.\\
A) 12,8\\
B) 32\\
C) 64\\
D) 48\\
27. 60 g suvga 44,8 litr metallda o'lchangan HF eritildi. Eritmaning foiz konsentratsiyasini hisoblang.\\
A) 36,5\\
B) 40\\
C) 73\\
D) 50\\
28. 180 g suvga 22,4 litr metallda o'lchangan HF eritildi. Eritmaning foiz konsentratsiyasini hisoblang.\\
A) 40\\
B) 30\\
C) 10\\
D) 20\\
29. 166 g suvga 44,8 litr metallda o'lchangan $\mathrm{NH}_{3}$ eritildi. Eritmadagi\\
ammiakning foiz konsentratsiyasini hisoblang.\\
A) 34\\
B) 8,5\\
C) 17\\
D) 68\\
30. 33 g suvga 22,4 litr metallda o'lchangan $\mathrm{NH}_{3}$ eritildi. Eritmadagi ammiakning foiz konsentratsiyasini hisoblang.\\
A) 34\\
B) 8.5\\
C) 17\\
D) 68
  \item $200 \mathrm{~g} 60 \%$ li tuz eritmasidagi tuz va suvning massalari ayirmasini hisoblang.\\
A) 40\\
B) 30\\
C) 10\\
D) 20
32. $150 \mathrm{~g} 40 \%$ li tuz eritmasidagi tuz va suvning massalari ayirmasini hisoblang.\\
A) 40\\
B) 30\\
C) 10\\
D) 20\\
33. $300 \mathrm{~g} 30 \%$ li tuz eritmasidagi tuz va suvning massalari ayirmasini hisoblang.\\
A) 140\\
B) 130\\
C) 120\\
D) 200\\
34. $140 \mathrm{~g} 20 \%$ li tuz eritmasidagi tuz va suvning massalari ayirmasini hisoblang.\\
A) 72\\
B) 50\\
C) 64\\
D) 84\\
35. $250 \mathrm{~g} 25 \%$ li tuz eritmasidagi tuz va suvning massalari ayirmasini hisoblang.\\
A) 130\\
B) 125\\
C) 110\\
D) 120\\
36. $400 \mathrm{~g} 35 \%$ li tuz eritmasidagi tuz va suvning massalari ayirmasini hisoblang.\\
A) 130\\
B) 125\\
C) 110\\
D) 120\\
37. $120 \mathrm{~g} 60 \%$ li tuz eritmasidagi tuz va suvning massalari ayirmasini hisoblang.\\
A) 48\\
B) 32\\
C) 24\\
D) 28\\
38. $160 \mathrm{~g} 20 \%$ li tuz eritmasidagi tuz va suvning massalari ayirmasini hisoblang.\\
A) 96\\
B) 36\\
C) 74\\
D) 27\\
39. $350 \mathrm{~g} 20 \%$ li tuz eritmasidagi tuz va suvning massalari ayirmasini hisoblang.\\
A) 240\\
B) 230\\
C) 210\\
D) 220\\
40. $140 \mathrm{~g} 25 \%$ li tuz eritmasidagi tuz va suvning massalari ayirmasini hisoblang.\\
A) 40\\
B) 70\\
C) 50\\
D) 60
  \item $110 \mathrm{~g} 50 \%$ li va $90 \mathrm{~g} 20 \%$ li eritmalar aralashtirildi. Hosil bo'lgan eritmaning foiz konsentratsiyasini hisoblang.\\
A) 36,5\\
B) 30\\
C) 44,2\\
D) 40
  \item $200 \mathrm{~g} 40 \%$ li va $200 \mathrm{~g} 60 \%$ li eritmalar aralashtirildi. Hosil bo'lgan eritmaning foiz konsentratsiyasini hisoblang.\\
A) 50\\
B) 30\\
C) 20\\
D) 40
  \item $120 \mathrm{~g} 20 \%$ li va $80 \mathrm{~g} 50 \%$ li eritmalar aralashtirildi. Hosil bo'lgan eritmaning foiz konsentratsiyasini hisoblang.\\
A) 36\\
B) 32\\
C) 48\\
D) 44
  \item $60 \mathrm{~g} 30 \%$ li va $40 \mathrm{~g} 25 \%$ li eritmalar aralashtirildi. Hosil bo'lgan eritmaning foiz konsentratsiyasini hisoblang.\\
A) 25\\
B) 24\\
C) 28\\
D) 22
  \item $55 \mathrm{~g} 40 \%$ li va $45 \mathrm{~g} 20 \%$ li eritmalar aralashtirildi. Hosil bo'lgan eritmaning foiz konsentratsiyasini hisoblang.\\
A) 33\\
B) 31\\
C) 34\\
D) 32
  \item $30 \mathrm{~g} 28 \%$ li va $70 \mathrm{~g} 15 \%$ li eritmalar aralashtirildi. Hosil bo'lgan eritmaning foiz konsentratsiyasini hisoblang.\\
A) 26,5\\
B) 20\\
C) 18\\
D) 25
  \item $44 \mathrm{~g} 25 \%$ li va $56 \mathrm{~g} 50 \%$ li eritmalar aralashtirildi. Hosil bo'lgan eritmaning foiz konsentratsiyasini hisoblang.\\
A) 36\\
B) 39\\
C) 44\\
D) 40
  \item $20 \mathrm{~g} 50 \%$ li va $80 \mathrm{~g} 80 \%$ li eritmalar aralashtirildi. Hosil bo'lgan eritmaning foiz konsentratsiyasini hisoblang.\\
A) 74\\
B) 68\\
C) 56\\
D) 70
  \item $110 \mathrm{~g} 20 \%$ li va $90 \mathrm{~g} 30 \%$ li eritmalar aralashtirildi. Hosil bo'lgan eritmaning foiz konsentratsiyasini hisoblang.\\
A) 36,5\\
B) 30\\
C) 24,5\\
D) 40
  \item $10 \mathrm{~g} 50 \%$ li va $90 \mathrm{~g} 20 \%$ li eritmalar aralashtirildi. Hosil bo'lgan eritmaning foiz konsentratsiyasini hisoblang.\\
A) 27\\
B) 23\\
C) 22\\
D) 25
  \item $200 \mathrm{~g} 20 \%$ li ishqor eritmasiga necha g $50 \%$ eritma qo'shilganda $30 \%$ li eritma hosil bo'ladi?\\
A) 100\\
B) 150\\
C) 50\\
D) 60\\
  \item $150 \mathrm{~g} 30 \%$ li ishqor eritmasiga necha g $40 \%$ eritma qo'shilganda $32 \%$ li eritma hosil bo'ladi?\\
A) 42\\
B) 25,6\\
C) 37,5\\
D) 18,4
  \item $300 \mathrm{~g} 10 \%$ li ishqor eritmasiga necha g 80 \% eritma qo'shilganda 40 \% li eritma hosil bo'ladi?\\
A) 150\\
B) 225\\
C) 400\\
D) 260
  \item $250 \mathrm{~g} 15 \%$ li ishqor eritmasiga necha g 40 \% eritma qo'shilganda, 20 \% li eritma hosil bo'ladi?\\
A) 44,6\\
B) 74,3\\
C) 55,4\\
D) 62,5
  \item $400 \mathrm{~g} 10 \%$ li ishqor eritmasiga necha g $70 \%$ eritma qo'shilganda $40 \%$ li eritma hosil bo'ladi?\\
A) 400\\
B) 150\\
C) 500\\
D) 300
  \item $180 \mathrm{~g} 50 \%$ li ishqor eritmasiga necha g $30 \%$ eritma qo'shilganda $25 \%$ li eritma hosil bo'ladi?\\
A) 600\\
B) 500\\
C) 900\\
D) 400
  \item $150 \mathrm{~g} 35 \%$ li ishqor eritmasiga necha g $50 \%$ eritma qo'shilganda $40 \%$ li eritma hosil bo'ladi?\\
A) 75\\
B) 50\\
C) 90\\
D) 40
  \item $300 \mathrm{~g} 20 \%$ li ishqor eritmasiga necha g $50 \%$ eritma qo'shilganda $40 \%$ li eritma hosil bo'ladi?\\
A) 100\\
B) 150\\
C) 500\\
D) 600
  \item $150 \mathrm{~g} 25 \%$ li ishqor eritmasiga necha g $50 \%$ eritma qo'shilganda $35 \%$ li eritma hosil bo'ladi?\\
A) 100\\
B) 150\\
C) 50\\
D) 60
  \item $200 \mathrm{~g} 5 \%$ li ishqor eritmasiga necha g 50 \% eritma qo'shilganda $30 \%$ li eritma hosil bo'ladi?\\
A) 200\\
B) 250\\
C) 150\\
D) 160
  \item $300 \mathrm{gr} 50 \%$ li eritma tayyorlash uchun necha g dan $60 \%$ li va $30 \%$ li eritmalarni áralashtirish kerak?\\
A) $200 ; 100$\\
B) $250 ; 50$\\
C) $150 ; 150$\\
D) $160 ; 140$
  \item $250 \mathrm{gr} 40 \%$ li eritma tayyorlash uchun necha g dan $60 \%$ li va $20 \%$ li eritmalarni aralashtirish kerak?\\
A) $150 ; 100$\\
B) $100 ; 150$\\
C) $125 ; 125$\\
D) $110 ; 140$
  \item $400 \mathrm{gr} 20 \%$ li eritma tayyorlash uchun necha g dan $50 \%$ li va $10 \%$ li eritmalarni aralashtirish kerak?\\
A) $200 ; 200$\\
B) $250 ; 150$\\
C) $150 ; 250$\\
D) $100 ; 300$
  \item $150 \mathrm{gr} 25 \%$ li eritma tayyorlash uchun necha g dan $35 \%$ li va $15 \%$ li eritmalarni aralashtirish kerak?\\
A) $100 ; 50$\\
B) $80 ; 70$\\
C) $75 ; 75$\\
D) $90 ; 60$
  \item $360 \mathrm{gr} 50 \%$ li eritma tayyorlash uchun necha $g$ dan $70 \%$ li va $30 \%$ li eritmalarni aralashtirish kerak?\\
A) $250 ; 110$\\
B) $180 ; 180$\\
C) $150 ; 210$\\
D) $200 ; 160$
  \item $180 \mathrm{gr} 30 \%$ li eritma tayyorlash uchun necha g dan $40 \%$ li va $25 \%$ li eritmalarni aralashtirish kerak?\\
A) $120 ; 60$\\
B) $60 ; 120$\\
C) $80 ; 100$\\
D) $100 ; 80$
  \item $800 \mathrm{gr} 30 \%$ li eritma tayyorlash uchun necha g dan $20 \%$ li va $60 \%$ li eritmalarni aralashtirish kerak?\\
A) $200 ; 600$\\
B) $400 ; 400$\\
C) $600 ; 200$\\
D) $500 ; 300$
  \item $600 \mathrm{gr} 30 \%$ li eritma tayyorlash uchun necha g dan 60 \% li va 20 \% li eritmalarni aralashtirish kerak?\\
A) $200 ; 400$\\
B) $150 ; 450$\\
C) $450 ; 150$\\
D) $260 ; 340$
  \item $340 \mathrm{gr} 44 \%$ li eritma tayyorlash uchun necha g dan $64 \%$ li va $30 \%$ li eritmalarni aralashtirish kerak?\\
A) $140 ; 200$\\
B) $200 ; 140$\\
C) $160 ; 180$\\
D) $170 ; 170$
  \item $450 \mathrm{gr} 50 \%$ li eritma tayyorlash uchun necha g dan $60 \%$ li va $30 \%$ li eritmalarni aralashtirish kerak?\\
A) $200 ; 250$\\
B) $250 ; 200$\\
C) $300 ; 150$\\
D) $225 ; 225$
  \item $100 \mathrm{ml} 20 \% \mathrm{li}, \mathrm{d}=1,2 \mathrm{~g} / \mathrm{ml}$ bo'lgan eritmaga $120 \mathrm{ml} 40 \% \mathrm{li}, \mathrm{d}=1,5 \mathrm{~g} / \mathrm{ml}$ bo'lgan eritma aralashtirilsa necha \% li eritma hosil bo'ladi?\\
A) 32\\
B) 25\\
C) 36\\
D) 30
72. $250 \mathrm{ml} 20 \% \mathrm{li}, \mathrm{d}=1 \mathrm{~g} / \mathrm{ml}$ bo'lgan eritmaga $140 \mathrm{ml} 40 \% \mathrm{li}, \mathrm{d}=1,5 \mathrm{~g} / \mathrm{ml}$ bo'lgan eritma aralashtirilsa necha \% li eritma hosil bo'ladi?\\
A) 35\\
B) 25\\
C) 29\\
D) 30\\
73. $400 \mathrm{ml} 30 \% \mathrm{li}, \mathrm{d}=1,4 \mathrm{~g} / \mathrm{ml}$ bo'lgan eritmaga $200 \mathrm{ml} 20 \% \mathrm{li}, \mathrm{d}=1,2 \mathrm{~g} / \mathrm{ml}$ bo'lgan eritma aralashtirilsa necha \% li eritma hosil bo'ladi?\\
A) 25\\
B) 27\\
C) 23\\
D) 24\\
74. $300 \mathrm{ml} 50 \% \mathrm{li}, \mathrm{d}=1,2 \mathrm{~g} / \mathrm{ml}$ bo'lgan eritmaga $220 \mathrm{ml} 40 \% \mathrm{li}, \mathrm{d}=1,5 \mathrm{~g} / \mathrm{ml}$ bo'lgan eritma aralashtirilsa necha \% li eritma hosil bo'ladi?\\
A) 40\\
B) 48\\
C) 45,2\\
D) 42,5\\
75. $500 \mathrm{ml} 35 \% \mathrm{li}, \mathrm{d}=1 \mathrm{~g} / \mathrm{ml}$ bo'lgan eritmaga $250 \mathrm{ml} 25 \% \mathrm{li}, \mathrm{d}=1,5 \mathrm{~g} / \mathrm{ml}$ bo'lgan eritma aralashtirilsa necha \% li eritma hosil bo'ladi?\\
A) 30,7\\
B) 28,8\\
C) 36\\
D) 34\\
76. $120 \mathrm{ml} 20 \%$ li, $\mathrm{d}=1,6 \mathrm{~g} / \mathrm{ml}$ bo'lgan eritmaga $140 \mathrm{ml} 40 \% \mathrm{li}, \mathrm{d}=1,2 \mathrm{~g} / \mathrm{ml}$\\
bo'lgan eritma aralashtirilsa necha \% li eritma hosil bo'ladi?\\
A) 32,22\\
B) 29,33\\
C) 28,88\\
D) 27,77\\
77. $180 \mathrm{ml} 25 \% \mathrm{li}, \mathrm{d}=1,5 \mathrm{~g} / \mathrm{ml}$ bo'lgan eritmaga $100 \mathrm{ml} 30 \% \mathrm{li}, \mathrm{d}=1,1 \mathrm{~g} / \mathrm{ml}$ bo'lgan eritma aralashtirilsa necha \% li eritma hosil bo'ladi?\\
A) 27,22\\
B) 28,88\\
C) 26,44\\
D) 29\\
78. $400 \mathrm{ml} 20 \% \mathrm{li}, \mathrm{d}=1,2 \mathrm{~g} / \mathrm{ml}$ bo'lgan eritmaga $420 \mathrm{ml} 40 \% \mathrm{li}, \mathrm{d}=1,5 \mathrm{~g} / \mathrm{ml}$ bo'lgan eritma aralashtirilsa necha \% li eritma hosil bo'ladi?\\
A) 31,35\\
B) 25\\
C) 34,22\\
D) 30\\
79. $200 \mathrm{ml} 30 \%$ li, $\mathrm{d}=1,1 \mathrm{~g} / \mathrm{ml}$ bo'lgan eritmaga $220 \mathrm{ml} 40 \% \mathrm{li}, \mathrm{d}=1,2 \mathrm{~g} / \mathrm{ml}$ bo'lgan eritma aralashtirilsa necha \% li eritma hosil bo'ladi?\\
A) 32,22\\
B) 36,7\\
C) 37,77\\
D) 35,45\\
80. $200 \mathrm{ml} 20 \%$ li, $\mathrm{d}=1,2 \mathrm{~g} / \mathrm{ml}$ bo'lgan eritmaga $140 \mathrm{ml} 40 \% \mathrm{li}, \mathrm{d}=1,5 \mathrm{~g} / \mathrm{ml}$ bo'lgan eritma aralashtirilsa necha \% li eritma hosil bo'ladi?\\
A) 32,22\\
B) 29,33\\
C) 28,88\\
D) 27,77
  \item 3 M li $\mathrm{NaOH}(\mathrm{d}=1,2 \mathrm{~g} / \mathrm{ml})$ eritmasining \% konsentratsiyasini hisoblang.\\
A) 10\\
B) 25\\
C) 20\\
D) 30\\
  \item $4 \mathrm{M} \mathrm{li} \mathrm{HNO}{ }_{3}(\mathrm{~d}=1,26 \mathrm{~g} / \mathrm{ml})$ eritmasining \% konsentratsiyasini hisoblang.\\
A) 10\\
B) 25\\
C) 20\\
D) 30
  \item $2,5 \mathrm{M}$ li $\mathrm{KOH}(\mathrm{d}=1,12 \mathrm{~g} / \mathrm{ml})$ eritmasining \% konsentratsiyasini hisoblang.\\
A) 14\\
B) 12,5\\
C) 5,6\\
D) 30
  \item $1,5 \mathrm{M}$ li HF ( $\mathrm{d}=1,5 \mathrm{~g} / \mathrm{ml}$ ) eritmasining \% konsentratsiyasini hisoblang.\\
A) 7\\
B) 5\\
C) 2\\
D) 3
  \item $10 \mathrm{M} \mathrm{li} \mathrm{H}_{2} \mathrm{SO}_{4}(\mathrm{~d}=1,96 \mathrm{~g} / \mathrm{ml})$ eritmasining \% konsentratsiyasini hisoblang.\\
A) 50\\
B) 40\\
C) 20\\
D) 30
  \item $5 \mathrm{M} \mathrm{li} \mathrm{CH}_{3} \mathrm{COOH}(\mathrm{d}=1,2 \mathrm{~g} / \mathrm{ml})$ eritmasining \% konsentratsiyasini hisoblang.\\
A) 10\\
B) 25\\
C) 20\\
D) 30
  \item 8 M li LiOH ( $\mathrm{d}=1.5 \mathrm{~g} / \mathrm{ml}$ ) eritmasining \% konsentratsiyasini hisoblang?\\
A) 11,2\\
B) 25\\
C) 14,7\\
D) 12,8
  \item 6 M li $\mathrm{NH}_{3}(\mathrm{~d}=1,7 \mathrm{~g} / \mathrm{ml})$ eritmasining \% konsentratsiyasini hisoblang.\\
A) 10\\
B) 5\\
C) 6\\
D) 3
  \item 3 M li $\mathrm{NaOH}(\mathrm{d}=1,5 \mathrm{~g} / \mathrm{ml})$ eritmasining \% konsentratsiyasini hisoblang?\\
A) 8\\
B) 9\\
C) 10\\
D) 12
  \item 4 M li HF ( $\mathrm{d}=1,6 \mathrm{~g} / \mathrm{ml}$ ) eritmasining \% konsentratsiyasini hisoblang.\\
A) 10\\
B) 5\\
C) 2\\
D) 3
  \item 3 N li NaOH ( $\mathrm{d}=1,2 \mathrm{~g} / \mathrm{ml}$ ) eritmasining \% konsentratsiyasini hisoblang.\\
A) 10\\
B) 25\\
C) 20\\
D) 30\\
  \item $4 \mathrm{~N} \mathrm{li} \mathrm{HNO}_{3}(\mathrm{~d}=1,26 \mathrm{~g} / \mathrm{ml})$ eritmasining \% konsentratsiyasini hisoblang.\\
A) 10\\
B) 25\\
C) 20\\
D) 30
  \item $2,5 \mathrm{~N}$ li KOH ( $\mathrm{d}=1,12 \mathrm{~g} / \mathrm{ml}$ ) eritmasining \% konsentratsiyasini hisoblang.\\
A) 14\\
B) 12,5\\
C) 5,6\\
D) 30
  \item $1,5 \mathrm{~N}$ li HF ( $\mathrm{d}=1,5 \mathrm{~g} / \mathrm{ml}$ ) eritmasining \% konsentratsiyasini hisoblang.\\
A) 7\\
B) 5\\
C) 2\\
D) 3
  \item 10 N li $\mathrm{H}_{2} \mathrm{SO}_{4}(\mathrm{~d}=1,96 \mathrm{~g} / \mathrm{ml})$ eritmasining \% konsentratsiyasini hisoblang.\\
A) 50\\
B) 40\\
C) 25\\
D) 30
  \item 5 N li $\mathrm{CH}_{3} \mathrm{COOH}(\mathrm{d}=1,2 \mathrm{~g} / \mathrm{ml})$ eritmasining \% konsentratsiyasini hisoblang.\\
A) 10\\
B) 25\\
C) 20\\
D) 30
  \item 8 N li LiOH ( $\mathrm{d}=1.5 \mathrm{~g} / \mathrm{ml}$ ) eritmasining \% konsentratsiyasini hisoblang.\\
A) 11,2\\
B) 25\\
C) 14,7\\
D) 12,8
  \item $6 \mathrm{~N} \mathrm{li} \mathrm{NH}_{3}(\mathrm{~d}=1,7 \mathrm{~g} / \mathrm{ml})$ eritmasining \% konsentratsiyasini hisoblang.\\
A) 10\\
B) 5\\
C) 6\\
D) 3
  \item 3 N li $\mathrm{NaOH}(\mathrm{d}=1,5 \mathrm{~g} / \mathrm{ml})$ eritmasining \% konsentratsiyasini hisoblang.\\
A) 8\\
B) 9\\
C) 10\\
D) 12
  \item 4 N li HF ( $\mathrm{d}=1,6 \mathrm{~g} / \mathrm{ml}$ ) eritmasining \% konsentratsiyasini hisoblang.\\
A) 10\\
B) 5\\
C) 2\\
D) 3
  \item Tarkibida $3 \mathrm{~mol} \mathrm{H}_{2} \mathrm{SO}_{4}$ saqlagan 500 ml eritmaning molyar konsentratsiyasini toping.\\
A) 10\\
B) 5\\
C) 6\\
D) 3
  \item Tarkibida $2 \mathrm{~mol} \mathrm{H}_{2} \mathrm{CrO}_{4}$ saqlagan 400 ml eritmaning molyar konsentratsiyasini toping.\\
A) 10\\
B) 5\\
C) 6\\
D) 3
  \item Tarkibida $3 \mathrm{~mol} \mathrm{H}_{3} \mathrm{PO}_{4}$ saqlagan 300 ml eritmaning molyar konsentratsiyasini toping.\\
A) 10\\
B) 5\\
C) 6\\
D) 3
  \item Tarkibida $2,5 \mathrm{~mol} \mathrm{HNO}_{3}$ saqlagan 500 ml eritmaning molyar konsentratsiyasini toping.\\
A) 10\\
B) 5\\
C) 6\\
D) 3
  \item Tarkibida $5 \mathrm{~mol} \mathrm{HClO}_{4}$ saqlagan 500 ml eritmaning molyar konsentratsiyasini toping.\\
A) 10\\
B) 5\\
C) 6\\
D) 3
  \item Tarkibida $0,6 \mathrm{~mol} \mathrm{H}_{2} \mathrm{SO}_{3}$ saqlagan 200 ml eritmaning molyar konsentratsiyasini toping.\\
A) 10\\
B) 5\\
C) 6\\
D) 3
  \item Tarkibida $0,4 \mathrm{~mol} \mathrm{H}_{2} \mathrm{CO}_{3}$ saqlagan 400 ml eritmaning molyar konsentratsiyasini toping.\\
A) 1\\
B) 5\\
C) 6\\
D) 3
  \item Tarkibida 6 mol NaOH saqlagan 1000 ml eritmaning molyar konsentratsiyasini toping.\\
A) 10\\
B) 5\\
C) 6\\
D) 3
  \item Tarkibida $3 \mathrm{~mol} \mathrm{Na}_{2} \mathrm{SO}_{4}$ saqlagan 400 ml eritmaning molyar konsentratsiyasini toping.\\
A) 7,5\\
B) 5,5\\
C) 6\\
D) 3
  \item Tarkibida $2 \mathrm{~mol} \mathrm{MgSO}_{4}$ saqlagan 200 ml eritmaning molyar konsentratsiyasini toping.\\
A) 10\\
B) 5\\
C) 6\\
D) 3
  \item Tarkibida $98 \mathrm{~g} \mathrm{H}_{2} \mathrm{SO}_{4}$ saqlagan 500 ml eritmaning molyar konsentratsiyasini toping.\\
A) 1\\
B) 4\\
C) 2\\
D) 3\\
  \item Tarkibida $59 \mathrm{~g} \mathrm{H}_{2} \mathrm{CrO}_{4}$ saqlagan 400 ml eritmaning molyar konsentratsiyasini toping.\\
A) 1\\
B) 1,25\\
C) 2,5\\
D) 3
  \item Tarkibida $196 \mathrm{~g} \mathrm{H}_{3} \mathrm{PO}_{4}$ saqlagan 800 ml eritmaning molyar konsentratsiyasini toping.\\
A) 1\\
B) 1,25\\
C) 2,5\\
D) 3
  \item Tarkibida $63 \mathrm{~g} \mathrm{HNO}_{3}$ saqlagan 2000 ml eritmaning molyar konsentratsiyasini toping.\\
A) 10\\
B) 0,5\\
C) 6\\
D) 3
  \item Tarkibida $402 \mathrm{~g} \mathrm{HClO}_{4}$ saqlagan 800 ml eritmaning molyar konsentratsiyasini toping.\\
A) 10\\
B) 5\\
C) 6\\
D) 3
  \item Tarkibida $41 \mathrm{~g} \mathrm{H}_{2} \mathrm{SO}_{3}$ saqlagan 500 ml eritmaning molyar konsentratsiyasini toping.\\
A) 1\\
B) 5\\
C) 6\\
D) 3
  \item Tarkibida $31 \mathrm{~g} \mathrm{H}_{2} \mathrm{CO}_{3}$ saqlagan 400 ml eritmaning molyar konsentratsiyasini toping.\\
A) 1\\
B) 1,25\\
C) 2,5\\
D) 3
  \item Tarkibida 80 g NaOH saqlagan 1000 ml eritmaning molyar konsentratsiyasini toping.\\
A) 10\\
B) 4\\
C) 6\\
D) 2
  \item Tarkibida $284 \mathrm{~g} \mathrm{Na}_{2} \mathrm{SO}_{4}$ saqlagan 400 ml eritmaning molyar konsentratsiyasini toping.\\
A) 7\\
B) 5\\
C) 6\\
D) 3
  \item Tarkibida $180 \mathrm{~g} \mathrm{MgSO}_{4}$ saqlagan 300 ml eritmaning molyar konsentratsiyasini toping.\\
A) 10\\
B) 5\\
C) 6\\
D) 3
  \item 3 M li 300 ml NaOH eritmasiga necha ml suv qo'shganimizda 2 M li eritma hosil bo'ladi?\\
A) 150\\
B) 50\\
C) 200\\
D) 180
122. 4 M li 500 ml NaOH eritmasiga necha ml suv qo'shganimizda $2,5 \mathrm{M}$ li eritma hosil bo'ladi?\\
A) 250\\
B) 150\\
C) 300\\
D) 180\\
123. 2 M li 400 ml NaOH eritmasiga necha ml suv qo'shganimizda 1 M li eritma hosil bo'ladi?\\
A) 350\\
B) 400\\
C) 600\\
D) 100\\
124. 6 M li 200 ml NaOH eritmasiga necha ml suv qo'shganimizda 4 M li eritma hosil bo'ladi?\\
A) 350\\
B) 400\\
C) 600\\
D) 100\\
125. 4 M li 100 ml NaOH eritmasiga necha ml suv qo'shganimizda 1 M li eritma hosil bo'ladi?\\
A) 250\\
B) 150\\
C) 300\\
D) 180\\
126. 5 M li 250 ml NaOH eritmasiga necha ml suv qo'shganimizda 2 M li eritma hosil bo'ladi?\\
A) 375\\
B) 350\\
C) 300\\
D) 100\\
127. 4 M li 600 ml NaOH eritmasiga necha ml suv qo'shganimizda 3 M li eritma hosil bo'ladi?\\
A) 150\\
B) 50\\
C) 200\\
D) 180\\
128. 4 M li 450 ml NaOH eritmasiga necha ml suv qo'shganimizda $2,5 \mathrm{M}$ li eritma hosil bo'ladi?\\
A) 270\\
B) 250\\
C) 240\\
D) 280\\
129. 6 M li 350 ml NaOH eritmasiga necha ml suv qo'shganimizda 4 M li eritma hosil bo'ladi?\\
A) 250\\
B) 150\\
C) 230\\
D) 175\\
130. 4 M li 300 ml NaOH eritmasiga necha ml suv qo'shganimizda 3 M li eritma hosil bo'ladi?\\
A) 150\\
B) 50\\
C) 100\\
D) 180
  \item $\mathrm{HNO}_{3}$ ning 1 M li 200 ml eritmasiga 4 M 400 ml eritmasi aralashtirildi. Hosil bo'lgan eritmaning $\mathrm{mol} / \mathrm{l}$ ni aniqlang.\\
A) 3\\
B) 2\\
C) 2,5\\
D) 1,5
  \item $\mathrm{HNO}_{3}$ ning 2 M li 200 ml eritmasiga 4 M 200 ml eritmasi aralashtirildi. Hosil bo'lgan eritmaning $\mathrm{mol} / \mathrm{l}$ ni aniqlang.\\
A) 3\\
B) 2\\
C) 2,5\\
D) 1,5
  \item $\mathrm{HNO}_{3}$ ning 5 M li 500 ml eritmasiga 4 M 500 ml eritmasi aralashtirildi. Hosil bo'lgan eritmaning $\mathrm{mol} / \mathrm{l}$ ni aniqlang.\\
A) 4,4\\
B) 4,2\\
C) 4,5\\
D) 5,5
  \item $\mathrm{HNO}_{3}$ ning 1 M li 100 ml eritmasiga 3 M 900 ml eritmasi aralashtirildi. Hosil bo'lgan eritmaning $\mathrm{mol} / \mathrm{l}$ ni aniqlang.\\
A) 2,5\\
B) 2\\
C) 2,4\\
D) 2,8
  \item $\mathrm{HNO}_{3}$ ning 2 M li 500 ml eritmasiga 3 M 500 ml eritmasi aralashtirildi. Hosil bo'lgan eritmaning $\mathrm{mol} / \mathrm{l}$ ni aniqlang.\\
A) 3\\
B) 2\\
C) 2,5\\
D) 1,5
  \item $\mathrm{HNO}_{3}$ ning 1 M li 900 ml eritmasiga 5 M 100 ml eritmasi aralashtirildi. Hosil bo'lgan eritmaning $\mathrm{mol} / \mathrm{l}$ ni aniqlang.\\
A) 3\\
B) 2\\
C) 2,2\\
D) 1,4
  \item $\mathrm{HNO}_{3}$ ning 2 M li 200 ml eritmasiga 6 M 800 ml eritmasi aralashtirildi. Hosil bo'lgan eritmaning $\mathrm{mol} / \mathrm{l}$ ni aniqlang.\\
A) 5,2\\
B) 4,2\\
C) 3,5\\
D) 2,5
  \item $\mathrm{HNO}_{3}$ ning 1 M li 300 ml eritmasiga 5 M 700 ml eritmasi aralashtirildi. Hosil bo'lgan eritmaning $\mathrm{mol} / \mathrm{l}$ ni aniqlang.\\
A) 3\\
B) 3,8\\
C) 2,6\\
D) 4,5
  \item $\mathrm{HNO}_{3}$ ning $2,5 \mathrm{M}$ li 200 ml eritmasiga 4 M 800 ml eritmasi aralashtirildi. Hosil bo'lgan eritmaning $\mathrm{mol} / \mathrm{l}$ ni aniqlang.\\
A) 3,2\\
B) 2,8\\
C) 2,7\\
D) 3,7
  \item $\mathrm{HNO}_{3}$ ning 4 M li 250 ml eritmasiga 3 M 750 ml eritmasi aralashtirildi. Hosil bo'lgan eritmaning $\mathrm{mol} / \mathrm{l}$ ni aniqlang.\\
A) 3,25\\
B) 2\\
C) 2,5\\
D) 1,5
  \item NaOH ning 120 g miqdori 1000 g suvga eritildi. Eritmaning zichligi 1,12 $\mathrm{g} / \mathrm{ml}$ bo'lsa, uning molyar konsentratsiyasi qanchaga teng?\\
A) 3\\
B) 2\\
C) 2,5\\
D) 1,5\\
  \item $\mathrm{HNO}_{3}$ ning 126 g miqdori 1000 g suvga eritildi. Eritmaning zichligi 1,126 g/ml bo'lsa, uning molyar konsentratsiyasi qanchaga teng?\\
A) 3\\
B) 2\\
C) 2,5\\
D) 1,5
  \item KOH ning 56 g miqdori 1000 g suvga eritildi. Eritmaning zichligi $1,056 \mathrm{~g} / \mathrm{ml}$ bo'lsa, uning molyar konsentratsiyasi qanchaga teng?\\
A) 3\\
B) 2\\
C) 5\\
D) 1
  \item HCl ning 142 g miqdori 1000 g suvga eritildi. Eritmaning zichligi $1,142 \mathrm{~g} / \mathrm{ml}$ bo'lsa, uning molyar konsentratsiyasi qanchaga teng?\\
A) 2\\
B) 4\\
C) 2,5\\
D) 3,5
  \item $\mathrm{H}_{2} \mathrm{SO}_{4}$ ning 294 g miqdori 1000 g suvga eritildi. Eritmaning zichligi 1,294 $\mathrm{g} / \mathrm{ml}$ bo'lsa, uning molyar konsentratsiyasi qanchaga teng?\\
A) 3\\
B) 2\\
C) 2,5\\
D) 1,5
  \item $\mathrm{NH}_{3}$ ning 34 g miqdori 1000 g suvga eritildi. Eritmaning zichligi $1,034 \mathrm{~g} / \mathrm{ml}$\\
bo'lsa. uning molyar konsentratsiyasi qanchaga teng?\\
A) 3\\
B) 2\\
C) 2.5\\
D) 1.5
  \item HBr ning 202.5 g miqdori 1000 g surga eritildi. Eritmaning zichligi 1,2025 $\mathrm{g} / \mathrm{ml}$ bo'lsa, uning molyar konsentratsiyasi qanchaga teng?\\
A) 3\\
B) 2\\
C) 2.5\\
D) 1,5
  \item LiOH ning 120 g miqdori 1000 g surga eritildi. Eritmaning zichligi 1,12 $\mathrm{g} / \mathrm{ml}$ bo'lsa, uning molyar konsentratsiyasi qanchaga teng?\\
A) 4\\
B) 3\\
C) 5\\
D) 2,5
  \item HI ning 128 g miqdori 1000 g surga eritildi. Eritmaning zichligi $1,128 \mathrm{~g} / \mathrm{ml}$ bo'lsa, uning molyar konsentratsiyasi qanchaga teng?\\
A) 3\\
B) 2\\
C) 2\\
D) 1
  \item NaOH ning 80 g miqdori 1000 g suvga eritildi. Eritmaning zichligi 1.08 $\mathrm{g} / \mathrm{ml}$ bo'lsa, uning molyar konsentratsiyasi qanchaga teng?\\
A) 3\\
B) 2\\
C) 2,5\\
D) 1.5
  \item HCl ning metallda o'lchangan 44,8 litr hajmi 1000 g suvga eritildi. Eritmaning zichligi $1,073 \mathrm{~g} / \mathrm{ml}$ bo'lsa, uning molyar konsentratsiyasi qanchaga teng?\\
A) 3\\
B) 2\\
C) 2,5\\
D) 1,5
  \item HBr ning metallda o'lchangan 22,4 litr hajmi 1000 g suvga eritildi. Eritmaning zichligi $1,081 \mathrm{~g} / \mathrm{ml}$ bo'lsa, uning molyar konsentratsiyasi qanchaga teng?\\
A) 1\\
B) 2\\
C) 3\\
D) 4
  \item HI ning metallda o'lchangan 67,2 litr hajmi 1000 g suvga eritildi. Eritmaning zichligi $1,384 \mathrm{~g} / \mathrm{ml}$ bo'lsa, uning molyar konsentratsiyasi qanchaga teng?\\
A) 1\\
B) 2\\
C) 3\\
D) 4
  \item $\mathrm{NH}_{3}$ ning metallda o'lchangan 33.6 litr hajmi 1000 g suvga eritildi. Eritmaning zichligi $1.0255 \mathrm{~g} / \mathrm{ml}$ bo'lsa. uning molyar konsentratsiyasi qaichasa tang?\\
A) 3\\
B) 2\\
C) 2.5\\
D) 1.5
  \item HF ning metallda o'lchangan 56 litr hajmi 1000 g surga eritildi. Eritmaning zichligi $1.05 \mathrm{~g} / \mathrm{ml}$ bo'lsa, uning molyar konsentratsivasi qanchaga teng?)\\
A) 3\\
B) 2\\
C) $2, \overline{0}$\\
D) 1.5
  \item $\mathrm{H}_{3} \mathrm{~S}$ ning metallda o'lchangan 89.6 litr hajmi 1000 g suvga eritildi. Eritmaning zichligi $1.136 \mathrm{~g} / \mathrm{ml}$ bo'lsa, uning molyar konsentratsiyasi qanchaga teng?\\
A) 4\\
B) 2\\
C) 3\\
D) 1.5
  \item HI ning metallda o'lchangan 44.8 litr hajmi 1000 g surga eritildi. Eritmaning zichligi $1,256 \mathrm{~g} / \mathrm{ml}$ bo'lsa, uning molyar konsentratsiyasi qanchaga teng?\\
A) 3\\
B) 2\\
C) 2.5\\
D) 1.5
  \item HCl ning n .sh da o'lchangan 11.2 litr hajmi 1000 g suvga eritildi. Eritmaning zichligi $1,01825 \mathrm{~g} / \mathrm{ml}$ bo'lsa, uning molyar konsentratsiyasi qanchaga teng?\\
A) 3\\
B) 1\\
C) 0.5\\
D) 1,5
  \item $\mathrm{NH}_{3}$ ning metallda o'lchangan 44.8 litr hajmi 1000 g suvga eritildi. Eritmaning zichligi $1,034 \mathrm{~g} / \mathrm{ml}$ bo'lsa, uning molvar konsentratsiyasi qanchaga teng?\\
A) 3\\
B) 2\\
C) $2, \overline{0}$\\
D) 1.5
  \item HF ning metallda o'lchangan 22.4 litr hajmi 1000 g suvga eritildi. Eritmaning zichligi $1,02 \mathrm{~g} / \mathrm{ml}$ bo'lsa, uning molyar konsentratsiyasi qanchaga teng?\\
A) 3\\
B) 2\\
C) 5\\
D) 1
  \item NaOH ning 3 mol miqdori 1000 g suvga eritildi. Eritmaning zichligi 1.12 $\mathrm{g} / \mathrm{ml}$ bo'lsa, uning molyar konsentratsiyasi qanchaga teng?\\
A) 3\\
B) 2\\
C) 2,5\\
D) 1.5\\
  \item $\mathrm{HNO}_{3}$ ning 2 mol miqdori 1000 g suvga eritildi. Eritmaning zichligi 1,126 g/ml bo'lsa, uning molyar konsentratsiyasi qanchaga teng?\\
A) 3\\
B) 2\\
C) 2,5\\
D) 1,5
  \item KOH ning 1 mol miqdori 1000 g suvga eritildi. Eritmaning zichligi 1,056 $\mathrm{g} / \mathrm{ml}$ bo'lsa, uning molyar konsentratsiyasi qanchaga teng?\\
A) 3\\
B) 2\\
C) 5\\
D) 1
  \item HCl ning 4 mol miqdori 1000 g suvga eritildi. Eritmaning zichligi $1,142 \mathrm{~g} / \mathrm{ml}$ bo'lsa, uning molyar konsentratsiyasi qanchaga teng?\\
A) 2\\
B) 4\\
C) 2,5\\
D) 3,5
  \item $\mathrm{H}_{2} \mathrm{SO}_{4}$ ning 3 mol miqdori 1000 g suvga eritildi. Eritmaning zichligi 1,294 $\mathrm{g} / \mathrm{ml}$ bo'lsa, uning molyar konsentratsiyasi qanchaga teng?\\
A) 3\\
B) 2\\
C) 2,5\\
D) 1,5
  \item $\mathrm{NH}_{3}$ ning 2 mol miqdori 1000 g suvga eritildi. Eritmaning zichligi $1,034 \mathrm{~g} / \mathrm{ml}$ bo'lsa, uning molyar konsentratsiyasi qanchaga teng?\\
A) 3\\
B) 2\\
C) 2,5\\
D) 1,5
  \item HBr ning $2,5 \mathrm{~mol}$ miqdori 1000 g suvga eritildi. Eritmaning zichligi 1,2025 $\mathrm{g} / \mathrm{ml}$ bo'lsa, uning molyar konsentratsiyasi qanchaga teng?\\
A) 3\\
B) 2\\
C) 2,5\\
D) 1,5
  \item LiOH ning 5 mol miqdori 1000 g suvga eritildi. Eritmaning zichligi 1,12 $\mathrm{g} / \mathrm{ml}$ bo'lsa, uning molyar konsentratsiyasi qanchaga teng?\\
A) 4\\
B) 3\\
C) 5\\
D) 2,5
  \item HI ning 1 mol miqdori 1000 g suvga eritildi. Eritmaning zichligi $1,128 \mathrm{~g} / \mathrm{ml}$ bo'lsa, uning molyar konsentratsiyasi qanchaga teng?\\
A) 3\\
B) 2\\
C) 2\\
D) 1
  \item NaOH ning 2 mol miqdori 1000 g suvga eritildi. Eritmaning zichligi 1,08 $\mathrm{g} / \mathrm{ml}$ bo'lsa, uning molyar konsentratsiyasi qanchaga teng?\\
A) 3\\
B) 2\\
C) 2,5\\
D) 1,5
171. Zichligi $1,2 \mathrm{~g} / \mathrm{ml}$ bo'lgan, $20 \% \mathrm{li}$ NaOH eritmasining molyar konsentratsiyasini hisoblang.\\
A) 6\\
B) 2\\
C) 2,5\\
D) 1,5\\
172. Zichligi $1,26 \mathrm{~g} / \mathrm{ml}$ bo'lgan, $10 \%$ li $\mathrm{HNO}_{3}$ eritmasining molyar konsentratsiyasini hisoblang.\\
A) 6\\
B) 2\\
C) 2,5\\
D) 1,5\\
173. Zichligi $1,225 \mathrm{~g} / \mathrm{ml}$ bo'lgan, $40 \% \mathrm{li}$ $\mathrm{H}_{2} \mathrm{SO}_{4}$ eritmasining molyar konsentratsiyasini hisoblang.\\
A) 6\\
B) 3\\
C) 2\\
D) 5\\
174. Zichligi $1,12 \mathrm{~g} / \mathrm{ml}$ bo'lgan, $15 \%$ li KOH eritmasining molyar konsentratsiyasini hisoblang.\\
A) 3\\
B) 2\\
C) 2,5\\
D) 3,5\\
175. Zichligi $1,2 \mathrm{~g} / \mathrm{ml}$ bo'lgan, $25 \% \mathrm{li}$ $\mathrm{MgSO}_{4}$ eritmasining molyar konsentratsiyasini hisoblang.\\
A) 6\\
B) 2\\
C) 2,5\\
D) 1,5\\
176. Zichligi $1,49 \mathrm{~g} / \mathrm{ml}$ bo'lgan, $20 \%$ li KCl eritmasining molyar konsentratsiyasini hisoblang.\\
A) 6\\
B) 2\\
C) 4\\
D) 5\\
177. Zichligi $1,17 \mathrm{~g} / \mathrm{ml}$ bo'lgan, $40 \%$ li NaCl eritmasining molyar konsentratsiyasini hisoblang.\\
A) 8\\
B) 7\\
C) 5\\
D) 6\\
178. Zichligi $1,9 \mathrm{~g} / \mathrm{ml}$ bo'lgan, $60 \% \mathrm{li}$ $\mathrm{MgCl}_{2}$ eritmasining molyar konsentratsiyasini hisoblang.\\
A) 9\\
B) 8\\
C) 12\\
D) 15\\
179. Zichligi $1,2 \mathrm{~g} / \mathrm{ml}$ bo'lgan, $35 \% \mathrm{li}$ NaOH eritmasining molyar konsentratsiyasini hisoblang.\\
A) 16\\
B) 12\\
C) 12,5\\
D) 10,5\\
180. Zichligi $1,2 \mathrm{~g} / \mathrm{ml}$ bo'lgan, $5 \% \mathrm{li} \mathrm{LiOH}$ eritmasining molyar konsentratsiyasini hisoblang.\\
A) 6\\
B) 2\\
C) 2,5\\
D) 1,5\\
181. $8 \mathrm{~N} \mathrm{li} \mathrm{H}_{2} \mathrm{SO}_{4}$ eritmasining molyar konsentratsiyasini hisoblang.\\
A) 4\\
B) 3\\
C) 2,5\\
D) 4,5
  \item $4,5 \mathrm{~N}$ li $\mathrm{HNO}_{3}$ eritmasining molyar konsentratsiyasini hisoblang.\\
A) 4\\
B) 3\\
C) 2,5\\
D) 4,5
  \item $9 \mathrm{Nli} \mathrm{H}_{3} \mathrm{PO}_{4}$ eritmasining molyar konsentratsiyasini hisoblang.\\
A) 4\\
B) 3\\
C) 2,5\\
D) 4,5
  \item 5 N li $\mathrm{H}_{2} \mathrm{SO}_{3}$ eritmasining molyar konsentratsiyasini hisoblang.\\
A) 4\\
B) 3\\
C) 2,5\\
D) 4,5
  \item $6 \mathrm{~N} \mathrm{li} \mathrm{H}_{3} \mathrm{AsO}_{4}$ eritmasining molyar konsentratsiyasini hisoblang.\\
A) 5\\
B) 3\\
C) 2\\
D) 4
  \item 9 N li $\mathrm{Ca}(\mathrm{OH})_{2}$ eritmasining molyar konsentratsiyasini hisoblang.\\
A) 4\\
B) 3\\
C) 2,5\\
D) 4,5
  \item 2 N li HCl eritmasining molyar konsentratsiyasini hisoblang.\\
A) 1\\
B) 3\\
C) 2\\
D) 4
  \item 5 N li $\mathrm{H}_{2} \mathrm{CrO}_{4}$ eritmasining molyar konsentratsiyasini hisoblang.\\
A) 4\\
B) 3\\
C) 2,5\\
D) 4,5
  \item 12 N li $\mathrm{H}_{3} \mathrm{PO}_{4}$ eritmasining molyar konsentratsiyasini hisoblang.\\
A) 4\\
B) 3\\
C) 2,5\\
D) 4,5
  \item $7 \mathrm{~N} \mathrm{li} \mathrm{H}_{2} \mathrm{SO}_{4}$ eritmasining molyar konsentratsiyasini hisoblang.\\
A) 5\\
B) 6\\
C) 3,5\\
D) 4,5
  \item Titr konsentratsiyasi $0,6 \mathrm{~g} / \mathrm{ml}$ bo'lgan NaOH eritmasining molyar konsentratsiyasini toping.\\
A) 15\\
B) 16\\
C) 13,5\\
D) 14,5
  \item Titr konsentratsiyasi $0,126 \mathrm{~g} / \mathrm{ml}$ bo'lgan $\mathrm{HNO}_{3}$ eritmasining molyar konsentratsiyasini toping.\\
A) 1\\
B) 2\\
C) 3\\
D) 4
  \item Titr konsentratsiyasi $0,49 \mathrm{~g} / \mathrm{ml}$ bo'lgan $\mathrm{H}_{2} \mathrm{SO}_{4}$ eritmasining molyar konsentratsiyasini toping.\\
A) 1\\
B) 6\\
C) 5\\
D) 4,5
  \item Titr konsentratsiyasi $0,168 \mathrm{~g} / \mathrm{ml}$ bo'lgan KOH eritmasining molyar konsentratsiyasini toping.\\
A) 1\\
B) 2\\
C) 3\\
D) 4
  \item Titr konsentratsiyasi $0,24 \mathrm{~g} / \mathrm{ml}$ bo'lgan LiOH eritmasining molyar konsentratsiyasini toping.\\
A) 10\\
B) 16\\
C) 13\\
D) 14
  \item Titr konsentratsiyasi $0,081 \mathrm{~g} / \mathrm{ml}$ bo'lgan HBr eritmasining molyar konsentratsiyasini toping.\\
A) 1\\
B) 2\\
C) 3\\
D) 4
  \item Titr konsentratsiyasi $0,6 \mathrm{~g} / \mathrm{ml}$ bo'lgan $\mathrm{MgSO}_{4}$ eritmasining molyar konsentratsiyasini toping.\\
A) 1\\
B) 6\\
C) 3\\
D) 5
  \item Titr konsentratsiyasi $0,365 \mathrm{~g} / \mathrm{ml}$ bo'lgan HCl eritmasining molyar konsentratsiyasini toping.\\
A) 10\\
B) 16\\
C) 13\\
D) 14
  \item Titr konsentratsiyasi $0,04 \mathrm{~g} / \mathrm{ml}$ bo'lgan HF eritmasining molyar konsentratsiyasini toping.\\
A) 1\\
B) 2\\
C) 3\\
D) 4
  \item Titr konsentratsiyasi $0,189 \mathrm{~g} / \mathrm{ml}$ bo'lgan $\mathrm{HNO}_{3}$ eritmasining molyar konsentratsiyasini toping.\\
A) 1\\
B) 2\\
C) 3\\
D) 4
  \item Tarkibida 80 g NaOH saqlagan 500 ml eritmaning normal konsentratsiyasini toping.\\
A) 1\\
B) 2\\
C) 3\\
D) 4
  \item Tarkibida $98 \mathrm{~g} \mathrm{H}_{2} \mathrm{SO}_{4}$ saqlagan 1000 ml eritmaning normal konsentratsiyasini toping.\\
A) 1\\
B) 2\\
C) 3\\
D) 4
  \item Tarkibida $31,5 \mathrm{~g} \mathrm{HNO}_{3}$ saqlagan 500 ml eritmaning normal konsentratsiyasini toping.\\
A) 1\\
B) 2\\
C) 3\\
D) 4
  \item Tarkibida $120 \mathrm{~g} \mathrm{MgSO}_{4}$ saqlagan 2000 ml eritmaning normal konsentratsiyasini toping.\\
A) 0,5\\
B) 1\\
C) 1,5\\
D) 2
  \item Tarkibida $171 \mathrm{~g} \mathrm{Al}_{2}\left(\mathrm{SO}_{4}\right)_{3}$ saqlagan 400 ml eritmaning normal konsentratsiyasini toping.\\
A) 2,5\\
B) 5\\
C) 7,5\\
D) 2
  \item Tarkibida $142 \mathrm{~g} \mathrm{Na}_{2} \mathrm{SO}_{4}$ saqlagan 250 ml eritmaning normal konsentratsiyasini toping.\\
A) 2\\
B) 4\\
C) 6\\
D) 8
  \item Tarkibida $126 \mathrm{~g} \mathrm{HNO}_{3}$ saqlagan 800 ml eritmaning normal konsentratsiyasini toping.\\
A) 2,5\\
B) 5\\
C) 7,5\\
D) 2
  \item Tarkibida $98 \mathrm{~g} \mathrm{H}_{3} \mathrm{PO}_{4}$ saqlagan 500 ml eritmaning normal konsentratsiyasini toping.\\
A) 2\\
B) 4\\
C) 6\\
D) 8
  \item Tarkibida $164 \mathrm{~g} \mathrm{H}_{2} \mathrm{SO}_{3}$ saqlagan 400 ml eritmaning normal konsentratsiyasini toping.\\
A) 10\\
B) 12\\
C) 13\\
D) 14
  \item Tarkibida 120 g NaOH saqlagan 2500 ml eritmaning normal konsentratsiyasini toping.\\
A) 1,2\\
B) 2,4\\
C) 3\\
D) 1,5
  \item $3 \mathrm{~mol} \mathrm{H}_{2} \mathrm{SO}_{4} 1000$ g suvga eritildi. Hosil bo'lgan eritmaning zichligi 1,294 $\mathrm{g} / \mathrm{ml}$ bo'lsa, eritmaning normal konsentratsiyasini aniqlang.\\
A) 2\\
B) 4\\
C) 6\\
D) 8\\
  \item 2 mol NaOH 1000 g suvga eritildi. Hosil bo'lgan eritmaning zichligi $1,08 \mathrm{~g} / \mathrm{ml}$ bo'lsa, eritmaning normal konsentratsiyasini aniqlang.\\
A) 2\\
B) 4\\
C) 6\\
D) 8
  \item $2,5 \mathrm{~mol} \mathrm{Ca}(\mathrm{OH})_{2} 1000 \mathrm{~g}$ suvga eritildi. Hosil bo'lgan eritmaning zichligi 1,185 g/ml bo'lsa, eritmaning normal konsentratsiyasini aniqlang.\\
A) 4\\
B) 5\\
C) 3,5\\
D) 2,5
  \item $1,5 \mathrm{~mol} \mathrm{H}_{3} \mathrm{PO}_{4} 1000 \mathrm{~g}$ suvga eritildi. Hosil bo'lgan eritmaning zichligi 1,147 $\mathrm{g} / \mathrm{ml}$ bo'lsa, eritmaning normal konsentratsiyasini aniqlang.\\
A) 3,5\\
B) 4\\
C) 5\\
D) 4,5
  \item $1 \mathrm{~mol} \mathrm{H}_{2} \mathrm{SO}_{3} 1000 \mathrm{~g}$ suvga eritildi. Hosil bo'lgan eritmaning zichligi 1,082 $\mathrm{g} / \mathrm{ml}$ bo'lsa, eritmaning normal konsentratsiyasini aniqlang.\\
A) 2\\
B) 4\\
C) 6\\
D) 8
  \item $2 \mathrm{~mol} \mathrm{MgSO}{ }_{4} 1000 \mathrm{~g}$ suvga eritildi. Hosil bo'lgan eritmaning zichligi $1,24 \mathrm{~g} / \mathrm{ml}$ bo'lsa, eritmaning normal konsentratsiyasini aniqlang.\\
A) 2\\
B) 4\\
C) 6\\
D) 8
218. $1 \mathrm{~mol} \mathrm{Al}_{2}\left(\mathrm{SO}_{4}\right)_{3} 1000 \mathrm{~g}$ suvga eritildi. Hosil bo'lgan eritmaning zichligi 1,342 g/ml bo'lsa, eritmaning normal konsentratsiyasini aniqlang.\\
A) 2\\
B) 4\\
C) 6\\
D) 8\\
219. $4 \mathrm{~mol} \mathrm{Ba}(\mathrm{OH})_{2} 1000 \mathrm{~g}$ suvga eritildi. Hosil bo'lgan eritmaning zichligi 1,684 $\mathrm{g} / \mathrm{ml}$ bo'lsa, eritmaning normal konsentratsiyasini aniqlang.\\
A) 2\\
B) 4\\
C) 6\\
D) 8\\
220. $2 \mathrm{~mol} \mathrm{H}_{2} \mathrm{CrO}_{4} 1000 \mathrm{~g}$ suvga eritildi.
Hosil bo'lgan eritmaning zichligi 1,236\\
g/ml bo'lsa, critmaning normal konsentratsiyasini aniqlang.\\
B) 4\\
C) 6\\
D) 8\\
N) 2
  \item Zichligi $1,2 \mathrm{~g} / \mathrm{ml}$ bo'lgnn, $20 \% \mathrm{li}$ NaOH critmasining normal konsontratsiyasini hisoblang.\\
A) 6\\
B) 2\\
C) 2,5\\
D) 1,5
  \item Zichligi $1,26 \mathrm{~g} / \mathrm{ml}$ bo'lgan, $10 \% \mathrm{li}$ $\mathrm{HNO}_{3}$ eritmasining normal konsentratsiyasini hisoblang.\\
A) 6\\
B) 2\\
C) 2,5\\
D) 1,5
  \item Zichligi $1,225 \mathrm{~g} / \mathrm{ml}$ bo'lgan, $40 \% \mathrm{li}$ $\mathrm{H}_{2} \mathrm{SO}_{4}$ eritmasining normal konsentratsiyasini hisoblang.\\
A) 6\\
B) 3\\
C) 2\\
D) 10
  \item Zichligi $1,12 \mathrm{~g} / \mathrm{ml}$ bo'lgan, $15 \% \mathrm{li}$ KOH eritmasining normal konsentratsiyasini hisoblang.\\
A) 3\\
B) 2\\
C) 2,5\\
D) 3,5
  \item Zichligi $1,2 \mathrm{~g} / \mathrm{ml}$ bo'lgan, $25 \% \mathrm{li}$ $\mathrm{MgSO}_{4}$ eritmasining normal konsentratsiyasini hisoblang.\\
A) 6\\
B) 2\\
C) 5\\
D) 1,5
  \item Zichligi $1,49 \mathrm{~g} / \mathrm{ml}$ bo'lgan, $20 \% \mathrm{li} \mathrm{KCl}$ eritmasining normal konsentratsiyasini hisoblang.\\
A) 6\\
B) 2\\
C) 4\\
D) 5
  \item Zichligi $1,17 \mathrm{~g} / \mathrm{ml}$ bo'lgan, $40 \%$ li NaCl eritmasining normal konsentratsiyasini hisoblang.\\
A) 8\\
B) 7\\
C) 5\\
D) 6
  \item Zichligi $1,9 \mathrm{~g} / \mathrm{ml}$ bo'lgan, $60 \% \mathrm{li}$ $\mathrm{MgCl}_{2}$ eritmasining normal konsentratsiyasini hisoblang.\\
A) 9\\
B) 8\\
C) 24\\
D) 15
  \item Zichligi $1,2 \mathrm{~g} / \mathrm{ml}$ bo'lgan, $35 \% \mathrm{li}$ NaOH eritmasining normal konsentratsiyasini hisoblang.\\
A) 16\\
B) 12\\
C) 12,5\\
D) 10,5
  \item Zichligi $1,2 \mathrm{~g} / \mathrm{ml}$ bo'lgan, $5 \%$ li LiOH eritmasining normal konsentrataíyasini himoblang.\\
A) 6\\
B) 2\\
C) 2,5\\
D) $1, \sqrt{5}$
  \item 8 M li $\mathrm{H}_{2} \mathrm{SO}_{4}$ eritmasining normal konsentratsiyasini hisoblang.\\
A) 16\\
B) 13\\
C) 12,5\\
D) 14,5\\
  \item $4,5 \mathrm{M}$ li $\mathrm{HNO}_{3}$ eritmasining normal konsentratsiyasini hisoblang.\\
A) 4\\
B) 3\\
C) 2,5\\
D) 4,5
  \item 1 M li $\mathrm{H}_{3} \mathrm{PO}_{4}$ eritmasining normal konsentratsiyasini hisoblang.\\
A) 4\\
B) 3\\
C) 2,5\\
D) 4,5
  \item 5 M li $\mathrm{H}_{2} \mathrm{SO}_{3}$ eritmasining normal konsentratsiyasini hisoblang.\\
A) 14\\
B) 3\\
C) 10\\
D) 4,5
  \item 2 M li $\mathrm{H}_{3} \mathrm{AsO}_{4}$ eritmasining normal konsentratsiyasini hisoblang.\\
A) 5\\
B) 3\\
C) 6\\
D) 4
  \item $4,5 \mathrm{M} \mathrm{li} \mathrm{Ca}(\mathrm{OH})_{2}$ eritmasining normal konsentratsiyasini hisoblang.\\
A) 4\\
B) 3\\
C) 4,5\\
D) 9
  \item 2 M li HCl eritmasining normal konsentratsiyasini hisoblang.\\
A) 1\\
B) 3\\
C) 2\\
D) 4
  \item 5 M li $\mathrm{H}_{2} \mathrm{CrO}_{4}$ eritmasining normal konsentratsiyasini hisoblang.\\
A) 4\\
B) 3\\
C) 10\\
D) 5
  \item 1 M li $\mathrm{H}_{3} \mathrm{PO}_{4}$ eritmasining normal konsentratsiyasini hisoblang.\\
A) 4\\
B) 3\\
C) 2,5\\
D) 4,5
  \item $2,5 \mathrm{M} \mathrm{li} \mathrm{H}_{2} \mathrm{SO}_{4}$ eritmasining normal konsentratsiyasini hisoblang.\\
A) 5\\
B) 6\\
C) 3,5\\
D) 4,5
  \item Titr konsentratsiyasi $0,6 \mathrm{~g} / \mathrm{ml}$ bo'lgan NaOH eritmasining normal konsentratsiyasini toping.\\
A) 15\\
B) 16\\
C) 13,5\\
D) 14,5\\
  \item Titr konsentratsiyasi $0,126 \mathrm{~g} / \mathrm{ml}$ bo'lgan $\mathrm{HNO}_{3}$ eritmasining normal konsentratsiyasini toping.\\
A) 1\\
B) 2\\
C) 3\\
D) 4
  \item Titr konsentratsiyasi $0,49 \mathrm{~g} / \mathrm{ml}$ bo'lgan $\mathrm{H}_{2} \mathrm{SO}_{4}$ eritmasining normal konsentratsiyasini toping.\\
A) 1\\
B) 6\\
C) 10\\
D) 4,5
  \item Titr konsentratsiyasi $0,168 \mathrm{~g} / \mathrm{ml}$ bo'lgan KOH eritmasining normal konsentratsiyasini toping.\\
A) 1\\
B) 2\\
C) 3\\
D) 4
  \item Titr konsentratsiyasi $0,24 \mathrm{~g} / \mathrm{ml}$ bo'lgan LiOH eritmasining normal konsentratsiyasini toping.\\
A) 10\\
B) 16\\
C) 13\\
D) 14
  \item Titr konsentratsiyasi $0,081 \mathrm{~g} / \mathrm{ml}$ bo'lgan HBr eritmasining normal konsentratsiyasini toping.\\
A) 1\\
B) 2\\
C) 3\\
D) 4
  \item Titr konsentratsiyasi $0,6 \mathrm{~g} / \mathrm{ml}$ bo'lgan $\mathrm{MgSO}_{4}$ eritmasining normal konsentratsiyasini toping.\\
A) 1\\
B) 6\\
C) 3\\
D) 10
  \item Titr konsentratsiyasi $0,365 \mathrm{~g} / \mathrm{ml}$ bo'lgan HCl eritmasining normal konsentratsiyasini toping.\\
A) 10\\
B) 16\\
C) 13\\
D) 14
  \item Titr konsentratsiyasi $0,04 \mathrm{~g} / \mathrm{ml}$ bo'lgan HF eritmasining normal konsentratsiyasini toping.\\
A) 1\\
B) 2\\
C) 3\\
D) 4
  \item Titr konsentratsiyasi $0,189 \mathrm{~g} / \mathrm{ml}$ bo'lgan $\mathrm{HNO}_{3}$ eritmasining normal konsentratsiyasini toping.\\
A) 1\\
B) 2\\
C) 3\\
D) 
  \item 250 ml eritma tarkibida 25 g NaOH bo'lsa, eritmaning titr konsentratsiyasini toping.\\
A) 0,1\\
B) 0,2\\
C) 0,3\\
D) 0,4\\
  \item 200 ml eritma tarkibida $80 \mathrm{~g} \mathrm{HNO}_{3}$ bo'lsa, eritmaning titr konsentratsiyasini toping.\\
A) 0,1\\
B) 0,2\\
C) 0,3\\
D) 0,4
  \item 500 ml eritma tarkibida 100 g KOH bo'lsa, eritmaning titr konsentratsiyasini toping.\\
A) 0,1\\
B) 0,2\\
C) 0,3\\
D) 0,4
  \item 600 ml eritma tarkibida 180 g NaCl bo'lsa, eritmaning titr konsentratsiyasini toping.\\
A) 0,1\\
B) 0,2\\
C) 0,3\\
D) 0,4
  \item 400 ml eritma tarkibida 160 g LiOH bo'lsa, eritmaning titr konsentratsiyasini toping.\\
A) 0,1\\
B) 0,2\\
C) 0,3\\
D) 0,4
  \item 300 ml eritma tarkibida $90 \mathrm{~g} \mathrm{MgSO}_{4}$ bo'lsa, eritmaning titr konsentratsiyasini toping.\\
A) 0,1\\
B) 0,2\\
C) 0,3\\
D) 0,4
  \item 250 ml eritma tarkibida 40 g NaOH bo'lsa, eritmaning titr konsentratsiyasini toping.\\
A) 0,16\\
B) 0,12\\
C) 0,3\\
D) 0,14
  \item 150 ml eritma tarkibida 15 g HCl bo'lsa, eritmaning titr konsentratsiyasini toping.\\
A) 0,1\\
B) 0,2\\
C) 0,3\\
D) 0,4
  \item 360 ml eritma tarkibida 72 g KCl bo'lsa, eritmaning titr konsentratsiyasini toping.\\
A) 0,1\\
B) 0,2\\
C) 0,3\\
D) 0,4
  \item 180 ml eritma tarkibida 54 g RbOH bo'lsa, eritmaning titr konsentratsiyasini toping.\\
A) 0,1\\
B) 0,2\\
C) 0,3\\
D) 0,4
  \item $3 \mathrm{~mol} \mathrm{H}_{2} \mathrm{SO}_{4} 1000 \mathrm{~g}$ suvga eritildi. Hosil bo'lgan eritmaning zichligi 1,294\\
$\mathrm{g} / \mathrm{ml}$ bo'lsa, eritmaning titr konsentratsiyasini aniqlang.\\
A) 0.98\\
B) 0,49\\
C) 0,294\\
D) 0,196
  \item 2 mol NaCl 1000 g suvga eritildi. Hosil bo'lgan eritmaning zichligi 1,117 $\mathrm{g} / \mathrm{ml}$ bo'lsa, eritmaning titr konsentratsiyasini aniqlang.\\
A) 0,585\\
B) 0,234\\
C) 0,117\\
D) 0,312
  \item $2,5 \mathrm{~mol} \mathrm{Ca}(\mathrm{OH})_{2} 1000 \mathrm{~g}$ suvga eritildi. Hosil bo'lgan eritmaning zichligi 1,185 $\mathrm{g} / \mathrm{ml}$ bo'lsa, eritmaning titr konsentratsiyasini aniqlang.\\
A) 0,74\\
B) 0185\\
C) 0,37\\
D) 0,25
  \item $1,5 \mathrm{~mol} \mathrm{H}_{3} \mathrm{PO}_{4} 1000 \mathrm{~g}$ suvga eritildi. Hosil bo'lgan eritmaning zichligi 1,147 g/ml bo'lsa, eritmaning titr konsentratsiyasini aniqlang.\\
A) 0,147\\
B) 0,49\\
C) 0,98\\
D) 0,245
  \item $1 \mathrm{~mol} \mathrm{H}_{2} \mathrm{SO}_{3} 1000 \mathrm{~g}$ suvga eritildi. Hosil bo'lgan eritmaning zichligi 1,082 $\mathrm{g} / \mathrm{ml}$ bo'lsa, eritmaning titr konsentratsiyasini aniqlang.\\
A) 0,041\\
B) 0,41\\
C) 0,082\\
D) 0,82
  \item $2 \mathrm{~mol} \mathrm{MgSO}_{4} 1000 \mathrm{~g}$ suvga eritildi. Hosil bo'lgan eritmaning zichligi $1,24 \mathrm{~g} / \mathrm{ml}$ bo'lsa, eritmaning titr konsentratsiyasini aniqlang.\\
A) 2,4\\
B) 0,12\\
C) 1,2\\
D) 0,24
  \item 5 mol LiOH 1000 g suvga eritildi. Hosil bo'lgan eritmaning zichligi $1,12 \mathrm{~g} / \mathrm{ml}$ bo'lsa, eritmaning titr konsentratsiyasini aniqlang.\\
A) 0,12\\
B) 0,24\\
C) 6\\
D) 0,012
  \item $1 \mathrm{~mol} \mathrm{Al}_{2}\left(\mathrm{SO}_{4}\right)_{3} 1000 \mathrm{~g}$ suvga eritildi. Hosil bo'lgan eritmaning zichligi 1,342 $\mathrm{g} / \mathrm{ml}$ bo'lsa, eritmaning titr konsentratsiyasini aniqlang.\\
A) 0,342\\
B) 1,71\\
C) 0,171\\
D) 3,42
  \item $4 \mathrm{~mol} \mathrm{Ba}(\mathrm{OH})_{2} 1000 \mathrm{~g}$ suvga eritildi. Hosil bo'lgan eritmaning zichligi 1,684\\
g/ml bo'lsa, eritmaning titr konsentratsiyasini aniqlang.\\
A) 0,2\\
B) 0,4\\
C) 0,171\\
D) 0,684
  \item $2 \mathrm{~mol} \mathrm{H}_{2} \mathrm{CrO}_{4} 1000 \mathrm{~g}$ suvga eritildi. Hosil bo'lgan eritmaning zichligi 1,236 $\mathrm{g} / \mathrm{ml}$ bo'lsa, eritmaning titr konsentratsiyasini aniqlang.\\
A) 2\\
B) 0,4\\
C) 0,236\\
D) 0,816
  \item Zichligi $1,2 \mathrm{~g} / \mathrm{ml}$ bo'lgan $20 \%$ li NaOH eritmasining titr konsentratsiyasini hisoblang.\\
A) 0,24\\
B) 0,2\\
C) 0,25\\
D) 0,15\\
  \item Zichligi $1,26 \mathrm{~g} / \mathrm{ml}$ bo'lgan $10 \%$ li $\mathrm{HNO}_{3}$ eritmasining titr konsentratsiyasini hisoblang.\\
A) 0,63\\
B) 0,315\\
C) 0,252\\
D) 0,126
  \item Zichligi $1,225 \mathrm{~g} / \mathrm{ml}$ bo'lgan $40 \% \mathrm{li}$ $\mathrm{H}_{2} \mathrm{SO}_{4}$ eritmasining titr konsentratsiyasini hisoblang.\\
A) 0,245\\
B) 0,98\\
C) 0,1225\\
D) 0,49
  \item Zichligi $1,12 \mathrm{~g} / \mathrm{ml}$ bo'lgan $15 \% \mathrm{li} \mathrm{KOH}$ eritmasining titr konsentratsiyasini hisoblang.\\
A) 0,168\\
B) 0,56\\
C) 1,12\\
D) 0,28
  \item Zichligi $1,2 \mathrm{~g} / \mathrm{ml}$ bo'lgan $25 \% \mathrm{li}$ $\mathrm{MgSO}_{4}$ eritmasining titr konsentratsiyasini hisoblang.\\
A) 0,6\\
B) 0,2\\
C) 0,3\\
D) 0,15
  \item Zichligi $1,49 \mathrm{~g} / \mathrm{ml}$ bo'lgan $20 \% \mathrm{li} \mathrm{KCl}$ eritmasining titr konsentratsiyasini hisoblang.\\
A) 0,149\\
B) 0,298\\
C) 0,745\\
D) 0,375
  \item Zichligi $1,17 \mathrm{~g} / \mathrm{ml}$ bo'lgan $40 \%$ li NaCl eritmasining titr konsentratsiyasini hisoblang.\\
A) 0,468\\
B) 0,585\\
C) 0,117\\
D) 0,234
  \item Zichligi $1,9 \mathrm{~g} / \mathrm{ml}$ bo'lgan $60 \% \mathrm{li}$ $\mathrm{MgCl}_{2}$ eritmasining titr konsentratsiyasini hisoblang.\\
A) 1,58\\
B) 1,18\\
C) 1,14\\
D) 1,9
  \item Zichligi $1,2 \mathrm{~g} / \mathrm{ml}$ bo'lgan $35 \% \mathrm{li} \mathrm{NaOH}$ eritmasining titr konsentratsiyasini hisoblang.\\
A) 0,16\\
B) 0,12\\
C) 0,125\\
D) 0,42
  \item Zichligi $1.2 \mathrm{~g} / \mathrm{ml}$ bo'lgan $5 \%$ li LiOH eritmasining titr konsentratsiyasini hisoblang.\\
A) 0,06\\
B) 0,03\\
C) 2,5\\
D) 0,15
  \item 8 M li $\mathrm{H}_{2} \mathrm{SO}_{4}$ eritmasining titr konsentratsiyasini hisoblang.\\
A) 0,116\\
B) 0,13\\
C) 0,784\\
D) 0,145\\
  \item $4,5 \mathrm{M}$ li $\mathrm{HNO}_{3}$ eritmasining titr konsentratsiyasini hisoblang.\\
A) 0,334\\
B) 0,3563\\
C) 0,245\\
D) 0,2835
  \item 1 M li $\mathrm{H}_{3} \mathrm{PO}_{4}$ eritmasining titr konsentratsiyasini hisoblang.\\
A) 0,0245\\
B) 0,049\\
C) 0,098\\
D) 0,98
  \item 5 M li $\mathrm{H}_{2} \mathrm{SO}_{3}$ eritmasining titr konsentratsiyasini hisoblang.\\
A) 0,41\\
B) 0,049\\
C) 0,098\\
D) 0,98
  \item 2 M li $\mathrm{H}_{3} \mathrm{AsO}_{4}$ eritmasining titr konsentratsiyasini hisoblang.\\
A) 0,71\\
B) 0,284\\
C) 0,142\\
D) 0,412
  \item $4,5 \mathrm{M}$ li $\mathrm{Ca}(\mathrm{OH})_{2}$ eritmasining titr konsentratsiyasini hisoblang.\\
A) 0,74\\
B) 0,333\\
C) 0,142\\
D) 0,999
  \item 2 M li HCl eritmasining titr konsentratsiyasini hisoblang.\\
A) 0,73\\
B) 0,365\\
C) 0,073\\
D) 0,0365
  \item 5 M li $\mathrm{H}_{2} \mathrm{CrO}_{4}$ eritmasining titr konsentratsiyasini hisoblang.\\
A) 0,118\\
B) 0,059\\
C) 0,59\\
D) 0,0118
  \item 1 M li $\mathrm{H}_{3} \mathrm{PO}_{4}$ eritmasining titr konsentratsiyasini hisoblang.\\
A) 0,98\\
B) 0,098\\
C) 0,49\\
D) 0,049
  \item $2,5 \mathrm{M}$ li $\mathrm{H}_{2} \mathrm{SO}_{4}$ eritmasining titr konsentratsiyasini hisoblang.\\
A) 0,245\\
B) 0,0245\\
C) 0,49\\
D) 0,049
  \item $8 \mathrm{~N} \mathrm{li} \mathrm{H}_{2} \mathrm{SO}_{4}$ eritmasining titr konsentratsiyasini hisoblang.\\
A) 0,392\\
B) 0,245\\
C) 0,490\\
D) 0.98\\
  \item 4,5 N li HNO ${ }_{3}$ eritmasining titr konsentratsiyasini hisoblang.\\
A) 0,2835\\
B) 0,063\\
C) 0,126\\
D) 0,45
  \item 9 N li $\mathrm{H}_{3} \mathrm{PO}_{4}$ eritmasining titr konsentratsiyasini hisoblang.\\
A) 0,098\\
B) 0,245\\
C) 2,5\\
D) 0,294
  \item $5 \mathrm{~N} \mathrm{li} \mathrm{H}_{2} \mathrm{SO}_{3}$ eritmasining titr konsentratsiyasini hisoblang.\\
A) 0,205\\
B) 0,041\\
C) 0,41\\
D) 0,082
  \item 6 N li $\mathrm{H}_{3} \mathrm{AsO}_{4}$ eritmasining titr konsentratsiyasini hisoblang.\\
A) 0,576\\
B) 0,342\\
C) 0,284\\
D) 0,432
  \item 9 N li $\mathrm{Ca}(\mathrm{OH})_{2}$ eritmasining titr konsentratsiyasini hisoblang.\\
A) 0,333\\
B) 0,125\\
C) 0,666\\
D) 0,074
  \item 2 N li HCl eritmasining titr konsentratsiyasini hisoblang.\\
A) 0,365\\
B) 0,073\\
C) 0,73\\
D) 0,365
  \item 5 N li $\mathrm{H}_{2} \mathrm{CrO}_{4}$ eritmasining titr konsentratsiyasini hisoblang.\\
A) 0,59\\
B) 0,045\\
C) 0,295\\
D) 0,0152
  \item 12 N li $\mathrm{H}_{3} \mathrm{PO}_{4}$ eritmasining titr konsentratsiyasini hisoblang.\\
A) 0,145\\
B) 0,098\\
C) 0,245\\
D) 0,392
  \item $7 \mathrm{~N} \mathrm{li} \mathrm{H}_{2} \mathrm{SO}_{4}$ eritmasining titr konsentratsiyasini hisoblang.\\
A) 0,49\\
B) 0,45\\
C) 0,343\\
D) 0,245
  \item 500 g suvga 117 g NaCl eritildi. Hosil bo'lgan eritmaning molyal konsentratsiyasini aniqlang.\\
A) 1\\
B) 2\\
C) 3\\
D) 4
302. 2500 g suvga $245 \mathrm{~g} \mathrm{H}_{2} \mathrm{SO}_{4}$ eritildi. Hosil bo'lgan eritmaning molyal konsentratsiyasini aniqlang.\\
A) 1\\
B) 2\\
C) 3\\
D) 4\\
303. 600 g suvga $89,4 \mathrm{~g} \mathrm{KCl}$ eritildi. Hosil bo'lgan eritmaning molyal konsentratsiyasini aniqlang.\\
A) 1\\
B) 2\\
C) 3\\
D) 4\\
304. 300 g suvga $75,6 \mathrm{~g} \mathrm{HNO}_{3}$ eritildi. Hosil bo'lgan eritmaning molyal konsentratsiyasini aniqlang.\\
A) 1\\
B) 2\\
C) 3\\
D) 4\\
305. 800 g suvga 16 g HF eritildi. Hosil bo'lgan eritmaning molyal konsentratsiyasini aniqlang.\\
A) 1\\
B) 2\\
C) 3\\
D) 4\\
306. 350 g suvga 42 g NaOH eritildi. Hosil bo'lgan eritmaning molyal konsentratsiyasini aniqlang.\\
A) 1\\
B) 2\\
C) 3\\
D) 4\\
307. 400 g suvga $44,8 \mathrm{~g} \mathrm{KOH}$ eritildi. Hosil bo'lgan eritmaning molyal konsentratsiyasini aniqlang.\\
A) 1\\
B) 2\\
C) 3\\
D) 4\\
308. 250 g suvga 24 g LiOH eritildi. Hosil bo'lgan eritmaning molyal konsentratsiyasini aniqlang.\\
A) 1\\
B) 2\\
C) 3\\
D) 4\\
309. 650 g suvga 26 g NaOH eritildi. Hosil bo'lgan eritmaning molyal konsentratsiyasini aniqlang.\\
A) 1\\
B) 2\\
C) 3\\
D) 4\\
310. 450 g suvga $115,2 \mathrm{~g}$ HI eritildi. Hosil bo'lgan eritmaning molyal konsentratsiyasini aniqlang.\\
A) 1\\
B) 2\\
C) 3\\
D) 4\\
311. $63 \%$ li $\mathrm{HNO}_{3}$ eritmaşining molyal konsentratsiyasini aniqlang.\\
A) 11\\
B) 27\\
C) 13\\
D) 14
312. $40 \%$ li NaOH eritmasining molyal konsentratsiyasini aniqlang.\\
A) 16,67\\
B) 18,88\\
C) 13,33\\
D) 14,44\\
313. 20 \% li HF eritmasining molyal konsentratsiyasini aniqlang.\\
A) 14,6\\
B) 12,7\\
C) 12,5\\
D) 14,5\\
314. $49 \% \mathrm{li} \mathrm{H}_{2} \mathrm{SO}_{4}$ eritmasining molyal konsentratsiyasini aniqlang.\\
A) 9,8\\
B) 11,2\\
C) 13\\
D) 14\\
315. $41 \%$ li $\mathrm{H}_{2} \mathrm{SO}_{3}$ eritmasining molyal konsentratsiyasini aniqlang.\\
A) 11,1\\
B) 9,2\\
C) 13,3\\
D) 8,5\\
316. $56 \%$ li KOH eritmasining molyal konsentratsiyasini aniqlang.\\
A) 11,6\\
B) 22,7\\
C) 13,8\\
D) 14,4\\
317. $69 \%$ li $\mathrm{LiNO}_{3}$ eritmasining molyal konsentratsiyasini aniqlang.\\
A) 32,2\\
B) 15,8\\
C) 18,7\\
D) 14,6\\
318. $60 \% \mathrm{li} \mathrm{MgSO}_{4}$ eritmasining molyal konsentratsiyasini aniqlang.\\
A) 14,6\\
B) 12,7\\
C) 12,5\\
D) 14,5\\
319. $85 \%$ li $\mathrm{NaNO}_{3}$ eritmasining molyal konsentratsiyasini aniqlang.\\
A) 11,11\\
B) 66,67\\
C) 13,33\\
D) 33,33\\
320. $28 \%$ li KOH eritmasining molyal konsentratsiyasini aniqlang.\\
A) 1\\
B) 7\\
C) 3\\
D) 4
  \item Ma'lum haroratda NaOH ning eruvchanlik koeffitsiyenti 60 ga teng. Shu haroratdagi eritmaning molyal konsentratsiyasi qanchaga teng?\\
A) 15\\
B) 12\\
C) 13\\
D) 14
  \item Ma'lum haroratda KOH ning eruvchanlik koeffitsiyenti 67,2 ga teng. Shu haroratdagi eritmaning molyal konsentratsiyasi qanchaga teng?\\
A) 15\\
B) 12\\
C) 13\\
D) 14
  \item Ma'lum haroratda $\mathrm{LiNO}_{3}$ ning eruvchanlik koeffitsiyenti 34,5 ga teng. Shu haroratdagi eritmaning molyal konsentratsiyasi qanchaga teng?\\
A) 5\\
B) 2\\
C) 3\\
D) 4
  \item Ma'lum haroratda NaCl ning eruvchanlik koeffitsiyenti $58,5 \mathrm{ga}$ teng. Shu haroratdagi eritmaning molyal konsentratsiyasi qanchaga teng?\\
A) 15\\
B) 11\\
C) 13\\
D) 10
  \item Ma'lum haroratda $\mathrm{Ca}(\mathrm{OH})_{2}$ ning eruvchanlik koeffitsiyenti 7,4 ga teng. Shu haroratdagi eritmaning molyal konsentratsiyasi qanchaga teng?\\
A) 1\\
B) 2\\
C) 3\\
D) 4
  \item Ma'lum haroratda HF ning eruvchanlik koeffitsiyenti 4 ga teng. Shu haroratdagi eritmaning molyal konsentratsiyasi qanchaga teng?\\
A) 1\\
B) 2\\
C) 3\\
D) 4
  \item Ma'lum haroratda HCl ning eruvchanlik koeffitsiyenti 7,3 ga teng. Shu haroratdagi eritmaning molyal konsentratsiyasi qanchaga teng?\\
A) 1\\
B) 2\\
C) 3\\
D) 4
  \item Ma'lum haroratda LiOH ning eruvchanlik koeffitsiyenti 24 ga teng. Shu haroratdagi eritmaning molyal konsentratsiyasi qanchaga teng?\\
A) 15\\
B) 11\\
C) 13\\
D) 10
  \item Ma'lum haroratda $\mathrm{MgSO}_{4}$ ning eruvchanlik koeffitsiyenti 60 ga teng. Shu haroratdagi eritmaning molyal konsentratsiyasi qanchaga teng?\\
A) 5\\
B) 2\\
C) 3\\
D) 4
  \item Ma'lum haroratda $\mathrm{Na}_{2} \mathrm{SO}_{4}$ ning eruvchanlik koeffitsiyenti $35,5 \mathrm{ga}$ teng. Shu haroratdagi eritmaning molyal konsentratsiyasi qanchaga teng?\\
A) 1,5\\
B) 1,2\\
C) 2,5\\
D) 1,4
  \item 3 M li zichligi $1,12 \mathrm{~g} / \mathrm{ml}$ bo'lgan NaOH eritmasining molyal konsentratsiyasini aniqlang?\\
A) 5\\
B) 2\\
C) 3\\
D) 4
332. 2 M li zichligi $1,126 \mathrm{~g} / \mathrm{ml}$ bo'lgan $\mathrm{HNO}_{3}$ eritmasining molyal konsentratsiyasini aniqlang?\\
A) 5\\
B) 2\\
C) 3\\
D) 4\\
333. 2,5 M li zichligi $1,06 \mathrm{~g} / \mathrm{ml}$ bo'lgan LiOH eritmasining molyal konsentratsiyasini aniqlang?\\
A) 4\\
B) 2,5\\
C) 3,2\\
D) 4\\
334. 4 M li zichligi $1,568 \mathrm{~g} / \mathrm{ml}$ bo'lgan $\mathrm{Na}_{2} \mathrm{SO}_{4}$ eritmasining molyal konsentratsiyasini aniqlang?\\
A) 5\\
B) 2\\
C) 3\\
D) 4\\
335. $1,5 \mathrm{M}$ li zichligi $1,18 \mathrm{~g} / \mathrm{ml}$ bo'lgan $\mathrm{MgSO}_{4}$ eritmasining molyal konsentratsiyasini aniqlang?\\
A) 1,5\\
B) 2,5\\
C) 1,3\\
D) 1,4\\
336. 2 M li zichligi $1,117 \mathrm{~g} / \mathrm{ml}$ bo'lgan NaCl eritmasining molyal konsentratsiyasini aniqlang?\\
A) 5\\
B) 2\\
C) 3\\
D) 4\\
337. $2,5 \mathrm{M}$ li zichligi $1,14 \mathrm{~g} / \mathrm{ml}$ bo'lgan KOH eritmasining molyal konsentratsiyasini aniqlang?\\
A) 1,5\\
B) 2,5\\
C) 1,3\\
D) 1,4\\
338. 5 M li zichligi $1,1 \mathrm{~g} / \mathrm{ml}$ bo'lgan HF eritmasining molyal konsentratsiyasini aniqlang?\\
A) 5\\
B) 2\\
C) 3\\
D) 4\\
339. $2,5 \mathrm{M}$ li zichligi $1,1 \mathrm{~g} / \mathrm{ml}$ bo'lgan NaOH eritmasining molyal konsentratsiyasini aniqlang?\\
A) 1,5\\
B) 2.5\\
C) 1.3\\
D) 1.4\\
340. 4 M li zichligi $1.146 \mathrm{~g} / \mathrm{ml}$ bo'lgan HCl eritmasining molyal konsentratsiyasini aniqlang?\\
A) 5\\
B) 2\\
C) 3\\
D) 4
  \item 44,8 litr da o'lchangan HCl 500 g suvga eritildi. Hosil bo'lgan eritmaning molyal konsentratsiyasini aniqlang.\\
A) 5\\
B) 2\\
C) 3\\
D) 4\\
  \item 22,4 litr da o'lchangan HF 800 g suvga eritildi. Hosil bo'lgan eritmaning molyal konsentratsiyasini aniqlang.\\
A) 1,25\\
B) 2,5\\
C) 1,5\\
D) 1,4
  \item 11,2 litr da o'lchangan HI 200 g suvga eritildi. Hosil bo'lgan eritmaning molyal konsentratsiyasini aniqlang.\\
A) 1,25\\
B) 2,5\\
C) 1,5\\
D) 1,4
  \item 33,6 litr da o'lchangan $\mathrm{NH}_{3} 500$ g suvga eritildi. Hosil bo'lgan eritmaning molyal konsentratsiyasini aniqlang.\\
A) 5\\
B) 2\\
C) 3\\
D) 4
  \item 44,8 litr da o'lchangan HCN 1000 g suvga eritildi. Hosil bo'lgan eritmaning molyal konsentratsiyasini aniqlang.\\
A) 5\\
B) 2\\
C) 3\\
D) 4
  \item 44,8 litr da o'lchangan $\mathrm{H}_{2} \mathrm{~S} 800$ g suvga eritildi. Hosil bo'lgan eritmaning molyal konsentratsiyasini aniqlang.\\
A) 1,25\\
B) 2,5\\
C) 1,5\\
D) 1,4
  \item 22,4 litr da o'lchangan HBr 200 g suvga eritildi. Hosil bo'lgan eritmaning molyal konsentratsiyasini aniqlang.\\
A) 5\\
B) 2\\
C) 3\\
D) 4
  \item 67.2 litr da o'lchangan HCl 1000 g suvga eritildi. Hosil bo'lgan eritmaning molyal konsentratsiyasini aniqlang.\\
A) 5\\
B) 2\\
C) 3\\
D) 4
349, 11,2 litr da o'lchangan HCl 500 g suvga eritildi. Hosil bo'lgan eritmaning molyal konsentratsiyasini aniqlang.\\
A) 1\\
B) 2\\
C) 3\\
D) 4\\
350. 44,8 litr da o'lchangan $\mathrm{NH}_{3}$ 2500 g suvga eritildi. Hosil bo'lgan eritmaning molyal konsentratsiyasini aniqlang.\\
A) 1,5\\
B) 1,2\\
C) 0,8\\
D) 0,4
  \item $30^{\circ} \mathrm{C}$ haroratda NaCl ning to'yingan eritmasini tayyorlash uchun 400 g suvga 120 g NaCl eritildi. Eritmaning $30^{\circ} \mathrm{C}$ haroratdagi eruvchanlik koeffitsiyentini aniqlang.\\
A) 10\\
B) 20\\
C) 30\\
D) 40\\
  \item $40^{\circ} \mathrm{C}$ haroratda NaOH ning to'yingan eritmasini tayyorlash uchun 200 g suvga 120 g NaOH eritildi. Eritmaning $40^{\circ} \mathrm{C}$ haroratdagi eruvchanlik koeffitsiyentini aniqlang.\\
A) 60\\
B) 40\\
C) 30\\
D) 50
  \item $50^{\circ} \mathrm{C}$ haroratda KCl ning to'yingan eritmasini tayyorlash uchun 300 g suvga 120 g KCl eritildi. Eritmaning $50^{\circ} \mathrm{C}$ haroratdagi eruvchanlik koeffitsiyentini aniqlang.\\
A) 10\\
B) 20\\
C) 30\\
D) 40
  \item $25^{\circ} \mathrm{C}$ haroratda $\mathrm{MgSO}_{4}$ ning to'yingan eritmasini tayyorlash uchun 500 g suvga $120 \mathrm{~g} \mathrm{MgSO}_{4}$ eritildi. Eritmaning $25^{\circ} \mathrm{C}$ haroratdagi eruvchanlik koeffitsiyentini aniqlang.\\
A) 14\\
B) 24\\
C) 35\\
D) 42
  \item $20^{\circ} \mathrm{C}$ haroratda $\mathrm{Ca}(\mathrm{OH})_{2}$ ning to'yingan eritmasini tayyorlash uchun 400 g suvga $40 \mathrm{~g} \mathrm{Ca}(\mathrm{OH})_{2}$ eritildi. Eritmaning\\
$20^{\circ} \mathrm{C}$ haroratdagi eruvchanlik koeffitsiyentini aniqlang.\\
A) 10\\
B) 20\\
C) 30\\
D) 40
  \item $10{ }^{\circ} \mathrm{C}$ haroratda LiCl ning to'yingan eritmasini tayyorlash uchun 250 g suvga 120 g LiCl eritildi. Eritmaning $10{ }^{\circ} \mathrm{C}$ haroratdagi eruvchanlik koeffitsiyentini aniqlang.\\
A) 17\\
B) 28\\
C) 34\\
D) 48
  \item $70^{\circ} \mathrm{C}$ haroratda $\mathrm{Na}_{2} \mathrm{SO}_{4}$ ning to'yingan eritmasini tayyorlash uchun 500 g suvga $150 \mathrm{~g} \mathrm{Na}_{2} \mathrm{SO}_{4}$ eritildi. Eritmaning $70^{\circ} \mathrm{C}$ haroratdagi eruvchanlik koeffitsiyentini aniqlang.\\
A) 10\\
B) 20\\
C) 30\\
D) 40
  \item $30^{\circ} \mathrm{C}$ haroratda RbCl ning to'yingan eritmasini tayyorlash uchun 300 g suvga 60 g RbCl eritildi. Eritmaning $30^{\circ} \mathrm{C}$ haroratdagi eruvchanlik koeffitsiyentini aniqlang.\\
A) 10\\
B) 20\\
C) 30\\
D) 40
  \item $15^{\circ} \mathrm{C}$ haroratda KOH ning to'yingan eritmasini tayyorlash uchun 400 g suvga 160 g KOH eritildi. Eritmaning $15^{\circ} \mathrm{C}$ haroratdagi eruvchanlik koeffitsiyentini aniqlang.\\
A) 10\\
B) 20\\
C) 30\\
D) 40
  \item $25^{\circ} \mathrm{C}$ haroratda $\mathrm{NaNO}_{3}$ ning to'yingan eritmasini tayyorlash uchun 300 g suvga $150 \mathrm{~g} \mathrm{NaNO}_{3}$ eritildi. Eritmaning $25^{\circ} \mathrm{C}$ haroratdagi eruvchanlik koeffitsiyentini aniqlang.\\
A) 60\\
B) 50\\
C) 30\\
D) 40
  \item $25^{\circ} \mathrm{C} \mathrm{NaOH}$ ning 150 g to'yingan eritmasida 75 g NaOH erigan bo'lsa, shu haroratdagi to'yingan eritmaning eruvchanlik koeffitsiyentini aniqlang.\\
A) 60\\
B) 100\\
C) 30\\
D) 40
  \item $20^{\circ} \mathrm{C} \mathrm{KOH}$ ning 150 g to'yingan eritmasida 50 g KOH erigan bo'lsa, shu\\
haroratdagi to'yingan eritmaning eruvchanlik koeffitsiyentini aniqlang.\\
A) 50\\
B) 60\\
C) 70\\
D) 80
  \item $15^{\circ} \mathrm{C} \mathrm{NaCl}$ ning 120 g to'yingan eritmasida 40 g NaCl erigan bo'lsa, shu haroratdagi to'yingan eritmaning eruvchanlik koeffitsiyentini aniqlang.\\
A) 50\\
B) 60\\
C) 70\\
D) 80
  \item $40^{\circ} \mathrm{C} \mathrm{KCl}$ ning 80 g to'yingan eritmasida 30 g KCl erigan bo'lsa, shu haroratdagi to'yingan eritmaning eruvchanlik koeffitsiyentini aniqlang.\\
A) 50\\
B) 60\\
C) 70\\
D) 80
  \item $25^{\circ} \mathrm{C} \mathrm{RbOH}$ ning 90 g to'yingan eritmasida 40 g RbOH erigan bo'lsa, shu haroratdagi to'yingan eritmaning eruvchanlik koeffitsiyentini aniqlang.\\
A) 50\\
B) 60\\
C) 70\\
D) 80
  \item $15{ }^{\circ} \mathrm{C} \mathrm{NaNO}_{3}$ ning 60 g to'yingan eritmasida $20 \mathrm{~g} \mathrm{NaNO}_{3}$ erigan bo'lsa, shu haroratdagi to'yingan eritmaning eruvchanlik koeffitsiyentini aniqlang.\\
A) 50\\
B) 60\\
C) 70\\
D) 80
  \item $20^{\circ} \mathrm{C} \mathrm{Na}_{2} \mathrm{SO}_{3}$ ning 100 g to'yingan eritmasida $60 \mathrm{~g} \mathrm{Na}_{2} \mathrm{SO}_{3}$ erigan bo'lsa, shu haroratdagi to'yingan eritmaning eruvchanlik koeffitsiyentini aniqlang.\\
A) 160\\
B) 100\\
C) 150\\
D) 140
  \item $28^{\circ} \mathrm{C} \mathrm{Na}{ }_{2} \mathrm{~S}$ ning 70 g to'yingan eritmasida 20 g NaOH erigan bo'lsa, shu haroratdagi to'yingan eritmaning eruvchanlik koeffitsiyentini aniqlang.\\
A) 60\\
B) 100\\
C) 30\\
D) 40
  \item $30^{\circ} \mathrm{C} \mathrm{NaClO}_{3}$ ning 180 g to'yingan eritmasida $60 \mathrm{~g} \mathrm{NaClO}_{3}$ erigan bo'lsa, shu haroratdagi to'yingan eritmaning eruvchanlik koeffitsiyentini aniqlang.\\
A) 50\\
B) 60\\
C) 70\\
D) 80
  \item $25^{\circ} \mathrm{C} \mathrm{KCl}$ ning 90 g to'yingan eritmasida 10 g KCl erigan bo'lsa, shu haroratdagi to'yingan eritmaning eruvchanlik koeffitsiyentini aniqlang.\\
A) 14\\
B) 12,5\\
C) 15\\
D) 14,6
  \item $80^{\circ} \mathrm{C}$ dagi NaCl ning 330 g eritmasi $20^{\circ} \mathrm{C}$ gacha sovitilganda necha g tuz cho'kmaga tushadi?\\
$\mathrm{S}\left(80^{\circ} \mathrm{C}\right)=120, \mathrm{~S}\left(20^{\circ} \mathrm{C}\right)=55$\\
A) 67,5\\
B) 70\\
C) 125\\
D) 97,5\\
  \item $60^{\circ} \mathrm{C}$ dagi KCl ning 360 g eritmasi 30\\
${ }^{\circ} \mathrm{C}$ gacha sovitilganda necha g tuz cho'kmaga tushadi?\\
$\mathrm{S}\left(60^{\circ} \mathrm{C}\right)=80, \mathrm{~S}\left(30^{\circ} \mathrm{C}\right)=30$\\
A) 60\\
B) 100\\
C) 115\\
D) 90
  \item $50^{\circ} \mathrm{C}$ dagi $\mathrm{NaNO}_{3}$ ning 320 g eritmasi $10^{\circ} \mathrm{C}$ gacha sovitilganda necha g tuz cho'kmaga tushadi?\\
$\mathrm{S}\left(50^{\circ} \mathrm{C}\right)=60, \mathrm{~S}\left(10^{\circ} \mathrm{C}\right)=25$\\
A) 67,5\\
B) 70\\
C) 125\\
D) 97,5
  \item $90^{\circ} \mathrm{C}$ dagi $\mathrm{MgSO}_{4}$ ning 400 g eritmasi $50^{\circ} \mathrm{C}$ gacha sovitilganda necha g tuz cho'kmaga tushadi?\\
$\mathrm{S}\left(90^{\circ} \mathrm{C}\right)=100, \mathrm{~S}\left(50^{\circ} \mathrm{C}\right)=40$\\
A) 60\\
B) 100\\
C) 110\\
D) 120
  \item $60^{\circ} \mathrm{C}$ dagi LiCl ning 450 g eritmasi 10 ${ }^{\circ} \mathrm{C}$ gacha sovitilganda necha g tuz cho'kmaga tushadi?\\
$\mathrm{S}\left(60^{\circ} \mathrm{C}\right)=50, \mathrm{~S}\left(10^{\circ} \mathrm{C}\right)=15$\\
A) 65\\
B) 70\\
C) 105\\
D) 95
  \item $40^{\circ} \mathrm{C}$ dagi $\mathrm{MgSO}_{4}$ ning 310 g eritmasi $20^{\circ} \mathrm{C}$ gacha sovitilganda necha g tuz cho'kmaga tushadi?\\
$\mathrm{S}\left(40^{\circ} \mathrm{C}\right)=55, \mathrm{~S}\left(20^{\circ} \mathrm{C}\right)=25$\\
A) 50\\
B) 60\\
C) 70\\
D) 80
  \item $30^{\circ} \mathrm{C}$ dagi $\mathrm{Na}_{2} \mathrm{CrO}_{4}$ ning 360 g eritmasi $10^{\circ} \mathrm{C}$ gacha sovitilganda necha g tuz cho'kmaga tushadi?\\
$\mathrm{S}\left(30^{\circ} \mathrm{C}\right)=20, \mathrm{~S}\left(10^{\circ} \mathrm{C}\right)=5$\\
A) 60\\
B) 10\\
C) 35\\
D) 45
  \item $35^{\circ} \mathrm{C}$ dagi $\mathrm{AlF}_{3}$ ning 360 g eritmasi 15 ${ }^{\circ} \mathrm{C}$ gacha sovitilganda necha g tuz cho'kmaga tushadi?\\
$\mathrm{S}\left(35^{\circ} \mathrm{C}\right)=80, \mathrm{~S}\left(15^{\circ} \mathrm{C}\right)=15$\\
A) 65\\
B) 110\\
C) 125\\
D) 97
  \item $70^{\circ} \mathrm{C}$ dagi $\mathrm{Na}_{2} \mathrm{CO}_{3}$ ning 570 g eritmasi $20^{\circ} \mathrm{C}$ gacha sovitilganda necha g tuz cho'kmaga tushadi?\\
$\mathrm{S}\left(70^{\circ} \mathrm{C}\right)=90, \mathrm{~S}\left(20^{\circ} \mathrm{C}\right)=30$\\
A) 180\\
B) 120\\
C) 125\\
D) 250
  \item $60^{\circ} \mathrm{C}$ dagi NaF ning 300 g eritmasi 25 ${ }^{\circ} \mathrm{C}$ gacha sovitilganda necha g tuz cho'kmaga tushadi?\\
$\mathrm{S}\left(60^{\circ} \mathrm{C}\right)=50, \mathrm{~S}\left(25^{\circ} \mathrm{C}\right)=25$\\
A) 67,5\\
B) 50\\
C) 25\\
D) 75
  \item $80^{\circ} \mathrm{C}$ dagi NaCl ning to'yingan eritmasi $20^{\circ} \mathrm{C}$ gacha sovitilganda 195 g tuz cho'kmaga tushsa, $80^{\circ} \mathrm{C}$ dagi to'yingan eritma massasini toping.\\
$\mathrm{S}\left(80^{\circ} \mathrm{C}\right)=120, \mathrm{~S}\left(20^{\circ} \mathrm{C}\right)=55$\\
A) 675\\
B) 700\\
C) 625\\
D) 660
  \item $60^{\circ} \mathrm{C}$ dagi KCl ning to'yingan eritmasi $30^{\circ} \mathrm{C}$ gacha sovitilganda 200 g tuz cho'kmaga tushsa, $60^{\circ} \mathrm{C}$ dagi to'yingan eritma massasini toping.\\
$\mathrm{S}\left(60^{\circ} \mathrm{C}\right)=80, \mathrm{~S}\left(30^{\circ} \mathrm{C}\right)=30$\\
A) 610\\
B) 800\\
C) 720\\
D) 900
  \item $50^{\circ} \mathrm{C}$ dagi $\mathrm{NaNO}_{3}$ ning to'yingan eritmasi $10^{\circ} \mathrm{C}$ gacha sovitilganda 140 g tuz cho'kmaga tushsa, $50^{\circ} \mathrm{C}$ dagi to'yingan eritma massasini toping.\\
$\mathrm{S}\left(50^{\circ} \mathrm{C}\right)=60, \mathrm{~S}\left(10^{\circ} \mathrm{C}\right)=25$\\
A) 675\\
B) 640\\
C) 625\\
D) 975
  \item $90^{\circ} \mathrm{C}$ dagi $\mathrm{MgSO}_{4}$ ning to'yingan eritmasi $50^{\circ} \mathrm{C}$ gacha sovitilganda 240 g tuz cho'kmaga tushsa, $90^{\circ} \mathrm{C}$ dagi to'yingan eritma massasini toping.\\
$\mathrm{S}\left(90^{\circ} \mathrm{C}\right)=100, \mathrm{~S}\left(50^{\circ} \mathrm{C}\right)=40$\\
A) 800\\
B) 700\\
C) 710\\
D) 720
  \item $60^{\circ} \mathrm{C}$ dagi LiCl ning to'yingan eritmasi $10^{\circ} \mathrm{C}$ gacha sovitilganda 210 g tuz cho'kmaga tushsa, $90^{\circ} \mathrm{C}$ dagi to'yingan eritma massasini toping.\\
$\mathrm{S}\left(60^{\circ} \mathrm{C}\right)=50, \mathrm{~S}\left(10^{\circ} \mathrm{C}\right)=15$\\
A) 650\\
B) 702\\
C) 805\\
D) 900
  \item $40^{\circ} \mathrm{C}$ dagi $\mathrm{MgSO}_{4}$ ning to'yingan eritmasi $20^{\circ} \mathrm{C}$ gacha sovitilganda 120 g tuz cho'kmaga tushsa, $40^{\circ} \mathrm{C}$ dagi to'yingan eritma massasini toping.\\
$\mathrm{S}\left(40^{\circ} \mathrm{C}\right)=55 . \mathrm{S}\left(20^{\circ} \mathrm{C}\right)=25$\\
A) 620\\
B) 600\\
C) 700\\
D) 820
  \item $30^{\circ} \mathrm{C}$ dagi $\mathrm{Na}_{2} \mathrm{CrO}_{4}$ ning to'yingan eritmasi $10^{\circ} \mathrm{C}$ gacha sovitilganda 90 g tuz cho'kmaga tushsa, $30^{\circ} \mathrm{C}$ dagi to'yingan eritma massasini toping.\\
$\mathrm{S}\left(30^{\circ} \mathrm{C}\right)=20, \mathrm{~S}\left(10^{\circ} \mathrm{C}\right)=5$\\
A) 600\\
B) 720\\
C) 350\\
D) 450
  \item $35^{\circ} \mathrm{C}$ dagi $\mathrm{AlF}_{3}$ ning to'yingan eritmasi $15^{\circ} \mathrm{C}$ gacha sovitilganda 260 g tuz cho'kmaga tushsa, $35^{\circ} \mathrm{C}$ dagi to'yingan eritma massasini toping.\\
$\mathrm{S}\left(35^{\circ} \mathrm{C}\right)=80, \mathrm{~S}\left(15^{\circ} \mathrm{C}\right)=15$\\
A) 765\\
B) 710\\
C) 725\\
D) 720
  \item $70^{\circ} \mathrm{C}$ dagi $\mathrm{Na}_{2} \mathrm{CO}_{3}$ ning to'yingan eritmasi $20^{\circ} \mathrm{C}$ gacha sovitilganda 180 g tuz cho'kmaga tushsa, $70^{\circ} \mathrm{C}$ dagi to'yingan eritma massasini toping.\\
$\mathrm{S}\left(70^{\circ} \mathrm{C}\right)=90, \mathrm{~S}\left(20^{\circ} \mathrm{C}\right)=30$\\
A) 570\\
B) 520\\
C) 525\\
D) 550
  \item $60^{\circ} \mathrm{C}$ dagi NaF ning to'yingan eritmasi $25^{\circ} \mathrm{C}$ gacha sovitilganda 100 g tuz cho'kmaga tushsa, $60^{\circ} \mathrm{C}$ dagi to'yingan eritma massasini toping.\\
$\mathrm{S}\left(60^{\circ} \mathrm{C}\right)=50, \mathrm{~S}\left(25^{\circ} \mathrm{C}\right)=25$\\
A) 67,5\\
B) 600\\
C) 25\\
D) 75
  \item $80^{\circ} \mathrm{C}$ dagi NaCl ning to'yingan eritmasi $20^{\circ} \mathrm{C}$ gacha sovitilganda 195 g tuz cho'kmaga tushsa, $80^{\circ} \mathrm{C}$ dagi to'yingan eritma tarkibidagi suvning massasini toping.\\
$\mathrm{S}\left(80^{\circ} \mathrm{C}\right)=120, \mathrm{~S}\left(20^{\circ} \mathrm{C}\right)=55$\\
A) 300\\
B) 400\\
C) 325\\
D) 360
  \item $60^{\circ} \mathrm{C}$ dagi KCl ning to'yingan eritmasi $30^{\circ} \mathrm{C}$ gacha sovitilganda 200 g tuz cho'kmaga tushsa, $60^{\circ} \mathrm{C}$ dagi to'yingan eritma tarkibidagi suvning massasini toping.\\
$\mathrm{S}\left(60^{\circ} \mathrm{C}\right)=80, \mathrm{~S}\left(30^{\circ} \mathrm{C}\right)=30$\\
A) 310\\
B) 500\\
C) 420\\
D) 400
  \item $50^{\circ} \mathrm{C}$ dagi $\mathrm{NaNO}_{3}$ ning to'yingan eritmasi $10^{\circ} \mathrm{C}$ gacha sovitilganda 140 g tuz cho'kmaga tushsa, $50^{\circ} \mathrm{C}$ dagi to'yingan eritma tarkibidagi suvning massasini toping.\\
$\mathrm{S}\left(50^{\circ} \mathrm{C}\right)=60, \mathrm{~S}\left(10^{\circ} \mathrm{C}\right)=25$\\
A) 310\\
B) 500\\
C) 420\\
D) 400
  \item $90^{\circ} \mathrm{C}$ dagi $\mathrm{MgSO}_{4}$ ning to'yingan eritmasi $50^{\circ} \mathrm{C}$ gacha sovitilganda 120 g tuz cho'kmaga tushsa, $90^{\circ} \mathrm{C}$ dagi to'yingan eritma tarkibidagi suvning massasini toping.\\
$\mathrm{S}\left(90^{\circ} \mathrm{C}\right)=100, \mathrm{~S}\left(50^{\circ} \mathrm{C}\right)=40$\\
A) 200\\
B) 300\\
C) 410\\
D) 520
  \item $60^{\circ} \mathrm{C}$ dagi LiCl ning to'yingan eritmasi $10^{\circ} \mathrm{C}$ gacha sovitilganda 210 g tuz cho'kmaga tushsa, $60^{\circ} \mathrm{C}$ dagi to'yingan eritma tarkibidagi suvning massasini toping.\\
$\mathrm{S}\left(60^{\circ} \mathrm{C}\right)=50, \mathrm{~S}\left(10^{\circ} \mathrm{C}\right)=15$\\
A) 610\\
B) 302\\
C) 505\\
D) 600
  \item $40^{\circ} \mathrm{C}$ dagi $\mathrm{MgSO}_{4}$ ning to'yingan eritmasi $20^{\circ} \mathrm{C}$ gacha sovitilganda 75 g tuz cho'kmaga tushsa, $40^{\circ} \mathrm{C}$ dagi to'yingan eritma tarkibidagi suvning massasini toping.\\
$\mathrm{S}\left(40^{\circ} \mathrm{C}\right)=55, \mathrm{~S}\left(20^{\circ} \mathrm{C}\right)=25$\\
A) 220\\
B) 200\\
C) 250\\
D) 320
  \item $30^{\circ} \mathrm{C}$ dagi $\mathrm{Na}_{2} \mathrm{CrO}_{4}$ ning to'yingan eritmasi $10^{\circ} \mathrm{C}$ gacha sovitilganda 90 g tuz cho'kmaga tushsa, $30^{\circ} \mathrm{C}$ dagi to'yingan eritma tarkibidagi suvning massasini toping.\\
$\mathrm{S}\left(30^{\circ} \mathrm{C}\right)=20, \mathrm{~S}\left(10^{\circ} \mathrm{C}\right)=5$\\
A) 600\\
B) 720\\
C) 350\\
D) 450
  \item $35^{\circ} \mathrm{C}$ dagi $\mathrm{AlF}_{3}$ ning to'yingan eritmasi $15^{\circ} \mathrm{C}$ gacha sovitilganda 220 g tuz cho'kmaga tushsa, $35^{\circ} \mathrm{C}$ dagi to'yingan\\
oritima Larkihidagi auvning тянния lopink.\\
$\mathrm{S}\left(35^{\circ} \mathrm{C}\right)=70,8\left(1 \beta^{\prime} \mathrm{C}\right) \neg 1$ ค\\
A) 310\\
B) 500\\
C) 420\\
l)) $4(0$
  \item $70^{\circ} \mathrm{C}$ dagi $\mathrm{Na}_{2} \mathrm{CO}_{3}$ ning to'yingan eritimnas $20^{\circ} \mathrm{C}$ ancha sovitilganda 180 kg tuz cho'kmagn turhan, $70^{\circ} \mathrm{C}$ dagi to'yingan oritma tarkibídngi suvníng marasaminí toping,\\
$\mathrm{S}\left(70^{\circ} \mathrm{C}\right)=90, \mathrm{~S}\left(20^{\circ} \mathrm{C}\right)=30$\\
A) 300\\
B) 400\\
C) 500\\
D) 600
  \item $60^{\circ} \mathrm{C}$ dingi NaF ning to'yingon oritmasi $25^{\circ} \mathrm{C}$ gacha sovitilganda 50 g tuz cho'kmaga tushsa, $60^{\circ} \mathrm{C}$ dagi to'yingan eritma tarkibidagi suvning massasini toping.\\
$\mathrm{S}\left(60^{\circ} \mathrm{C}\right)=50, \mathrm{~S}\left(25^{\circ} \mathrm{C}\right)=25$\\
A) 220\\
B) 200\\
C) 250\\
D) 320
  \item 129 g oleum eritmasi tarkibida 80 g $\mathrm{SO}_{3}$ bo'lsa, oleum tarkibini toping.\\
A) $\mathrm{H}_{2} \mathrm{SO}_{4} \cdot 2 \mathrm{SO}_{3}$\\
B) $\mathrm{H}_{2} \mathrm{SO}_{4} \cdot 1,5 \mathrm{SO}_{3}$\\
C) $2 \mathrm{H}_{2} \mathrm{SO}_{4} \cdot \mathrm{SO}_{3}$\\
D) $\mathrm{H}_{2} \mathrm{SO}_{4} \cdot \mathrm{SO}_{3}$
  \item 89 g oleum eritmasi tarkibida 40 g $\mathrm{SO}_{3}$ bo'lsa, oleum tarkibini toping.\\
A) $\mathrm{H}_{2} \mathrm{SO}_{4} \cdot 2 \mathrm{SO}_{3}$\\
B) $\mathrm{H}_{2} \mathrm{SO}_{4} \cdot 1,5 \mathrm{SO}_{3}$\\
C) $2 \mathrm{H}_{2} \mathrm{SO}_{4} \cdot \mathrm{SO}_{3}$\\
D) $\mathrm{H}_{2} \mathrm{SO}_{4} \cdot \mathrm{SO}_{3}$
  \item 138 g oleum eritmasi tarkibida 40 g $\mathrm{SO}_{3}$ bo'lsa, oleum tarkibini toping.\\
A) $\mathrm{H}_{2} \mathrm{SO}_{4} \cdot 2 \mathrm{SO}_{3}$\\
B) $\mathrm{H}_{2} \mathrm{SO}_{4} \cdot 1,5 \mathrm{SO}_{3}$\\
C) $2 \mathrm{H}_{2} \mathrm{SO}_{4} \cdot \mathrm{SO}_{3}$\\
D) $\mathrm{H}_{2} \mathrm{SO}_{4} \cdot \mathrm{SO}_{3}$
  \item 109 g oleum eritmasi tarkibida 60 g $\mathrm{SO}_{3}$ bo'lsá, oleum tarkibini toping.\\
A) $\mathrm{H}_{2} \mathrm{SO}_{4} \cdot 2 \mathrm{SO}_{3}$\\
B) $\mathrm{H}_{2} \mathrm{SO}_{4} \cdot 1,5 \mathrm{SO}_{3}$\\
C) $2 \mathrm{H}_{2} \mathrm{SO}_{4} \cdot \mathrm{SO}_{3}$\\
D) $\mathrm{H}_{2} \mathrm{SO}_{4} \cdot \mathrm{SO}_{3}$
  \item $44,5 \mathrm{~g}$ oleum eritmasi tarkibida 20 g $\mathrm{SO}_{3}$ bo'lsa, oleum tarkibini toping.\\
A) $\mathrm{H}_{2} \mathrm{SO}_{4} \cdot 2 \mathrm{SO}_{3}$\\
B) $\mathrm{H}_{2} \mathrm{SO}_{4} \cdot 1,5 \mathrm{SO}_{3}$\\
C) $2 \mathrm{H}_{2} \mathrm{SO}_{4} \cdot \mathrm{SO}_{3}$\\
D) $\mathrm{H}_{2} \mathrm{SO}_{4} \cdot \mathrm{SO}_{3}$
408, 109 « olcum eritmasi tarkíhida 651 \% $\left.\mathrm{SO}_{3} \mathrm{bo}^{\prime}\right]_{\text {ка, olcum tarkibini toping. }}$\\
A) $\mathrm{H}_{2,} \mathrm{SO}_{4} \cdot 3 \mathrm{SO}_{3}$\\
B) $\mathrm{H}_{2} \mathrm{SO}_{4} \cdot 5 \mathrm{SO}_{2}$\\
C) $2 \mathrm{H}_{2} \mathrm{SO}_{4} \cdot 3 \mathrm{SO}_{3}$\\
D) $\mathrm{H}_{2} \mathrm{SO}_{4} \cdot \mathrm{SO} \geqslant$\\
109. 258 g oleum critmasi tarkibida $160 \%$ $\mathrm{SO}_{3}$ bo'lsa, oleum tarkíbini toping.\\
A) $\mathrm{H}_{2} \mathrm{SO}_{4} \cdot 2 \mathrm{SO}_{3}$\\
B) $\mathrm{H}_{2} \mathrm{SO}_{4} \cdot 1,5 \mathrm{SO}_{2}$,\\
C) $2 \mathrm{H}_{2} \mathrm{SO}_{4} \cdot \mathrm{SO}_{3}$\\
D) $\mathrm{H}_{2} \mathrm{SO}_{4} \cdot \mathrm{SO}_{3}$\\
$410_{2} 178 \mathrm{~g}$ oleum erítmasí tarkíbída 80 g $\mathrm{SO}_{3}$ bo'lsa, oleum tarkibini toping.\\
A) $\mathrm{H}_{2} \mathrm{SO}_{4} \cdot 2 \mathrm{SO}_{3}$\\
B) $\mathrm{H}_{2} \mathrm{SO}_{4} \cdot 1,5 \mathrm{SO}_{3}$\\
C) $2 \mathrm{H}_{2} \mathrm{SO}_{4} \cdot \mathrm{SO}_{3}$\\
D) $\mathrm{H}_{2} \mathrm{SO}_{4} \cdot \mathrm{SO}_{3}$
  \item Oleum tarkibida oltingugurning massa ulushi $16 / 44,5$ ga teng bo'lsa, oleum tarkibini toping.\\
A) $\mathrm{H}_{2} \mathrm{SO}_{4} \cdot 2 \mathrm{SO}_{3}$\\
B) $\mathrm{H}_{2} \mathrm{SO}_{4} \cdot 1,5 \mathrm{SO}_{3}$\\
C) $2 \mathrm{H}_{2} \mathrm{SO}_{4} \cdot \mathrm{SO}_{3}$\\
D) $\mathrm{H}_{2} \mathrm{SO}_{4} \cdot \mathrm{SO}_{3}$
  \item Oleum tarkibida oltingugurning massa ulushi $24 / 64,5 \mathrm{ga}$ teng bo'lsa, oleum tarkibini toping.\\
A) $\mathrm{H}_{2} \mathrm{SO}_{4} \cdot 2 \mathrm{SO}_{3}$\\
B) $\mathrm{H}_{2} \mathrm{SO}_{4} \cdot 1,5 \mathrm{SO}_{3}$\\
C) $2 \mathrm{H}_{2} \mathrm{SO}_{4} \cdot \mathrm{SO}_{3}$\\
D) $\mathrm{H}_{2} \mathrm{SO}_{4} \cdot \mathrm{SO}_{3}$
  \item Oleum tarkibida oltingugurning massa ulushi 56/149 ga teng bo'lsa, oleum tarkibini toping.\\
A) $3 \mathrm{H}_{2} \mathrm{SO}_{4} \cdot 2 \mathrm{SO}_{3}$\\
B) $2 \mathrm{H}_{2} \mathrm{SO}_{4} \cdot 1,5 \mathrm{SO}_{3}$\\
C) $2 \mathrm{H}_{2} \mathrm{SO}_{4} \cdot 5 \mathrm{SO}_{3}$\\
D) $2 \mathrm{H}_{2} \mathrm{SO}_{4} \cdot \mathrm{SO}_{3}$
  \item Oleum tarkibida oltingugurning massa ulushi 20/54,5 ga teng bolsa, oleum tarkibini toping.\\
A) $\mathrm{H}_{2} \mathrm{SO}_{4} \cdot 2 \mathrm{SO}_{3}$\\
B) $\mathrm{H}_{2} \mathrm{SO}_{4} \cdot 1,5 \mathrm{SO}_{3}$\\
C) $2 \mathrm{H}_{2} \mathrm{SO}_{4} \cdot \mathrm{SO}_{3}$\\
D) $\mathrm{H}_{2} \mathrm{SO}_{4} \cdot \mathrm{SO}_{3}$
  \item Oleum tarkibida oltingugurning massa ulushi 40/113,5 ga teng bo'lsa, oleum tarkibini toping.\\
A) $3 \mathrm{H}_{2} \mathrm{SO}_{4} \cdot 2 \mathrm{SO}_{3}$\\
B) $2 \mathrm{H}_{2} \mathrm{SO}_{4} \cdot 1,5 \mathrm{SO}_{3}$\\
C) $2 \mathrm{H}_{2} \mathrm{SO}_{4} \cdot 5 \mathrm{SO}_{3}$\\
D) $2 \mathrm{H}_{2} \mathrm{SO}_{4} \cdot \mathrm{SO}_{3}$
  \item Oleum tarkibida oltingugurning massa ulushi 24/69 ga teng bo'lsa, oleum tarkibini toping.\\
A) $\mathrm{H}_{2} \mathrm{SO}_{4} \cdot 2 \mathrm{SO}_{3}$\\
B) $\mathrm{H}_{2} \mathrm{SO}_{4} \cdot 1,5 \mathrm{SO}_{3}$\\
C) $2 \mathrm{H}_{2} \mathrm{SO}_{4} \cdot \mathrm{SO}_{3}$\\
D) $\mathrm{H}_{2} \mathrm{SO}_{4} \cdot \mathrm{SO}_{3}$
  \item Oleum tarkibida oltingugurning massa ulushi 72/198 ga teng bo'lsa, oleum tarkibini toping.\\
A) $4 \mathrm{H}_{2} \mathrm{SO}_{4} \cdot 5 \mathrm{SO}_{3}$\\
B) $2 \mathrm{H}_{2} \mathrm{SO}_{4} \cdot 1,5 \mathrm{SO}_{3}$\\
C) $2 \mathrm{H}_{2} \mathrm{SO}_{4} \cdot 3 \mathrm{SO}_{3}$\\
D) $\mathrm{H}_{2} \mathrm{SO}_{4} \cdot 2 \mathrm{SO}_{3}$
  \item Oleum tarkibida oltingugurning massa ulushi 28/79 ga teng bo'lsa, oleum tarkibini toping.\\
A) $4 \mathrm{H}_{2} \mathrm{SO}_{4} \cdot 5 \mathrm{SO}_{3}$\\
B) $2 \mathrm{H}_{2} \mathrm{SO}_{4} \cdot 1,5 \mathrm{SO}_{3}$\\
C) $2 \mathrm{H}_{2} \mathrm{SO}_{4} \cdot 3 \mathrm{SO}_{3}$\\
D) $\mathrm{H}_{2} \mathrm{SO}_{4} \cdot 2 \mathrm{SO}_{3}$
  \item Oleum tarkibida oltingugurning massa ulushi $8 / 22,25$ ga teng bo'lsa, oleum tarkibini toping.\\
A) $\mathrm{H}_{2} \mathrm{SO}_{4} \cdot 2 \mathrm{SO}_{3}$\\
B) $\mathrm{H}_{2} \mathrm{SO}_{4} \cdot 1,5 \mathrm{SO}_{3}$\\
C) $2 \mathrm{H}_{2} \mathrm{SO}_{4} \cdot \mathrm{SO}_{3}$\\
D) $\mathrm{H}_{2} \mathrm{SO}_{4} \cdot \mathrm{SO}_{3}$
  \item Oleum tarkibida oltingugurning massa ulushi 48/138 ga teng bo'lsa, oleum tarkibini toping.\\
A) $\mathrm{H}_{2} \mathrm{SO}_{4} \cdot 2 \mathrm{SO}_{3}$\\
B) $\mathrm{H}_{2} \mathrm{SO}_{4} \cdot 1,5 \mathrm{SO}_{3}$\\
C) $2 \mathrm{H}_{2} \mathrm{SO}_{4} \cdot \mathrm{SO}_{3}$\\
D) $\mathrm{H}_{2} \mathrm{SO}_{4} \cdot \mathrm{SO}_{3}$
422. 180 g suvga $820 \mathrm{~g} \mathrm{SO}_{3}$ qo'shilganda necha \% li oleum eritmasi hosil bo'ladi?\\
A) 3\\
B) 4\\
C) 2\\
D) 7\\
423. 90 g suvga $910 \mathrm{~g} \mathrm{SO}_{3}$ qo'shilganda necha \% li oleum eritmasi hosil bo'ladi?\\
A) 51\\
B) 49\\
C) 63\\
D) 71\\
424. 72 g suvga $928 \mathrm{~g} \mathrm{SO}_{3}$ qo'shilganda necha \% li oleum eritmasi hosil bo'ladi?\\
A) 62\\
B) 35\\
C) 60,8\\
D) 71,4\\
425. 144 g suvga $856 \mathrm{~g} \mathrm{SO}_{3}$ qo'shilganda necha \% li oleum eritmasi hosil bo'ladi?\\
A) 32,3\\
B) 42,5\\
C) 18,4\\
D) 21,6\\
426. 126 g suvga $874 \mathrm{~g} \mathrm{SO}_{3}$ qo'shilganda necha \% li oleum eritmasi hosil bo'ladi?\\
A) 42,5\\
B) 33,6\\
C) 31,4\\
D) 19,8\\
427. 108 g suvga $892 \mathrm{~g} \mathrm{SO}_{3}$ qo'shilganda necha \% li oleum eritmasi hosil bo'ladi?\\
A) 41,2\\
B) 54,3\\
C) 56,3\\
D) 45,4\\
428. 54 g suvga $446 \mathrm{~g} \mathrm{SO}_{3}$ qo'shilganda necha \% li oleum eritmasi hosil bo'ladi?\\
A) 41,2\\
B) 54,3\\
C) 56,3\\
D) 45,4\\
429. 216 g suvga $1284 \mathrm{~g} \mathrm{SO}_{3}$ qo'shilganda necha \% li oleum eritmasi hosil bo'ladi?\\
A) 32,3\\
B) 42,5\\
C) 18,4\\
D) 21,6\\
430. 18 g suvga $182 \mathrm{~g} \mathrm{SO}_{3}$ qo'shilganda necha \% li oleum eritmasi hosil bo'ladi?\\
A) 51\\
B) 49\\
C) 63\\
D) 71
  \item 54 g suyga necha g $\mathrm{SO}_{3}$ eritilganda 44 $\%$ li oleum eritmasi hosil bo'ladi?\\
A) 446\\
B) 471\\
C) 542\\
D) 846
  \item 72 g suvga necha $\mathrm{g} \mathrm{SO}_{3}$ eritilganda 20 $\%$ li oleum eritmasi hosil bo'ladi?\\
A) 418\\
B) 450\\
C) 532\\
D) 646
  \item 90 g suvga necha $\mathrm{g} \mathrm{SO}_{3}$ eritilganda 35 $\%$ li oleum eritmasi hosil bo'ladi?\\
A) 546\\
B) 471\\
C) 664\\
D) 846
  \item 36 g suvga necha $\mathrm{g} \mathrm{SO}_{3}$ eritilganda 48 $\%$ li oleum eritmasi hosil bo'ladi?\\
A) 446\\
B) 341\\
C) 542\\
D) 846
  \item 27 g suvga necha $\mathrm{g} \mathrm{SO}_{3}$ eritilganda 56 \% li oleum eritmasi hosil bo'ladi?\\
A) 246\\
B) 452\\
C) 475\\
D) 307
  \item 81 g suvga necha $\mathrm{g} \mathrm{SO}_{3}$ eritilganda 60 $\%$ li oleum eritmasi hosil bo'ladi?\\
A) 1021,5\\
B) 1324,6\\
C) 542,4\\
D) 847,8
  \item 108 g suvga necha $\mathrm{g} \mathrm{SO}_{3}$ eritilganda $50 \%$ li oleum eritmasi hosil bo'ladi?\\
A) 1075\\
B) 1420\\
C) 1068\\
D) 1821
  \item 144 g suvga necha $\mathrm{g} \mathrm{SO}_{3}$ eritilganda $40 \%$ li oleum eritmasi hosil bo'ladi?\\
A) 1446\\
B) 1162,67\\
C) 1542\\
D) 1244,34
  \item 216 g suvga necha $\mathrm{g} \mathrm{SO}_{3}$ eritilganda $36 \%$ li oleum eritmasi hosil bo'ladi?\\
A) 1621,5\\
B) 1478,8\\
C) 1342,4\\
D) 846,4
  \item 180 g suvga necha $\mathrm{g} \mathrm{SO}_{3}$ eritilganda $50 \%$ li oleum eritmasi hosil bo'ladi?\\
A) 1446\\
B) 1483\\
C) 1780\\
D) 1846
  \item 178 g oleumni neytrallash uchun 160 g NaOH sarflangan bo'lsa, oleum tarkibini aniqlang.\\
A) $\mathrm{H}_{2} \mathrm{SO}_{4} \cdot 2 \mathrm{SO}_{3}$\\
B) $\mathrm{H}_{2} \mathrm{SO}_{4} \cdot 1,5 \mathrm{SO}_{3}$\\
C) $2 \mathrm{H}_{2} \mathrm{SO}_{4} \cdot \mathrm{SO}_{3}$\\
D) $\mathrm{H}_{2} \mathrm{SO}_{4} \cdot \mathrm{SO}_{3}$\\
  \item 129 g oleumni neytrallash uchun 120 g NaOH sarflangan bo'lsa, oleum tarkibini aniqlang.\\
A) $\mathrm{H}_{2} \mathrm{SO}_{4} \cdot 2 \mathrm{SO}_{3}$\\
B) $\mathrm{H}_{2} \mathrm{SO}_{4} \cdot 1,5 \mathrm{SO}_{3}$\\
C) $2 \mathrm{H}_{2} \mathrm{SO}_{4} \cdot \mathrm{SO}_{3}$\\
D) $\mathrm{H}_{2} \mathrm{SO}_{4} \cdot \mathrm{SO}_{3}$
  \item 138 g oleumni neytrallash uchum 120 g NaOH sarflangan bo'lsa. oleum turkibini aniqlang.\\
A) $\mathrm{H}_{2} \mathrm{SO}_{4} \cdot 2 \mathrm{SO}_{3}$\\
B) $\mathrm{H}_{2} \mathrm{SO}_{4} \cdot 1,5 \mathrm{SO}_{3}$\\
C) $2 \mathrm{H}_{2} \mathrm{SO}_{4} \cdot \mathrm{SO}_{8}$\\
D) $\mathrm{H}_{2} \mathrm{SO}_{4} \cdot \mathrm{SO}_{4}$
  \item 218 g oleumni neytrallash uchun 200 g NaOH sarflangan bo'lsa, oloum tarkibini aniqlang.\\
A) $\mathrm{H}_{2} \mathrm{SO}_{4} \cdot 2 \mathrm{SO}_{3}$\\
B) $\mathrm{H}_{2} \mathrm{SO}_{4} \cdot 1,5 \mathrm{SO}_{3}$\\
C) $2 \mathrm{H}_{2} \mathrm{SO}_{4} \cdot \mathrm{SO}_{3}$\\
D) $\mathrm{H}_{2} \mathrm{SO}_{4} \cdot \mathrm{SO}_{3}$
  \item $44,5 \mathrm{~g}$ oleumni neytrallash uchun 40 g NaOH sarflangan bo'lsa, oleum tarkibini aniqlang.\\
A) $\mathrm{H}_{2} \mathrm{SO}_{4} \cdot 2 \mathrm{SO}_{3}$\\
B) $\mathrm{H}_{2} \mathrm{SO}_{4} \cdot 1,5 \mathrm{SO}_{3}$\\
C) $2 \mathrm{H}_{2} \mathrm{SO}_{4} \cdot \mathrm{SO}_{3}$\\
D) $\mathrm{H}_{2} \mathrm{SO}_{4} \cdot \mathrm{SO}_{3}$
  \item $64,5 \mathrm{~g}$ oleunni neytrallash uchun 60 g NaOH sarflangan bo'lsa, oleum tarkibini aniqlang.\\
A) $\mathrm{H}_{2} \mathrm{SO}_{4} \cdot 2 \mathrm{SO}_{3}$\\
B) $\mathrm{H}_{2} \mathrm{SO}_{4} \cdot 1,5 \mathrm{SO}_{3}$\\
C) $2 \mathrm{H}_{3} \mathrm{SO}_{4} \cdot \mathrm{SO}_{3}$\\
D) $\mathrm{H}_{2} \mathrm{SO}_{4} \cdot \mathrm{SO}_{3}$
  \item 92 g oleumni neytrallash uchum 80 g NaOH sarflangan bo'lsa, oleum tarkibini aniqlang.\\
A) $\mathrm{H}_{2} \mathrm{SO}_{4} \cdot 2 \mathrm{SO}_{3}$\\
B) $\mathrm{H}_{2} \mathrm{SO}_{4} \cdot 1,5 \mathrm{SO}_{3}$\\
C) $2 \mathrm{H}_{2} \mathrm{SO}_{4} \cdot \mathrm{SO}_{3}$\\
D) $\mathrm{H}_{2} \mathrm{SO}_{4} \cdot \mathrm{SO}_{5}$
  \item 90.8 g oleumni neytrallash uchun 80 g NaOH sarflangan bo'lsa, oleum tarkibini aniqlang.\\
A) $3 \mathrm{H}_{2} \mathrm{SO}_{4} \cdot 2 \mathrm{SO}_{3}$\\
B) $2 \mathrm{H}_{2} \mathrm{SO}_{4} \cdot 1,5 \mathrm{SO}_{3}$\\
C) $2 \mathrm{H}_{2} \mathrm{SO}_{4} \cdot 5 \mathrm{SO}_{3}$\\
D) $\mathrm{H}_{2} \mathrm{SO}_{4} \cdot \mathrm{SO}_{3}$
  \item 118 g oleumni neytrallash uchun 100 g NaOH sarflangan bo'lsa, oleum tarkibini aniqlang.\\
A) $7 \mathrm{H}_{2} \mathrm{SO}_{4} \cdot 2 \mathrm{SO}_{3}$\\
B) $2,5 \mathrm{H}_{2} \mathrm{SO}_{4} \cdot 1,5 \mathrm{SO}_{3}$\\
C) $2 \mathrm{H}_{2} \mathrm{SO}_{4} \cdot \mathrm{SO}_{3}$\\
D) $4 \mathrm{H}_{2} \mathrm{SO}_{4} \cdot \mathrm{SO}_{3}$
  \item $173,5 \mathrm{~g}$ oleumni neytrallash uchun 160 g NaOH sartlangan bo'lsa, oleum tarkibini aniqlang.\\
A) $3 \mathrm{H}_{2} \mathrm{SO}_{4} \cdot 5 \mathrm{SO}_{3}$\\
B) $\mathrm{H}_{2} \mathrm{SO}_{4} \cdot 1,5 \mathrm{SO}_{3}$\\
C) $2 \mathrm{H}_{2} \mathrm{SO}_{4} \cdot \mathrm{SO}_{3}$\\
D) $\mathrm{H}_{2} \mathrm{SO}_{4} \cdot 3 \mathrm{SO}_{3}$
  \item Ushbu jarayonlarning qaysi birida plastinka massasi ortadi?\\
A) $\mathrm{FeSO}_{4}+\mathrm{Cu} \rightarrow$\\
B) $\mathrm{CuSO}_{4}+\mathrm{Fe} \rightarrow$\\
C) $\mathrm{ZnSO}_{4}+\mathrm{Cr} \rightarrow$\\
D) $\mathrm{MnSO}_{4}+\mathrm{Zn} \rightarrow$\\
  \item Ushbu jarayonlarning qaysi birida plastinka massasi o'zgarmaydi?\\
A) $\mathrm{FeSO}_{4}+\mathrm{Cu} \rightarrow$\\
B) $\mathrm{CuSO}_{4}+\mathrm{Fe} \rightarrow$\\
C) $\mathrm{ZnSO}_{4}+\mathrm{Mn} \rightarrow$\\
D) $\mathrm{MnSO}_{4}+\mathrm{Al} \rightarrow$
  \item Ushbu jarayonlarning qaysi birida plastinka massasi kamayadi?\\
A) $\mathrm{FeSO}_{4}+\mathrm{Pb} \rightarrow$\\
B) $\mathrm{CuSO}_{4}+\mathrm{Pb} \rightarrow$\\
C) $\mathrm{ZnSO}_{4}+\mathrm{Al} \rightarrow$\\
D) $\mathrm{MnSO}_{4}+\mathrm{Al} \rightarrow$
  \item Ushbu jarayonlarning qaysi birida plastinka massasi ortadi?\\
A) $\mathrm{FeSO}_{4}+\mathrm{Cu} \rightarrow$\\
B) $\mathrm{CuSO}_{4}+\mathrm{Pb} \rightarrow$\\
C) $\mathrm{ZnSO}_{4}+\mathrm{Mn} \rightarrow$\\
D) $\mathrm{MnSO}_{4}+\mathrm{Zn} \rightarrow$
  \item Ushbu jarayonlarning qaysi birida plastinka massasi kamayadi?\\
A) $\mathrm{CdSO}_{4}+\mathrm{Fe} \rightarrow$\\
B) $\mathrm{CuSO}_{4}+\mathrm{Fe} \rightarrow$\\
C) $\mathrm{CrSO}_{4}+\mathrm{Mn} \rightarrow$\\
D) $\mathrm{MnSO}_{4}+\mathrm{Zn} \rightarrow$
  \item Ushbu jarayonlarning qaysi birida plastinka massasi o'zgarmaydi?\\
A) $\mathrm{FeSO}_{4}+\mathrm{Cd} \rightarrow$\\
B) $\mathrm{CuSO}_{4}+\mathrm{Fe} \rightarrow$\\
C) $\mathrm{ZnSO}_{4}+\mathrm{Al} \rightarrow$\\
D) $\mathrm{TiSO}_{4}+\mathrm{Zn} \rightarrow$
  \item Ushbu jarayonlarning qaysi birida plastinka massasi ortadi?\\
A) $\mathrm{CoSO}_{4}+\mathrm{Cu} \rightarrow$\\
B) $\mathrm{CuSO}_{4}+\mathrm{Cr} \rightarrow$\\
C) $\mathrm{ZnSO}_{4}+\mathrm{Cr} \rightarrow$\\
D) $\mathrm{MnSO}_{4}+\mathrm{Zn} \rightarrow$
  \item Ushbu jarayonlarning qaysi birida plastinka massasi kamayadi?\\
A) $\mathrm{CoSO}_{4}+\mathrm{Cu} \rightarrow$\\
B) $\mathrm{CuSO}_{4}+\mathrm{Fe} \rightarrow$\\
C) $\mathrm{CoSO}_{4}+\mathrm{Cr} \rightarrow$\\
D) $\mathrm{TiSO}_{4}+\mathrm{Zn} \rightarrow$
  \item Ushbu jarayonlarning qaysi birida plastinka massasi o'zgarmaydi?\\
A) $\mathrm{FeSO}_{4}+\mathrm{Co} \rightarrow$\\
B) $\mathrm{CuSO}_{4}+\mathrm{Fe} \rightarrow$\\
C) $\mathrm{ZnSO}_{4}+\mathrm{Mn} \rightarrow$\\
D) $\mathrm{CoSO}_{4}+\mathrm{Zn} \rightarrow$
  \item Ushbu jarayonlarning qaysi birida plastinka massasi ortadi?\\
A) $\mathrm{FeSO}_{4}+\mathrm{Cu} \rightarrow$\\
B) $\mathrm{CuSO}_{4}+\mathrm{Mn} \xrightarrow{+}$\\
C) $\mathrm{ZnSO}_{4}+\mathrm{Cr} \rightarrow$\\
D) $\mathrm{MnSO}_{4}+\mathrm{Zn} \rightarrow$
  \item $\mathrm{CuSO}_{4}$ eritmasiga Fe dan yasalgan plastinka ma'lum muddat botirib qo'yildi. Natijada plastinka massasi $0,8 \mathrm{~g}$ ga ortgan bo'lsa, eritma tarkibidagi $\mathrm{FeSO}_{4}$ massasini toping.\\
A) 32,3\\
B) 42,5\\
C) 18,4\\
D) 15,2
  \item $\mathrm{CdSO}_{4}$ eritmasiga Mn dan yasalgan plastinka ma'lum muddat botirib qo'yildi. Natijada plastinka massasi $11,4 \mathrm{~g}$ ga ortgan bo'lsa, eritma tarkibidagi $\mathrm{MnSO}_{4}$ massasini toping.\\
A) 30,2\\
B) 18,6\\
C) 18,8\\
D) 15,1
  \item $\mathrm{Zn}\left(\mathrm{NO}_{3}\right)_{2}$ eritmasiga Mn dan yasalgan plastinka ma'lum muddat botirib qo'yildi. Natijada plastinka massasi 1 g ga ortgan bo'lsa, eritma tarkibidagi $\mathrm{Mn}\left(\mathrm{NO}_{3}\right)_{2}$ massasini toping.\\
A) 14,4\\
B) 12,5\\
C) 17,9\\
D) 25,2
  \item $\mathrm{FeCl}_{2}$ eritmasiga Mn dan yasalgan plastinka ma'lum muddat botirib qo'yildi.\\
Natijada plastinka massasi $0,1 \mathrm{~g}$ ga ortgan bo'lsa, eritma tarkibidagi $\mathrm{MnCl}_{2}$ massasini toping.\\
A) 12,6\\
B) 14,8\\
C) 18,4\\
D) 13,2
  \item $\mathrm{AgNO}_{3}$ eritmasiga Fe dan yasalgan plastinka ma'lum muddat botirib qo'yildi. Natijada plastinka massasi 16 g ga ortgan bo'lsa, eritma tarkibidagi $\mathrm{Fe}\left(\mathrm{NO}_{3}\right)_{2}$ massasini toping.\\
A) 32,3\\
B) 42,5\\
C) 18\\
D) 15,2
  \item $\mathrm{NiSO}_{4}$ eritmasiga Ti dan yasalgan plastinka ma'lum muddat botirib qo'yildi.\\
Natijada plastinka massasi $1,1 \mathrm{~g}$ ga ortgan bo'lsa, eritma tarkibidagi $\mathrm{TiSO}_{4}$ massasini toping.\\
A) 14,4\\
B) 12,5\\
C) 18,1\\
D) 25,2
  \item $\mathrm{CoSO}_{4}$ eritmasiga Cr dan yasalgan plastinka ma'lum muddat botirib qo'yildi. Natijada plastinka massasi $1,4 \mathrm{~g}$ ga ortgan bo'lsa, eritma tarkibidagi $\mathrm{CrSO}_{4}$ massasini toping.\\
A) 29,6\\
B) 42,5\\
C) 18,4\\
D) 15,2
  \item $\mathrm{CrCl}_{3}$ eritmasiga Al dan yasalgan plastinka ma'lum muddat botirib qo'yildi. Natijada plastinka massasi $2,5 \mathrm{~g}$ ga ortgan bo'lsa, eritma tarkibidagi $\mathrm{AlCl}_{3}$ massasini toping.\\
A) 32,35\\
B) 13,35\\
C) 18,46\\
D) 14,2
  \item $\mathrm{Pb}\left(\mathrm{NO}_{3}\right)_{2}$ eritmasiga Fe yasalgan plastinka ma'lum muddat botirib qo'yildi. Natijada plastinka massasi $15,1 \mathrm{~g}$ ga ortgan bo'lsa, eritma tarkibidagi $\mathrm{Fe}\left(\mathrm{NO}_{3}\right)_{2}$ massasini toping.\\
A) 37,3\\
B) 44,5\\
C) 18\\
D) 11
  \item $\mathrm{AgNO}_{3}$ eritmasiga Cu yasalgan plastinka ma'lum muddat botirib qo'yildi. Natijada plastinka massasi $15,2 \mathrm{~g}$ ga ortgan bo'lsa, eritma tarkibidagi $\mathrm{Cu}\left(\mathrm{NO}_{3}\right)_{2}$ massasini toping.\\
A) 18,8\\
B) 42,8\\
C) 16,4\\
D) 17,2
  \item $\mathrm{CuSO}_{4}$ eritmasiga Fe dan yasalgan 16 g plastinka ma'lum muddat botirib qo'yildi. Natijada plastinka massasi $10 \%$ ga ortgan bo'lsa, eritma tarkibidagi $\mathrm{FeSO}_{4}$ massasini toping.\\
A) 32,3\\
B) 42,5\\
C) 30,4\\
D) 30,2
  \item $\mathrm{CdSO}_{4}$ eritmasiga Mn dan yasalgan 57 g plastinka ma'lum muddat botirib qo'yildi. Natijada plastinka massasi $20 \%$ ga ortgan bo'lsa, eritma tarkibidagi $\mathrm{MnSO}_{4}$ massasini toping.\\
A) 30,2\\
B) 18,6\\
C) 18,8\\
D) 15,1
  \item $\mathrm{Zn}\left(\mathrm{NO}_{3}\right)_{2}$ eritmasiga Mn dan yasalgan 100 g plastinka ma'lum muddat botirib\\
qo'yildi. Natijada plastinka massasi 1\% ga ortgan bo'lsa, eritma tarkibidagi $\mathrm{Mn}\left(\mathrm{NO}_{3}\right)_{2}$ massasini toping.\\
A) 14,4\\
B) 12,5\\
C) 17,9\\
D) 25,2
  \item $\mathrm{FeCl}_{2}$ eritmasiga Mn dan yasalgan 10 g plastinka ma'lum muddat botirib qo'yildi. Natijada plastinka massasi $2 \%$ ga ortgan bo'lsa, eritma tarkibidagi $\mathrm{MnCl}_{2}$ massasini toping.\\
A) 25,2\\
B) 12,6\\
C) 18,4\\
D) 13,2
  \item $\mathrm{AgNO}_{3}$ eritmasiga Fe dan yasalgan 160 g plastinka ma'lum muddat botirib qo'yildi. Natijada plastinka massasi $20 \%$ ga ortgan bo'lsa, eritma tarkibidagi $\mathrm{Fe}\left(\mathrm{NO}_{3}\right)_{2}$ massasini toping.\\
A) 36\\
B) 27\\
C) 18\\
D) 15
  \item $\mathrm{NiSO}_{4}$ eritmasiga Ti dan yasalgan 22 g plastinka ma'lum muddat botirib qo'yildi. Natijada plastinka massasi 5\% ga ortgan bo'lsa, eritma tarkibidagi $\mathrm{TiSO}_{4}$ massasini toping.\\
A) 14,4\\
B) 12,5\\
C) 18,1\\
D) 25,2
  \item $\mathrm{CoSO}_{4}$ eritmasiga Cr dan yasalgan 14 g plastinka ma'lum muddat botirib qo'yildi. Natijada plastinka massasi $20 \%$ ortgan bo'lsa, eritma tarkibidagi $\mathrm{CrSO}_{4}$ massasini toping.\\
A) 29,6\\
B) 42,5\\
C) 18,4\\
D) 59,2
  \item $\mathrm{CrCl}_{3}$ eritmasiga Al dan yasalgan 25 g plastinka ma'lum muddat botirib qo'yildi. Natijada plastinka massasi $10 \%$ ga ortgan bo'lsa, eritma tarkibidagi $\mathrm{AlCl}_{3}$ massasini toping.\\
A) 32,35\\
B) 13,35\\
C) 18,46\\
D) 14,2
  \item $\mathrm{Pb}\left(\mathrm{NO}_{3}\right)_{2}$ eritmasiga Fe yasalgan 151 g plastinka ma'lum muddat botirib qo'yildi. Natijada plastinka massasi $20 \%$ ga ortgan bo'lsa, eritma tarkibidagi $\mathrm{Fe}\left(\mathrm{NO}_{3}\right)_{2}$ massasini toping.\\
A) 34\\
B) 36\\
C) 18\\
D) 17
  \item $\mathrm{AgNO}_{3}$ eritmasiga Cu yasalgan 304 g plastinka ma'lum muddat botirib qo'yildi. Natijada plastinka massasi 5\% ga ortgan bo'lsa, eritma tarkibidagi $\mathrm{Cu}\left(\mathrm{NO}_{3}\right)_{2}$ massasini toping.\\
A) 18,8\\
B) 42,8\\
C) 16,4\\
D) 17,2
  \item $\mathrm{CuSO}_{4}$ eritmasiga Fe dan yasalgan plastinka ma'lum muddat botirib qo'yildi. Natijada plastinka massasi $0,8 \mathrm{~g}$ ortgan bo'lsa, plastinka tarkibiga o'tgan Cu massasini toping.\\
A) 64\\
B) 128\\
C) 6,4\\
D) 12,8
  \item $\mathrm{CdSO}_{4}$ eritmasiga Mn dan yasalgan plastinka ma'lum muddat botirib qo'yildi. Natijada plastinka massasi $11,4 \mathrm{~g}$ ortgan bo'lsa, plastinka tarkibiga o'tgan Cd massasini toping.\\
A) 22,4\\
B) 11,2\\
C) 18,8\\
D) 5,6
  \item $\mathrm{Zn}\left(\mathrm{NO}_{3}\right)_{2}$ eritmasiga Mn dan yasalgan plastinka ma'lum muddat botirib qo'yildi. Natijada plastinka massasi 1 g ortgan bo'lsa, plastinka tarkibiga o'tgan Zn massasini toping.\\
A) 13\\
B) 6,5\\
C) 65\\
D) 25,2
  \item $\mathrm{FeCl}_{2}$ eritmasiga Mn dan yasalgan plastinka ma'lum muddat botirib qo'yildi. Natijada plastinka massasi $0,2 \mathrm{~g}$ ortgan bo'lsa, plastinka tarkibiga o'tgan Fe massasini toping.\\
A) 22,4\\
B) 11,2\\
C) 18,8\\
D) 5,6
  \item $\mathrm{AgNO}_{3}$ eritmasiga Fe dan yasalgan plastinka ma'lum muddat botirib qo'yildi. Natijada plastinka massasi $3,2 \mathrm{~g}$ ortgan bo'lsa, plastinka tarkibiga o'tgan Ag massasini toping.\\
A) 12,8\\
B) 11,2\\
C) 5,6\\
D) 4,32
  \item $\mathrm{NiSO}_{4}$ eritmasiga Ti dan yasalgan plastinka ma'lum muddat botirib qo'yildi. Natijada plastinka massasi $2,2 \mathrm{~g}$ ortgan bo'lsa, plastinka tarkibiga o'tgan Ni massasini toping.\\
A) 5,9\\
B) 12,6\\
C) 11,8\\
D) 25,2
  \item $\mathrm{CoSO}_{4}$ eritmasiga Cr dan yasalgan plastinka ma'lum muddat botirib qo'yildi. Natijada plastinka massasi $0,7 \mathrm{~g}$ ortgan bo'lsa, plastinka tarkibiga o'tgan Co massasini toping.\\
A) 5,9\\
B) 12,6\\
C) 11,8\\
D) 25,2\\
488. $\mathrm{CrCl}_{3}$ eritmasiga Al dan yasalgan plastinka ma'lum muddat botirib qo'yildi. Natijada plastinka massasi 5 g ortgan bo'lsa, plastinka tarkibiga o'tgan Cr massasini toping.\\
A) 20,8\\
B) 5,2\\
C) 18,4\\
D) 10,4\\
489. $\mathrm{Pb}\left(\mathrm{NO}_{3}\right)_{2}$ eritmasiga Fe yasalgan plastinka ma'lum muddat botirib qo'yildi. Natijada plastinka massasi $15,1 \mathrm{~g}$ ortgan bo'lsa, plastinka tarkibiga o'tgan Pb massasini toping.\\
A) 30,3\\
B) 10,4\\
C) 20,7\\
D) 41,6\\
490. $\mathrm{AgNO}_{3}$ eritmasiga Cu yasalgan plastinka ma'lum muddat botirib qo'yildi. Natijada plastinka massasi $15,2 \mathrm{~g}$ ortgan bo'lsa, plastinka tarkibiga o'tgan Ag massasini toping.\\
A) 21,6\\
B) 10,8\\
C) 16,4\\
D) 17,2
  \item Ikki valentli metaldan yasalgan teng massali ikkita plastinka biri $\mathrm{AgNO}_{3}$ eritmasiga ikkinchisi $\mathrm{CuSO}_{4}$ eritmasiga tushirilganda birinchi plastinka massasi $16 \%$ ga ikkinchi plastinka massasi $0,8 \%$ ga ortgan bo'lsa, plastinka qaysi metaldan yasalgan?\\
A) Fe\\
B) Cr\\
C) Mn\\
D) Mg
  \item Ikki valentli metaldan yasalgan teng massali ikkita plastinka biri $\mathrm{AgNO}_{3}$ eritmasiga ikkinchisi $\mathrm{CuSO}_{4}$ eritmasiga tushirilganda birinchi plastinka massasi\\ $16,1 \%$ ga ikkinchi plastinka massasi $0,9 \%$ ga ortgan bo'lsa, plastinka qaysi metaldan yasalgan?\\
A) Fe\\
B) Cr\\
C) Mn\\
D) Mg
  \item Ikki valentli metaldan yasalgan teng massali ikkita plastinka biri $\mathrm{AgNO}_{3}$ eritmasiga ikkinchisi $\mathrm{CuSO}_{4}$ eritmasiga tushirilganda birinchi plastinka massasi $16,4 \%$ ga ikkinchi plastinka massasi $1,2 \%$ ga ortgan bo'lsa, plastinka qaysi metaldan yasalgan?\\
A) Ti\\
B) Cr\\
C) Mn\\
D) Ni
  \item Ikki valentli metaldan yasalgan teng massali ikkita plastinka biri $\mathrm{AgNO}_{3}$ eritmasiga ikkinchisi $\mathrm{CuSO}_{4}$ eritmasiga tushirilganda birinchi plastinka massasi 16\% ga ikkinchi plastinka massasi 0,8\% ga ortgan bo'lsa, plastinka qaysi metaldan yasalgan?\\
A) Fe\\
B) Cr\\
C) Mn\\
D) Mg
  \item Ikki valentli metaldan yasalgan teng massali ikkita plastinka biri $\mathrm{AgNO}_{3}$ eritmasiga ikkinchisi $\mathrm{CuSO}_{4}$ eritmasigą tushirilganda birinchi plastinka massasi $19,2 \%$ ga ikkinchi plastinka massasi $4 \%$ ga ortgan bo'lsa, plastinka qaysi metaldan yasalgan?\\
A) Fe\\
B) Cr\\
C) Mn\\
D) Mg
  \item Ikki valentli metaldan yasalgan teng massali ikkita plastinka biri $\mathrm{AgNO}_{3}$ eritmasiga ikkinchisi $\mathrm{CuSO}_{4}$ eritmasiga tushirilganda birinchi plastinka massasi 32\% ga ikkinchi plastinka massasi 1,6\% ga ortgan bo'lsa, plastinka qaysi metaldan yasalgan?\\
A) Fe\\
B) Cr\\
C) Mn\\
D) Mg
  \item Ikki valentli metaldan yasalgan teng massali ikkita plastinka biri $\mathrm{AgNO}_{3}$ eritmasiga ikkinchisi $\mathrm{CuSO}_{4}$ eritmasiga tushirilganda birinchi plastinka massasi 9,6\% ga ikkinchi plastinka massasi $2 \%$ ga ortgan bo'lsa, plastinka qaysi metaldan yasalgan?\\
A) Fe\\
B) Cr\\
C) Mn\\
D) Mg
  \item Ikki valentli metaldan yasalgan teng massali ikkita plastinka biri $\mathrm{AgNO}_{3}$\\
eritmasiga ikkinchisi $\mathrm{CuSO}_{4}$ eritmasiga tushirilganda birinchi plastinka massasi $15,7 \%$ ga ikkinchi plastinka massasi $0,5 \%$ ga ortgan bo'lsa, plastinka qaysi metaldan yasalgan?\\
A) Ti\\
B) Cr\\
C) Mn\\
D) Ni
  \item Ikki valentli metaldan yasalgan teng massali ikkita plastinka biri $\mathrm{AgNO}_{3}$ eritmasiga ikkinchisi $\mathrm{CuSO}_{4}$ eritmasiga tushirilganda birinchi plastinka massasi $16,8 \%$ ga ikkinchi plastinka massasi $1,6 \%$ ga ortgan bo'lsa, plastinka qaysi metaldan yasalgan?\\
A) Ti\\
B) Cr\\
C) Mn\\
D) Ni
  \item Ikki valentli metaldan yasalgan teng massali ikkita plastinka biri $\mathrm{AgNO}_{3}$ eritmasiga ikkinchisi $\mathrm{CuSO}_{4}$ eritmasiga tushirilganda birinchi plastinka massasi $31,4 \%$ ga ikkinchi plastinka massasi $1 \%$ ga ortgan bo'lsa, plastinka qaysi metaldan yasalgan?\\
A) Ti\\
B) Cr\\
C) Mn\\
D) Co
  \item $\mathrm{A}+\mathrm{B} \rightarrow \mathrm{C}$ ushbu reaksiya bo'yicha 120 sekund vaqt ichida A moddaning miqdori 5 moldan 3 mol kamaygan bo'lsa, reaksiya tezligini mol/l•minutda toping. Reaktor hajmi 5 litrga teng.\\
A) 0,3\\
B) 0,2\\
C) 0,15\\
D) 1,5
  \item $\mathrm{A}+\mathrm{B} \rightarrow \mathrm{C}$ ushbu reaksiya bo'yicha 120 sekund vaqt ichida A moddaning miqdori 4 moldan 2 mol kamaygan bo'lsa, reaksiya tezligini $\mathrm{mol} / \mathrm{l} \cdot$ minutda toping. Reaktor hajmi 5 litrga teng.\\
A) 0,3\\
B) 0,2\\
C) 0,15\\
D) 1,5
  \item $\mathrm{A}+\mathrm{B} \rightarrow \mathrm{C}$ ushbu reaksiya bo'yicha 240 sekund vaqt ichida A moddaning miqdori 6 moldan 3 mol kamaygan bo'lsa, reaksiya tezligini $\mathrm{mol} / \mathrm{l} \cdot$ minutda toping. Reaktor hajmi 3 litrga teng.\\
A) 0,35\\
B) 0,25\\
C) 0,15\\
D) 1,5
  \item $\mathrm{A}+\mathrm{B} \rightarrow \mathrm{C}$ ushbu reaksiya bo'yicha 60 sekund vaqt ichida A moddaning miqdori 10 moldan 8 mol kamaygan bo'lsa, reaksiya tezligini $\mathrm{mol} / \cdot$ minutda toping. Reaktor hajmi 8 litrga teng.\\
A) 3\\
B) 2\\
C) 5\\
D) 1
  \item $\mathrm{A}+\mathrm{B} \rightarrow \mathrm{C}$ ushbu reaksiya bo'yicha 240 sekund vaqt ichida A moddaning miqdori 15 moldan 9 mol kamaygan bo'lsa, reaksiya tezligini $\mathrm{mol} / \mathrm{l} \cdot$ minutda toping. Reaktor hajmi 4,5 litrga teng.\\
A) 0,3\\
B) 0,4\\
C) 0,5\\
D) 1
  \item $\mathrm{A}+\mathrm{B} \rightarrow \mathrm{C}$ ushbu reaksiya bo'yicha 72 sekund vaqt ichida A moddaning miqdori 10 moldan 6 mol kamaygan bo'lsa, reaksiya tezligini $\mathrm{mol} / \mathrm{l} \cdot$ minutda toping. Reaktor hajmi 5 litrga teng.\\
A) 0,3\\
B) 0,4\\
C) 0,5\\
D) 1
  \item $\mathrm{A}+\mathrm{B} \rightarrow \mathrm{C}$ ushbu reaksiya bo'yicha 90 sekund vaqt ichida A moddaning miqdori 5 moldan 3 mol kamaygan bo'lsa, reaksiya tezligini mol/l•minutda toping. Reaktor hajmi 5 litrga teng.\\
A) 0,3\\
B) 0,4\\
C) 0,5\\
D) 1
  \item $\mathrm{A}+\mathrm{B} \rightarrow \mathrm{C}$ ushbu reaksiya bo'yicha 30 sekund vaqt ichida A moddaning miqdori 8 moldan 5 mol kamaygan bo'lsa, reaksiya tezligini mol/l•minutda toping. Reaktor hajmi 5 litrga teng.\\
A) 3\\
B) 2\\
C) 5\\
D) 1
  \item $\mathrm{A}+\mathrm{B} \rightarrow \mathrm{C}$ ushbu reaksiya bo'yicha 360 sekund vaqt ichida A moddaning miqdori 10 moldan 6 mol kamaygan bo'lsa, reaksiya tezligini $\mathrm{mol} / \mathrm{l} \cdot$ minutda toping. Reaktor hajmi 2 litrga teng.\\
A) 0,3\\
B) 0,4\\
C) 0,5\\
D) 1
  \item $\mathrm{A}+\mathrm{B} \rightarrow \mathrm{C}$ ushbu reaksiya bo'yicha 45 sekund vaqt ichida A moddaning miqdori 5 moldan 3 mol kamaygan bo'lsa, reaksiya tezligini $\mathrm{mol} / \mathrm{l} \cdot$ minutda toping. Reaktor hajmi 4 litrga teng.\\
A) 3\\
B) 2\\
C) 5\\
D) 1
  \item $\mathrm{A}+\mathrm{B} \rightarrow \mathrm{C}$ ushbu reaksiya bo'yicha 2 minut vaqt ichida A moddaning konsentratsiyasi $16 \mathrm{~mol} / \mathrm{l}$ dan $4 \mathrm{~mol} / \mathrm{l}$ gacha kamaygan bo'lsa, reaksiya tezligini $\mathrm{mol} / \mathrm{l} \cdot$ sekundda toping.\\
A) 0,3\\
B) 0,2\\
C) 0,5\\
D) 0,1\\
  \item $\mathrm{A}+\mathrm{B} \rightarrow \mathrm{C}$ ushbu reaksiya bo'yicha 1 minut vaqt ichida A moddaning konsentratsiyasi $20 \mathrm{~mol} / \mathrm{l}$ dan $2 \mathrm{~mol} / \mathrm{l}$ gacha kamaygan bo'lsa, reaksiya tezligini $\mathrm{mol} / \mathrm{l} \cdot$ sekundda toping.\\
A) 0,3\\
B) 0,2\\
C) 0,5\\
D) 0,1
  \item $\mathrm{A}+\mathrm{B} \rightarrow \mathrm{C}$ ushbu reaksiya bo'yicha 1,5 minut vaqt ichida A moddaning konsentratsiyasi $50 \mathrm{~mol} / /$ dan $5 \mathrm{~mol} / /$ gacha kamaygan bo'lsa, reaksiya tezligini $\mathrm{mol} / \mathrm{l} \cdot$ sekundda toping.\\
A) 0,3\\
B) 0,2\\
C) 0,5\\
D) 0,1
  \item $\mathrm{A}+\mathrm{B} \rightarrow \mathrm{C}$ ushbu reaksiya bo'yicha 0,5 minut vaqt ichida A moddaning konsentratsiyasi $16 \mathrm{~mol} / \mathrm{l}$ dan $10 \mathrm{~mol} / \mathrm{l}$ gacha kamaygan bo'lsa, reaksiya tezligini $\mathrm{mol} / \mathrm{l} \cdot$ sekundda toping.\\
A) 0,3\\
B) 0,2\\
C) 0,5\\
D) 0,1
  \item $\mathrm{A}+\mathrm{B} \rightarrow \mathrm{C}$ ushbu reaksiya bo'yicha 3 minut vaqt ichida A moddaning konsentratsiyasi $20 \mathrm{~mol} / \mathrm{l}$ dan $2 \mathrm{~mol} / \mathrm{l}$ gacha kamaygan bo'lsa, reaksiya tezligini $\mathrm{mol} / \mathrm{l} \cdot$ sekundda toping.\\
A) 0,3\\
B) 0,2\\
C) 0,5\\
D) 0,1
  \item $\mathrm{A}+\mathrm{B} \rightarrow \mathrm{C}$ ushbu reaksiya bo'yicha 0,167 minut vaqt ichida A moddaning konsentratsiyasi $7 \mathrm{~mol} / \mathrm{l}$ dan $4 \mathrm{~mol} / \mathrm{l}$ gacha kamaygan bo'lsa, reaksiya tezligini $\mathrm{mol} / \mathrm{l} \cdot$ sekundda toping.\\
A) 0,3\\
B) 0,2\\
C) 0,5\\
D) 0,1
  \item $\mathrm{A}+\mathrm{B} \rightarrow \mathrm{C}$ ushbu reaksiya bo'yicha 2,5 minut vaqt ichida A moddaning konsentratsiyasi $20 \mathrm{~mol} / \mathrm{l}$ dan $5 \mathrm{~mol} / \mathrm{l}$ gacha kamaygan bo'lsa, reaksiya tezligini mol/l•sekundda toping.\\
A) 0,3\\
B) 0,2\\
C) 0,5\\
D) 0,1
  \item $\mathrm{A}+\mathrm{B} \rightarrow \mathrm{C}$ ushbu reaksiya bo'yicha 3,5 minut vaqt ichida A moddaning konsentratsiyasi $50 \mathrm{~mol} / \mathrm{l}$ dan $8 \mathrm{~mol} / \mathrm{l}$ gacha kamaygan bo'lsa, reaksiya tezligini $\mathrm{mol} / \mathrm{l} \cdot$ sekundda toping.\\
A) 0,3\\
B) 0,2\\
C) 0,5\\
D) 0,1
  \item $\mathrm{A}+\mathrm{B} \rightarrow \mathrm{C}$ ushbu reaksiya bo'yicha 5 minut vaqt ichida A moddaning\\
konsentratsiyasi $100 \mathrm{~mol} / \mathrm{l}$ dan $10 \mathrm{~mol} / \mathrm{l}$ gacha kamaygan bo'lsa, reaksiya tezligini $\mathrm{mol} / 1 \cdot$ sekundda toping.\\
A) 0,3\\
B) 0.2\\
C) 0,5\\
D) 0,1
  \item $\mathrm{A}+\mathrm{B} \rightarrow \mathrm{C}$ ushbu reaksiya bo'yicha 0,2 minut vaqt ichida A moddaning konsentratsiyasi $2,4 \mathrm{~mol} / \mathrm{l}$ dan $1,2 \mathrm{~mol} / \mathrm{l}$ gacha kamaygan bo'lsa, reaksiya tezligini $\mathrm{mol} / \mathrm{l} \cdot$ sekundda toping.\\
A) 0.3\\
B) 0,2\\
C) 0.5\\
D) 0,1
  \item $2 \mathrm{~A}+\mathrm{B} \rightarrow \mathrm{C}$ ushbu reaksiya bo'yicha A va B moddalarning dastlabki konsentratsiyalari mos ravishda 2 va 3 $\mathrm{mol} / \mathrm{l}$ bo'lsa, $\mathrm{k}=1,5$ bo'lgandagi tezligini ( $\mathrm{mol} / \cdot \cdot \mathrm{min}$ ) toping.\\
A) 18\\
B) 12\\
C) 15\\
D) 10\\
\item B moddalarning dastlabki konsentratsiyalari mos ravishda 1 va 2 $\mathrm{mol} / \mathrm{l}$ bo'lsa, $\mathrm{k}=1$ bo'lgandagi tezligini ( $\mathrm{mol} / \mathrm{l} \cdot \mathrm{min}$ ) toping.\\
A) 6\\
B) 7\\
C) 8\\
D) 10\\
23. $2 \mathrm{~A}+2 \mathrm{~B} \rightarrow \mathrm{C}$ ushbu reaksiya bo'yicha A va B moddalarning dastlabki konsentratsiyalari mos ravishda 2 va 2 $\mathrm{mol} / \mathrm{l}$ bo'lsa, $\mathrm{k}=2$ bo'lgandagi tezligini ( $\mathrm{mol} / \mathrm{l} \cdot \mathrm{min}$ ) toping.\\
A) 28\\
B) 32\\
C) 25\\
D) 30\\
24. $\mathrm{A}+2 \mathrm{~B} \rightarrow \mathrm{C}$ ushbu reaksiya bo'yicha A va B moddalarning dastlabki konsentratsiyalari mos ravishda 2 va 1 $\mathrm{mol} / \mathrm{l}$ bo'lsa, $\mathrm{k}=1$ bo'lgandagi tezligini ( $\mathrm{mol} / \mathrm{l} \cdot \mathrm{min}$ ) toping.\\
A) 2\\
B) 5\\
C) 6\\
D) 7\\
25. $3 \mathrm{~A}+\mathrm{B} \rightarrow \mathrm{C}$ ushbu reaksiya bo'yicha A va B moddalarning dastlabki konsentratsiyalari mos ravishda 1 va 3 $\mathrm{mol} / \mathrm{l}$ bo'lsa, $\mathrm{k}=3$ bo'lgandagi tezligini ( $\mathrm{mol} / \mathrm{l} \cdot \mathrm{min}$ ) toping.\\
A) 6\\
B) 5\\
C) 8\\
D) 9\\
26. $2 \mathrm{~A}+3 \mathrm{~B} \rightarrow \mathrm{C}$ ushbu reaksiya bo'yicha A va B moddalarning dastlabki\\
konsentratsiyalari mos ravishda 2 va 1 $\mathrm{mol} / \mathrm{l}$ bo'lsa, $\mathrm{k}=1$ bo'lgandagi tezligini ( $\mathrm{mol} / / \cdot \min$ ) toping.\\
A) 4\\
B) 5\\
C) 6\\
D) 7\\
27. $3 \mathrm{~A}+3 \mathrm{~B} \rightarrow \mathrm{C}$ ushbu reaksiya bo'yicha A va B moddalarning dastlabki konsentratsiyalari mos ravishda 1 va 2 $\mathrm{mol} / \mathrm{l}$ bo'lsa, $\mathrm{k}=1,5$ bo'lgandagi tezligini ( $\mathrm{mol} / \mathrm{l} \cdot \mathrm{min}$ ) toping.\\
A) 18\\
B) 12\\
C) 15\\
D) 10\\
28. $\mathrm{A}+\mathrm{B} \rightarrow \mathrm{C}$ ushbu reaksiya bo'yicha A va B moddalarning dastlabki konsentratsiyalari mos ravishda 2 va 3 $\mathrm{mol} / \mathrm{l}$ bo'lsa, $\mathrm{k}=2,5$ bo'lgandagi tezligini ( $\mathrm{mol} / \mathrm{l} \cdot \mathrm{min}$ ) toping.\\
A) 18\\
B) 12\\
C) 15\\
D) 10\\
29. $2 \mathrm{~A}+\mathrm{B} \rightarrow \mathrm{C}$ ushbu reaksiya bo'yicha A va B moddalarning dastlabki konsentratsiyalari mos ravishda 2,5 va 3 $\mathrm{mol} / \mathrm{l}$ bo'lsa, $\mathrm{k}=0,4$ bo'lgandagi tezligini ( $\mathrm{mol} / / \cdot \mathrm{min}$ ) toping.\\
A) 7,5\\
B) 6,8\\
C) 1,5\\
D) 4,5\\
30. $2 \mathrm{~A}+4 \mathrm{~B} \rightarrow \mathrm{C}$ ushbu reaksiya bo'yicha A va B moddalarning dastlabki konsentratsiyalari mos ravishda 2 va 1 $\mathrm{mol} / \mathrm{l}$ bo'lsa, $\mathrm{k}=1,5$ bo'lgandagi tezligini ( $\mathrm{mol} / 1 \cdot \mathrm{~min}$ ) toping.\\
A) 6\\
B) 7\\
C) 8\\
D) 10
31. $2 \mathrm{~A}+\mathrm{B} \rightarrow \mathrm{C}$ ushbu reaksiya bo'yicha A va B moddalarning dastlabki konsentratsiyalari mos ravishda 2 va 3 $\mathrm{mol} / \mathrm{l}$ bo'lsa, to'g'ri reaksiya tezligi 18 $\mathrm{mol} / \mathrm{l} \cdot$ min bo'lsa tezlik konstantasini toping.\\
A) 1,8\\
B) 1,2\\
C) 1,5\\
D) 1\\
32. $\mathrm{A}+\mathrm{B} \rightarrow \mathrm{C}$ ushbu reaksiya bo'yicha A va B moddalarning dastlabki konsentratsiyalari mos ravishda 2 va 2 $\mathrm{mol} / \mathrm{l}$ bo'lsa, to'g'ri reaksiya tezligi 16 $\mathrm{mol} / \mathrm{l} \cdot \mathrm{min}$ bo'lsa tezlik konstantasini toping.\\
A) 2\\
B) 3\\
C) 4\\
D) 1\\
33. $\mathrm{A}+2 \mathrm{~B} \rightarrow \mathrm{C}$ ushbu reaksiya bo'yicha A va B moddalarning dastlabki konsentratsiyalari mos ravishda 2 va 4 $\mathrm{mol} / \mathrm{l}$ bo'lsa, to'g'ri reaksiya tezligi 16 $\mathrm{mol} / \cdot \cdot \min$ bo'lsa tezlik konstantasini toping.\\
A) 1.1\\
B) 0,5\\
C) 1,5\\
D) 0,6\\
34. $2 \mathrm{~A}+2 \mathrm{~B} \rightarrow \mathrm{C}$ ushbu reaksiya bo'yicha A va B moddalarning dastlabki konsentratsiyalari mos ravishda 2 va 1 $\mathrm{mol} / \mathrm{l}$ bo'lsa, to'g'ri reaksiya tezligi 8 $\mathrm{mol} / \cdot \cdot \min$ bo'lsa tezlik konstantasini toping.\\
A) 2\\
B) 3\\
C) 4\\
D) 1\\
35. $2 \mathrm{~A}+\mathrm{B} \rightarrow \mathrm{C}$ ushbu reaksiya bo'yicha A va B moddalarning dastlabki konsentratsiyalari mos ravishda 2 va 5 $\mathrm{mol} / \mathrm{l}$ bo'lsa, to'g'ri reaksiya tezligi 20 $\mathrm{mol} / \cdot \cdot \min$ bo'lsa tezlik konstantasini toping.\\
A) 1,8\\
B) 1,2\\
C) 1,5\\
D) 1\\
36. $\mathrm{A}+3 \mathrm{~B} \rightarrow \mathrm{C}$ ushbu reaksiya bo'yicha A va B moddalarning dastlabki konsentratsiyalari mos ravishda 2 va 1 $\mathrm{mol} / \mathrm{l}$ bo'lsa, to'g'ri reaksiya tezligi 14 $\mathrm{mol} / \cdot \cdot \min$ bo'lsa tezlik konstantasini toping.\\
A) 7\\
B) 6\\
C) 5\\
D) 4\\
37. $\mathrm{A}+\mathrm{B} \rightarrow \mathrm{C}$ ushbu reaksiya bo'yicha A va B moddalarning dastlabki konsentratsiyalari mos ravishda 5 va 5 $\mathrm{mol} / \mathrm{l}$ bo'lsa, to'g'ri reaksiya tezligi 50 $\mathrm{mol} / \mathrm{l} \cdot \min$ bo'lsa tezlik konstantasini toping.\\
A) 8\\
B) 2\\
C) 5\\
D) 1\\
38. $2 \mathrm{~A}+\mathrm{B} \rightarrow \mathrm{C}$ ushbu reaksiya bo'yicha A va B moddalarning dastlabki konsentratsiyalari mos ravishda 1 va 5 $\mathrm{mol} / \mathrm{l}$ bo'lsa, to'g'ri reaksiya tezligi 7,5 $\mathrm{mol} / \mathrm{l} \cdot$ min bo'lsa tezlik konstantasini toping.\\
A) 1,8\\
B) 1,2\\
C) 1,5\\
D) 1\\
39. $2 \mathrm{~A}+2 \mathrm{~B} \rightarrow \mathrm{C}$ ushbu reaksiya bo'yicha A va B moddalarning dastlabki konsentratsiyalari mos ravishda 2 va 3 $\mathrm{mol} / \mathrm{l}$ bo'lsa, to'g'ri reaksiya tezligi 18\\
mol/l•min bo'lsa tezlik konstantasini toping.\\
A) 1,1\\
B) 0,5\\
C) 1,5\\
D) 0,6\\
40. $2 \mathrm{~A}+4 \mathrm{~B} \rightarrow \mathrm{C}$ ushbu reaksiya bo'yicha A va B moddalarning dastlabki konsentratsiyalari mos ravishda 2 va 2 $\mathrm{mol} / \mathrm{l}$ bo'lsa, to'g'ri reaksiya tezligi 16 $\mathrm{mol} / \mathrm{l} \cdot \min$ bo'lsa tezlik konstantasini toping.\\
A) 0,8\\
B) 0,2\\
C) 0,5\\
D) 0,25
  \item $\mathrm{A}+2 \mathrm{~B} \rightarrow \mathrm{C}$ ushbu reaksiya bo'yicha A va B moddalardan mos ravishda 2 va $3 \mathrm{~mol} / \mathrm{l}$ olingandagi tezligi $36 \mathrm{~mol} / 1 \cdot \mathrm{~min}$ ga teng. A moddadan $1 \mathrm{~mol} / 1$ sarflangandan keyingi reaksiya tezligini $\mathrm{mol} / \mathrm{l} \cdot \min$ da toping.\\
A) 2\\
B) 1\\
C) 1,5\\
D) 0,6
  \item A+2B $\rightarrow$ C ushbu reaksiya bo'yicha A va B moddalardan mos ravishda 2 va $4 \mathrm{~mol} / \mathrm{l}$ olingandagi tezligi $32 \mathrm{~mol} / \cdot \mathrm{min}$ ga teng. A moddadan $1 \mathrm{~mol} / \mathrm{l}$ sarflangandan keyingi reaksiya tezligini $\mathrm{mol} / \mathrm{l} \cdot \min$ da toping.\\
A) 7\\
B) 6\\
C) 5\\
D) 4
  \item $2 \mathrm{~A}+\mathrm{B} \rightarrow \mathrm{C}$ ushbu reaksiya bo'yicha A va B moddalardan mos ravishda 4 va $2 \mathrm{~mol} / \mathrm{l}$ olingandagi tezligi $64 \mathrm{~mol} / \cdot \cdot \mathrm{min}$ ga teng. A moddadan $2 \mathrm{~mol} / \mathrm{l}$ sarflangandan keyingi reaksiya tezligini $\mathrm{mol} / \cdot \cdot \min$ da toping.\\
A) 5\\
B) 6\\
C) 8\\
D) 4
  \item $\mathrm{A}+2 \mathrm{~B} \rightarrow \mathrm{C}$ ushbu reaksiya bo'yicha A va B moddalardan mos ravishda 4 va $4 \mathrm{~mol} / \mathrm{l}$ olingandagi tezligi $32 \mathrm{~mol} / \cdot \mathrm{min}$ ga teng. A moddadan $1 \mathrm{~mol} / 1$ sarflangandan keyingi reaksiya tezligini $\mathrm{mol} / \mathrm{l} \cdot \min$ da toping.\\
A) 5\\
B) 6\\
C) 8\\
D) 4
  \item $2 \mathrm{~A}+2 \mathrm{~B} \rightarrow \mathrm{C}$ ushbu reaksiya bo'yicha A va B moddalardan mos ravishda 4 va 4 $\mathrm{mol} / \mathrm{l}$ olingandagi tezligi $128 \mathrm{~mol} / \cdot \min$ ga teng. A moddadan $2 \mathrm{~mol} / \mathrm{l}$ sarflangandan keyingi reaksiya tezligini $\mathrm{mol} / \cdot \cdot \min$ da toping.\\
A) 5\\
B) 6\\
C) 8\\
D) 4
  \item $3 \mathrm{~A}+2 \mathrm{~B} \rightarrow \mathrm{C}$ ushbu reaksiya bo'yicha A va B moddalardan mos ravishda 4 va 3 $\mathrm{mol} / \mathrm{l}$ olingandagi tezligi $144 \mathrm{~mol} / \cdot \cdot \min$ ga teng. A moddadan $3 \mathrm{~mol} / \mathrm{l}$ sarflangandan keyingi reaksiya tezligini $\mathrm{mol} / \cdot \cdot \min$ da toping.\\
A) 0,8\\
B) 0,2\\
C) 0,5\\
D) 0,25
  \item $\mathrm{A}+\mathrm{B} \rightarrow \mathrm{C}$ ushbu reaksiya bo'yicha A va B moddalardan mos ravishda 4 va $8 \mathrm{~mol} / \mathrm{l}$ olingandagi tezligi $32 \mathrm{~mol} / \cdot \cdot \mathrm{min}$ ga teng. A moddadan $3 \mathrm{~mol} / / \mathrm{s}$ arflangandan keyingi reaksiya tezligini $\mathrm{mol} / \cdot \cdot \min$ da toping.\\
A) 5\\
B) 6\\
C) 8\\
D) 4
  \item $2 \mathrm{~A}+2 \mathrm{~B} \rightarrow \mathrm{C}$ ushbu reaksiya bo'yicha A va $B$ moddalardan mos ravishda 6 va 6 $\mathrm{mol} / \mathrm{l}$ olingandagi tezligi $324 \mathrm{~mol} / \cdot \cdot \min$ ga teng. A moddadan $4 \mathrm{~mol} / \mathrm{l}$ sarflangandan keyingi reaksiya tezligini $\mathrm{mol} / \cdot \cdot \min$ da toping.\\
A) 5\\
B) 6\\
C) 8\\
D) 4
  \item $3 \mathrm{~A}+\mathrm{B} \rightarrow \mathrm{C}$ ushbu reaksiya bo'yicha A va B moddalardan mos ravishda 4 va $4 \mathrm{~mol} / \mathrm{l}$ olingandagi tezligi $128 \mathrm{~mol} / \cdot \mathrm{min}$ ga teng. A moddadan $3 \mathrm{~mol} / 1$ sarflangandan keyingi reaksiya tezligini $\mathrm{mol} / \cdot \cdot \min \mathrm{da}$ toping.\\
A) 2\\
B) 1\\
C) 1,5\\
D) 0,6
  \item $\mathrm{A}+\mathrm{B} \rightarrow \mathrm{C}$ ushbu reaksiya bo'yicha A va B moddalardan mos ravishda 8 va $8 \mathrm{~mol} / \mathrm{l}$ olingandagi tezligi $64 \mathrm{~mol} / \cdot \cdot \mathrm{min}$ ga teng. A moddadan $4 \mathrm{~mol} / \mathrm{l}$ sarflangandan keyingi reaksiya tezligini $\mathrm{mol} / \mathrm{l} \cdot \min$ da toping.\\
A) 16\\
B) 14\\
C) 15\\
D) 6
  \item $\mathrm{Fe}_{(\mathrm{q})}+\mathrm{Cl}_{2(\mathrm{~g})} \rightarrow \mathrm{FeCl}_{3(\mathrm{q})}$ ushbu reaksiya bo'yicha sistema bosimi 2 marta oshirilsa, to'g'ri reaksiya tezligi necha marta ortadi?\\
A) 5\\
B) 6\\
C) 8\\
D) 4
  \item $\mathrm{A}_{(\mathrm{g})}+2 \mathrm{~B}_{(\mathrm{g})} \rightarrow \mathrm{C}_{(\mathrm{g})}$ ushbu reaksiya bo'yicha sistema bosimi 3 marta oshirilsa, to'g'ri reaksiya tezligi necha marta ortadi?\\
A) 27\\
B) 18\\
C) 9\\
D) 3
  \item $\mathrm{H}_{2(\mathrm{~g})}+\mathrm{Cl}_{2(\mathrm{~g})} \rightarrow \mathrm{HCl}_{(\mathrm{g})}$ ushbu reaksiya bo'yicha sistema bosimi 4 marta oshirilsa, to'g'ri reaksiya tezligi necha marta ortadi?\\
A) 15\\
B) 16\\
C) 18\\
D) 14
  \item $\mathrm{N}_{2(\mathrm{~g})}+\mathrm{H}_{2(\mathrm{~g})} \rightarrow \mathrm{NH}_{3(\mathrm{~g})}$ ushbu reaksiya bo'yicha sistema bosimi 3 marta oshirilsa, to'g'ri reaksiya tezligi necha marta ortadi?\\
A) 18\\
B) 36\\
C) 27\\
D) 81
  \item $\mathrm{S}_{(\mathrm{q})}+\mathrm{H}_{2(\mathrm{~g})} \rightarrow \mathrm{H}_{2} \mathrm{~S}(\mathrm{~g})$ ushbu reaksiya bo'yicha sistema bosimi 6 marta oshirilsa, to'g'ri reaksiya tezligi necha marta ortadi?\\
A) 5\\
B) 6\\
C) 8\\
D) 4
  \item $\mathrm{CO}_{(\mathrm{g})}+\mathrm{Cl}_{2(\mathrm{~g})} \rightarrow \mathrm{COCl}_{2(\mathrm{~g})}$ ushbu reaksiya bo'yicha sistema bosimi 3 marta oshirilsa, to'g'ri reaksiya tezligi necha marta ortadi?\\
A) 9\\
B) 8\\
C) 7\\
D) 6
  \item $\mathrm{NO}_{2(\mathrm{~g})} \rightarrow \mathrm{N}_{2} \mathrm{O}_{4(\mathrm{~g})}$ ushbu reaksiya bo'yicha sistema bosimi 5 marta oshirilsa, to'g'ri reaksiya tezligi necha marta ortadi?\\
A) 18\\
B) 15\\
C) 5\\
D) 25
  \item $\mathrm{Al}_{(\mathrm{q})}+\mathrm{Cl}_{2(\mathrm{~g})} \rightarrow \mathrm{AlCl}_{3(\mathrm{q})}$ ushbu reaksiya bo'yicha sistema bosimi 3 marta oshirilsa, to'g'ri reaksiya tezligi necha marta ortadi?\\
A) 18\\
B) 27\\
C) 9\\
D) 3
  \item $\mathrm{Mg}_{(q)}+\mathrm{O}_{2(g)} \rightarrow \mathrm{MgO}_{(q)}$ ushbu reaksiya bo'yicha sistema bosimi 4 marta oshirilsa, to'g'ri reaksiya tezligi necha marta ortadi?\\
A) 5\\
B) 6\\
C) 8\\
D) 4
  \item $\mathrm{A}_{(\mathrm{g})}+\mathrm{B}_{(\mathrm{g})} \rightarrow \mathrm{A}_{3} \mathrm{~B}_{2(\mathrm{~g})}$ ushbu reaksiya bo'yicha sistema bosimi 2 marta oshirilsa, to'g'ri reaksiya tezligi necha marta ortadi?\\
A) 32\\
B) 16\\
C) 8\\
D) 4
  \item $\mathrm{Fe}_{(\mathrm{q})}+\mathrm{Cl}_{2(\mathrm{~g})} \rightarrow \mathrm{FeCl}_{3(\mathrm{q})}$ ushbu reaksiya bo'yicha $\mathrm{Cl}_{2}$ ning sarflanish tezligi 2 $\mathrm{mol} / \cdot \cdot$ min teng. Sistema bosimi 2 marta oshirilsa, $\mathrm{Cl}_{2}$ ning sarflanish tezligi qancha $\mathrm{mol} / \cdot \cdot \min$ ga teng bo'ladi?\\
A) 16\\
B) 8\\
C) 9\\
D) 4
  \item $\mathrm{A}_{(\mathrm{g})}+2 \mathrm{~B}_{(\mathrm{g})} \rightarrow \mathrm{C}_{(\mathrm{g})}$ ushbu reaksiya bo'yicha A ning sarflanish tezligi $1 \mathrm{~mol} / \cdot \mathrm{min}$ teng. Sistema bosimi 3 marta oshirilsa, A ning sarflanish tezligi qancha mol/l•min ga teng bo'ladi?\\
A) 3\\
B) 9\\
C) 27\\
D) 18
  \item $\mathrm{H}_{2(\mathrm{~g})}+\mathrm{Cl}_{2(\mathrm{~g})} \rightarrow \mathrm{HCl}_{(\mathrm{g})}$ ushbu reaksiya bo'yicha $\mathrm{Cl}_{2}$ ning sarflanish tezligi 1,5 $\mathrm{mol} / \cdot \cdot \min$ teng. Sistema bosimi 4 marta oshirilsa, $\mathrm{Cl}_{2}$ ning sarflanish tezligi qancha $\mathrm{mol} / \cdot \cdot \min$ ga teng bo'ladi?\\
A) 4\\
B) 48\\
C) 16\\
D) 24
  \item $\mathrm{N}_{2(\mathrm{~g})}+\mathrm{H}_{2(\mathrm{~g})} \rightarrow \mathrm{NH}_{3(\mathrm{~g})}$ ushbu reaksiya bo'yicha $\mathrm{N}_{2}$ ning sarflanish tezligi 2,5 $\mathrm{mol} / \cdot$ min teng. Sistema bosimi 2 marta oshirilsa, $\mathrm{N}_{2}$ ning sarflanish tezligi qancha $\mathrm{mol} / \cdot \cdot \min$ ga teng bo'ladi?\\
A) 40\\
B) 20\\
C) 50\\
D) 24
  \item $\mathrm{S}_{(\mathrm{q})}+\mathrm{H}_{2(\mathrm{~g})} \rightarrow \mathrm{H}_{2} \mathrm{~S}_{(\mathrm{g})}$ ushbu reaksiya bo'yicha $\mathrm{H}_{2}$ ning sarflanish tezligi 3 $\mathrm{mol} / \cdot$ - min teng. Sistema bosimi 4 marta oshirilsa, $\mathrm{H}_{2}$ ning sarflanish tezligi qancha $\mathrm{mol} / \cdot \cdot \min$ ga teng bo'ladi?\\
A) 16\\
B) 8\\
C) 9\\
D) 12
  \item $\mathrm{CO}_{(\mathrm{g})}+\mathrm{Cl}_{2(\mathrm{~g})} \rightarrow \mathrm{COCl}_{2(\mathrm{~g})}$ ushbu reaksiya bo'yicha $\mathrm{Cl}_{2}$ ning sarflanish tezligi 2 $\mathrm{mol} / 1 \cdot \mathrm{~min}$ teng. Sistema bosimi 2 marta oshirilsa, $\mathrm{Cl}_{2}$ ning sarflanish tezligi qancha $\mathrm{mol} / \cdot \mathrm{min}$ ga teng bo'ladi?\\
A) 16\\
B) 8\\
C) 9\\
D) 4
  \item $\mathrm{NO}_{2(\mathrm{~g})} \rightarrow \mathrm{N}_{2} \mathrm{O}_{4(\mathrm{~g})}$ ushbu reaksiya bo'yicha $\mathrm{NO}_{2}$ ning sarflanish tezligi $4 \mathrm{~mol} / \cdot \cdot \mathrm{min}$ teng. Sistema bosimi 3 marta oshirilsa, $\mathrm{NO}_{2}$ ning sarflanish tezligi qancha $\mathrm{mol} / \cdot \mathrm{min}$ ga teng bo'ladi?\\
A) 27\\
B) 48\\
C) 36\\
D) 54
  \item $\mathrm{Al}_{(\mathrm{q})}+\mathrm{Cl}_{2(\mathrm{~g})} \rightarrow \mathrm{AlCl}_{3(\mathrm{q})}$ ushbu reaksiya bo'yicha $\mathrm{Cl}_{2}$ ning sarflanish tezligi 1 $\mathrm{mol} / \cdot \cdot \mathrm{min}$ teng. Sistema bosimi 2 marta oshirilsa, $\mathrm{Cl}_{2}$ ning sarflanish tezligi qancha $\mathrm{mol} / \cdot$ min ga teng bo'ladi?\\
A) 16\\
B) 8\\
C) 9\\
D) 4
  \item $\mathrm{Mg}_{(\mathrm{q})}+\mathrm{O}_{2(\mathrm{~g})} \rightarrow \mathrm{MgO}_{(\mathrm{q})}$ ushbu reaksiya bo'yicha $\mathrm{O}_{2}$ ning sarflanish tezligi 3 $\mathrm{mol} / \mathrm{l} \cdot \mathrm{min}$ teng. Sistema bosimi 4 marta oshirilsa, $\mathrm{O}_{2}$ ning sarflanish tezligi qancha $\mathrm{mol} / \cdot \cdot \min$ ga teng bo'ladi?\\
A) 16\\
B) 8\\
C) 9\\
D) 12
  \item $\mathrm{A}_{(\mathrm{g})}+\mathrm{B}_{(\mathrm{g})} \rightarrow \mathrm{A}_{3} \mathrm{~B}_{2(\mathrm{~g})}$ ushbu reaksiya bo'yicha A ning sarflanish tezligi 2 $\mathrm{mol} / \mathrm{l} \cdot \mathrm{min}$ teng. Sistema bosimi 2 marta oshirilsa, A ning sarflanish tezligi qancha $\mathrm{mol} / \cdot \min$ ga teng bo'ladi?\\
A) 64\\
B) 8\\
C) 9\\
D) 72
  \item $\mathrm{Fe}(\mathrm{q})+\mathrm{Cl}_{2(\mathrm{~g})} \rightarrow \mathrm{FeCl}_{3(\mathrm{q})}$ ushbu reaksiya bo'yicha $\mathrm{Cl}_{2}$ ning sarflanish tezligi 2 $\mathrm{mol} / \cdot$ min teng. Sistema hajmi 2 marta oshirilsa, $\mathrm{Cl}_{2}$ ning sarflanish tezligi qancha $\mathrm{mol} / \cdot \cdot \min$ ga teng bo'ladi?\\
A) 0,25\\
B) 8\\
C) 9\\
D) 4
  \item $\mathrm{A}_{(\mathrm{g})}+2 \mathrm{~B}_{(\mathrm{g})} \rightarrow \mathrm{C}_{(\mathrm{g})}$ ushbu reaksiya bo'yicha A ning sarflanish tezligi $27 \mathrm{~mol} / \cdot \cdot \min$ teng. Sistema hajmi 3 marta oshirilsa, A ning sarflanish tezligi qancha $\mathrm{mol} / \cdot \cdot \min$ ga teng bo'ladi?\\
A) 1\\
B) 9\\
C) 2\\
D) 18
  \item $\mathrm{H}_{2(\mathrm{~g})}+\mathrm{Cl}_{2(\mathrm{~g})} \rightarrow \mathrm{HCl}_{(\mathrm{g})}$ ushbu reaksiya bo'yicha $\mathrm{Cl}_{2}$ ning sarflanish tezligi 48 $\mathrm{mol} / \mathrm{l} \cdot$ min teng. Sistema hajmi 4 marta oshirilsa, $\mathrm{Cl}_{2}$ ning sarflanish tezligi qancha $\mathrm{mol} / \cdot \min$ ga teng bo'ladi?\\
A) 4\\
B) 2\\
C) 6\\
D) 3
  \item $\mathrm{N}_{2(\mathrm{~g})}+\mathrm{H}_{2(\mathrm{~g})} \rightarrow \mathrm{NH}_{3(\mathrm{~g})}$ ushbu reaksiya bo'yicha $\mathrm{N}_{2}$ ning sarflanish tezligi 32 $\mathrm{mol} / \cdot \cdot \min$ teng. Sistema hajmi 2 marta\\ oshirilsa, $\mathrm{N}_{2}$ ning sarflanish tezligi qancha $\mathrm{mol} / \cdot \cdot \min$ ga teng bo'ladi?\\
A) 4\\
B) 2\\
C) 3\\
D) 5
  \item $\mathrm{S}_{(q)}+\mathrm{H}_{2(q)} \rightarrow \mathrm{H}_{2} \mathrm{~S}_{(k)}$ ushbu reaksiya bo'yicha $\mathrm{H}_{2}$ ning sarflanish tezligí 16 $\mathrm{mol} / \cdot$ min teng. Sistema hajmi 4 marta oshirilsa, $\mathrm{H}_{2}$ ning sarflanish tezligí qancha $\mathrm{mol} / \cdot \min$ ga teng bo'ladi?\\
A) 6\\
B) 4\\
C) 8\\
D) 2
  \item $\mathrm{CO}_{(\mathrm{g})}+\mathrm{Cl}_{2(\mathrm{~g})} \rightarrow \mathrm{COCl}_{2(\mathrm{~g})}$ ushbu reaksiya bo'yicha $\mathrm{Cl}_{2}$ ning sarflanish tezligi 24 $\mathrm{mol} / \cdot \cdot$ min teng. Sistema hajmi 2 marta oshirilsa, $\mathrm{Cl}_{2}$ ning sarflanish tezligi qancha $\mathrm{mol} / \cdot \cdot \mathrm{min}$ ga teng bo'ladi?\\
A) 6\\
B) 4\\
C) 8\\
D) 2
  \item $\mathrm{NO}_{2(\mathrm{~g})} \rightarrow \mathrm{N}_{2} \mathrm{O}_{4(\mathrm{~g})}$ ushbu reaksiya bo'yicha $\mathrm{NO}_{2}$ ning sarflanish tezligi $27 \mathrm{~mol} / \cdot \min$ teng. Sistema hajmi 3 marta oshirilsa, $\mathrm{NO}_{2}$ ning sarflanish tezligi qancha $\mathrm{mol} / \cdot \cdot \min$ ga teng bo'ladi?\\
A) 7\\
B) 8\\
C) 3\\
D) 5
  \item $\mathrm{Al}_{(\mathrm{q})}+\mathrm{Cl}_{2(\mathrm{~g})} \rightarrow \mathrm{AlCl}_{3(\mathrm{q})}$ ushbu reaksiya bo'yicha $\mathrm{Cl}_{2}$ ning sarflanish tezligi 32 $\mathrm{mol} / \cdot \cdot \min$ teng. Sistema hajmi 2 marta oshirilsa, $\mathrm{Cl}_{2}$ ning sarflanish tezligi qancha $\mathrm{mol} / \mathrm{l} \cdot \mathrm{min}$ ga teng bo'ladi?\\
A) 16\\
B) 8\\
C) 9\\
D) 4
  \item $\mathrm{Mg}_{(q)}+\mathrm{O}_{2(\mathrm{~g})} \rightarrow \mathrm{MgO}_{(q)}$ ushbu reaksiya bo'yicha $\mathrm{O}_{2}$ ning sarflanish tezligi 12 $\mathrm{mol} / \cdot \min$ teng. Sistema hajmi 4 marta oshirilsa, $\mathrm{O}_{2}$ ning sarflanish tezligi qancha $\mathrm{mol} / \cdot \cdot \min$ ga teng bo'ladi?\\
A) 3\\
B) 9\\
C) 27\\
D) 12
  \item $\mathrm{A}_{(\mathrm{g})}+\mathrm{B}_{(\mathrm{g})} \rightarrow \mathrm{A}_{3} \mathrm{~B}_{2(\mathrm{~g})}$ ushbu reaksiya bo'yicha A ning sarflanish tezligi 32 $\mathrm{mol} / \cdot$ min teng. Sistema hajmi 2 marta oshirilsa, A ning sarflanish tezligi qancha $\mathrm{mol} / \cdot \min$ ga teng bo'ladi?\\
A) 1\\
B) 8\\
C) 9\\
D) 72
  \item $\mathrm{H}_{2(\mathrm{~g})}+\mathrm{Cl}_{2(\mathrm{y})} \rightarrow \mathrm{HCl}_{(\mathrm{g})}$ ushbu reaksiya bo'yicha $\mathrm{H}_{2}$ ning konsentratsiyasi 3 marta oshírilsa, to'g'ri reaksiya tezligi necha marta ortadi?\\
A) 3\\
B) 8\\
C) 9\\
D) 7
82. $\mathrm{Fe}_{(q)}+\mathrm{Cl}_{2(g)} \rightarrow \mathrm{FeCl}_{3(q)}$ ushbu reaksiya bo'yicha $\mathrm{Cl}_{2}$ ning konsentratsiyasi 2 marta oshirilsa, to'g'ri reaksiya tezligi necha marta ortadi?\\
A) 3\\
B) 8\\
C) 9\\
D) 7\\
83. $\mathrm{A}_{(\mathrm{g})}+2 \mathrm{~B}_{(\mathrm{g})} \rightarrow \mathrm{C}_{(\mathrm{g})}$ ushbu reaksiya bo'yicha B ning konsentratsiyasi 4 marta oshirilsa, to'g'ri reaksiya tezligi necha marta ortadi?\\
A) 14\\
B) 18\\
C) 15\\
D) 16\\
84. $\mathrm{N}_{2(\mathrm{~g})}+\mathrm{H}_{2(\mathrm{~g})} \rightarrow \mathrm{NH}_{3(\mathrm{~g})}$ ushbu reaksiya bo'yicha $\mathrm{N}_{2}$ ning konsentratsiyasi 3 marta oshirilsa, to'g'ri reaksiya tezligi necha marta ortadi?\\
A) 3\\
B) 8\\
C) 9\\
D) 7\\
85. $\mathrm{S}_{(\mathrm{q})}+\mathrm{H}_{2(\mathrm{~g})} \rightarrow \mathrm{H}_{2} \mathrm{~S}_{(\mathrm{g})}$ ushbu reaksiya bo'yicha $\mathrm{H}_{2}$ ning konsentratsiyasi 7 marta oshirilsa, to'g'ri reaksiya tezligi necha marta ortadi?\\
A) 3\\
B) 8\\
C) 9\\
D) 7\\
86. $\mathrm{CO}_{(\mathrm{g})}+\mathrm{Cl}_{2(\mathrm{~g})} \rightarrow \mathrm{COCl}_{2(\mathrm{~g})}$ ushbu reaksiya bo'yicha CO ning konsentratsiyasi 8 marta oshirilsa, to'g'ri reaksiya tezligi necha marta ortadi?\\
A) 3\\
B) 8\\
C) 9\\
D) 7\\
87. $\mathrm{NO}_{2(\mathrm{~g})} \rightarrow \mathrm{N}_{2} \mathrm{O}_{4(\mathrm{~g})}$ ushbu reaksiya bo'yicha $\mathrm{NO}_{2}$ ning konsentratsiyasi 3 marta oshirilsa, to'g'ri reaksiya tezligi necha marta ortadi?\\
A) 3\\
B) 8\\
C) 9\\
D) 7\\
88. $\mathrm{Al}_{(q)}+\mathrm{Cl}_{2(g)} \rightarrow \mathrm{AlCl}_{3(q)}$ ushbu reaksiya bo'yicha $\mathrm{Cl}_{2}$ ning konsentratsiyasi 2 marta oshirilsa, to'g'ri reaksiya tezligi necha marta ortadi?\\
A) 3\\
B) 8\\
C) 9\\
D) 7\\
89. $\mathrm{Mg}_{(q)}+\mathrm{O}_{2(\mathrm{~g})} \rightarrow \mathrm{MgO}_{(\mathrm{q})}$ ushbu reaksiya bo'yicha $\mathrm{O}_{2}$ ning konsentratsiyasi 6 marta oshirilsa, to'g'ri reaksiya tezligi necha marta ortadi?\\
A) 3\\
B) 5\\
C) 6\\
D) 7\\
90. $\mathrm{A}_{(\mathrm{g})}+\mathrm{B}_{(\mathrm{g})} \rightarrow \mathrm{A}_{3} \mathrm{~B}_{2(\mathrm{~g})}$ ushbu reaksiya bo'yicha A ning konsentratsiyasi 3 marta oshirilsa, to'g'ri reaksiya tezligi necha marta ortadi?\\
A) 23\\
B) 28\\
C) 29\\
D) 27
  \item $\mathrm{H}_{2(\mathrm{~g})}+\mathrm{Cl}_{2(\mathrm{~g})} \rightarrow \mathrm{HCl}_{(\mathrm{g})}$ ushbu reaksiya bo'yicha HCl ning hosil bo'lish tezligi 27 $\mathrm{mol} / \mathrm{l} \cdot \min$ ga teng. $\mathrm{H}_{2}$ ning konsentratsiyasi 3 marta kamaytirilsa, HCl ning hosil bo'lish tezligi qanchaga teng bo'ladi?\\
A) 3\\
B) 8\\
C) 9\\
D) 7
92. $\mathrm{Fe}{ }_{(\mathrm{q})}+\mathrm{Cl}_{2(\mathrm{~g})} \rightarrow \mathrm{FeCl}_{3(\mathrm{q})}$ ushbu reaksiya bo'yicha $\mathrm{Cl}_{2}$ ning sarf bo'lish tezligi 8 $\mathrm{mol} / \sqrt{ } \cdot \min$ ga teng. $\mathrm{Cl}_{2}$ ning konsentratsiyasi 2 marta kamaytirilsa, $\mathrm{Cl}_{2}$ ning sarf bo'lish tezligi qanchaga teng bo'ladi?\\
A) 4\\
B) 3\\
C) 2\\
D) 1\\
93. $\mathrm{A}_{(\mathrm{g})}+2 \mathrm{~B}_{(\mathrm{g})} \rightarrow \mathrm{C}_{(\mathrm{g})}$ ushbu reaksiya bo'yicha C ning hosil bo'lish tezligi $27 \mathrm{~mol} / \cdot \min$ ga teng. A ning konsentratsiyasi 9 marta kamaytirilsa, C ning hosil bo'lish tezligi qanchaga teng bo'ladi?\\
A) 3\\
B) 8\\
C) 9\\
D) 7\\
94. $\mathrm{N}_{2(\mathrm{~g})}+\mathrm{H}_{2(\mathrm{~g})} \rightarrow \mathrm{NH}_{3(\mathrm{~g})}$ ushbu reaksiya bo'yicha $\mathrm{NH}_{3}$ ning hosil bo'lish tezligi 64 $\mathrm{mol} / 1 \cdot \min$ ga teng. $\mathrm{H}_{2}$ ning konsentratsiyasi 2 marta kamaytirilsa, $\mathrm{NH}_{3}$ ning hosil bo'lish tezligi qanchaga teng bo'ladi?\\
A) 3\\
B) 8\\
C) 9\\
D) 7\\
95. $\mathrm{S}_{(\mathrm{q})}+\mathrm{H}_{2(\mathrm{q})} \rightarrow \mathrm{H}_{2} \mathrm{~S}_{(\mathrm{g})}$ ushbu reaksiya bo'yicha $\mathrm{H}_{2} \mathrm{~S}$ ning hosil bo'lish tezligi 49 $\mathrm{mol} / / \cdot \min$ ga teng. $\mathrm{H}_{2}$ ning konsentratsiyasi 7 marta kamaytirilsa, $\mathrm{H}_{2} \mathrm{~S}$ ning hosil bo'lish tezligi qanchaga teng bo'ladi?\\
A) 3\\
B) 8\\
C) 9\\
D) 7\\
96. $\mathrm{CO}_{(\mathrm{g})}+\mathrm{Cl}_{2(\mathrm{~g})} \rightarrow \mathrm{COCl}_{2(\mathrm{~g})}$ ushbu reaksiya bo'yicha $\mathrm{COCl}_{2}$ ning hosil bo'lish tezligi 25 $\mathrm{mol} / \cdot \cdot \min$ ga teng. $\mathrm{Cl}_{2}$ ning konsentratsiyasi 5 marta kamaytirilsa, $\mathrm{COCl}_{2}$ ning hosil bo'lish tezligi qanchaga teng bo'ladi?\\
A) 5\\
B) 6\\
C) 7\\
D) 8\\
97. $\mathrm{NO}_{2(\mathrm{~g})} \rightarrow \mathrm{N}_{2} \mathrm{O}_{4(\mathrm{~g})}$ ushbu reaksiya bo'yicha $\mathrm{N}_{2} \mathrm{O}_{4}$ ning hosil bo'lish tezligi 27 $\mathrm{mol} / \cdot \cdot$ min ga teng. $\mathrm{NO}_{2}$ ning konsentratsiyasi 3 marta kamaytirilsa, $\mathrm{N}_{2} \mathrm{O}_{4}$ ning hosil bo'lish tezligi qanchaga teng bo'ladi?\\
A) 3\\
B) 8\\
C) 9\\
D) 7\\
98. $\mathrm{Al}_{(\mathrm{q})}+\mathrm{Cl}_{2(\mathrm{~g})} \rightarrow \mathrm{AlCl}_{3(\mathrm{q})}$ ushbu reaksiya bo'yicha $\mathrm{Cl}_{2}$ ning sarf bo'lish tezligi 54 $\mathrm{mol} / 1 \cdot \min$ ga teng. $\mathrm{Cl}_{2}$ ning konsentratsiyasi 3 marta kamaytirilsa, $\mathrm{Cl}_{2}$ ning sarf bo'lish tezligi qanchaga teng bo'ladi?\\
A) 3\\
B) 2\\
C) 9\\
D) 1\\
99. $\mathrm{Mg}_{(q)}+\mathrm{O}_{2(\mathrm{~g})} \rightarrow \mathrm{MgO}_{(\mathrm{q})}$ ushbu reaksiya bo'yicha' $\mathrm{O}_{2}$ ning sarf bo'lish tezligi 18 $\mathrm{mol} / \mathrm{l} \cdot \min$ ga teng. $\mathrm{O}_{2}$ ning konsentratsiyasi 2 marta kamaytirilsa, $\mathrm{O}_{2}$ ning sarf bo'lish tezligi qanchaga teng bo'ladi?\\
A) 3\\
B) 8\\
C) 9\\
D) 7\\
100. $\mathrm{A}_{(\mathrm{g})}+\mathrm{B}_{(\mathrm{g})} \rightarrow \mathrm{A}_{3} \mathrm{~B}_{2(\mathrm{~g})}$ ushbu reaksiya bo'yicha $\mathrm{A}_{3} \mathrm{~B}_{2}$ ning hosil bo'lish tezligi 81 $\mathrm{mol} / \cdot \cdot \min$ ga teng. B ning konsentratsiyasi 3 marta kamaytirilsa, $\mathrm{A}_{3} \mathrm{~B}_{2}$ ning hosil bo'lish tezligi qanchaga teng bo'ladi?\\
A) 3\\
B) 8\\
C) 9\\
D) 7
  \item $30^{\circ} \mathrm{C}$ da borayotgan reaksiya haroratini $70^{\circ} \mathrm{C}$ gacha qizdirilsa, reaksiya tezligi necha marta ortadi? ( $\mathrm{Y}=2$ )\\
A) 16\\ 
B) 8\\
C) 24\\
D) 48
102. $40^{\circ} \mathrm{C}$ da borayotgan reaksiya haroratini $60^{\circ} \mathrm{C}$ gacha qizdirilsa, reaksiya tezligi necha marta ortadi? $(\gamma=3)$\\
A) 3\\
B) 8\\
C) 9\\
D) 7\\
103. $20^{\circ} \mathrm{C}$ da borayotgan reaksiya haroratini $50^{\circ} \mathrm{C}$ gacha qizdirilsa, reaksiya tezligi necha marta ortadi? $(\gamma=4)$\\
A) 64\\
B) 32\\
C) 16\\
D) 8\\
104. $25^{\circ} \mathrm{C}$ da borayotgan reaksiya haroratini $45^{\circ} \mathrm{C}$ gacha qizdirilsa, reaksiya tezligi necha marta ortadi? $(\gamma=2)$\\
A) 64\\
B) 32\\
C) 8\\
D) 4\\
105. $10^{\circ} \mathrm{C}$ da borayotgan reaksiya haroratini $50^{\circ} \mathrm{C}$ gacha qizdirilsa, reaksiya tezligi necha marta ortadi? $(\gamma=3)$\\
A) 3\\
B) 9\\
C) 27\\
D) 81\\
106. $15^{\circ} \mathrm{C}$ da borayotgan reaksiya haroratini $25^{\circ} \mathrm{C}$ gacha qizdirilsa, reaksiya tezligi necha marta ortadi? $(\gamma=8)$\\
A) 64\\
B) 8\\
C) 9\\
D) 7\\
107. $90^{\circ} \mathrm{C}$ da borayotgan reaksiya haroratini $110^{\circ} \mathrm{C}$ gacha qizdirilsa, reaksiya tezligi necha marta ortadi? ( $\gamma=2$ )\\
A) 4\\
B) 8\\
C) 16\\
D) 32\\
108. $10^{\circ} \mathrm{C}$ da borayotgan reaksiya haroratini $20^{\circ} \mathrm{C}$ gacha qizdirilsa, reaksiya tezligi necha marta ortadi? $(\gamma=7)$\\
A) 49\\
B) 8\\
C) 9\\
D) 7\\
109. $55^{\circ} \mathrm{C}$ da borayotgan reaksiya haroratini $105^{\circ} \mathrm{C}$ gacha qizdirilsa, reaksiya tezligi necha marta ortadi? ( $\mathrm{Y}=1$ )\\
A) 1\\
B) 5\\
C) 2\\
D) 3\\
110. $20^{\circ} \mathrm{C}$ da borayotgan reaksiya haroratini $80^{\circ} \mathrm{C}$ gacha qizdirilsa, reaksiya tezligi necha marta ortadi? $(\gamma=2)$\\
A) 64\\
B) 8\\
C) 9\\
D) 7
  \item $70^{\circ} \mathrm{C}$ da borayotgan reaksiya tezligi $16 \mathrm{~mol} / / \cdot \min$ teng. Agar harorat $30^{\circ} \mathrm{C}$ gacha sovitilsa, reaksiya tezligi necha $\mathrm{mol} / \mathrm{l} \cdot$ min ga teng bo'ladi? $(\gamma=2)$\\
A) 1\\
B) 2\\
C) 3\\
D) 4\\
  \item $50^{\circ} \mathrm{C}$ da borayotgan reaksiya tezligi $27 \mathrm{~mol} / 1 \cdot \mathrm{~min}$ teng. Agar harorat $30^{\circ} \mathrm{C}$ gacha sovitilsa, reaksiya tezligi necha $\mathrm{mol} / \mathrm{l} \cdot$ min ga teng bo'ladi? $(\gamma=3)$\\
A) 1\\
B) 2\\
C) 3\\
D) 4
  \item $60^{\circ} \mathrm{C}$ da borayotgan reaksiya tezligi $64 \mathrm{~mol} /{ }^{\cdot} \cdot \mathrm{min}$ teng. Agar harorat $20^{\circ} \mathrm{C}$ gacha sovitilsa, reaksiya tezligi necha $\mathrm{mol} / \mathrm{l} \cdot \min$ ga teng bo'ladi? ( $\mathrm{Y}=2$ )\\
A) 1\\
B) 2\\
C) 3\\
D) 4
  \item $40^{\circ} \mathrm{C}$ da borayotgan reaksiya tezligi $64 \mathrm{~mol} /{ }^{\cdot} \min$ teng. Agar harorat $10^{\circ} \mathrm{C}$ gacha sovitilsa, reaksiya tezligi necha $\mathrm{mol} / \mathrm{l} \cdot$ min ga teng bo'ladi? $(\gamma=4)$\\
A) 1\\
B) 2\\
C) 3\\
D) 4
  \item $10^{\circ} \mathrm{C}$ da borayotgan reaksiya tezligi $36 \mathrm{~mol} / \mathrm{l} \cdot \min$ teng. Agar harorat $-10^{\circ} \mathrm{C}$ gacha sovitilsa, reaksiya tezligi necha $\mathrm{mol} / \mathrm{l} \cdot \mathrm{min}$ ga teng bo'ladi? $(\gamma=3)$\\
A) 1\\
B) 2\\
C) 3\\
D) 4
  \item $100^{\circ} \mathrm{C}$ da borayotgan reaksiya tezligi $32 \mathrm{~mol} /{ }^{\cdot} \cdot \min$ teng. Agar harorat $60^{\circ} \mathrm{C}$ gacha sovitilsa, reaksiya tezligi necha $\mathrm{mol} / \mathrm{l} \cdot$ min ga teng bo'ladi? ( $\mathrm{Y}=2$ )\\
A) 1\\
B) 2\\
C) 3\\
D) 4
  \item $90^{\circ} \mathrm{C}$ da borayotgan reaksiya tezligi $162 \mathrm{~mol} / \mathrm{l} \cdot \mathrm{min}$ teng. Agar harorat $50^{\circ} \mathrm{C}$ gacha sovitilsa, reaksiya tezligi necha $\mathrm{mol} / \mathrm{l} \cdot \mathrm{min}$ ga teng bo'ladi? ( $\gamma=3$ )\\
A) 1\\
B) 2\\
C) 3\\
D) 4
  \item $20^{\circ} \mathrm{C}$ da borayotgan reaksiya tezligi $81 \mathrm{~mol} / \mathrm{l} \cdot \min$ teng. Agar harorat $-10{ }^{\circ} \mathrm{C}$ gacha sovitilsa, reaksiya tezligi necha $\mathrm{mol} / \mathrm{l} \cdot \mathrm{min}$ ga teng bo'ladi? $(\gamma=3)$\\
A) 1\\
B) 2\\
C) 3\\
D) 4
  \item $35^{\circ} \mathrm{C}$ da borayotgan reaksiya tezligi $16 \mathrm{~mol} / 1 \cdot \mathrm{~min}$ teng. Agar harorat $25^{\circ} \mathrm{C}$ gacha sovitilsa, reaksiya tezligi necha $\mathrm{mol} / \mathrm{l} \cdot \mathrm{min}$ ga teng bo'ladi? ( $\gamma=4$ )\\
A) 1\\
B) 2\\
C) 3\\
D) 4
  \item $10^{\circ} \mathrm{C}$ da borayotgan reaksiya tezligi $20 \mathrm{~mol} / \mathrm{l} \cdot \mathrm{min}$ teng. Agar harorat $-20^{\circ} \mathrm{C}$\\
gacha sovitilsa, reaksiya tezligi necha $\mathrm{mol} / \mathrm{l} \cdot \mathrm{min}$ ga teng bo'ladi? ( $\gamma=2$ )\\
A) 1,5\\
B) 2,5\\
C) 3\\
D) 4
  \item $40^{\circ} \mathrm{C}$ da reaksiya 10 sekundda $20^{\circ} \mathrm{C}$ da esa 90 sekundda tugasa, reaksiyaning temperatura koeffitsiyentini toping.\\
A) 1\\
B) 2\\
C) 3\\
D) 4
122. $50^{\circ} \mathrm{C}$ da reaksiya 10 sekundda $20^{\circ} \mathrm{C}$ da esa 80 sekundda tugasa, reaksiyaning temperatura koeffitsiyentini toping.\\
A) 1\\
B) 2\\
C) 3\\
D) 4\\
123. $70^{\circ} \mathrm{C}$ da reaksiya 1 sekundda $40^{\circ} \mathrm{C}$ da esa 64 sekundda tugasa, reaksiyaning temperatura koeffitsiyentini toping.\\
A) 1\\
B) 2\\
C) 3\\
D) 4\\
124. $45^{\circ} \mathrm{C}$ da reaksiya 10 sekundda $25^{\circ} \mathrm{C}$ da esa 40 sekundda tugasa, reaksiyaning temperatura koeffitsiyentini toping.\\
A) 1\\
B) 2\\
C) 3\\
D) 4\\
125. $50^{\circ} \mathrm{C}$ da reaksiya 5 sekundda $30^{\circ} \mathrm{C}$ da esa 125 sekundda tugasa, reaksiyaning temperatura koeffitsiyentini toping.\\
A) 5\\
B) 4\\
C) 3\\
D) 2\\
126. $80^{\circ} \mathrm{C}$ da reaksiya 10 sekundda $40^{\circ} \mathrm{C}$ da esa 160 sekundda tugasa, reaksiyaning temperatura koeffitsiyentini toping.\\
A) 5\\
B) 4\\
C) 3\\
D) 2\\
127. $10^{\circ} \mathrm{C}$ da reaksiya 10 sekundda $-20^{\circ} \mathrm{C}$ da esa 80 sekundda tugasa, reaksiyaning temperatura koeffitsiyentini toping.\\
A) 1\\
B) 2\\
C) 3\\
D) 4\\
128. $50^{\circ} \mathrm{C}$ da reaksiya 10 sekundda $10^{\circ} \mathrm{C}$ da esa 810 sekundda tugasa, reaksiyaning temperatura koeffitsiyentini toping.\\
A) 5\\
B) 2\\
C) 3\\
D) 4\\
129. $20^{\circ} \mathrm{C}$ da reaksiya 10 sekundda $-20^{\circ} \mathrm{C}$ da esa 160 sekundda tugasa, reaksiyaning temperatura koeffitsiyentini toping.\\
A) 1\\
B) 2\\
C) 3\\
D) 4\\
130. $40^{\circ} \mathrm{C}$ da reaksiya 1 sekundda $0^{\circ} \mathrm{C}$ da esa 256 sekundda tugasa, reaksiyaning temperatura koeffitsiyentini toping.\\
A) 5\\
B) 4\\
C) 3\\
D) 2
  \item Agarda reaksiya tezligi $70^{\circ} \mathrm{C}$ da 90 $\mathrm{mol} / \mathrm{l} \cdot \min \mathrm{ga} 50^{\circ} \mathrm{C}$ da esa $10 \mathrm{~mol} / \mathrm{l} \cdot \min$ ga teng bo'lsa, reaksiyaning temperatura koeffitsiyentini toping.\\
A) 1\\
B) 2\\
C) 3\\
D) 4\\
  \item Agarda reaksiya tezligi $50^{\circ} \mathrm{C}$ da 80 $\mathrm{mol} / \mathrm{l} \cdot \min$ ga $20^{\circ} \mathrm{C}$ da esa $10 \mathrm{~mol} / \mathrm{l} \cdot \min$ ga teng bo'lsa, reaksiyaning temperatura koeffitsiyentini toping.\\
A) 5\\
B) 2\\
C) 3\\
D) 4
  \item Agarda reaksiya tezligi $10^{\circ} \mathrm{C}$ da 40 $\mathrm{mol} / \mathrm{l} \cdot \min$ ga $\cdot 10^{\circ} \mathrm{C}$ da esa $10 \mathrm{~mol} / \cdot \cdot \min$ ga teng bo'lsa, reaksiyaning temperatura koeffitsiyentini toping.\\
A) 2,5\\
B) 2\\
C) 3\\
D) 4
  \item Agarda reaksiya tezligi $20^{\circ} \mathrm{C}$ da 80 $\mathrm{mol} / \mathrm{l} \cdot \min$ ga $10^{\circ} \mathrm{C}$ da esa $20 \mathrm{~mol} / \mathrm{l} \cdot \min$ ga teng bo'lsa, reaksiyaning temperatura koeffitsiyentini toping.\\
A) 1\\
B) 2\\
C) 3\\
D) 4
  \item Agarda reaksiya tezligi $80^{\circ} \mathrm{C}$ da 25 $\mathrm{mol} / 1 \cdot \min$ ga $70^{\circ} \mathrm{C}$ da esa $10 \mathrm{~mol} / 1 \cdot \min$ ga teng bo'lsa, reaksiyaning temperatura koeffitsiyentini toping.\\
A) 2,5\\
B) 2\\
C) 3\\
D) 4
  \item Agarda reaksiya tezligi $70^{\circ} \mathrm{C}$ da 25 $\mathrm{mol} / \sqrt{ } \cdot \min \mathrm{ga} 60^{\circ} \mathrm{C}$ da esa $5 \mathrm{~mol} / \cdot \min \mathrm{ga}$ teng bo'lsa, reaksiyaning temperatura koeffitsiyentini toping.\\
A) 5\\
B) 4\\
C) 3\\
D) 2
  \item Agarda reaksiya tezligi $45^{\circ} \mathrm{C}$ da 80 $\mathrm{mol} / \mathrm{l} \cdot \min \mathrm{ga} 25^{\circ} \mathrm{C}$ da esa $5 \mathrm{~mol} / \mathrm{l} \cdot \min \mathrm{ga}$ teng bo'lsa, reaksiyaning temperatura koeffitsiyentini toping.\\
A) 1\\
B) 2\\
C) 3\\
D) 4
  \item Agarda reaksiya tezligi $40^{\circ} \mathrm{C}$ da 27 $\mathrm{mol} /{ }^{\cdot} \cdot \min$ ga $10^{\circ} \mathrm{C}$ da esa $1 \mathrm{~mol} / \cdot \cdot \min$ ga teng bo'lsa, reaksiyaning temperatura koeffitsiyentini toping.\\
A) 1\\
B) 2\\
C) 3\\
D) 4
  \item Agarda reaksiya tezligi $100^{\circ} \mathrm{C}$ da 50 $\mathrm{mol} / \cdot \min$ ga $90^{\circ} \mathrm{C}$ da esa $10 \mathrm{~mol} / \cdot \min$ ga teng bo'lsa, reaksiyaning temperatura koeffitsiyentini toping.\\
A) 5\\
B) 4\\
C) 3\\
D) 2
  \item Agarda reaksiya tezligi $0^{\circ} \mathrm{C}$ da 16 $\mathrm{mol} / \cdot \min \mathrm{ga}-20^{\circ} \mathrm{C}$ da esa $1 \mathrm{~mol} / \cdot \min \mathrm{ga}$ teng bo'lsa, reaksiyaning temperatura koeffitsiyentini toping.\\
A) 1\\
B) 2\\
C) 3\\
D) 4
\item $20^{\circ} \mathrm{C}$ da reaksiya 15 minut davom etadi. Reaksiya davomiyligi 5 minut bo'lishi uchun harorat necha gradusgacha oshirilishi kerak? $(\mathrm{Y}=3)$\\
A) 40\\
B) 60\\
C) 20\\
D) 30
  \item $10^{\circ} \mathrm{C}$ da reaksiya 80 minut davom etadi. Reaksiya davomiyligi 10 minut bo'lishi uchun harorat necha gradusgacha oshirilishi kerak? $\quad(\mathrm{Y}=2)$\\
A) 40\\
B) 60\\
C) 20\\
D) 30
  \item $20^{\circ} \mathrm{C}$ da reaksiya 160 minut davom etadi. Reaksiya davomiyligi 10 minut bo'lishi uchun harorat necha gradusgacha oshirilishi kerak? $\quad(\mathrm{Y}=2)$\\
A) 40\\
B) 60\\
C) 20\\
D) 30
  \item $20^{\circ} \mathrm{C}$ da reaksiya 40 minut davom etadi. Reaksiya davomiyligi 10 minut bo'lishi uchun harorat necha gradusgacha oshirilishi kerak? $\quad(\mathrm{Y}=4)$\\
A) 40\\
B) 60\\
C) 20\\
D) 30
  \item $20^{\circ} \mathrm{C}$ da reaksiya 125 minut davom etadi. Reaksiya davomiyligi 5 minut bo'lishi uchun harorat necha gradusgacha oshirilishi kerak? $\quad(\mathrm{Y}=5)$\\
A) 40\\
B) 60\\
C) 20\\
D) 30
  \item $70^{\circ} \mathrm{C}$ da reaksiya 45 minut davom etadi. Reaksiya davomiyligi 5 minut bo'lishi uchun harorat neche gradusgacha oshirilishi kerak? $\quad(\mathrm{Y}=3)$\\
A) 50\\
B) 60\\
C) 90\\
D) 80
  \item $-10{ }^{\circ} \mathrm{C}$ da reaksiya 25 minut davom etadi. Reaksiya davomiyligi 1 minut bo'lishi uchun harorat necha gradusgacha oshirilishi kerak? $\quad(\gamma=5)$\\
A) 40\\
B) 10\\
C) 20\\
D) 30
  \item $50^{\circ} \mathrm{C}$ da reaksiya 28 minut davom etadi. Reaksiya davomiyligi 7 minut bo'lishi uchun harorat necha gradusgacha oshirilishi kerak? $\quad(\gamma=2)$\\
A) 70\\
B) 80\\
C) 90\\
D) 100
  \item $20^{\circ} \mathrm{C}$ da reaksiya 64 minut davom etadi. Reaksiya davomiyligi 8 minut bo'lishi uchun harorat necha gradusgacha oshirilishi kerak? $\quad(\mathrm{Y}=2)$\\
A) 50\\
B) 60\\
C) 70\\
D) 80
  \item $20^{\circ} \mathrm{C}$ da reaksiya 60 minut davom etadi. Reaksiya davomiyligi 15 minut bo"lishi uchun harorat necha gradusgacha oshirilishi kerak? $\quad(\mathrm{\gamma}=4)$\\
A) 40\\
B) 60\\
C) 20\\
D) 30
  \item $\mathrm{H}_{2(\mathrm{~g})}+\mathrm{Cl}_{2(\mathrm{~g})} \leftrightarrow 2 \mathrm{HCl}_{(\mathrm{g})}$ ushbu reaksiya sistemasiga $\mathrm{H}_{2}$ gazi qo'shilsa, muvozanat qaysi tomonga siljiydi?\\
A) o'ngga\\
B) chapga\\
C) siljimaydi\\
D) ikki tomonga bír xil siljiydi
2. $\mathrm{H}_{2(\mathrm{~g})}+\mathrm{Cl}_{2(\mathrm{~g})} \leftrightarrow 2 \mathrm{HCl}_{(\mathrm{g})}$ ushbu reaksiya sistemasiga $\mathrm{Cl}_{2}$ gazi qo'shilsa, muvozanat qaysi tomonga siljiydi?\\
A) o'ngga\\
B) chapga\\
C) siljimaydi\\
D) ikki tomonga bir xil siljiydi\\
3. $\mathrm{H}_{2(\mathrm{~g})}+\mathrm{Cl}_{2(\mathrm{~g})} \leftrightarrow 2 \mathrm{HCl}_{(\mathrm{g})}$ ushbu reaksiya sistemasiga HCl gazi qo'shilsa, muvozanat qaysi tomonga siljiydi?\\
A) o'ngga\\
B) chapga\\
C) siljimaydi\\
D) ikki tomonga bir xil siljiydi\\
4. $\mathrm{H}_{2(\mathrm{~g})}+\mathrm{Cl}_{2(\mathrm{~g})} \leftrightarrow 2 \mathrm{HCl}_{(\mathrm{g})}$ ushbu reaksiya sistemasdan $\mathrm{H}_{2}$ gazi chiqarilib yuborilsa, muvozanat qaysi tomonga siljiydi?\\
A) o'ngga\\
B) chapga\\
C) siljimaydi\\
D) ikki tomonga bir xil siljiydi\\
5. $\mathrm{H}_{2(\mathrm{~g})}+\mathrm{Cl}_{2(\mathrm{~g})} \leftrightarrow 2 \mathrm{HCl}_{(\mathrm{g})}$ ushbu reaksiya sistemasdan $\mathrm{Cl}_{2}$ gazi chiqarilib yuborilsa, muvozanat qaysi tomonga siljiydi?\\
A) o'ngga\\
B) chapga\\
C) siljimaydi\\
D) ikki tomonga bir xil siljiydi\\
6. $\mathrm{H}_{2(\mathrm{~g})}+\mathrm{Cl}_{2(\mathrm{~g})} \leftrightarrow 2 \mathrm{HCl}_{(\mathrm{g})}$ ushbu reaksiya sistemasdan HCl gazi chiqarilib yuborilsa, muvozanat qaysi tomonga siljiydi?\\
A) o'ngga\\
B) chapga\\
C) siljimaydi\\
D) ikki tomonga bir xil siljiydi\\
7. $\mathrm{Fe}\left(_{(\mathrm{q})}+\mathrm{Cl}_{2(\mathrm{~g})} \leftrightarrow \mathrm{FeCl}_{3(\mathrm{q})}\right.$ ushbu reaksiya sistemasiga $\mathrm{Cl}_{2}$ gazi qo'shilsa, muvozanat qaysi tomonga siljiydi?\\
A) o'ngga\\
B) chapga\\
C) siljimaydi 
D) ikki tomonga bir xil siljiydi\\
8. $\mathrm{Fe}_{(\mathrm{q})}+\mathrm{Cl}_{2(\mathrm{~g})} \leftrightarrow \mathrm{FeCl}_{3(\mathrm{q})}$ ushbu reaksiya sistemasiga Fe qo'shilsa, muvozanat qaysi tomonga siljiydi?\\
A) o'ngga\\
B) chapga\\
C) siljimaydi\\
D) ikki tomonga bir xil siljiydi\\
9. $\mathrm{Fe}(\mathrm{q})+\mathrm{Cl}_{2(\mathrm{~g})} \leftrightarrow \mathrm{FeCl}_{3(\mathrm{q})}$ ushbu reaksiya sistemasiga $\mathrm{FeCl}_{3}$ qo'shilsa, muvozanat qaysi tomonga siljiydi?\\
A) o'ngga\\
B) chapga\\
C) siljimaydi\\
D) ikki tomonga bir xil siljiydi\\
10. $\mathrm{Fe}_{(\mathrm{q})}+\mathrm{Cl}_{2(\mathrm{~g})} \leftrightarrow \mathrm{FeCl}_{3(\mathrm{q})}$ ushbu reaksiya sistemasidan $\mathrm{FeCl}_{3}$ chiqarilib yuborilsa, muvozanat qaysi tomonga siljiydi?\\
A) o'ngga\\
B) chapga\\
C) siljimaydi\\
D) ikki tomonga bir xil siljiydi
  \item $\mathrm{A}_{(\mathrm{g})}+\mathrm{B}_{(\mathrm{g})} \leftrightarrow \mathrm{C}_{(\mathrm{g})}$ ushbu reaksiya sistemasiga bosimni oshirsak muvozanat qaysi tomonga siljiydi?\\
A) o'ngga\\
B) chapga\\
C) siljimaydi\\
D) ikki tomonga bir xil siljiydi
12. $\mathrm{N}_{2(\mathrm{~g})}+\mathrm{H}_{2(\mathrm{~g})} \leftrightarrow \mathrm{NH}_{3(\mathrm{~g})}$ ushbu reaksiya sistemasiga bosimni oshirsak muvozanat qaysi tomonga siljiydi?\\
A) o'ngga\\
B) chapga\\
C) siljimaydi\\
D) ikki tomonga bir xil siljiydi\\
13. $\mathrm{NH}_{3(\mathrm{~g})} \leftrightarrow \mathrm{N}_{2(\mathrm{~g})}+\mathrm{H}_{2(\mathrm{~g})}$ ushbu reaksiya sistemasiga bosimni oshirsak muvozanat qaysi tomonga siljiydi?\\
A) o'ngga\\
B) chapga\\
C) siljimaydi\\
D) ikki tomonga bir xil siljiydi\\
16. $\mathrm{NH}_{3(\mathrm{~g})} \leftrightarrow \mathrm{N}_{2(\mathrm{~g})}+\mathrm{H}_{2(\mathrm{~g})}$ ushbu reaksiya sistemasiga bosimni kamaytirsak muvozanat qaysi tomonga siljiydi. siljiydi?\\
A) o'ngga\\
B) chapga\\
C) siljimaydi\\
D) ikki tomonga bir xil siljiydi\\
17. $\mathrm{N}_{2(\mathrm{~g})}+\mathrm{H}_{2(\mathrm{~g})} \leftrightarrow \mathrm{NH}_{3(\mathrm{~g})}$ ushbu reaksiya sistemasiga bosimni kamaytirsak muvozanat qaysi tomonga siljiydi?\\
A) o'ngga\\
B) chapga\\
C) siljimaydi\\
D) ikki tomonga bir xil siljiydi\\
18. $\mathrm{Fe}_{(\mathrm{q})}+\mathrm{Cl}_{2(\mathrm{~g})} \leftrightarrow \mathrm{FeCl}_{3(\mathrm{q})}$ ushbu reaksiya sistemasiga bosimni oshirsak muvozanat qaysi tomonga siljiydi?\\
A) o'ngga\\
B) chapga\\
C) siljimaydi\\
D) ikki tomonga bir xil siljiydi\\
19. $\mathrm{A}_{(\mathrm{g})}+\mathrm{B}_{(\mathrm{g})} \leftrightarrow 2 \mathrm{C}_{(\mathrm{g})}$ ushbu reaksiya sistemasiga bosimni oshirsak muvozanat qaysi tomonga siljiydi?\\
A) o'ngga\\
B) chapga\\
C) siljimaydi\\
D) ko'proq o'ngga\\
20. $2 \mathrm{~A}_{(\mathrm{g})}+3 \mathrm{~B}_{(\mathrm{g})} \leftrightarrow \mathrm{C}_{(\mathrm{g})}$ ushbu reaksiya sistemasiga bosimni kamaytirsak muvozanat qaysi tomonga siljiydi?\\
A) o'ngga\\
B) chapga\\
C) siljimaydi\\
D) ikki tomonga bir xil siljiydi
  \item $\mathrm{A}_{(\mathrm{g})}+\mathrm{B}_{(\mathrm{g})} \leftrightarrow \mathrm{C}_{(\mathrm{g})}$ ushbu reaksiya sistemasiga reaksiya o'tkazilayotgan idish hajmini oshirsak muvozanat qaysi tomonga siljiydi?\\
A) o'ngga\\
B) chapga\\
C) siljimaydi\\
D) ikki tomonga bir xil siljiydi
22. $\mathrm{N}_{2(\mathrm{~g})}+\mathrm{H}_{2(\mathrm{~g})} \leftrightarrow \mathrm{NH}_{3(\mathrm{~g})}$ ushbu reaksiya sistemasiga reaksiya o'tkazilayotgan idish hajmini oshirsak muvozanat qaysi tomonga siljiydi?\\
A) o'ngga\\
B) chapga\\
C) siljimaydi\\
D) ikki tomonga bir xil siljiydi\\
23. $\mathrm{NH}_{3(\mathrm{~g})} \leftrightarrow \mathrm{N}_{2(\mathrm{~g})}+\mathrm{H}_{2(\mathrm{~g})}$ ushbu reaksiya sistemasiga reaksiya o'tkazilayotgan idish hajmini oshirsak muvozanat qaysi tomonga siljiydi?\\
A) o'ngga\\
B) chapga\\
C) siljimaydi\\
D) ikki tomonga bir xil siljiydi\\
24. $\mathrm{H}_{2(\mathrm{~g})}+\mathrm{Cl}_{2(\mathrm{~g})} \leftrightarrow 2 \mathrm{HCl}_{(\mathrm{g})}$ ushbu reaksiya sistemasiga reaksiya o'tkazilayotgan idish hajmini oshirsak muvozanat qaysi tomonga siljiydi?\\
A) o'ngga\\
B) chapga\\
C) siljimaydi\\
D) ko'proq o'ngga\\
25. $\mathrm{A}_{(\mathrm{g})}+\mathrm{B}_{(\mathrm{g})} \leftrightarrow \mathrm{C}_{(\mathrm{g})}$ ushbu reaksiya sistemasiga reaksiya o'tkazilayotgan idish hajmini kamaytirsak muvozanat qaysi tomonga siljiydi?\\
A) o'ngga\\
B) chapga\\
C) siljimaydi\\
D) ikki tomonga bir xil siljiydi\\
26. $\mathrm{NH}_{3(\mathrm{~g})} \leftrightarrow \mathrm{N}_{2(\mathrm{~g})}+\mathrm{H}_{2(\mathrm{~g})}$ ushbu reaksiya sistemasiga reaksiya o'tkazilayotgan idish hajmini kamaytirsak muvozanat qaysi tomonga siljiydi?\\
A) o'ngga\\
B) chapga\\
C) siljimaydi\\
D) ikki tomonga bir xil siljiydi\\
27. $\mathrm{N}_{2(\mathrm{~g})}+\mathrm{H}_{2(\mathrm{~g})} \leftrightarrow \mathrm{NH}_{3(\mathrm{~g})}$ ushbu reaksiya sistemasiga reaksiya o'tkazilayotgan idish hajmini kamaytirsak muvozanat qaysi tomonga siljiydi?\\
A) o'ngga\\
B) chapga\\
C) siljimaydi\\
D) ikki tomonga bir xil siljiydi\\
29. $\mathrm{A}_{(\text {g })}+\mathrm{B}_{(\text {k })} \leftrightarrow 2 \mathrm{C}_{(\text {g })}$ ushbu ronksiyn sistemasiga reaksiya o'tkazilayotgan idish hajmini oshirsak muvozanat quysi tomonge siljiydi?\\
A) o'ngga\\
B) chapga\\
C) siljimaydi\\
D) ko'proq o'ugga\\
30. $2 \mathrm{~A}_{(k)}+3 \mathrm{~B}_{(k)} \leftrightarrow \mathrm{C}_{(k)}$ ushbu reaksiya sistemasiga reaksiya o'tknzilayotgan idish hajmini kamaytirsak muvozanat qaysi tomonga siljiydi?\\
A) o'ngga\\
B) chapga\\
C) siljimaydi\\
D) ikki tomonga bir xil siljiydi
  \item $2 \mathrm{~A}_{(\mathrm{g})}+\mathrm{B}_{(\mathrm{g})} \leftrightarrow \mathrm{C}_{(\mathrm{g})}+3 \mathrm{D}_{(\mathrm{g})}$ ushbu reaksiya bo'yicha moddalarning muvozanat konsentratsiyalari mos ravishda $2,4,4,2$ $\mathrm{mol} / \mathrm{l}$ dan bo'lsa, muvozanat konstantasini toping.\\
A) 2\\
B) 3\\
C) 1\\
D) 4
42. $\mathrm{N}_{2(\mathrm{~g})}+\mathrm{H}_{2(\mathrm{~g})} \leftrightarrow \mathrm{NH}_{3(\mathrm{~g})}$ ushbu reaksiya bo'yicha moddalarning muvozanat konsentratsiyalari mos ravishda $2,2,4$ $\mathrm{mol} / \mathrm{l}$ dan bo'lsa, muvozanat konstantasini toping.\\
A) 2\\
B) 3\\
C) 1\\
D) 4\\
43. $\mathrm{H}_{2(\alpha)}+\mathrm{Cl}_{2(\alpha)} \leftrightarrow \mathrm{HCl}_{(\alpha)}$ ushbu reaksiya bo'yicha moddalarning muvozanat konsentratsiyalari mos ravishdu 2, 1, 1 mol/l dan bo'lsa, muvozanat konstantarini toping.\\
A) 2\\
B) 3\\
C) 1\\
D) 4\\
44. $\mathrm{CO}_{(\mathrm{g})}+\mathrm{H}_{2} \mathrm{O}_{(\mathrm{k})} \leftrightarrow \mathrm{CO}_{2(\mathrm{k})}+\mathrm{H}_{2(\mathrm{~h})}$ ushbu reaksiya bo'yicha moddalarning muvozanat konsentratsiyalari mos ravishda $1,4,1,4 \mathrm{~mol} / \mathrm{l}$ dan bo'lsa, muvozanat konstantasini toping.\\
A) 2\\
B) 3\\
C) 1\\
D) 4\\
45. $\mathrm{CO}_{(\mathrm{g})}+\mathrm{Cl}_{2(\mathrm{~g})} \leftrightarrow \mathrm{COCl}_{2(\mathrm{~g})}$ ushbu reaksiya bo'yicha moddalarning muvozanat konsentratsiyalari mos ravishda $1,1,2$ $\mathrm{mol} / \mathrm{l}$ dan bo'lsa, muvozanat konstantasini toping.\\
A) 2\\
B) 3\\
C) 1\\
D) 4\\
46. $\mathrm{SO}_{2(\mathrm{~g})}+\mathrm{O}_{2(\mathrm{~g})} \leftrightarrow \mathrm{SO}_{3(\mathrm{~g})}$ ushbu reaksiya bo'yicha moddalarning muvozanat. konsentratsiyalari mos ravishda $2,1,2$ $\mathrm{mol} / \mathrm{l}$ dan bo'lsa, muvozanat konstantasini toping.\\
A) 2\\
B) 3\\
C) 1\\
D) 4\\
47. $\mathrm{Fe}_{(\mathrm{q})}+\mathrm{H}_{2} \mathrm{O}_{(\mathrm{g})} \leftrightarrow \mathrm{Fe}_{3} \mathrm{O}_{4(\mathrm{q})}+\mathrm{H}_{2(\mathrm{~g})}$ ushbu reaksiya bo'yicha $\mathrm{H}_{2} \mathrm{O}$ va $\mathrm{H}_{2}$ ning muvozanat konsentratsiyalari mos ravishda $2,4 \mathrm{~mol} / \mathrm{l}$ dan bo'lsa, muvozanat konstantasini toping.\\
A) 14\\
B) 15\\
C) 17\\
D) 16\\
48. $\mathrm{NO}_{2(\mathrm{~g})} \leftrightarrow \mathrm{N}_{2} \mathrm{O}_{4(\mathrm{~g})}$ ushbu reaksiya bo'yicha moddalarning muvozanat konsentratsiyalari mos ravishda $3,9 \mathrm{~mol} / \mathrm{l}$ dan bo'lsa, muvozanat konstantasini toping.\\
A) 2\\
B) 3\\
C) 1\\
D) 4\\
49. $\mathrm{A}_{(\mathrm{g})}+\mathrm{B}_{(\mathrm{g})} \leftrightarrow \mathrm{C}_{(\mathrm{g})}+\mathrm{D}_{(\mathrm{g})}$ ushbú reaksiya bo'yicha moddalarning muvozanat konsentratsiyalari mos ravishda $2,2,5,2$ $\mathrm{mol} / \mathrm{l}$ dan bo'lsa, muvozanat konstantasini toping.\\
A) 2\\
B) 2,5\\
C) 1,5\\
D) 4\\
50. $\mathrm{CH}_{4(\mathrm{~g})}+\mathrm{O}_{2(\mathrm{~g})} \leftrightarrow \mathrm{CO}_{2(\mathrm{~g})}+\mathrm{H}_{2} \mathrm{O}_{(\mathrm{g})}$ ushbu reaksiya bo'yicha moddalarning muvozanat konsentratsiyalari mos\\
ravishda $2,4,4,2 \mathrm{~mol} / \mathrm{l}$ dan bo'lsa, muvozanat konstantasini toping.\\
A) 2\\
B) 3\\
C) 1\\
D) 4
  \item $2 \mathrm{~A}_{(\mathrm{g})}+\mathrm{B}_{(\mathrm{k})} \leftrightarrow \mathrm{C}_{(\mathrm{g})}+3 \mathrm{D}_{(\mathrm{g})}$ ushbu reaksiya bo'yicha moddalarning muvoząnat holatdagi miqdorlari mos ravishda $2,4,4$, 2 mol dan bo'lsa, muvozanat konstantasini toping.\\ Reaktor hajmi 2 litrga teng.\\
A) 2\\
B) 3\\
C) 1\\
D) 4\\
  \item $\mathrm{N}_{2(\mathrm{~g})}+\mathrm{H}_{2(\mathrm{~g})} \leftrightarrow \mathrm{NH}_{3(\mathrm{~g})}$ ushbu reaksiya bo'yicha moddalarning muvozanat holatdagi miqdorlari mos ravishda $3,3,3$ mol dan bo'lsa, muvozanat konstantasini toping.\\Reaktor hajmi 3 litrga teng.\\
A) 2\\
B) 3\\
C) 1\\
D) 4
  \item $\mathrm{H}_{2(\mathrm{~g})}+\mathrm{Cl}_{2(\mathrm{~g})} \leftrightarrow \mathrm{HCl}_{(\mathrm{g})}$ ushbu reaksiya bo'yicha moddalarning muvozanat holatdagi miqdorlari mos ravishda $2,4,4$ mol dan bo'lsa, muvozanat konstantasini toping.\\
Reaktor hajmi 2 litrga teng.\\
A) 2\\
B) 3\\
C) 1\\
D) 4
  \item $\mathrm{CO}_{(\mathrm{g})}+\mathrm{H}_{2} \mathrm{O}_{(\mathrm{g})} \leftrightarrow \mathrm{CO}_{2(\mathrm{~g})}+\mathrm{H}_{2(\mathrm{~g})}$ ushbu reaksiya bo'yicha moddalarning muvozanat holatdagi miqdorlari mos ravishda $5,25,10,20 \mathrm{~mol}$ dan bo'lsa, muvozanat konstantasini toping. Reaktor hajmi 5 litrga teng.\\
A) 2,5\\
B) 3\\
C) 1,2\\
D) 1,6
  \item $\mathrm{CO}_{(\mathrm{g})}+\mathrm{Cl}_{2(\mathrm{~g})} \leftrightarrow \mathrm{COCl}_{2(\mathrm{~g})}$ ushbu reaksiya bo'yicha moddalarning muvozanat holatdagi miqdorlari mos ravishda $6,3,18$, mol dan bo'lsa, muvozanat konstantasini toping.\\
Reaktor hajmi 3 litrga teng.\\
A) 2\\
B) 3\\
C) 1\\
D) 4
  \item $\mathrm{SO}_{2(\mathrm{~g})}+\mathrm{O}_{2(\mathrm{~g})} \leftrightarrow \mathrm{SO}_{3(\mathrm{~g})}$ ushbu reaksiya bo'yicha moddalarning muvozanat\\
holatdagi migdorlari mos ravishda $8,8,16$, mol dan bo'lsa, muvozanat konstantasini toping.\\
Reaktor hajmi 4 litrga teng.\\
A) 2\\
B) 3\\
C) 1\\
D) 4
  \item $\mathrm{Fe}_{(q)}+\mathrm{H}_{2} \mathrm{O}_{(q)} \leftrightarrow \mathrm{Fe}_{3} \mathrm{O}_{4(q)}+\mathrm{H}_{2(\xi)}$ ushbu reaksiya bo'yicha $\mathrm{H}_{2} \mathrm{O}$ va $\mathrm{H}_{2}$ moddalarning muvozanat holatdagi miqdorlari mos ravishda $4,4 \mathrm{~mol}$ dan bo'lsa, muvozanat konstantasini toping. Reaktor hajmi 2 litrga teng.\\
A) 2\\
B) 3\\
C) 1\\
D) 4
  \item $\mathrm{NO}_{2(g)} \leftrightarrow \mathrm{N}_{2} \mathrm{O}_{4(g)}$ ushbu reaksiya bo'yícha moddalarning muvozanat holatdagi miqdorlari mos ravishda 4,10 , mol dan bo'lsa, muvozanat konstantasini toping. Reaktor hajmi 4 litrga teng.\\
A) 2,5\\
B) 3\\
C) 1,5\\
D) 4
  \item $\mathrm{A}_{(\mathrm{g})}+\mathrm{B}_{(\mathrm{g})} \leftrightarrow \mathrm{C}_{(\mathrm{g})}+\mathrm{D}_{(\mathrm{k})}$ ushbu reaksiya bo'yicha moddalarning muvozanat holatdagi miqdorlari mos ravishda $2,4,4$, 2 mol dan bo'lsa, muvozanat konstantasini toping.\\
Reaktor hajmi 1 litrga teng.\\
A) 2\\
B) 3\\
C) 1\\
D) 4
  \item $\mathrm{CH}_{4(\mathrm{~g})}+\mathrm{O}_{2(\mathrm{~g})} \leftrightarrow \mathrm{CO}_{2(\mathrm{~g})}+\mathrm{H}_{2} \mathrm{O}_{(\mathrm{g})}$ ushbu reaksiya bo'yicha moddalarning muvozanat holatdagi miqdorlari mos ravishda $8,8,8,8 \mathrm{~mol}$ dan bo'lsa, muvozanat konstantasini toping. Reaktor hajmi 4 litrga teng.\\
A) 2\\
B) 3\\
C) 1\\
D) 4
  \item $2 \mathrm{~A}_{(\mathrm{g})}+\mathrm{B}_{(\mathrm{g})} \leftrightarrow \mathrm{C}_{(\mathrm{g})}+3 \mathrm{D}_{(\mathrm{g})}$ ushbu reaksiya bo'yicha moddalarning muvozanat konsentratsiyalari mos ravishda $2,4,4,12$ $\mathrm{mol} / \mathrm{l}$ dan bo'lsa, A va B ning dastlabki konsentratsiyalarini toping.\\
A) $10 ; 8$\\
B) $6 ; 4$\\
C) $5 ; 5$\\
D) $4 ; 4$
  \item $\mathrm{N}_{2(\mathrm{~g})}+\mathrm{H}_{2(\mathrm{~g})} \leftrightarrow \mathrm{NH}_{3(\mathrm{~g})}$ ushbu reaksiya boyicha moddalarning muvozanat konsentratsiyalari mos ravishda 2, 2, 4 $\mathrm{mol} / \mathrm{l}$ dan bo'lsa, $\mathrm{N}_{2}$ va $\mathrm{H}_{2}$ ning dastlabki konsentratsiyalarini toping.\\
A) $2 ; 4$\\
B) $3 ; 6$\\
C) $1 ; 3$\\
D) $4 ; 8$
  \item $\mathrm{H}_{2(\mathrm{~g})}+\mathrm{Cl}_{2(\mathrm{~g})} \leftrightarrow \mathrm{HCl}_{(\mathrm{g})}$ ushbu reaksiya bo'yicha moddalarning muvozanat konsentratsiyalari mos ravishda $4,4,4$ $\mathrm{mol} / \mathrm{l}$ dan bo'lsa, $\mathrm{Cl}_{2}$ va $\mathrm{H}_{2}$ ning dastlabki konsentratsiyalarini toping.\\
A) $5: 5$\\
B) $7 ; 7$\\
C) $6 ; 6$\\
D) $6 ; 5$
  \item $\mathrm{CO}_{(\mathrm{g})}+\mathrm{H}_{2} \mathrm{O}_{(\mathrm{g})} \leftrightarrow \mathrm{CO}_{2(\mathrm{~g})}+\mathrm{H}_{2(\mathrm{~g})}$ ushbu reaksiya bo'yicha moddalarning muvozanat konsentratsiyalari mos ravishda $1,4,4,4 \mathrm{~mol} / \mathrm{l}$ dan bo'lsa, CO va $\mathrm{H}_{2} \mathrm{O}$ ning dastlabki konsentratsiyalarini toping.\\
A) $5 ; 8$\\
B) $5 ; 5$\\
C) $4 ; 6$\\
D) $6 ; 6$
  \item $\mathrm{CO}_{(\mathrm{g})}+\mathrm{Cl}_{2(\mathrm{~g}) \leftrightarrow} \mathrm{COCl}_{2(\mathrm{~g})}$ ushbu reaksiya bo'yicha moddalarning muvozanat konsentratsiyalari mos ravishda $1,1,2$ $\mathrm{mol} / 1$ dan bo'lsa, CO va $\mathrm{Cl}_{2}$ ning dastlabki konsentratsiyalarini toping.\\
A) $2 ; 2$\\
B) $3 ; 3$\\
C) $4 ; 4$\\
D) $5 ; 5$
  \item $\mathrm{SO}_{2(\mathrm{~g})}+\mathrm{O}_{2(\mathrm{~g})} \leftrightarrow \mathrm{SO}_{3(\mathrm{~g})}$ ushbu reaksiya bo'yicha moddalarning muvozanat konsentratsiyalari mos ravishda $2,1,2$ $\mathrm{mol} / \mathrm{l}$ dan bo'lsa, $\mathrm{SO}_{2}$ va $\mathrm{O}_{2}$ ning dastlabki konsentratsiyalarini toping.\\
A) $2 ; 4$\\
B) $3 ; 4$\\
C) $1 ; 2$\\
D) $4 ; 2$
  \item $\mathrm{Fe}_{(\mathrm{q})}+\mathrm{H}_{2} \mathrm{O}_{(\mathrm{g})} \leftrightarrow \mathrm{Fe}_{3} \mathrm{O}_{4(\mathrm{q})}+\mathrm{H}_{2(\mathrm{~g})}$ ushbu reaksiya bo'yicha $\mathrm{H}_{2} \mathrm{O}$ va $\mathrm{H}_{2}$ ning muvozanat konsentratsiyalari mos ravishda $2,4 \mathrm{~mol} / \mathrm{l}$ dan bo'lsa, $\mathrm{H}_{2} \mathrm{O}$ ning dastlabki konsentratsiyasini toping.\\
A) 4\\
B) 5\\
C) 7\\
D) 6
  \item $\mathrm{NO}_{2(\mathrm{~g})} \leftrightarrow \mathrm{N}_{2} \mathrm{O}_{4(\mathrm{~g})}$ ushbu reaksiya bo'yicha moddalarning muvozanat konsentratsiyalari mos ravishda $3,4 \mathrm{~mol} / \mathrm{l}$ dan bo'lsa, $\mathrm{NO}_{2}$ ning dastlabki konsentratsiyasini toping.\\
A) 12\\
B) 13\\
C) 11\\
D) 14
  \item $\mathrm{A}_{(\mathrm{g})}+\mathrm{B}_{(\mathrm{g})} \leftrightarrow \mathrm{C}_{(\mathrm{g})}+\mathrm{D}_{(\mathrm{g})}$ ushbu reaksiya bo'yicha moddalarning muvozanat konsentratsiyalari mos ravishda $2,2,5,5$ mol/l dan bo'lsa, A va B ning dastlabki konsentratsiyalarini toping.\\
A) $6 ; 6$\\
B) $5 ; 5$\\
C) $7 ; 7$\\
D) $4 ; 4$
  \item $\mathrm{CH}_{4(\mathrm{~g})}+\mathrm{O}_{2(\mathrm{~g})} \leftrightarrow \mathrm{CO}_{2(\mathrm{~g})}+\mathrm{H}_{2} \mathrm{O}_{(\mathrm{g})}$ ushbu reaksiya bo'yicha moddalarning muvozanat konsentratsiyalari mos ravishda $2,4,2,4 \mathrm{~mol} / \mathrm{l}$ dan bo'lsa, $\mathrm{CH}_{4}$ va $\mathrm{O}_{2}$ ning dastlabki konsentratsiyalarini toping.\\
A) $2 ; 4$\\
B) $3 ; 6$\\
C) $1 ; 6$\\
D) $4 ; 8$
71. $2 \mathrm{~A}_{(\mathrm{g})}+\mathrm{B}_{(\mathrm{g})} \leftrightarrow \mathrm{C}_{(\mathrm{g})}+3 \mathrm{D}_{(\mathrm{g})}$ ushbu reaksiya bo'yicha mos ravishda A va B moddalardan $4,5 \mathrm{~mol} / \mathrm{l}$ dan olindi. A moddaning 25 \% qismi sarflangan bo'lsa, C va D moddalarning muvozanat konsentratsiyalarini toping.\\
A) 0,$5 ; 1,5$\\
B) $1 ; 1,5$\\
C) $1 ; 2$\\
D) $2 ; 1$
  \item $\mathrm{N}_{2(\mathrm{~g})}+\mathrm{H}_{2(\mathrm{~g})} \leftrightarrow \mathrm{NH}_{3(\mathrm{~g})}$ ushbu reaksiya bo'yicha mos ravishda $\mathrm{N}_{2}$ va $\mathrm{H}_{2}$ lardan 6,6 $\mathrm{mol} / \mathrm{l}$ dan olindi. $\mathrm{H}_{2}$ ning $50 \%$ qismi sarflangan bo'lsa, $\mathrm{NH}_{3}$ ning muvozanat konsentratsiyasini toping.\\
A) 2\\
B) 3\\
C) 1\\
D) 4
  \item $\mathrm{H}_{2(\mathrm{~g})}+\mathrm{Cl}_{2(\mathrm{~g})} \leftrightarrow \mathrm{HCl}_{(\mathrm{g})}$ ushbu reaksiya bo'yicha mos ravishda $\mathrm{H}_{2}$ va $\mathrm{Cl}_{2}$ lardan 4, $4 \mathrm{~mol} / \mathrm{l}$ dan olindi. $\mathrm{Cl}_{2}$ ning $40 \%$ qismi sarflangan bo'lsa, HCl ning muvozanat konsentratsiyasini toping.\\
A) 4,8\\
B) 3,2\\
C) 1,8\\
D) 1,6
  \item $\mathrm{CO}_{(\mathrm{g})}+\mathrm{H}_{2} \mathrm{O}_{(\mathrm{g})} \leftrightarrow \mathrm{CO}_{2(\mathrm{~g})}+\mathrm{H}_{2(\mathrm{~g})}$ ushbu reaksiya bo'yicha mos ravishda CO va $\mathrm{H}_{2} \mathrm{O}$ lardan $6,5 \mathrm{~mol} / \mathrm{l}$ dan olindi. $\mathrm{H}_{2} \mathrm{O}$ ning $20 \%$\\qismi sarflangan bo'lsa, $\mathrm{CO}_{2}$ va $\mathrm{H}_{2}$ larning muvozanat konsentratsiyalarini toping.\\
A) $4 ; 3$\\
B) $3 ; 3$\\
C) $1 ; 1$\\
D) $2 ; 2$
  \item $\mathrm{CO}_{(\mathrm{g})}+\mathrm{Cl}_{2(\mathrm{~g})} \leftrightarrow \mathrm{COCl}_{2(\mathrm{~g})}$ ushbu reaksiya bo'yicha mos ravishda CO va $\mathrm{Cl}_{2}$ lardan $10,5 \mathrm{~mol} / \mathrm{l}$ dan olindi. CO ning $30 \%$ qismi sarflangan bo'lsa, $\mathrm{COCl}_{2}$ ning muvozanat konsentratsiyasini toping.\\
A) 2\\
B) 3\\
C) 1\\
D) 4
  \item $\mathrm{SO}_{2(\mathrm{~g})}+\mathrm{O}_{2(\mathrm{~g})} \leftrightarrow \mathrm{SO}_{3(\mathrm{~g})}$ ushbu reaksiya bo'yicha mos ravishda $\mathrm{SO}_{2}$ va $\mathrm{O}_{2}$ moddalardan $8,9 \mathrm{~mol} / \mathrm{l}$ dan olindi. $\mathrm{O}_{2}$ ning $50 \%$ qismi sarflangan bo'lsa, $\mathrm{SO}_{3}$ ning muvozanat konsentratsiyasini toping.\\
A) 7\\
B) 8\\
C) 9\\
D) 5
  \item $\mathrm{Fe}_{(\mathrm{q})}+\mathrm{H}_{2} \mathrm{O}_{(\mathrm{g})} \leftrightarrow \mathrm{Fe}_{3} \mathrm{O}_{4(\mathrm{q})}+\mathrm{H}_{2(\mathrm{~g})}$ ushbu reaksiya bo'yicha $\mathrm{H}_{2} \mathrm{O}$ dan $5 \mathrm{~mol} / \mathrm{l}$ olindi. $\mathrm{H}_{2} \mathrm{O}$ ning $40 \%$ qismi sarflangan bo'lsa, $\mathrm{H}_{2}$ ning muvozanat konsentratsiyasini toping.\\
A) 2\\
B) 3\\
C) 1\\
D) 4
  \item $\mathrm{NO}_{2(\mathrm{~g})} \leftrightarrow \mathrm{N}_{2} \mathrm{O}_{4(\mathrm{~g})}$ ushbu reaksiya bo'yicha $\mathrm{NO}_{2}$ dan $8 \mathrm{~mol} / \mathrm{l}$ olindi. $\mathrm{NO}_{2}$ ning $50 \%$ qismi sarflangan bo'lsa, $\mathrm{N}_{2} \mathrm{O}_{4}$ ning muvozanat konsentratsiyasini toping.\\
A) 2\\
B) 3\\
C) 1\\
D) 4
  \item $\mathrm{A}_{(\mathrm{g})}+\mathrm{B}_{(\mathrm{g})} \leftrightarrow \mathrm{C}_{(\mathrm{g})}+\mathrm{D}_{(\mathrm{g})}$ ushbu reaksiya bo'yicha mos ravishda A va B moddalardan $4,5 \mathrm{~mol} / \mathrm{l}$ dan olindi. A moddaning 25 \% qismi sarflangan bo'lsa, C va D moddalarning muvozanat konsentratsiyalarini toping.\\
A) $2 ; 2$\\
B) $2 ; 5$\\
C) $1 ; 1$\\
D) $4 ; 2$
  \item $\mathrm{CH}_{4(\mathrm{~g})}+\mathrm{O}_{2(\mathrm{~g})} \leftrightarrow \mathrm{CO}_{2(\mathrm{~g})}+\mathrm{H}_{2} \mathrm{O}_{(\mathrm{g})}$ ushbu reaksiya bo'yicha mos ravishda $\mathrm{CH}_{4}$ va $\mathrm{O}_{2}$ lardan $8,8 \mathrm{~mol} / \mathrm{l}$ dan olindi. $\mathrm{CH}_{4}$ ning $25 \%$ qismi sarflangan bo'lsa, $\mathrm{CO}_{2}$ va $\mathrm{H}_{2} \mathrm{O}$ moddalarning muvozanat konsentratsiyalarini toping.\\
A) $2 ; 2$\\
B) $3 ; 2$\\
C) $1 ; 2$\\
D) $2 ; 4$
  \item $\mathrm{A}_{(\mathrm{g})}+\mathrm{B}_{(\mathrm{g})} \leftrightarrow \mathrm{C}_{(\mathrm{g})}+\mathrm{D}_{(\mathrm{g})}$ ushbu reaksiya bo'yicha mos ravishda A va B moddalardan $4,5 \mathrm{~mol} / \mathrm{l}$ dan olindi. $\mathrm{K}_{\mathrm{M}}=1$\\
bo'lsa, C va D ning muvozanat konsentratsiyasini aniqlang.\\
A) $2 ; 2$\\
B) 3,$2 ; 2,4$\\
C) $1 ; 2$\\
D) $2.22 ; 2.22$
  \item $\mathrm{A}_{(\mathrm{g})}+\mathrm{B}_{(\mathrm{g})} \leftrightarrow \mathrm{C}_{(\mathrm{g})}+\mathrm{D}_{(\mathrm{g})}$ ushbu reaksiya bo'yicha mos ravishda A va B moddalardan $6,6 \mathrm{~mol} / \mathrm{l}$ dan olindi. $\mathrm{K}_{\mathrm{M}}=1$ bo'lsa, B va D ning muvozanat konsentratsiyasini aniqlang.\\
A) $3 ; 3$\\
B) $3 ; 4$\\
C) $1 ; 2$\\
D) $2 ; 2$
  \item $\mathrm{A}_{(\mathrm{g})}+\mathrm{B}_{(\mathrm{g})} \leftrightarrow \mathrm{C}_{(\mathrm{g})}+\mathrm{D}_{(\mathrm{g})}$ ushbu reaksiya bo'yicha mos ravishda A va B moddalardan $4,4 \mathrm{~mol} / \mathrm{l}$ dan olindi. $\mathrm{K}_{\mathrm{M}}=1$ bo'lsa, A va C ning muvozanat konsentratsiyasini aniqlang.\\
A) $2 ; 2$\\
B) 3,$2 ; 2,4$\\
C) $1 ; 2$\\
D) 2,$22 ; 2,22$
  \item $\mathrm{A}_{(\mathrm{g})}+\mathrm{B}_{(\mathrm{g})} \leftrightarrow \mathrm{C}_{(\mathrm{g})}+\mathrm{D}_{(\mathrm{g})}$ ushbu reaksiya bo'yicha mos ravishda A va B moddalardan $2,2 \mathrm{~mol} / \mathrm{l}$ dan olindi. $\mathrm{K}_{\mathrm{M}}=1$ bo'lsa, A va B ning muvozanat konsentratsiyasini aniqlang.\\
A) $2 ; 2$\\
B) 1,$5 ; 1,5$\\
C) $1 ; 1$\\
D) 2,$22 ; 2,22$
  \item $\mathrm{A}_{(\mathrm{g})}+\mathrm{B}_{(\mathrm{g})} \leftrightarrow \mathrm{C}_{(\mathrm{g})}+\mathrm{D}_{(\mathrm{g})}$ ushbu reaksiya bo'yicha mos ravishda A va B moddalardan $6,4 \mathrm{~mol} / \mathrm{l}$ dan olindi. $\mathrm{K}_{\mathrm{M}}=1$ bo'lsa, A va D ning muvozanat konsentratsiyasini aniqlang.\\
A) $3 ; 3$\\
B) 3,$6 ; 2,4$\\
C) $2 ; 2$\\
D) 2,$32 ; 2,44$
  \item $\mathrm{A}_{(\mathrm{g})}+\mathrm{B}_{(\mathrm{g})} \leftrightarrow \mathrm{C}_{(\mathrm{g})}+\mathrm{D}_{(\mathrm{g})}$ ushbu reaksiya bo'yicha mos ravishda A va B moddalardan $3,3 \mathrm{~mol} / \mathrm{l}$ dan olindi. $\mathrm{K}_{\mathrm{M}}=1$ bo'lsa, B va C ning muvozanat konsentratsiyasini aniqlang.\\
A) $2 ; 2$\\
B) 1,$5 ; 1,5$\\
C) $1 ; 1$\\
D) 2,$22 ; 2,22$
  \item $\mathrm{A}_{(\mathrm{g})}+\mathrm{B}_{(\mathrm{g})} \leftrightarrow \mathrm{C}_{(\mathrm{g})}+\mathrm{D}_{(\mathrm{g})}$ ushbu reaksiya bo'yicha mos ravishda A va B moddalardan $8,4 \mathrm{~mol} / /$ dan olindi. $\mathrm{K}_{\mathrm{M}}=1$\\
bo'lsa, C va D ning muvozanat konsentratsjuasini aniqlang.\\
A) $2: 2$\\
B) 3,$2 ; 2,4$\\
C) $1: 2$\\
D) 2,$67 ; 2,67$
88, $\mathrm{A}_{(\Omega)}+\mathrm{B}_{(g)} \leftrightarrow \mathrm{C}_{(\Omega)}+\mathrm{D}_{(g)}$ ushbu reaksiya boticha mos ravishda A va B moddalardan $3.5 \mathrm{~mol} / /$ dan olindi. $\mathrm{K}_{\mathrm{N}}=1$ bolsa. A va D ning muvozanat konsentratsiyasini aniqlang.\\
A) $2,25: 1,875$\\
B) $3,225: 2,445$\\
C) $1,125: 1,875$\\
D) $2,22: 2,22$\\
89. $\mathrm{A}_{(\Omega)}+\mathrm{B}_{(\Omega)} \leftrightarrow \mathrm{C}_{(\Omega)}+\mathrm{D}_{(\Omega)}$ ushbu reaksiya boyicha mos ravishda A va B moddalardan $6,5 \mathrm{~mol} / \mathrm{l}$ dan olindi. $\mathrm{K}_{\mathrm{U}}=1$ bolsa, B va D ning muvozanat kon'sentratsivasini aniqlang.\\
A) 2,$27 ; 2.73$\\
B) 3,$24 ; 2,45$\\
C) $1 ; 2$\\
D) 2,$22 ; 2,22$\\
90. $\mathrm{A}_{(\mathrm{g})}+\mathrm{B}_{(\mathrm{g})} \leftrightarrow \mathrm{C}_{(\mathrm{g})}+\mathrm{D}_{(\mathrm{g})}$ ushbu reaksiya bo'yicha mos ravishda $A$ va $B$ moddalardan $2.6 \mathrm{~mol} / /$ dan olindi. $\mathrm{K}_{\mathrm{M}}=1$ bo'lsa. B va C ning muvozanat konsentratsiyasini aniqlang.\\
A) $2: 2$\\
B) 0,$5 ; 1,5$\\
C) $1 ; 2$\\
D) $1 ; 1,5$
  \item $\mathrm{A}_{(\mathrm{g})}+\mathrm{B}_{(\mathrm{g})} \leftrightarrow \mathrm{C}_{(\mathrm{g})}+\mathrm{D}_{(\mathrm{g})}$ ushbu reaksiya bo'yicha $\mathrm{A}, \mathrm{B}, \mathrm{C}, \mathrm{D}$ moddalarning muvozanat konsentratsiyasi mos ravishda $3,4,6,2 \mathrm{~mol} / \mathrm{l}$ dan bo'lsa, A moddadan 1 $\mathrm{mol} / \mathrm{l}$ qo'shilgandan keyingi C va D ning muvozanat konsentratsiyasini aniqlang.\\
A) 6,$25 ; 2,25$\\
B) $3,5 \div 2,45$\\
C) $1,5: 2,5$\\
D) $2,25: 2,85$
$\underline{92 .} \mathrm{A}_{(g)}+\mathrm{B}_{(g)} \leftrightarrow \mathrm{C}_{(g)}+\mathrm{D}_{(g)}$ ushbu reaksiya boyicha A. B. C. D moddalarning muvozanat konsentratsiyasi mos ravishda 4. 4. $8,2 \mathrm{~mol} / \mathrm{l}$ dan bo'lsa. B moddadan 2 mol/l qo'shilgandan keyingi A va D ning muvozanat konsentratsiyasini aniqlang.\\
A) 3,$25 ; 2,25$\\
B) $3,6: 2,4$\\
C) $4.5: 2,5$\\
D) 2,$25 ; 2,85$\\
93. $\mathrm{A}_{(\mathrm{g})}+\mathrm{B}_{(\mathrm{g})} \leftrightarrow \mathrm{C}_{(\mathrm{g})}+\mathrm{D}_{(\mathrm{g})}$ ushbu reaksiya bo'yicha $\mathrm{A}, \mathrm{B}, \mathrm{C}, \mathrm{D}$ moddalarning muvozanat konsentratsiyasi mos ravishda 5, $4,5,4 \mathrm{~mol} / \mathrm{l}$ dan bo'lsa, B moddadan 1 mol/l qo'shilgandan keyingi C va D ning muvozanat konsentratsiyasini aniqlang.\\
A) 5,$25 ; 4,25$\\
B) $3,52: 4,52$\\
C) 3,$5 ; 2,5$\\
D) 5,$26 ; 4,26$\\
94. $\mathrm{A}_{(\mathrm{g})}+\mathrm{B}_{(\mathrm{g})} \leftrightarrow \mathrm{C}_{(\mathrm{g})}+\mathrm{D}_{(\mathrm{g})}$ ushbu reaksiya bo'yicha A, B, C, D moddalarning muvozanat konsentratsiyasi mos ravishda $7,4,14,2 \mathrm{~mol} / \mathrm{l}$ dan bo'lsa, B moddadan 3 $\mathrm{mol} / \mathrm{l}$ qo'shilgandan keyingi C va D ning muvozanat konsentratsiyasini aniqlang.\\
A) 14,$25 ; 2,25$\\
B) 14,$5 ; 2,5$\\
C) 14,$7 ; 2,7$\\
D) $2,25 \div 2,85$\\
95. $\mathrm{A}_{(\mathrm{g})}+\mathrm{B}_{(\mathrm{g})} \leftrightarrow \mathrm{C}_{(\mathrm{g})}+\mathrm{D}_{(\mathrm{g})}$ ushbu reaksiya bo'yicha $\mathrm{A}, \mathrm{B}, \mathrm{C}, \mathrm{D}$ moddalarning muvozanat konsentratsiyasi mos ravishda 6, $4,8,3 \mathrm{~mol} / \mathrm{l}$ dan bo'lsa, B moddadan 2 mol/l qo'shilgandan keyingi B va C ning muvozanat konsentratsiyasini aniqlang.\\
A) 5,$12 ; 8,88$\\
B) 5,$48 ; 8,52$\\
C) 4,$5 ; 8,5$\\
D) 2,$25 ; 2.85$\\
96. $\mathrm{A}_{(\mathrm{g})}+\mathrm{B}_{(\mathrm{g})} \leftrightarrow \mathrm{C}_{(\mathrm{g})}+\mathrm{D}_{(\mathrm{g})}$ ushbu reaksiya bo'yicha $\mathrm{A}, \mathrm{B}, \mathrm{C}, \mathrm{D}$ moddalarning muvozanat konsentratsiyasi mos ravishda 6, 6, 9, $4 \mathrm{~mol} /$ dan bo'lsa, A moddadan 3 mol/l qo'shilgandan keyingi A va D ning muvozanat konsentratsiyasini aniqlang.\\
A) 6,$25 ; 2,25$\\
B) $8 ; 5$\\
C) 8,$357 ; 4,643$\\
D) 7,$25 ; 4,75$\\
  \item Fe metalida Fe ning oksidlanish darajasi nechaga teng?\\
A) 0\\
B) +2\\
C) +3\\
D) +1
2. Oq fosforda $\left(\mathrm{P}_{4}\right) \mathrm{P}$ ning oksidlanish darajasi nechaga teng?\\
A) 0\\
B) +2\\
C) +3\\
D) +1\\
3. $\mathrm{Fe}_{2} \mathrm{O}_{3}$ da Fe ning oksidlanish darajasi nechaga teng?\\
A) 0\\
B) +2\\
C) +3\\
D) +1\\
4. $\mathrm{As}_{2} \mathrm{O}_{5}$ metalida As ning oksidlanish darajasi nechaga teng?\\
A) 0\\
B) +2\\
C) +4\\
D) +5\\
5. $\mathrm{Fe}(\mathrm{OH})_{3}$ da Fe ning oksidlanish darajasi nechaga teng?\\
A) 0\\
B) +2\\
C) +3\\
D) +1\\
6. $\mathrm{Ca}(\mathrm{OH})_{2}$ da Ca ning oksidlanish darajasi nechaga teng?\\
A) 0\\
B) +2\\
C) +3\\
D) +1\\
7. $\mathrm{HNO}_{3}$ da N ning oksidlanish darajasi nechaga teng?\\
A) 0\\
B) +2\\
C) +4\\
D) +5\\
8. $\mathrm{H}_{2} \mathrm{Cr}_{2} \mathrm{O}_{7}$ da Cr ning oksidlanish darajasi nechaga teng?\\
A) +4\\
B) +5\\
C) +6\\
D) +7\\
9. $\mathrm{KMnO}_{4}$ da Mn ning oksidlanish darajasi nechaga teng?\\
A) +4\\
B) +5\\
C) +6\\
D) +7\\
10. $\mathrm{Al}_{2}\left(\mathrm{SO}_{4}\right)_{3}$ da S ning oksidlanish darajasi nechaga teng?\\
A) +4\\
B) +5\\
C) +6\\
D) +7
  \item Quyidagi moddalardan qaysi biri faqat oksidlovchi xususiyatga ega?\\
A) $\mathrm{F}_{2}$\\
B) $\mathrm{H}_{2} \mathrm{O}_{2}$\\
C) Na\\
D) S
12. Quyidagi moddalardan qaysi biri faqat qaytaruvehi xususiyatga ega?)\\
A) $\mathrm{F}_{2}$\\
B) $\mathrm{H}_{2} \mathrm{O}_{2}$\\
C) Na\\
D) S\\
13. Quyidagi moddalardan qaysi biri ham oksidlovchi ham qaytaruvchi xususiyatga ega?\\
A) $\mathrm{F}_{2}$\\
B) Fe\\
C) Na\\
D) S\\
14. Quyidagi ionlardan qaysi biri faqat oksidlovchi xususiyatga ega?\\
A) $\mathrm{Cl}^{-}$\\
B) $\mathrm{O}^{-2}$\\
C) $\mathrm{S}^{+4}$\\
D) $\mathrm{S}^{+6}$\\
15. Quyidagi ionlardan qaysi biri faqat qaytaruvchi xususiyatga ega?\\
A) $\mathrm{Cl}^{-}$\\
B) $\mathrm{O}^{+2}$\\
C) $\mathrm{S}^{+4}$\\
D) $\mathrm{S}^{+6}$\\
16. Quyidagi ionlardan qaysi biri ham oksidlovchi ham qaytaruvchi xususiyatga ega?\\
A) $\mathrm{Cl}^{-}$\\
B) $\mathrm{O}^{+2}$\\
C) $\mathrm{S}^{+4}$\\
D) $\mathrm{S}^{+6}$\\
17. Quyidagi moddalardan qaysi biri faqat qaytaruvchi xususiyatga ega?\\
A) $\mathrm{F}_{2}$\\
B) $\mathrm{H}_{2} \mathrm{O}$\\
C) Al\\
D) P\\
18. Quyidagi ionlardan qaysi biri ham oksidlovchi ham qaytaruvchi xususiyatga ega?\\
A) $\mathrm{Cl}^{+1}$\\
B) $\mathrm{O}^{+2}$\\
C) $\mathrm{C}^{+4}$\\
D) $\mathrm{S}^{-2}$\\
19. Quyidagi moddalardan qaysi biri faqat qaytaruvchi xususiyatga ega?\\
A) P\\
B) $\mathrm{H}_{2} \mathrm{O}$\\
C) Fe\\
D) S\\
20. Quyidagi moddalardan qaysi biri faqat oksidlovchi xususiyatga ega?\\
A) $\mathrm{F}_{2}$\\
B) $\mathrm{H}_{2} \mathrm{O}$\\
C) NaCl\\
D) $\mathrm{SO}_{2}$
  \item Quyidagi jarayonlardan qaysi biri qaytarilish jarayoni hisoblanadi?\\
A) $\mathrm{Mn}^{+2} \rightarrow \mathrm{MnO}_{4}{ }^{-}$\\
B) $\mathrm{CrO}_{4}{ }^{2-} \rightarrow \mathrm{Cr}^{+3}$\\
C) $\mathrm{MnO}_{2} \rightarrow \mathrm{MnO}_{4}^{-}$\\
D) $\mathrm{Cr}^{+3} \rightarrow \mathrm{CrO}_{4}{ }^{2-}$
22. Quyidagi jarayonlardan qaysi biri qaytarilish jarayoni hisoblanadi?\\
A) $\mathrm{Mn}^{+7} \rightarrow \mathrm{MnO}_{4}{ }^{2-}$\\
B) $\mathrm{Al}^{0} \rightarrow \mathrm{Al}^{+3}$\\
C) $\mathrm{Na}^{0} \rightarrow \mathrm{Na}^{+}$\\
D) $\mathrm{Cr}^{+3} \rightarrow \mathrm{Cr}^{+6}$\\
23. Quyidagi jarayonlardan qaysi biri qaytarilish jarnyoni hisoblanadi?\\
A) $\mathrm{CH}_{4} \rightarrow \mathrm{CO}_{2}$\\
B) $\mathrm{PH}_{3} \rightarrow \mathrm{H}_{9} \mathrm{PO}_{4}$\\
C) $\mathrm{Mn}^{+7} \rightarrow \mathrm{Mn}^{+2}$\\
D) $\mathrm{Cr}^{+3} \rightarrow \mathrm{CrO}_{4}{ }^{2-}$\\
24. Quyidagi jarayonlardan qnyai biri qaytarilish jarayoni hisoblanadi?\\
A) $\mathrm{Al}^{+3} \rightarrow \mathrm{Al}^{0}$\\
B) $\mathrm{C}_{2} \mathrm{H}_{2} \rightarrow \mathrm{H}_{2} \mathrm{C}_{2} \mathrm{O}_{4}$\\
C) $\mathrm{MnO}_{2} \rightarrow \mathrm{MnO}_{4}$\\
D) $\mathrm{Cr}^{+2} \rightarrow \mathrm{Cr}^{+3}$\\
25. Quyidagi jarayonlardan qaysi biri qaytarilish jarayoni hisoblanadi?\\
A) $\mathrm{Mn}^{+2} \rightarrow \mathrm{MnO}_{4}{ }^{-}$\\
B) $\mathrm{Cr}^{0} \rightarrow \mathrm{Cr}^{+3}$\\
C) $\mathrm{Cr}^{+3} \rightarrow \mathrm{CrO}_{4}{ }^{2-}$\\
D) $\mathrm{MnO}_{4} \rightarrow \mathrm{MnO}_{2}$\\
26. Quyidagi jarayonlardan qaysi biri oksidlanish jarayoni hisoblanadi?\\
A) $\mathrm{Mn}^{+2} \rightarrow \mathrm{MnO}_{4}$.\\
B) $\mathrm{CrO}_{4}{ }^{2-} \rightarrow \mathrm{Cr}^{+3}$\\
C) $\mathrm{MnO}_{4} \rightarrow \mathrm{MnO}_{2}$\\
D) $\mathrm{CrO}_{4}{ }^{2} \rightarrow \mathrm{Cr}^{+3}$\\
27. Quyidagi jarayonlardan qaysi biri oksidlanish jarayoni hisoblanadi?\\
A) $\mathrm{Al}^{+3} \rightarrow \mathrm{Al}^{0}$\\
B) $\mathrm{C}_{2} \mathrm{H}_{2} \rightarrow \mathrm{H}_{2} \mathrm{C}_{2} \mathrm{O}_{4}$\\
C) $\mathrm{MnO}_{4} \rightarrow \mathrm{MnO}_{2}$\\
D) $\mathrm{Cr}^{+3} \rightarrow \mathrm{Cr}^{+2}$\\
28. Quyidagi jarayonlardan qaysi biri oksidlanish jarayoni hisoblanadi?\\
A) $\mathrm{CO}_{2} \rightarrow \mathrm{CH}_{4}$\\
B) $\mathrm{H}_{3} \mathrm{PO}_{4} \rightarrow \mathrm{PH}_{3}$\\
C) $\mathrm{Mn}^{+2} \rightarrow \mathrm{Mn}^{+7}$\\
D) $\mathrm{CrO}_{4}{ }^{2-} \rightarrow \mathrm{Cr}^{+3}$\\
29. Quyidagi jarayonlardan qaysi biri oksidlanish jarayoni hisoblanadi?\\
A) $\mathrm{Mn}^{+2} \rightarrow \mathrm{Mn}^{+4}$\\
B) $\mathrm{CrO}_{4}{ }^{2} \rightarrow \mathrm{Cr}^{+3}$\\
C) $\mathrm{MnO}_{4} \rightarrow \mathrm{MnO}_{2}$\\
D) $\mathrm{Cr}_{2} \mathrm{O}_{7}{ }^{2-} \rightarrow \mathrm{Cr}^{+3}$\\
30. Quyidagi jarayonlardan qaysi biri oksidlanish jarayoni hisoblanadi?\\
A) $\mathrm{CO}_{2}$\\$\rightarrow \mathrm{CH}_{4}$\\
B) $\mathrm{PH}_{3} \rightarrow \mathrm{H}_{3} \mathrm{PO}_{4}$\\
C) $\mathrm{Mn}^{+7} \rightarrow \mathrm{Mn}^{+2}$\\
D) $\mathrm{CrO}_{4}{ }^{2} \rightarrow \mathrm{Cr}^{+3}$
  \item $\mathrm{K}_{2} \mathrm{Cr}_{2} \mathrm{O}_{7}+\mathrm{H}_{2} \mathrm{~S}+\mathrm{H}_{2} \mathrm{O} \rightarrow \mathrm{KOH}+\mathrm{Cr}(\mathrm{OH})_{3}+\mathrm{S}$ ushbu reaksiyada $\mathrm{K}_{2} \mathrm{Cr}_{2} \mathrm{O}_{7}$ ning ekvivalent massasini aniqlang.\\
A) 49\\
B) 52\\
C) 48\\
D) 47
32. $\mathrm{HNO}_{3}+\mathrm{SO}_{2}+\mathrm{H}_{2} \mathrm{O} \rightarrow \mathrm{H}_{2} \mathrm{SO}_{4}+\mathrm{NO}$ urhbu reakayada $\mathrm{SO}_{2}$ ning okvivalent mananaíni aniqlang.\\
A) 30\\
B) 16\\
C) 32\\
D) $(34$\\
33. $\mathrm{KClO}_{3}+\mathrm{HCl} \rightarrow \mathrm{KCl}+\mathrm{Cl}_{2}+\mathrm{H}_{2} \mathrm{O}$ ushbu roaksiyada $\mathrm{KClO}_{3}$ ning ukvivalent massasini aniqlang.\\
A) 25,2\\
B) 52,2\\
C) 16,6\\
D) 20,4\\
$\underline{34}, \mathrm{KMnO}_{4}+\mathrm{FeCl}_{2}+\mathrm{HCJ} \rightarrow \mathrm{FoCl}_{3}+\mathrm{MnCl}_{2}+\mathrm{KC}$ $\mathrm{l}++\mathrm{H}_{2} \mathrm{O}$ ushbu reaksiyada $\mathrm{FeCl}_{2}$ ning ekvivalent massasini aniqlang,\\
A) 63,5\\
B) 52\\
C) 127\\
D) 47\\
$35, \mathrm{KMnO}_{4}+\mathrm{KNO}_{2}+\mathrm{H}_{2} \mathrm{SO}_{4} \rightarrow \mathrm{~K}_{2} \mathrm{SO}_{4}+\mathrm{MnSO}_{4}$ $+\mathrm{KNO}_{3}+\mathrm{H}_{2} \mathrm{O}$ ushbu reaksiyada $\mathrm{KMnO}_{4}$ ning ekvivalent massasini aniqlang.\\
A) 31,6\\
B) 15,8\\
C) 158\\
D) 316\\
36. $\mathrm{K}_{2} \mathrm{Cr}_{2} \mathrm{O}_{7}+\mathrm{KBr}+\mathrm{H}_{2} \mathrm{SO}_{4} \rightarrow \mathrm{Cr}_{2}\left(\mathrm{SO}_{4}\right)_{3}=\mathrm{Br}_{2}+$ $+\mathrm{K}_{2} \mathrm{SO}_{4}+\mathrm{H}_{2} \mathrm{O}$ ushbu reaksiyada $\mathrm{K}_{2} \mathrm{Cr}_{2} \mathrm{O}_{7}$ ning ekvivalent massasini aniqlang.\\
A) 49\\
B) 52\\
C) 48\\
D) 47\\
37. $\mathrm{HNO}_{3}+\mathrm{P} \rightarrow \mathrm{NO}_{2}+\mathrm{H}_{3} \mathrm{PO}_{4}+\mathrm{H}_{2} \mathrm{O}$ ushbu reaksiyada P ning ekvivalent massasini aniqlang.\\
A) 6,2\\
B) 62\\
C) 15,5\\
D) 31\\
38. $\mathrm{Cl}_{2}+\mathrm{NH}_{3} \rightarrow \mathrm{~N}_{2}+\mathrm{HCl}$\\
ushbu reaksiyada $\mathrm{Cl}_{2}$ ning ekvivalent massasini aniqlang.\\
A) 17,25\\
B) 71\\
C) 142\\
D) 35,5\\
39. $\mathrm{Ca}_{3}\left(\mathrm{PO}_{4}\right)_{2}+\mathrm{C}+\mathrm{SiO}_{2} \rightarrow \mathrm{CaSiO}_{3}+\mathrm{P}+\mathrm{CO}$ ushbu reaksiyada C ning ekvivalent massasini aniqlang.\\
A) 9\\
B) 6\\
C) 12\\
D) 8\\
40. $\mathrm{KNO}_{3}+\mathrm{C}+\mathrm{S} \rightarrow \mathrm{K}_{2} \mathrm{~S}+\mathrm{CO}_{2}+\mathrm{N}_{2}$ ushbu reaksiyada S ning ekvivalent massasini aníqlang.\\
A) 8\\
B) 32\\
C) 16\\
D) 4
  \item $\mathrm{HNO}_{3}+\mathrm{P} \rightarrow \mathrm{NO}_{2}+\mathrm{H}_{3} \mathrm{PO}_{4}+\mathrm{H}_{2} \mathrm{O}$ ushbu reaksiya oksidlanish-qaytarilish reaksiyasining qaysi turiga kiradi?\\
A) molekulalararo\\
B) ichki molekulyar\\
C) disproporsiya\\
D) sinproporsiya\\
  \item $\mathrm{KMnO}_{4} \rightarrow \mathrm{~K}_{2} \mathrm{MnO}_{4}+\mathrm{MnO}_{2}+\mathrm{O}_{2}$ ushbu reaksiya oksidlanish-qaytarilish reaksiyasining qaysi turiga kiradi?\\
A) molekulalararo\\
B) ichki molekulyar\\
C) disproporsiya\\
D) sinproporsiya
  \item $\mathrm{NO}+\mathrm{NO}_{2} \rightarrow \mathrm{~N}_{2} \mathrm{O}_{3}$ ushbu reaksiya oksidlanish-qaytarilish reaksiyasining qaysi turiga kiradi?\\
A) molekulalararo\\
B) ichki molekulyar\\
C) disproporsiya\\
D) sinproporsiya
  \item $\mathrm{Cr}+\mathrm{H}_{2} \mathrm{SO}_{4} \rightarrow \mathrm{Cr}_{2}\left(\mathrm{SO}_{4}\right)_{3}+\mathrm{SO}_{2}+\mathrm{H}_{2} \mathrm{O}$ ushbu reaksiya oksidlanish -qaytarilish reaksiyasining qaysi turiga kiradi?\\
A) molekulalararo\\
B) ichki molekulyar\\
C) disproporsiya\\
D) sinproporsiya
  \item $\mathrm{H}_{2} \mathrm{O}_{2} \rightarrow \mathrm{H}_{2} \mathrm{O}+\mathrm{O}_{2}$ ushbu reaksiya oksidlanish-qaytarilish reaksiyasining qaysi turiga kiradi?\\
A) molekulalararo\\
B) ichki molekulyar\\
C) disproporsiya\\
D) sinproporsiya
  \item $\mathrm{AuCl}_{3} \rightarrow \mathrm{AuCl}+\mathrm{Cl}_{2}$ ushbu reaksiya oksidlanish-qaytarilish reaksiyasining qaysi turiga kiradi?\\
A) molekulalararo\\
B) ichki molekulyar\\
C) disproporsiya\\
D) sinproporsiya
  \item $\mathrm{S}+\mathrm{KOH} \rightarrow \mathrm{K}_{2} \mathrm{~S}+\mathrm{H}_{2} \mathrm{SO}_{3}+\mathrm{H}_{2} \mathrm{O}$ ushbu reaksiya oksidlanish-qaytarilish reaksiyasining qaysi turiga kiradi?\\
A) molekulalararo\\
B) ichki molekulyar\\
C) disproporsiya\\
D) sinproporsiya
  \item $\mathrm{NH}_{4} \mathrm{NO}_{2} \rightarrow \mathrm{~N}_{2}+\mathrm{H}_{2} \mathrm{O}$ ushbu reaksiya oksidlanish-qaytarilish reaksiyasining qaysi turiga kiradi?\\
A) molekulalararo\\
B) ichki molekulyar\\
C) disproporsiya\\
D) sinproporsiya
  \item $\mathrm{CuO} \rightarrow \mathrm{Cu}_{2} \mathrm{O}+\mathrm{O}_{2}$ ushbu reaksiya oksidlanish -qaytarilish reaksiyasining qaysi turiga kiradi?\\
A) molekulalararo\\
B) ichki molekulyar\\
C) disproporsiya\\
D) sinproporsiya
  \item $\mathrm{H}_{2} \mathrm{~S}+\mathrm{SO}_{2} \rightarrow \mathrm{H}_{2} \mathrm{O}+\mathrm{S}$ ushbu reaksiya oksidlanish-qaytarilish reaksiyasining qaysi turiga kiradi?\\
A) molekulalararo\\
B) ichki molekulyar\\
C) disproporsiya\\
D) sinproporsiya
  \item $\mathrm{Cu}+\mathrm{HNO}_{3} \rightarrow \mathrm{Cu}\left(\mathrm{NO}_{3}\right)_{2}+\mathrm{NO}+\mathrm{H}_{2} \mathrm{O}$ ushbu reaksiyani tenglashtiring hamda oksidlovchining koeffitsiyentini aniqlang.\\
A) 4\\
B) 5\\
C) 8\\
D) 7
  \item $\mathrm{Cu}+\mathrm{HNO}_{3} \rightarrow \mathrm{Cu}\left(\mathrm{NO}_{3}\right)_{2}+\mathrm{NO}_{2}+\mathrm{H}_{2} \mathrm{O}$ ushbu reaksiyani tenglashtiring hamda qaytaruvchining koeffitsiyentini aniqlang.\\
A) 1\\
B) 2\\
C) 3\\
D) 4
  \item $\mathrm{FeSO}_{4}+\mathrm{H}_{2} \mathrm{SO}_{4}+\mathrm{KMnO}_{4} \rightarrow \mathrm{Fe}_{2}\left(\mathrm{SO}_{4}\right)_{3}+$ $+\mathrm{K}_{2} \mathrm{SO}_{4}+\mathrm{MnSO}_{4}+\mathrm{H}_{2} \mathrm{O}$ ushbu reaksiyani tenglashtiring hamda suvning koeffitsiyentini aniqlang.\\
A) 4\\
B) 5\\
C) 8\\
D) 7
  \item $\mathrm{P}+\mathrm{HNO}_{3}+\mathrm{H}_{2} \mathrm{O} \rightarrow \mathrm{H}_{3} \mathrm{PO}_{4}+\mathrm{NO}$ ushbu reaksiyani tenglashtiring hamda oksidlovchining koeffitsiyentini aniqlang.\\
A) 4\\
B) 5\\
C) 8\\
D) 7
  \item $\mathrm{Ca}+\mathrm{HNO}_{3} \rightarrow \mathrm{Ca}\left(\mathrm{NO}_{3}\right)_{2}+\mathrm{N}_{2} \mathrm{O}+\mathrm{H}_{2} \mathrm{O}$ ushbu reaksiyani tenglashtiring hamda reaksiyaning o'ng tomonidagi koeffitsiyentlar yig'indisini aniqlang.\\
A) 10\\
B) 9\\
C) 8\\
D) 7
  \item $\mathrm{KOH}+\mathrm{Cl}_{2} \rightarrow \mathrm{KClO}_{3}+\mathrm{KCl}+\mathrm{H}_{2} \mathrm{O}$ ushbu reaksiyani tenglashtiring hamda reaksiyaning chap tomonidagi koeffitsiyentlar yig'indisini aniqlang.\\
A) 10\\
B) 9\\
C) 8\\
D) 7\\
58. $\mathrm{Ca}_{3}\left(\mathrm{PO}_{4}\right)_{2}+\mathrm{C}+\mathrm{SiO}_{2} \rightarrow \mathrm{CaSiO}_{3}+\mathrm{P}+\mathrm{CO}$ ushbu reaksiyani tenglashtiring hamda reaksiyaning jami koeffitsiyentlar yig'indisini aniqlang.\\
A) 18\\
B) 19\\
C) 20\\
D) 21\\
59. $\mathrm{Cl}_{2}+\mathrm{K}_{2} \mathrm{~S}+\mathrm{H}_{2} \mathrm{O} \rightarrow \mathrm{K}_{2} \mathrm{SO}_{4}+\mathrm{HCl}$ ushbu reaksiyani tenglashtiring hamda reaksiyaning o'ng tomonidagi koeffitsiyentlar yig'indisini aniqlang.\\
A) 10\\
B) 9\\
C) 8\\
D) 7\\
60. $\mathrm{H}_{2} \mathrm{SO}_{4}+\mathrm{Ca} \rightarrow \mathrm{H}_{2} \mathrm{~S}+\mathrm{CaSO}_{4}+\mathrm{H}_{2} \mathrm{O}$ ushbu reaksiyani tenglashtiring hamda reaksiyaning chap tomonidagi koeffitsiyentlar yig'indisini aniqlang.\\
A) 10\\
B) 9\\
C) 8\\
D) 7
  \item $\mathrm{C}_{6} \mathrm{H}_{5} \mathrm{CH}_{3}+\mathrm{KMnO}_{4}+\mathrm{H}_{2} \mathrm{SO}_{4} \rightarrow \mathrm{C}_{6} \mathrm{H}_{5} \mathrm{COOH}$ $++\mathrm{K}_{2} \mathrm{SO}_{4}+\mathrm{MnSO}_{4}+\mathrm{H}_{2} \mathrm{O}$ ushbu reaksiya bo'yicha 92 g qaytaruvchi oksidlangan bo'lsa, necha mol suv hosil bo'lgan?\\
A) 2,8\\
B) 3,2\\
C) 8\\
D) 7
  \item $\mathrm{C}_{6} \mathrm{H}_{4}\left(\mathrm{CH}_{3}\right)_{2}+\mathrm{KMnO}_{4}+\mathrm{H}_{2} \mathrm{SO}_{4} \rightarrow \mathrm{~K}_{2} \mathrm{SO}_{4}+\mathrm{H}$ ${ }_{2} \mathrm{O}+\mathrm{C}_{6} \mathrm{H}_{4}(\mathrm{COOH})_{2}+\mathrm{MnSO}_{4}$ ushbu reaksiya bo'yicha 212 g qaytaruvchi\\
oksidlangan bo'lsa, necha mol $\mathrm{MnSO}_{4}$ Һозі] bo'lgan?\\
A) 5,2\\
B) 6,4\\
C) 4,8\\
D) 7,6
  \item $\mathrm{HCOH}+\mathrm{KMnO}_{1}+\mathrm{H}_{2} \mathrm{SO}_{1} \rightarrow \mathrm{CO}_{2}+\mathrm{K}_{2} \mathrm{SO}_{1}+$ $+\mathrm{MnSO}_{4}+\mathrm{H}_{2} \mathrm{O}$ ushbu reaksiya bo'yicha 90 g qaytaruvchi oksidlangan bo'lsa, necha $\mathrm{mol} \mathrm{CO}_{2}$ hosil bo'lgan?\\
A) 2\\
B) 3\\
C) 8\\
D) 7
  \item $\mathrm{C}_{2} \mathrm{H}_{2}+\mathrm{K}_{2} \mathrm{Cr}_{2} \mathrm{O}_{7}+\mathrm{H}_{2} \mathrm{SO}_{4} \rightarrow \mathrm{CO}_{2}+\mathrm{K}_{2} \mathrm{SO}_{4}+$ $+\mathrm{Cr}_{2}\left(\mathrm{SO}_{4}\right)_{3}+\mathrm{H}_{2} \mathrm{O}$ ushbu reaksiya bo'yicha 26 g qaytaruvchi oksidlangan bo'lsa, necha mol suv hosil bo'lgan?\\
A) 5,35\\
B) 4,22\\
C) 8,16\\
D) 7,67
  \item $\mathrm{C}_{6} \mathrm{H}_{5} \mathrm{C}_{2} \mathrm{H}_{5}+\mathrm{KMnO}_{4}+\mathrm{H}_{2} \mathrm{SO}_{4} \rightarrow$ $\mathrm{C}_{6} \mathrm{H}_{5} \mathrm{COOH}+\mathrm{CO}_{2}+\mathrm{K}_{2} \mathrm{SO}_{4}+\mathrm{MnSO}_{4}+\mathrm{H}_{2} \mathrm{O}$ ushbu reaksiya bo'yicha 316 g oksidlovchi qaytarilangan bo'lsa, necha mol suv hosil bo'lgan?\\
A) 2,22\\
B) 4,667\\
C) 4\\
D) 3
  \item $\mathrm{C}_{4} \mathrm{H}_{10}+\mathrm{K}_{2} \mathrm{Cr}_{2} \mathrm{O}_{7}+\mathrm{H}_{2} \mathrm{SO}_{4} \rightarrow$ $\mathrm{CH}_{3} \mathrm{COOC}_{2} \mathrm{H}_{5}++\mathrm{K}_{2} \mathrm{SO}_{4}+\mathrm{Cr}_{2}\left(\mathrm{SO}_{4}\right)_{3}+\mathrm{H}_{2} \mathrm{O}$ ushbu reaksiya bo'yicha 58 g qaytaruvchi oksidlangan bo'lsa, necha mol suv hosil bo'lgan?\\
A) 4\\
B) 3\\
C) 5\\
D) 7
  \item $\mathrm{C}_{6} \mathrm{H}_{5} \mathrm{C}_{2} \mathrm{H}_{5}+\mathrm{KMnO}_{4}+\mathrm{H}_{2} \mathrm{SO}_{4} \rightarrow \mathrm{C}_{6} \mathrm{H}_{5} \mathrm{COO}$ $\mathrm{H}++\mathrm{CO}_{2}+\mathrm{K}_{2} \mathrm{SO}_{4}+\mathrm{MnSO}_{4}+\mathrm{H}_{2} \mathrm{O}$ ushbu reaksiya bo'yicha $148,4 \mathrm{~g}$ qaytaruvchi oksidlangan bo'lsa, necha mol suv hosil bo'lgan?\\
A) 4\\
B) 5,24\\
C) 6\\
D) 7,84
  \item $\mathrm{C}_{2} \mathrm{H}_{5} \mathrm{OH}+\mathrm{KMnO}_{4}+\mathrm{H}_{2} \mathrm{SO}_{4} \rightarrow \mathrm{CH}_{3} \mathrm{COOH}+$ $+\mathrm{K}_{2} \mathrm{SO}_{4}+\mathrm{MnSO}_{4}+\mathrm{H}_{2} \mathrm{O}$ ushbu reaksiya bo'yicha 92 g qaytaruvchi oksidlangan bo'lsa, necha mol suv hosil bo'lgan?\\
A) 2,8\\
B) 5,6\\
C) 8\\
D) 4,4
  \item $\mathrm{CH}_{3} \mathrm{OH}+\mathrm{KMnO}_{4}+\mathrm{H}_{2} \mathrm{SO}_{4} \rightarrow \mathrm{HCOOH}+$ $+\mathrm{K}_{2} \mathrm{SO}_{4}+\mathrm{MnSO}_{4}+\mathrm{H}_{2} \mathrm{O}$ ushbu reaksiya bo'yicha 64 g qaytaruvchi oksidlangan bo'lsa, necha mol chumoli kislota hosil bo'lgan?\\
A) 2\\
B) 3\\
C) 8\\
D) 7
  \item $\mathrm{P}_{1} \mathrm{~S}_{7}+\mathrm{HNO}_{3} \rightarrow \mathrm{H}_{3} \mathrm{PO}_{4}+\mathrm{H}_{2} \mathrm{SO}_{4}+\mathrm{NO}_{2}+\mathrm{H}_{2} \mathrm{O}$ ushbu reaksiya bo'yicha oksidlovchi va qaytaruvchining massa farqi $35,58 \mathrm{~g}$ bo'lsa, necha g suv hosil bo'lgan?\\
A) 3,24\\
B) 4,58\\
C) 18\\
D) 9
  \item $\mathrm{P}_{4} \mathrm{~S}_{6}+\mathrm{HNO}_{3} \rightarrow \mathrm{H}_{3} \mathrm{PO}_{4}+\mathrm{H}_{2} \mathrm{SO}_{4}+\mathrm{NO}_{2}+\mathrm{H}_{2} \mathrm{O}$ ushbu reaksiya bo'yicha oksidlovchi va qaytaruvchining massa farqi $32,12 \mathrm{~g}$ bo'lsa, necha g suv hosil bo'lgan?\\
A) 4,22\\
B) 3,44\\
C) 18\\
D) 2,88
  \item $\mathrm{P}_{2} \mathrm{~S}_{3}+\mathrm{HNO}_{3} \rightarrow \mathrm{H}_{3} \mathrm{PO}_{4}+\mathrm{H}_{2} \mathrm{SO}_{4}+\mathrm{NO}_{2}+\mathrm{H}_{2} \mathrm{O}$ ushbu reaksiya bo'yicha oksidlovchi va qaytaruvchining massa farqi $21,74 \mathrm{~g}$ bo'lsa, necha g suv hosil bo'lgan?\\
A) 1,64\\
B) 2,28\\
C) 1,44\\
D) 1,8
  \item $\mathrm{P}_{3} \mathrm{~S}_{5}+\mathrm{HNO}_{3} \rightarrow \mathrm{H}_{3} \mathrm{PO}_{4}+\mathrm{H}_{2} \mathrm{SO}_{4}+\mathrm{NO}_{2}+\mathrm{H}_{2} \mathrm{O}$ ushbu reaksiya bo'yicha oksidlovchi va qaytaruvchining massa farqi $25,82 \mathrm{~g}$ bo'lsa, necha g suv hosil bo'lgan?\\
A) 2,34\\
B) 4,58\\
C) 18\\
D) 9
  \item $\mathrm{As}_{2} \mathrm{~S}_{3}+\mathrm{HNO}_{3} \rightarrow \mathrm{H}_{3} \mathrm{AsO}_{4}+\mathrm{H}_{2} \mathrm{SO}_{4}+\mathrm{NO}_{2}+\mathrm{H}_{2}$ O ushbu reaksiya bo'yicha oksidlovchi va qaytaruvchining massa farqi $15,18 \mathrm{~g}$ bo'lsa, necha g suv hosil bo'lgan?\\
A) 1,64\\
B) 2,28\\
C) 1,44\\
D) 0,9
  \item $\mathrm{CuFeS}_{2}+\mathrm{HNO}_{3} \rightarrow \mathrm{Fe}\left(\mathrm{NO}_{3}\right)_{3}+\mathrm{Cu}\left(\mathrm{NO}_{3}\right)_{2}+$ $+\mathrm{H}_{2} \mathrm{SO}_{4}+\mathrm{NO}_{2}+\mathrm{H}_{2} \mathrm{O}$ ushbu reaksiya bo'yicha oksidlovchi va qaytaruvchining massa farqi $12,02 \mathrm{~g}$ bo'lsa, necha g suv hosil bo'lgan?\\
A) 3,24\\
B) 1,62\\
C) 2,88\\
D) 1,44
  \item $\mathrm{FeS}_{2}+\mathrm{HNO}_{3} \rightarrow \mathrm{Fe}\left(\mathrm{NO}_{3}\right)_{3}+\mathrm{H}_{2} \mathrm{SO}_{4}+\mathrm{NO}_{2}+\mathrm{H}$\\
${ }_{2} \mathrm{O}$ ushbu reaksiya bo'yicha oksidlovchi va\\
qaytaruvchining massa farqi $10,14 \mathrm{~g}$ bo'lsa, necha g suv hosil bo'lgan?'\\
A) 3,24\\
B) 2,88\\
C) 1,26\\
D) 1,44
  \item $\mathrm{Cu}_{3} \mathrm{~N}+\mathrm{HNO}_{3} \rightarrow \mathrm{Cu}\left(\mathrm{NO}_{3}\right)_{2}+\mathrm{NO}_{2}+\mathrm{H}_{2} \mathrm{O}$ ushbu reaksiya bo'yicha oksidlovchi va qaytaruvchining massa farqi $8,02 \mathrm{~g}$ bo'lsa, necha g suv hosil bo'lgan?\\
A) 3,24\\
B) 1,62\\
C) 2,88\\
D) 1,44
  \item $\mathrm{Cu}_{3} \mathrm{~N}+\mathrm{HNO}_{3} \rightarrow \mathrm{Cu}\left(\mathrm{NO}_{3}\right)_{2}+\mathrm{NO}+\mathrm{H}_{2} \mathrm{O}$ ushbu reaksiya bo'yicha oksidlovchi va qaytaruvchining massa farqi $2,13 \mathrm{~g}$ bo'lsa, necha g suv hosil bo'lgan?\\
A) 2,34\\
B) 4,58\\
C) 18\\
D) 9
  \item $\mathrm{Hg}_{3} \mathrm{~N}+\mathrm{HNO}_{3} \rightarrow \mathrm{Hg}\left(\mathrm{NO}_{3}\right)_{2}+\mathrm{NO}+\mathrm{H}_{2} \mathrm{O}$ ushbu reaksiya bo'yicha oksidlovchí va qaytaruvchining massa farqi $9,93 \mathrm{~g}$ bo'lsa, necha g suv hosil bo'lgan?\\
A) 2,34\\
B) 4,58\\
C) 18\\
D) 9
  \item Quyidagi moddalar orasidan kuchli elektrolitni tanlang.\\
A) NaCl\\
B) HF\\
C) HCN\\
D) $\mathrm{H}_{2} \mathrm{~S}$
2. Quyidagi moddalar orasidan kuchli elektrolitni tanlang.\\
A) $\mathrm{CaCO}_{3}$\\
B) $\mathrm{CH}_{3} \mathrm{COOH}$\\
C) HCl\\
D) $\mathrm{Fe}(\mathrm{OH})_{2}$\\
3. Quyidagi moddalar orasidan kuchli elektrolitni tanlang.\\
A) $\mathrm{CaF}_{2}$\\
B) $\mathrm{Al}_{2}\left(\mathrm{SO}_{4}\right)_{3}$\\
C) $\mathrm{Cr}(\mathrm{OH})_{3}$\\
D) $\mathrm{H}_{2} \mathrm{SO}_{3}$\\
4. Quyidagi moddalar orasidan kuchli elektrolitni tanlang.\\
A) AgCl\\
B) $\mathrm{H}_{2} \mathrm{O}$\\
C) $\mathrm{CaSO}_{4}$\\
D) KOH\\
5. Quyidagi moddalar orasidan kuchsiz elektrolitni tanlang.\\
A) NaCl\\
B) HF\\
C) NaCN\\
D) $\mathrm{K}_{2} \mathrm{~S}$\\
6. Quyidagi moddalar orasidan kuchsiz elektrolitni tanlang.\\
A) KOH\\
B) $\mathrm{MgSO}_{4}$\\
C) HCN\\
D) $\mathrm{H}_{2} \mathrm{SO}_{4}$\\
7. Quyidagi moddalar orasidan kuchsiz elektrolitni tanlang.\\
A) NaCl\\
B) KF\\
C) AgF\\
D) AgCl\\
8. Quyidagi moddalar orasidan noelektrolitni tanlang.\\
A) NaCl\\
B) HF\\
C) HCN\\
D) $\mathrm{C}_{6} \mathrm{H}_{12} \mathrm{O}_{6}$\\
9. Quyidagi moddalar orasidan noelektrolitni tanlang.\\
A) NaCl\\
B) HF\\
C) $\mathrm{C}_{5} \mathrm{H}_{12}$\\
D) $\mathrm{H}_{2} \mathrm{~S}$\\
10. Quyidagi moddalar orasidan noelektrolitni tanlang.\\
A) NaCl\\
B) HF\\
C) HCN\\
D) $\mathrm{C}_{6} \mathrm{H}_{6}$
  \item Eritmada $\mathrm{K}_{2} \mathrm{SO}_{1}$ ning 150 ta dissotsiolangan va 50 ta dissotsialanmagan molekulalari bo'lsa, dissotsiolanish darajasini \% da toping.\\
A) 75\\
B) 70\\
C) 65\\
D) 60
  \item Eritmada $\mathrm{H}_{2} \mathrm{SO}_{4}$ ning 180 ta dissotsiolangan va 20 ta dissotsialanmagan molekulalar bo'lsa, dissotsiolanish darajasini \% da toping.\\
A) 80\\
B) 70\\
C) 50\\
D) 90
  \item Eritmada NaCl ning 100 ta dissotsiolangan va 100 ta dissotsialanmagan molekulalar bo'lsa, dissotsiolanish darajasini \% da toping.\\
A) 80\\
B) 70\\
C) 50\\
D) 90
14 Eritma LiCl ning 90 ta dissotsiolangan va 110 ta dissotsialanmagan molekulalar bo'lsa, dissotsiolanish darajasini \% da toping.\\
A) 55\\
B) 40\\
C) 45\\
D) 60\\
15. Eritmada $\mathrm{KNO}_{3}$ ning 80 ta dissotsiolangan va 120 ta dissotsialanmagan molekulalar bo'lsa, dissotsiolanish darajasini \% da toping.\\
A) 55\\
B) 40\\
C) 45\\
D) 60\\
16. Eritmada $\mathrm{Na}_{3} \mathrm{PO}_{4}$ ning 70 ta dissotsiolangan va 30 ta dissotsialanmagan molekulalar bo'lsa, dissotsiolanish darajasini \% da toping.\\
A) 75\\
B) 70\\
C) 65\\
D) 60\\
17. Eritmada $\mathrm{HNO}_{3}$ ning 150 ta dissotsiolangan va 50 ta dissotsialanmagan molekulalar bo'lsa, dissotsiolanish darajasini \% da toping.\\
A) 75\\
B) 70\\
C) 65\\
D) 60\\
18. Eritmada $\mathrm{Al}\left(\mathrm{NO}_{3}\right)_{3}$ ning 160 ta dissotsialangan va 40 ta dissotsialanmagan molekulalar bo'lsa, dissotsiolanish darajasini \% da toping.\\
A) 80\\
B) 70\\
C) 50\\
D) 90\\
19. Eritmada KI ning 130 ta dissotsiolangan va 70 ta\\
dissotsialanmagan molekulalar bo'lsa, dissotsiolanish darajasini \% da toping.\\
A) 75\\
B) 70\\
C) 65\\
D) 60\\
20. Eritmada HBr ning 110 ta dissotsiolangan va 90 ta dissotsialanmagan molekulalar bo'lsa, dissotsiolanish darajasini \% da toping.\\
A) 55\\
B) 40\\
C) 45\\
D) 60
  \item $\mathrm{K}_{2} \mathrm{SO}_{4}$ eritmasida 120 ta ion bo'lsa, dissotsialanmagan molekulalar sonini toping. ( $\mathrm{a}=80 \%$ ) (Suvning dissotsialanishi hisobga olinmasin)\\
A) 10\\
B) 20\\
C) 30\\
D) 40
  \item $\mathrm{H}_{2} \mathrm{SO}_{4}$ eritmasida 180 ta ion bo'lsa, dissotsialanmagan molekulalar sonini toping. $\alpha=60 \%$ (Suvning dissotsialanishi hisobga olinmasin)\\
A) 70\\
B) 60\\
C) 50\\
D) 40
  \item NaCl eritmasida 100 ta ion bo'lsa, dissotsialanmagan molekulalar sonini toping. $\alpha=50 \%$ (Suvning dissotsialanishi hisobga olinmasin)\\
A) 80\\
B) 70\\
C) 50\\
D) 90
  \item LiCl eritmasida 150 ta ion bo'lsa, dissotsialanmagan molekulalar sonini toping. $\alpha=75 \%$ (Suvning dissotsialanishi hisobga olinmasin)\\
A) 55\\
B) 40\\
C) 25\\
D) 60
  \item $\mathrm{KNO}_{3}$ eritmasida 130 ta ion bo'lsa, dissotsialanmagan molekulalar sonini toping. $\mathrm{a}=65 \%$ (Suvning dissotsialanishi hisobga olinmasin)\\
A) 35\\
B) 40\\
C) 45\\
D) 50
  \item $\mathrm{Na}_{3} \mathrm{PO}_{4}$ eritmasida 120 ta ion bo'lsa, dissotsialanmagan molekulalar sonini toping. $\alpha=60 \%$ (Suvning dissotsialanishi hisobga olinmasin)\\
A) 25\\
B) 20\\
C) 35\\
D) 40
  \item $\mathrm{HNO}_{3}$ eritmasida 140 ta ion bo'lsa, dissotsialanmagan molekulalar sonini toping. $\alpha=70 \%$ (Suvning dissotsialanishi hisobga olinmasin)\\
A) 45\\
B) 40\\
C) 35\\
D) 30
  \item $\mathrm{Al}\left(\mathrm{NO}_{3}\right)_{3}$ eritmasida 160 ta ion bo'lsa, dissotsialanmagan molekulalar sonini toping. $\alpha=80 \%$ (Suvning dissotsialanishi hisobga olinmasin)\\
A) 10\\
B) 20\\
C) 30\\
D) 40
  \item KI eritmasida 110 ta ion bo'lsa, dissotsialanmagan molekulalar sonini toping. $\mathrm{a}=55 \%$ (Suvning dissotsialanishi hisobga olinmasin)\\
A) 55\\
B) 30\\
C) 45\\
D) 60
  \item HBr eritmasida 150 ta ion bo'lsa, dissotsialanmagan molekulalar sonini toping. $\alpha=75 \%$ (Suvning dissotsialanishi hisobga olinmasin)\\
A) 25\\
B) 20\\
C) 35\\
D) 40
  \item $\mathrm{K}_{2} \mathrm{SO}_{4}$ ning dissosialanmagan molekulalari soni 80 ta bo'lsa, eritmadagi ionlar sonini hisoblang. $a=75 \%$ (Suvning dissotsialanishi hisobga olinmasin)\\
A) 25\\
B) 240\\
C) 320\\
D) 720
  \item $\mathrm{H}_{2} \mathrm{SO}_{4}$ ning dissosialanmagan molekulalari soni 120 ta bo'lsa, eritmadagi ionlar sonini hisoblang. $\alpha=60 \%$ (Suvning dissotsialanishi hisobga olinmasin)\\
A) 540\\
B) 600\\
C) 550\\
D) 440
  \item NaCl ning dissosialanmagan molekulalari soni 27 ta bo'lsa, eritmadagi ionlar sonini hisoblang. $\alpha=90 \%$ (Suvning dissotsialanishi hisobga olinmasin)\\
A) 514\\
B) 536\\
C) 486\\
D) 444
  \item LiCl ning dissosialanmagan molekulalari soni 25 ta bo'lsa, eritmadagi ionlar sonini hisoblang. $a=75 \%$ (Suvning dissotsialanishi hisobga olinmasin)\\
A) 155\\
B) 140\\
C) 125\\
D) 150
35. $\mathrm{KNO}_{3}$ ning dianosialmamagan molokulalnri aoni 70 ta bolan,oritmadagi ionlar sonini hisoblang,$a=00 \%$(Suvning disaotajalanishi himobga olinmagin)\\
A) 250\\
B) 140\\
C) 210\\
D) 250
36. $\mathrm{Na}_{3} \mathrm{PO}_{4}$ ning dísgosínlanmagan molokulalari soni 120 ta bo'lsa,oritmadagj ionlar sonini hisoblang,$a=40 \%$(Suvning dissotsialanish\\
B i hísobga olinmasin)\\
A) 325\\
B) 320\\
C) 335\\
D) 340
37. $\mathrm{HNO}_{3}$ ning dissobialanmagan molekulalari soni 140 ta bo'lsa,eritmadagi ionlar sonini hisoblang.$a=65 \%$(Suvning dissotsialanisbi hisobga olínmasin)\\
A) 520\\
B) 540\\
C) 535\\
D) 530
38. $\mathrm{Al}\left(\mathrm{NO}_{3}\right)_{3}$ ning díssosialanmagan molekulalari soni 40 ta bo'lsa,eritmadagi ionlar sonini hisoblang.$a=75 \%$(Suvning díssotsialanishí hísobga olinmasin)\\
A) 510\\
B) 420\\
C) 480\\
D) 340
39.KI ning díssosialanmagan molekulalari soni 60 ta bo'lsa,eritmadagi ionlar sonini hisoblang,$\alpha=85 \%$(Suvning dissotsialanishí hísobga olinmasin)\\
A) 555\\
B) 680\\
C) 445\\
D) 640
40. HBr ning dissosialanmagan molekulalari soni 80 ta bo'lsa,erítmadagí ionlar sonini hísoblang.$a=60 \%$(Suvning dissotsialanishí hísobga olinmasin)\\
A) 225\\
B) 220\\
C) 235\\
D) 240
41. $\mathrm{Al}_{2}\left(\mathrm{SO}_{4}\right)_{3}$ eritmasida dissotsilanmagan molekulalar soni 70 ta bo'lsa,nechta sulfat ioní eritmada mavjud.$\alpha=65 \%$(Suv molekulalari soni va dissotsialanishí hisobga olinmasin)\\
A) 390\\
B) 350\\
C) 210\\
D) 140
42. $\mathrm{H}_{2} \mathrm{HO}_{1}$ critmosids discretailsnamagen molegulsalar soní 40 to bologi,nechese sulfat ioní eritmada maxviud,$x=\psi\left(y^{\prime}\right) \%$(先 $x y$ molekulalari somi vad diskutesivilynishi hissobga olinmasin)\\
A) 1.40\\
B) 2010\\
C) $1(8)$\\
D) 240
13.NaCl sritmasida dismatrilannaysorn molekulalar aoní 30 ta boyasa,nsohtata ylorid ioni eritmada mavjud,$\alpha=75 \% \%$( 5 fuy molekulalari sonj vas disseyatsiajanishi hisobga olinmasín)\\
A) 60\\
B) 90\\
C) 50\\
D) 70
44.LiCl eritmasida disectsilanmagan molekulalar roni 40 ta bo'lsa,nechta Jítíy ioní eritmada mavjud,$a=80 \%$(Suv molekulalarí soní va dissotsíalanishí hisobga olinmasin)\\
A) 155\\
B) 150\\
C) 125\\
D) 160
45. $\mathrm{KNO}_{3}$ erítmasida dissotsilanmagan molekulalar soni 35 ta bo'lsa,nechta nítrat ioni eritmada mavjud.$\alpha=65 \%$(Suv molekulalari soní va discotsialanishí hisobga olinmasin)\\
A) 65\\
B) 70\\
C) 85\\
D) 75
46. $\mathrm{Na}_{3} \mathrm{PO}_{4}$ erítmasida dissotsílanmagan molekulalar soni 90 ta bo'lsa,nechta natriy ioni eritmada mavjud.$a=55 \%$(Suv molekulalari soní va diseotsialanishi hisobga olinmasin)\\
A) 330\\
B) 320\\
C) 335\\
D) 340

47. $\mathrm{HNO}_{3}$ erítmasida dissotsilanmagan molekulalar soni 70 ta bo'lsa,nechta vodorod ioni eritmada mavjud.$a=65 \%$ (Suv molekulalari soni va díssotsialanishí hisobga olinmasin)\\
A) 120\\
B) 140\\
C) 135\\
D) 130

48. $\mathrm{Al}\left(\mathrm{NO}_{3}\right)_{3}$ eritmasida dissotsilanmagan molekulalar soni 40 ta bo'lsa,nechta nítrat ioni erítmada mavjud.$a=75 \%$(Suv molekulalari soní va dissotsialanishí hisobga olinmasin)\\
A) 310\\
B) 360\\
C) 380\\
D) 340

49.KI erítmasida dissotsilanmagan molekulalar soni 80 ta bo'lsa,nechta yodid ioni erítmada mavjud. $\mathbf{a}=60 \%$(Suv\\
molekulalari soni va dissotsialanishi hisobga olinmasin)\\
A) 90\\
B) 100\\
C) 120\\
D) 130\\
50. HBr eritmasida dissotsilanmagan molekulalar soni 70 ta bo'lsa, nechta bromid ioni eritmada mavjud. $\alpha=65 \%$ (Suv molekulalari soni va dissotsialanishi hisobga olinmasin)\\
A) 125\\
B) 120\\
C) 135\\
D) 130
  \item 2,5 litr $0,4 \mathrm{M}$ li HCl eritmasida $3,01 \cdot 10^{23}$ ta $\mathrm{H}^{+}$ioni mavjud bo'lsa, dissotsiolanish darajasini \% da aniqlang. (Suvning dissotsialanishi hisobga olinmasin)\\
A) 60\\
B) 90\\
C) 50\\
D) 70
  \item 5 litr $0,4 \mathrm{M}$ li $\mathrm{AlCl}_{3}$ eritmasida $9,03 \cdot 10^{23}$ ta Cl ioni mavjud bo'lsa, dissotsiolanish darajasini \% da aniqlang. (Suvning dissotsialanishi hisobga olinmasin)\\
A) 25\\
B) 30\\
C) 50\\
D) 40
  \item 2 litr $0,5 \mathrm{M}$ li $\mathrm{K}_{2} \mathrm{SO}_{4}$ eritmasida $6,02 \cdot 10^{23}$ ta $\mathrm{K}^{+}$ioni mavjud bo'lsa, dissotsiolanish darajasini \% da aniqlang. (Suvning dissotsialanishi hisobga olinmasin)\\
A) 60\\
B) 90\\
C) 50\\
D) 70
  \item 5 litr $0,8 \mathrm{M}$ li $\mathrm{Na}_{2} \mathrm{SO}_{4}$ eritmasida $18,06 \cdot 10^{23}$ ta $\mathrm{SO}_{4}{ }^{2-}$ ioni mavjud bo'lsa, dissotsiolanish darajasini \% da aniqlang. (Suvning dissotsialanishi hisobga olinmasin)\\
A) 60\\
B) 80\\
C) 70\\
D) 75
  \item 4 litr $0,4 \mathrm{M}$ li $\mathrm{HNO}_{3}$ eritmasida $4,816 \cdot 10^{23}$ ta $\mathrm{H}^{+}$ioni mavjud bo'lsa, dissotsiolanish darajasini \% da aniqlang.\\
(Suvning dissotsialanishi hisobga olinmasin)\\
A) 60\\
B) 90\\
C) 50\\
D) 70
  \item 3,5 litr $0,6 \mathrm{M}$ li NaCl eritmasida $4,214 \cdot 10^{23}$ ta $\mathrm{Na}^{+}$ioni mavjud bo'lsa, dissotsiolanish darajasini \% da aniqlang. (Suvning dissotsialanishi hisobga olinmasin)\\
A) 33,3\\
B) 66,6\\
C) 55,5\\
D) 44,4
  \item 2 litr $0,5 \mathrm{M}$ li $\mathrm{H}_{2} \mathrm{SO}_{4}$ eritmasida $3,01 \cdot 10^{23}$ ta $\mathrm{H}^{+}$ioni mavjud bo'lsa, dissotsiolanish darajasini \% da aniqlang. (Suvning dissotsialaníshi hisobga olinmasin)\\
A) 25\\
B) 30\\
C) 50\\
D) 40
  \item 2,5 litr $0,8 \mathrm{M}$ li $\mathrm{HMnO}_{4}$ eritmasida $7,224 \cdot 10^{23}$ ta $\mathrm{H}^{+}$ioni mavjud bo'lsa, dissotsiolanish darajasini \% da aniqlang. (Suvning dissotsialanishi hisobga olinmasin)\\
A) 60\\
B) 90\\
C) 50\\
D) 70
  \item 8 litr $0,25 \mathrm{M}$ li $\mathrm{AlCl}_{3}$ eritmasida $8,428 \cdot 10^{23}$ ta $\mathrm{Al}^{+}$ioni mavjud bo'lsa, dissotsiolanish darajasini \% da aniqlang. (Suvning dissotsialanishi hisobga olinmasin)\\
A) 60\\
B) 90\\
C) 50\\
D) 70
  \item 10 litr $0,3 \mathrm{M}$ li HCl eritmasida $9,03 \cdot 10^{23}$ ta $\mathrm{H}^{+}$ioni mavjud bo'lsa, dissotsiolanish darajasini \% da aniqlang. (Suvning dissotsialanishi hisobga olinmasin)\\
A) 60\\
B) 90\\
C) 50\\
D) 70
  \item Quyidagi moddalardan qaysi biri dissotsialansa eng ko'p ion hosil bo'ladi? ( $a=100 \%$ )\\
A) NaCl\\
B) $\mathrm{MgSO}_{4}$\\
C) $\mathrm{AlCl}_{3}$\\
D) $\mathrm{K}_{2} \mathrm{SO}_{4}$
  \item Quyidagi moddalardan qaysi biri dissotsialansa eng ko'p ion hosil bo'ladi? ( $\mathrm{a}=100 \%$ )\\
A) $\mathrm{Na}_{3} \mathrm{PO}_{4}$\\
B) $\mathrm{Al}_{2}\left(\mathrm{SO}_{4}\right)_{3}$\\
C) HCl\\
D) $\mathrm{Na}_{2} \mathrm{SO}_{4}$
  \item Quyidagi moddalardan qaysi biri dissotsialansa eng ko'p ion hosil bo'ladi? ( $\alpha=100 \%$ )\\
A) $\mathrm{CrCl}_{3}$\\
B) $\mathrm{H}_{2} \mathrm{SO}_{4}$\\
C) LiCl\\
D) $\mathrm{KMnO}_{4}$
  \item Quyidagi moddalardan qaysi biri dissotsialansa eng ko'p ion hosil bo'ladi? ( $a=100 \%$ )\\
A) $\mathrm{MgSO}_{4}$\\
B) $\mathrm{NaMnO}_{4}$\\
C) $\mathrm{CaCl}_{2}$\\
D) $\mathrm{Al}\left(\mathrm{MnO}_{4}\right)_{3}$
  \item Quyidagi moddalardan qaysi biri dissotsialansa eng ko'p ion hosil bo'ladi? ( $a=100 \%$ )\\
A) $\mathrm{HNO}_{3}$\\
B) $\mathrm{MgSO}_{4}$\\
C) $\mathrm{Cr}\left(\mathrm{NO}_{3}\right)_{3}$\\
D) $\mathrm{K}_{2} \mathrm{SO}_{4}$
  \item Quyidagi moddalardan qaysi biri dissotsialansa eng ko'p ion hosil bo'ladi? ( $\alpha=100 \%$ )\\
A) $\mathrm{K}_{3} \mathrm{PO}_{4}$\\
B) $\mathrm{Li}_{2} \mathrm{SO}_{4}$\\
C) $\mathrm{BaCl}_{2}$\\
D) $\mathrm{K}_{2} \mathrm{SO}_{4}$
  \item Quyidagi moddalardan qaysi biri dissotsialansa eng ko'p ion hosil bo'ladi? ( $\mathrm{a}=100 \%$ )\\
A) $\mathrm{CrCl}_{2}$\\
B) $\mathrm{FeCl}_{3}$\\
C) $\mathrm{HNO}_{2}$\\
D) KOH
  \item Quyidagi moddalardan qaysi biri dissotsialansa eng ko'p ion hosil bo'ladi? ( $\alpha=100 \%$ )\\
A) NaOH\\
B) $\mathrm{MgI}_{2}$\\
C) NaCl\\
D) KOH
  \item Quyidagi moddalardan qaysi biri dissotsialansa eng ko'p ion hosil bo'ladi? ( $\mathrm{a}=100 \%$ )\\
A) NaCl\\
B) $\mathrm{MgSO}_{4}$\\
C) $\mathrm{AlCl}_{3}$\\
D) $\mathrm{Al}_{2}\left(\mathrm{SO}_{4}\right)_{3}$
  \item Quyidagi moddalardan qaysi biri dissotsialansa eng ko'p ion hosil bo'ladi? ( $a=100 \%$ )\\
A) HCl\\
B) KOH\\
C) $\mathrm{Ba}\left(\mathrm{MnO}_{4}\right)_{2}$\\
D) $\mathrm{MgSO}_{4}$
  \item Quyidagi reaksiyaning ionli tenglamasini molekulyar shaklda yozish uchun keltírilgan ion juftlarning qaysilaridan foydalanish mumkin?\\
$3 \mathrm{Ca}^{2+}+2 \mathrm{PO}_{4}{ }^{3 \cdot} \rightarrow \mathrm{Ca}_{3}\left(\mathrm{PO}_{4}\right)_{2}$\\
A) $\mathrm{CH}_{3} \mathrm{COO}$ va $\mathrm{Na}^{+}$\\
B) $\mathrm{NO}_{3}{ }^{-} \mathrm{va} \mathrm{Li}^{+}$\\
C) $\mathrm{Cl} \cdot$ va $\mathrm{Li}^{+}$\\
D) $\mathrm{SO}_{4}{ }^{2-}$ va $\mathrm{NH}_{4}{ }^{+}$\\
  \item Quyidagi reaksiyaning ionli tenglamasini molekulyar shaklda yozish uchun keltirilgan ion juftlarning qaysilaridan foydalanish mumkin? $\mathrm{Ca}^{2+}+\mathrm{CO}_{3}{ }^{2-} \rightarrow \mathrm{CaCO}_{3}$\\
A) $\mathrm{Cl} \cdot$ va $\mathrm{Ba}^{2+}$\\
B) $\mathrm{NO}_{3}$ va $\mathrm{Mg}^{2+}$\\
C) $\mathrm{Cl}^{-}$va $\mathrm{Na}^{+}$\\
D) $\mathrm{PO}_{4}{ }^{3 \cdot}$ va $\mathrm{NH}_{4}{ }^{+}$
  \item Quyidagi reaksiyaning ionli tenglamasini molekulyar shaklda yozish uchun keltirilgan ion juftlarning qaysilaridan foydalanish mumkin? $\mathrm{CO}_{3}{ }^{2-}$ $+2 \mathrm{H}^{+} \rightarrow \mathrm{CO}_{2}+\mathrm{H}_{2} \mathrm{O}$\\
A) $\mathrm{Ba}^{2+}$ va $\mathrm{SiO}_{3}{ }^{2-}$\\
B) $\mathrm{Ca}^{2+}$ va $\mathrm{SiO}_{3}{ }^{2-}$\\
C) $\mathrm{Ca}^{2+}$ va $\mathrm{Li}^{+}$\\
D) $\mathrm{Na}^{+}$va $\mathrm{Cl} \cdot$
  \item Quyidagi reaksiyaning ionli tenglamasini molekulyar shaklda yozish uchun keltirilgan ion juftlarning qaysilaridan foydalanish mumkin? $\mathrm{Ag}^{+}+\mathrm{Cl} \rightarrow \mathrm{AgCl}$\\
A) I va $\mathrm{Na}^{+}$\\
B) $\mathrm{Br}^{-}$va $\mathrm{Li}^{+}$\\
C) $\mathrm{SO}_{3}{ }^{2-}$ va $\mathrm{Li}^{+}$\\
D) $\mathrm{NO}_{3}^{-}$va $\mathrm{Na}^{+}$
  \item Quyidagi reaksiyaning ionli tenglamasini molekulyar shaklda yozish uchun keltirilgan ion juftlarning qaysilaridan foydalanish mumkin? $3 \mathrm{Ba}^{2+}+2 \mathrm{PO}_{4}{ }^{3-} \rightarrow \mathrm{Ba}_{3}\left(\mathrm{PO}_{4}\right)_{2}$\\
A) $\mathrm{CH}_{3} \mathrm{COO}^{-}$va $\mathrm{Na}^{+}$\\
B) $\mathrm{NO}_{3}{ }^{-}$va $\mathrm{Li}^{+}$\\
C) $\mathrm{Cl}^{-}$va $\mathrm{Li}^{+}$\\
D) $\mathrm{SO}_{4}{ }^{2-}$ va $\mathrm{NH}_{4}{ }^{+}$
  \item Quyidagi reaksiyaning ionli tenglamasini molekulyar shaklda yozish uchun keltirilgan ion juftlarning qaysilaridan foydalanish mumkin? $\mathrm{Fe}^{2+}+2 \mathrm{OH}^{-} \rightarrow \mathrm{Fe}(\mathrm{OH})_{2}$\\
A) $\mathrm{PO}_{4}{ }^{3 \cdot}$ va $\mathrm{Na}^{+}$\\
B) $\mathrm{NO}_{3}{ }^{-}$va $\mathrm{Li}^{+}$\\
C) $\mathrm{F}^{-} \mathrm{va} \mathrm{Li}^{+}$\\
D) $\mathrm{SO}_{3}{ }^{2-}$ va $\mathrm{NH}_{4}{ }^{+}$
  \item Quyidagi reaksiyaning ionli tenglamasini molekulyar shaklda yozish uchun keltirilgan ion juftlarning qaysilaridan foydalanish mumkin? $\mathrm{Cu}^{2+}+\mathrm{CO}_{3}{ }^{2-} \rightarrow \mathrm{CuCO}_{3}$\\
A) $\mathrm{CH}_{3} \mathrm{COO}$ va $\mathrm{Na}^{+}$\\
B) $\mathrm{OH}^{-} \mathrm{va} \mathrm{Li}^{+}$\\
C) $\mathrm{F}^{-}$va $\mathrm{Li}^{+}$\\
D) $\mathrm{SO}_{3}{ }^{2-}$ va $\mathrm{NH}_{4}{ }^{+}$
  \item Quyidagi reaksiyaning ionli tenglamasini molekulyar shaklda yozish\\
uchun keltirilgan ion juftlarning qaysilaridan foydalanish mumkin? $\mathrm{SiO}_{3}{ }^{2-}$ $+2 \mathrm{H}^{+} \rightarrow \mathrm{SiO}_{2}+\mathrm{H}_{2} \mathrm{O}$\\
A) $\mathrm{Ag}^{+}$va Cl\\
B) $\mathrm{Na}^{+}$va $\mathrm{Cl}^{-}$\\
C) $\mathrm{Al}^{3+}$ va $\mathrm{NO}_{3}$.\\
D) $\mathrm{Ca}^{2+}$ va $\mathrm{NH}_{4}{ }^{+}$
  \item Quyidagi reaksiyaning ionli tenglamasini molekulyar shaklda yozish uchun keltirilgan ion juftlarning qaysilaridan foydalanish mumkin? $\mathrm{Fe}^{3+}+3 \mathrm{OH} \rightarrow \mathrm{Fe}(\mathrm{OH})_{3}$\\
A) $\mathrm{F}^{\cdot}$ va $\mathrm{Na}^{+}$\\
C) $\mathrm{Cl}^{-}$va Li\\
B) $\mathrm{NO}_{3}{ }^{\circ}$ va $\mathrm{Zn}^{2+}$\\
$\mathrm{C}^{+}$\\
D) $\mathrm{SO}_{4}{ }^{2-}$ va $\mathrm{Ag}^{+}$
  \item Quyidagi reaksiyaning ionli tenglamasini molekulyar shaklda yozish uchun keltirilgan ion juftlarning qaysilaridan foydalanish mumkin? $3 \mathrm{Mg}^{2+}+2 \mathrm{PO}_{4}{ }^{3-} \rightarrow \mathrm{Mg}_{3}\left(\mathrm{PO}_{4}\right)_{2}$\\
A) $\mathrm{CH}_{3} \mathrm{COO} \mathrm{va} \mathrm{Li}^{+}$\\
B) $\mathrm{NO}_{3}{ }^{-} \mathrm{va} \mathrm{Li}^{+}$\\
C) $\mathrm{Cl}^{\cdot} \mathrm{va} \mathrm{Li}^{+}$\\
D) $\mathrm{SO}_{4}{ }^{2-}$ va $\mathrm{Na}^{+}$
  \item $0,05 \mathrm{M}$ li $\mathrm{H}_{2} \mathrm{SO}_{4}$ eritmasining pOH qiymatini aniqlang. ( $\alpha=100 \%$ )\\
A) 13\\
B) 9\\
C) 12\\
D) 11
  \item $0,005 \mathrm{M} \mathrm{li} \mathrm{H}_{2} \mathrm{CrO}_{4}$ eritmasinîng pOH qiymatini aniqlang. ( $\alpha=100 \%$ )\\
A) 13\\
B) 9\\
C) 12\\
D) 11
  \item $0,0001 \mathrm{M}$ li HCl eritmasining pOH qiymatini aniqlang. ( $\mathrm{a}=100 \%$ )\\
A) 13\\
B) 10\\
C) 12\\
D) 7
  \item $0,01 \mathrm{M} \mathrm{li} \mathrm{HCl}$ eritmasining pOH qiymatini aniqlang. ( $\mathrm{a}=100 \%$ )\\
A) 13\\
B) 9\\
C) 12\\
D) 11
  \item $0,000333 \mathrm{M}$ li $\mathrm{H}_{3} \mathrm{PO}_{4}$ eritmasining pOH qiymatini aniqlang. ( $\alpha=100 \%$ )\\
A) 13\\
B) 9\\
C) 12\\
D) 11
  \item $0,05 \mathrm{M}$ li $\mathrm{Ba}(\mathrm{OH})_{2}$ eritmasining pH qiymatini aniqlang. ( $\alpha=100 \%$ )\\
A) 13\\
B) 9\\
C) 12\\
D) 11
  \item $0,00001 \mathrm{M}$ li NaOH eritmasining pOH qiymatini aniqlang. ( $\alpha=100 \%$ )\\
A) 9\\
B) 4\\
C) 5\\
D) 11
  \item $0,05 \mathrm{M} \mathrm{li} \mathrm{H}_{2} \mathrm{SO}_{4}$ eritmasining pH qiymatini aniqlang. ( $\alpha=100 \%$ )\\
A) 13\\
B) 9\\
C) 2\\
D) 1
  \item $0,005 \mathrm{M} \mathrm{li} \mathrm{Ba}(\mathrm{OH})_{2}$ eritmasining pOH qiymatini aniqlang. ( $\alpha=100 \%$ )\\
A) 12\\
B) 9\\
C) 2\\
D) 1
  \item $0,0005 \mathrm{M}$ li $\mathrm{H}_{2} \mathrm{SO}_{4}$ eritmasining pH qiymatini aniqlang. ( $\alpha=100 \%$ )\\
A) 3\\
B) 9\\
C) 2\\
D) 11
  \item pOH qiymati 13 bo'lgan 10 litr $\mathrm{H}_{2} \mathrm{SO}_{4}$ eritmasidagi dissotsialanmagan $\mathrm{H}_{2} \mathrm{SO}_{4}$ molekulalar sonini hisoblang. ( $\alpha=80 \%$ )\\
A) $0,125 \cdot \mathrm{~N}_{\mathrm{A}}$\\
B) $0,3 \cdot \mathrm{~N}_{\mathrm{A}}$\\
C) $2 \cdot N_{\mathrm{A}}$\\
D) $1 \cdot \mathrm{~N}_{\mathrm{A}}$
  \item pOH qiymati 12 bo'lgan 20 litr HCl eritmasidagi dissotsialanmagan HCl molekulalar sonini hisoblang. ( $\alpha=50 \%$ )\\
A) $0,1 \cdot \mathrm{~N}_{\mathrm{A}}$\\
B) $0,02 \cdot \mathrm{~N}_{\mathrm{A}}$\\
C) $0,2 \cdot \mathrm{~N}_{\mathrm{A}}$\\
D) $1,5 \cdot \mathrm{~N}_{\mathrm{A}}$
  \item pOH qiymati 11 bo'lgan 100 litr $\mathrm{HNO}_{3}$ eritmasidagi dissotsialanmagan $\mathrm{HNO}_{3}$ molekulalar sonini hisoblang. ( $\alpha=80 \%$ )\\
A) $0,125 \cdot \mathrm{~N}_{\mathrm{A}}$\\
B) $0,025 \cdot \mathrm{~N}_{\mathrm{A}}$\\
C) $0,2 \cdot \mathrm{~N}_{\Lambda}$\\
D) $0,1 \cdot \mathrm{~N}_{\mathrm{A}}$
  \item pH qiymati 1 bo'lgan 20 litr $\mathrm{H}_{2} \mathrm{SO}_{4}$ eritmasidagi dissotsialanmagan $\mathrm{H}_{2} \mathrm{SO}_{4}$ molekulalar sonini hisoblang. ( $\mathrm{a}=80 \%$ )\\
A) $0,25 \cdot \mathrm{~N}_{\mathrm{A}}$\\
B) $0,31 \cdot \mathrm{~N}_{\mathrm{A}}$\\
C) $2 \cdot \mathrm{~N}_{\mathrm{A}}$\\
D) $1 \cdot N_{A}$
  \item pH qiymati 2 bo'lgan 7 litr $\mathrm{HNO}_{3}$ eritmasidagi dissotsialanmagan $\mathrm{HNO}_{3}$ molekulalar sonini hisoblang. ( $\mathrm{a}=70 \%$ )\\
A) $0,125 \cdot \mathrm{~N}_{\Lambda}$\\
B) $0,3 \cdot \mathrm{~N}_{\mathrm{A}}$\\
C) $0,03 \cdot \mathrm{~N}_{\mathrm{A}}$\\
D) $0,01 \cdot \mathrm{~N}_{\mathrm{A}}$
  \item pH qiymati 13 bo'lgan 10 litr NaOH eritmasidagi dissotsialanmagan NaOH molekulalar sonini hisoblang. ( $\alpha=80 \%$ )\\
A) $0,25 \cdot \mathrm{~N}_{\mathrm{A}}$\\
B) $0,3 \cdot \mathrm{~N}_{\mathrm{A}}$\\
C) $2 \cdot N_{A}$\\
D) $1 \cdot N_{A}$
  \item pH qiymati 12 bo'lgan 12 litr $\mathrm{Ba}(\mathrm{OH})_{2}$ eritmasidagi dissotsialanmagan $\mathrm{Ba}(\mathrm{OH})_{2}$ molekulalar sonini hisoblang. ( $\alpha=60 \%$ )\\
A) $0,25 \cdot \mathrm{~N}_{\mathrm{A}}$\\
B) $0,03 \cdot \mathrm{~N}_{\mathrm{A}}$\\
C) $2 \cdot N_{A}$\\
D) $0,04 \cdot \mathrm{~N}_{\mathrm{A}}$
  \item pH qiymati 11 bo'lgan 100 litr $\mathrm{Ba}(\mathrm{OH})_{2}$ eritmasidagi dissotsialanmagan $\mathrm{Ba}(\mathrm{OH})_{2}$ molekulalar sonini hisoblang. ( $\alpha=50 \%$ )\\
A) $0,05 \cdot \mathrm{~N}_{\mathrm{A}}$\\
B) $0,2 \cdot \mathrm{~N}_{\mathrm{A}}$\\
C) $2 \cdot N_{A}$\\
D) $1 \cdot \mathrm{~N}_{\mathrm{A}}$
  \item pOH qiymati 1 bo'lgan 10 litr NaOH eritmasidagi dissotsialanmagan NaOH molekulalar sonini hisoblang. ( $\alpha=80 \%$ )\\
A) $0,125 \cdot \mathrm{~N}_{\mathrm{A}}$\\
B) $0,25 \cdot \mathrm{~N}_{\mathrm{A}}$\\
C) $0,2 \cdot \mathrm{~N}_{\mathrm{A}}$\\
D) $0,1 \cdot \mathrm{~N}_{\mathrm{A}}$
  \item pOH qiymati 3 bo'lgan 100 litr $\mathrm{Ba}(\mathrm{OH})_{2}$ eritmasidagi dissotsialanmagan $\mathrm{Ba}(\mathrm{OH})_{2}$ molekulalar sonini hisoblang. ( $\mathrm{a}=50 \%$ )\\
A) $0,125 \cdot \mathrm{~N}_{\mathrm{A}}$\\
B) $0,25 \cdot \mathrm{~N}_{\mathrm{A}}$\\
C) $0,05 \cdot \mathrm{~N}_{\mathrm{A}}$\\
D) $1 \cdot \mathrm{~N}_{\mathrm{A}}$
  \item Quyidagi tuzlarning qaysi biri gidrolizga uchraydi?\\
A) NaCl\\
B) $\mathrm{Ca}\left(\mathrm{NO}_{3}\right)_{2}$\\
C) $\mathrm{Al}\left(\mathrm{NO}_{3}\right)_{3}$\\
D) $\mathrm{BaCl}_{2}$
2. Quyidagi tuzlarning qaysi biri gidrolizga uchraydi?\\
A) NaCN\\
B) $\mathrm{CaCl}_{2}$\\
C) $\mathrm{Ba}\left(\mathrm{NO}_{3}\right)_{2}$\\
D) $\mathrm{K}_{2} \mathrm{SO}_{4}$\\
3. Quyidagi tuzlarning qaysi biri gidrolizga uchraydi?\\
A) $\mathrm{NaNO}_{3}$\\
B) $\mathrm{MgSO}_{4}$\\
C) $\mathrm{BaSO}_{4}$\\
D) $\mathrm{CaSO}_{4}$\\
4. Quyidagi tuzlarning qaysi biri gidrolizga uchraydi?\\
A) $\mathrm{LiNO}_{3}$\\
B) $\mathrm{Ca}\left(\mathrm{NO}_{3}\right)_{2}$\\
C) KCl\\
D) $\mathrm{Al}_{2} \mathrm{~S}_{3}$\\
5. Quyidagi tuzlarning qaysi biri gidrolizga uchraydi?\\
A) $\mathrm{Na}_{2} \mathrm{CrO}_{7}$\\
B) $\mathrm{Ca}\left(\mathrm{CH}_{3} \mathrm{COO}\right)_{2}$\\
C) $\mathrm{KMnO}_{4}$\\
D) $\mathrm{SrCl}_{2}$\\
6. Quyidagi tuzlarning qaysi biri gidrolizga uchraydi?\\
A) $\mathrm{CH}_{3} \mathrm{COONa}$\\
B) $\mathrm{Ca}\left(\mathrm{NO}_{3}\right)_{2}$\\
C) $\mathrm{Sr}\left(\mathrm{NO}_{3}\right)_{2}$\\
D) $\mathrm{BaCl}_{2}$\\
7. Quyidagi tuzlarning qaysi biri gidrolizga uchraydi?\\
A) KCl\\
B) $\mathrm{CaC}_{2}$\\
C) $\mathrm{RbNO}_{3}$\\
D) $\mathrm{BaCl}_{2}$\\
8. Quyidagi tuzlarning qaysi biri gidrolizga uchraydi?\\
A) $\mathrm{K}_{2} \mathrm{SO}_{4}$\\
B) $\mathrm{Ca}\left(\mathrm{NO}_{3}\right)_{2}$\\
C) $\mathrm{Cr}\left(\mathrm{NO}_{3}\right)_{3}$\\
D) $\mathrm{SrCl}_{2}$\\
9. Quyidagi tuzlarning qaysi biri gidrolizga uchraydi?\\
A) NaCl\\
B) $\mathrm{Ca}\left(\mathrm{NO}_{3}\right)_{2}$\\
C) $\mathrm{BaCl}_{2}$\\
D) $\mathrm{Ba}\left(\mathrm{NO}_{2}\right)_{2}$\\
10. Quyidagi tuzlarning qaysi biri gidrolizga uchraydi?\\
A) $\mathrm{NaNO}_{3}$\\
B) $\mathrm{Ca}\left(\mathrm{NO}_{3}\right)_{2}$\\
C) $\mathrm{NH}_{4} \mathrm{Cl}$\\
D) $\mathrm{BaCl}_{2}$
  \item Quyidagi tuzlarning qaysi biri anion bo'yicha gidrolizga uchraydi?\\
A) $\mathrm{NH}_{4} \mathrm{Cl}$\\
B) $\mathrm{Ca}\left(\mathrm{NO}_{3}\right)_{2}$\\
C) $\mathrm{Al}\left(\mathrm{NO}_{3}\right)_{3}$\\
D) $\mathrm{Ba}\left(\mathrm{NO}_{2}\right)_{2}$\\
  \item Quyidagi tuzlarning qaysi biri anion bo'yicha gidrolizga uchraydi?\\
A) $\mathrm{Cr}\left(\mathrm{NO}_{2}\right)_{3}$\\
B) $\mathrm{Ca}\left(\mathrm{CH}_{3} \mathrm{COO}\right)_{2}$\\
C) $\mathrm{CaCl}_{2}$\\
D) $\mathrm{Ba}\left(\mathrm{NO}_{2}\right)_{2}$
  \item Quyidagi tuzlarning qaysi biri kation bo'yicha gidrolizga uchraydi?\\
A) $\mathrm{Na}_{2} \mathrm{~S}$\\
B) $\mathrm{Ca}\left(\mathrm{NO}_{3}\right)_{2}$\\
C) $\mathrm{Al}\left(\mathrm{NO}_{3}\right)_{3}$\\
D) $\mathrm{Ba}\left(\mathrm{NO}_{2}\right)_{2}$
  \item Quyidagi tuzlarning qaysi biri kation bo'yicha gidrolizga uchraydi?\\
A) NaCN\\
B) $\mathrm{Al}\left(\mathrm{NO}_{2}\right)_{3}$\\
C) $\mathrm{MgSO}_{4}$\\
D) $\mathrm{K}_{2} \mathrm{SO}_{4}$
  \item Quyidagi tuzlarning qaysi biri kation bo'yicha gidrolizga uchraydi?\\
A) $\mathrm{Cr}\left(\mathrm{NO}_{3}\right)_{3}$\\
B) $\mathrm{Ca}\left(\mathrm{CH}_{3} \mathrm{COO}\right)_{2}$\\
C) $\mathrm{CaCl}_{2}$\\
D) $\mathrm{Ba}\left(\mathrm{NO}_{2}\right)_{2}$
  \item Quyidagi tuzlarning qaysi biri ham anion ham kation bo'yicha gidrolizga uchraydi?\\
A) $\mathrm{NH}_{4} \mathrm{Cl}$\\
B) $\mathrm{Ca}\left(\mathrm{NO}_{3}\right)_{2}$\\
C) $\mathrm{Al}\left(\mathrm{NO}_{2}\right)_{3}$\\
D) $\mathrm{Ba}\left(\mathrm{NO}_{2}\right)_{2}$
  \item Quyidagi tuzlarning qaysi biri ham anion ham kation bo'yicha gidrolizga uchraydi?\\
A) NaCN\\
B) $\mathrm{Al}\left(\mathrm{NO}_{2}\right)_{3}$\\
C) $\mathrm{MgSO}_{4}$\\
D) $\mathrm{K}_{2} \mathrm{SO}_{4}$
  \item Quyidagi tuzlarning qaysi biri ham anion ham kation bo'yicha gidrolizga uchraydi?\\
A) $\mathrm{Cr}\left(\mathrm{NO}_{2}\right)_{3}$\\
B) $\mathrm{Ca}\left(\mathrm{CH}_{3} \mathrm{COO}\right)_{2}$\\
C) $\mathrm{CaCl}_{2}$\\
D) $\mathrm{Ba}\left(\mathrm{NO}_{2}\right)_{2}$
  \item Quyidagi tuzlarning qaysi biri gidrolizga uchramaydi?\\
A) $\mathrm{NH}_{4} \mathrm{Cl}$\\
B) $\mathrm{Ca}\left(\mathrm{NO}_{3}\right)_{2}$\\
C) $\mathrm{Al}\left(\mathrm{NO}_{3}\right)_{3}$\\
D) $\mathrm{Ba}\left(\mathrm{NO}_{2}\right)_{2}$
  \item Quyidagi amallarning qaysi biri $\mathrm{Ca}\left(\mathrm{NO}_{2}\right)_{2}$ eritmasining gidrolizlanish jarayonini kuchaytiradi?\\
A) Eritmaga $\mathrm{Ca}\left(\mathrm{NO}_{2}\right)_{2}$ qo'shish\\
B) Eritmaga NaOH qo'shish\\
C) Eritmaga suv qo'shish\\
D) Eritmadagi suvni bug'latish
  \item Quyidagi amallarning qaysi biri $\mathrm{Al}_{2}\left(\mathrm{SO}_{4}\right)_{3}$ eritmasining gidrolizlanish jarayonini kuchaytiradi?\\
A) Eritmaga $\mathrm{ZnSO}_{4}$ qo'shish\\
B) Erimaga NaOH qo'shish\\
C) Eritmani sovitish\\
D) Eritmadagi suvni bug'latish
  \item Quyidagi amallarning qaysi biri $\mathrm{MgSO}_{4}$ eritmasining gidrolizlanish jarayonini kuchaytiradi?\\
A) Eritmaga HCl qo'shish\\
B) Eritmaga NaCl qo'shish\\
C) Eritmadagi suvni bug'latish\\
D) Eritmadagi KOH qo'shish
  \item Quyidagi amallarning qaysi biri NaCN eritmasining gidrolizlanish jarayonini kuchaytiradi?\\
A) Eritmaga HCl qo'shish\\
B) Eritmaga NaCl qo'shish\\
C) Eritmadagi suvni bug'latish\\
D) Eritmadagi KOH qo'shish
  \item Quyidagi amallarning qaysi biri $\mathrm{AlCl}_{3}$ eritmasining gidrolizlanish jarayonini kuchaytiradi?\\
A) Eritmaga $\mathrm{Ca}\left(\mathrm{NO}_{2}\right)_{2}$ qo'shish\\
B) Eritmaga HCl qo'shish\\
C) Eritmaga $\mathrm{AlCl}_{3}$ qo'shish\\
D) Eritmadagi suvni bug'latish
  \item Quyidagi amallarning qaysi biri $\mathrm{Ca}\left(\mathrm{NO}_{2}\right)_{2}$ eritmasining gidrolizlanish jarayonini susaytiradi?\\
A) Eritmaga $\mathrm{Ca}\left(\mathrm{NO}_{2}\right)_{2}$ qo'shish\\
B) Eritmaga HCl qo'shish\\
C) Eritmaga suv qo'shish\\
D) Eritmaga $\mathrm{ZnSO}_{4}$
  \item Quyidagi amallarning qaysi biri $\mathrm{AlCl}_{3}$ eritmasining gidrolizlanish jarayonini susaytiradi?\\
A) Eritmaga $\mathrm{Ca}\left(\mathrm{NO}_{3}\right)_{2}$ qo'shish\\
B) Eritmaga NaOH qo'shish\\
C) Eritmaga suv qo'shish\\
D) Eritmadagi suvni bug'latish
  \item Quyidagi amallarning qaysi biri $\mathrm{CH}_{3} \mathrm{COOK}$ eritmasining gidrolizlanish jarayonini susaytiradi?\\
A) Eritmaga $\mathrm{Ca}\left(\mathrm{NO}_{2}\right)_{2}$ qo'shish\\
B) Eritmaga NaOH qo'shish\\
C) Eritmaga suv qo'shish\\
D) Eritmadagi Eritmaga HCl qo'shish
  \item Quyidagi amallarning qaysi biri $\mathrm{NH}_{4} \mathrm{Cl}$ eritmasining gidrolizlanish jarayonini susaytiradi?\\
A) Eritmaga $\mathrm{Ca}\left(\mathrm{NO}_{2}\right)_{2}$ qo'shish\\
B) Eritmaga HCl qo'shish\\
C) Eritmaga suv qo'shish\\
D) Eritmaga NaOH qo'shish
  \item Quyidagi amallarning qaysi biri $\mathrm{K}_{3} \mathrm{PO}_{4}$ eritmasining gidrolizlanish jarayonini susaytiradi?\\
A) Eritmaga LiOH qo'shish\\
B) Eritmaga $\mathrm{H}_{2} \mathrm{SO}_{4}$ qo'shish\\
C) Eritmaga suv qo'shish\\
D) Eritmaga HCl qo'shish
  \item Quyidagi tuzlarning qaysi birining eritmasi ishqoriy muhitga ega bo'ladi?\\
A) $\mathrm{NH}_{4} \mathrm{Cl}$\\
B) $\mathrm{Ca}\left(\mathrm{NO}_{3}\right)_{2}$\\
C) $\mathrm{Al}\left(\mathrm{NO}_{3}\right)_{3}$\\
D) $\mathrm{Ba}\left(\mathrm{NO}_{2}\right)_{2}$\\
  \item Quyidagi tuzlarning qaysi birining eritmasi ishqoriy muhitga ega bo'ladi?\\
A) NaCN\\
B) $\mathrm{Al}\left(\mathrm{NO}_{2}\right)_{3}$\\
C) $\mathrm{MgSO}_{4}$\\
D) $\mathrm{K}_{2} \mathrm{SO}_{4}$
  \item Quyidagi tuzlarning qaysi birining eritmasi ishqoriy muhitga ega bo'ladi?\\
A) $\mathrm{Cr}\left(\mathrm{NO}_{2}\right)_{3}$\\
B) $\mathrm{Ca}\left(\mathrm{CH}_{3} \mathrm{COO}\right)_{2}$\\
C) $\mathrm{CaCl}_{2}$\\
D) $\mathrm{Ca}\left(\mathrm{NO}_{3}\right)_{2}$
  \item Quyidagi tuzlarning qaysi birining eritmasi kislotali muhitga ega bo'ladi?\\
A) $\mathrm{NH}_{4} \mathrm{CN}$\\
B) $\mathrm{Ca}\left(\mathrm{NO}_{3}\right)_{2}$\\
C) $\mathrm{Al}\left(\mathrm{NO}_{3}\right)_{3}$\\
D) $\mathrm{Ba}\left(\mathrm{NO}_{2}\right)_{2}$
  \item Quyidagi tuzlarning qaysi birining eritmasi kislotali muhitga ega bo'ladi?\\
A) NaCN\\
B) $\mathrm{Al}\left(\mathrm{NO}_{2}\right)_{3}$\\
C) $\mathrm{MgSO}_{4}$\\
D) $\mathrm{K}_{2} \mathrm{SO}_{4}$
  \item Quyidagi tuzlarning qaysi birining eritmasi kislotali muhitga ega bo'ladi?\\
A) $\mathrm{Cr}\left(\mathrm{NO}_{3}\right)_{3}$\\
B) $\mathrm{Ca}\left(\mathrm{CH}_{3} \mathrm{COO}\right)_{2}$\\
C) $\mathrm{CaCl}_{2}$\\
D) $\mathrm{Ba}\left(\mathrm{NO}_{2}\right)_{2}$
  \item Quyidagi tuzlarning qaysi birining eritmasi neytral muhitga ega bo'ladi?\\
A) $\mathrm{NH}_{4} \mathrm{Cl}$\\
B) $\mathrm{Ca}\left(\mathrm{NO}_{3}\right)_{2}$\\
C) $\mathrm{Al}\left(\mathrm{NO}_{3}\right)_{3}$\\
D) $\mathrm{Ba}\left(\mathrm{NO}_{2}\right)_{2}$
  \item Quyidagi tuzlarning qaysi birining eritmasi neytral muhitga ega bo'ladi?\\
A) NaCN\\
B) $\mathrm{Al}\left(\mathrm{NO}_{2}\right)_{3}$\\
C) $\mathrm{MgSO}_{4}$\\
D) $\mathrm{K}_{2} \mathrm{SO}_{4}$
  \item Quyidagi tuzlarning qaysi birining eritmasi neytral muhitga ega bo'ladi?\\
A) $\mathrm{Cr}\left(\mathrm{NO}_{3}\right)_{3}$\\
B) $\mathrm{Ca}\left(\mathrm{CH}_{3} \mathrm{COO}\right)_{2}$\\
C) $\mathrm{CaCl}_{2}$\\
D) $\mathrm{Ba}\left(\mathrm{NO}_{2}\right)_{2}$
  \item Quyidagi tuzlarning qaysi birining eritmasi neytral muhitga ega bo'ladi?\\
A) $\mathrm{NH}_{4} \mathrm{Cl}$\\
B) $\mathrm{Ca}\left(\mathrm{NO}_{3}\right)_{2}$\\
C) $\mathrm{Al}\left(\mathrm{NO}_{3}\right)_{3}$\\
D) $\mathrm{Ba}\left(\mathrm{NO}_{2}\right)_{2}$
  \item Quyidagi tuzlarning qaysi birining eritmasi lakmus rangini qizartiradi?\\
A) $\mathrm{NH}_{4} \mathrm{Cl}$\\
B) $\mathrm{Ca}\left(\mathrm{NO}_{3}\right)_{2}$\\
C) $\mathrm{Al}\left(\mathrm{NO}_{2}\right)_{3}$\\
D) $\mathrm{Ba}\left(\mathrm{NO}_{2}\right)_{2}$\\
  \item Quyidagi tuzlarning qaysi birining eritmasi lakmus rangini qizartiradi?\\
A) NaCN\\
B) $\mathrm{Al}\left(\mathrm{NO}_{2}\right)_{3}$\\
C) $\mathrm{MgSO}_{4}$\\
D) $\mathrm{K}_{2} \mathrm{SO}_{4}$
  \item Quyidagi tuzlarning qaysi birining eritmasi lakmus rangini qizartiradi?\\
A) $\mathrm{Cr}\left(\mathrm{NO}_{3}\right)_{3}$\\
B) $\mathrm{Ca}\left(\mathrm{CH}_{3} \mathrm{COO}\right)_{2}$\\
C) $\mathrm{CaCl}_{2}$\\
D) $\mathrm{Ba}\left(\mathrm{NO}_{2}\right)_{2}$
  \item Quyidagi tuzlarning qaysi birining eritmasi fenolftalein rangini qizartiradi?\\
A) $\mathrm{NH}_{4} \mathrm{CN}$\\
B) $\mathrm{Ca}\left(\mathrm{NO}_{3}\right)_{2}$\\
C) $\mathrm{Al}\left(\mathrm{NO}_{3}\right)_{3}$\\
D) $\mathrm{Ba}\left(\mathrm{NO}_{2}\right)_{2}$
  \item Quyidagi tuzlarning qaysi birining eritmasi fenolftalein rangini qizartiradi?\\
A) NaCN\\
B) $\mathrm{Al}\left(\mathrm{NO}_{2}\right)_{3}$\\
C) $\mathrm{MgSO}_{4}$\\
D) $\mathrm{K}_{2} \mathrm{SO}_{4}$
  \item Quyidagi tuzlarning qaysi birining eritmasi fenolftalein rangini qizartiradi?\\
A) $\mathrm{Cr}\left(\mathrm{NO}_{3}\right)_{3}$\\
B) $\mathrm{Ca}\left(\mathrm{CH}_{3} \mathrm{COO}\right)_{2}$\\
C) $\mathrm{CaCl}_{2}$\\
D) $\mathrm{Ba}\left(\mathrm{NO}_{3}\right)_{2}$
  \item Quyidagi tuzlarning qaysi birining eritmasi metil zarg'aldog'i rangini fushti rangga o'zgartiradi?\\
A) $\mathrm{NH}_{4} \mathrm{Cl}$\\
B) $\mathrm{Ca}\left(\mathrm{NO}_{3}\right)_{2}$\\
C) $\mathrm{Al}\left(\mathrm{NO}_{2}\right)_{3}$\\
D) $\mathrm{Ba}\left(\mathrm{NO}_{2}\right)_{2}$
  \item Quyidagi tuzlarning qaysi birining eritmasi metil zarg'aldog'i rangini fushti rangga o'zgartiradi?\\
A) NaCN\\
B) $\mathrm{Al}\left(\mathrm{NO}_{2}\right)_{3}$\\
C) $\mathrm{MgSO}_{4}$\\
D) $\mathrm{K}_{2} \mathrm{SO}_{4}$
  \item Quyidagi tuzlarning qaysi birining eritmasi lakmus rangini o'zgartirmaydi?\\
A) $\mathrm{Cr}\left(\mathrm{NO}_{3}\right)_{3}$\\
B) $\mathrm{Ca}\left(\mathrm{CH}_{3} \mathrm{COO}\right)_{2}$\\
C) $\mathrm{CaCl}_{2}$\\
D) $\mathrm{Ba}\left(\mathrm{NO}_{2}\right)_{2}$
  \item Quyidagi tuzlarning qaysi birining eritmasi lakmus rangini o'zgartirmaydi?\\
A) $\mathrm{NH}_{4} \mathrm{Cl}$\\
B) $\mathrm{Ca}\left(\mathrm{NO}_{3}\right)_{2}$\\
C) $\mathrm{Al}\left(\mathrm{NO}_{3}\right)_{3}$\\
D) $\mathrm{Ba}\left(\mathrm{NO}_{2}\right)_{2}$
  \item $500 \mathrm{~g} 53,4 \% \mathrm{li} \mathrm{AlCl}_{3}$ eritmasining ma'lum qismi gidrolizga uchradi. Natijada eritma tarkibida $1,5 \mathrm{~mol} \mathrm{Al}^{3+}$ ioni qolgan bo'lsa tuzning gidrolizlanish darajasini \% da toping. $(\mathrm{a}=100 \%)$\\
A) 25\\
B) 20\\
C) 30\\
D) 35
  \item $400 \mathrm{~g} 60 \% \mathrm{li} \mathrm{MgSO}_{4}$ eritmasining ma'lum qismi gidrolizga uchradi. Natijada eritma tarkibida $1,8 \mathrm{~mol} \mathrm{Mg}^{2+}$ ioni qolgan bo'lsa, tuzning gidrolizlanish darajasini \% da toping. $(\mathrm{a}=100 \%)$\\
A) 15\\
B) 20\\
C) 10\\
D) 25
  \item $500 \mathrm{~g} 65 \% \mathrm{li} \mathrm{FeCl}_{3}$ eritmasining ma'lum qismi gidrolizga uchradi. Natijada eritma tarkibida $1,7 \mathrm{~mol} \mathrm{Fe}^{3+}$ ioni qolgan\\
bo'lsa tuzning gidrolizlanish darajasini \% da toping. ( $\alpha=100 \%$ )\\
A) 15\\
B) 20\\
C) 10\\
D) 25
  \item $500 \mathrm{~g} 47,6 \%$ li $\mathrm{Cr}\left(\mathrm{NO}_{3}\right)_{3}$ eritmasining ma'lum qismi gidrolizga uchradi. Natijada eritma tarkibida $0,8 \mathrm{~mol} \mathrm{Cr}^{3+}$ ioni qolgan bo'lsa tuzning gidrolizlanish darajasini \% da toping. ( $\alpha=100 \%$ )\\
A) 25\\
B) 20\\
C) 30\\
D) 35
  \item $636 \mathrm{~g} 50 \% \mathrm{li} \mathrm{Na}_{2} \mathrm{CO}_{3}$ eritmasining ma'lum qismi gidrolizga uchradi. Natijada eritma tarkibida $2,7 \mathrm{~mol} \mathrm{CO}_{3}{ }^{2-}$ ioni qolgan bo'lsa tuzning gidrolizlanish darajasini \% da toping. $\alpha=100 \%$ )\\
A) 15\\
B) 20\\
C) 10\\
D) 25
  \item $400 \mathrm{~g} 61 \%$ li $\mathrm{Na}_{2} \mathrm{SiO}_{3}$ eritmasining ma'lum qismi gidrolizga uchradi. Natijada eritma tarkibida $1,5 \mathrm{~mol} \mathrm{SiO}_{3}{ }^{2-}$ ioni qolgan bo'lsa tuzning gidrolizlanish darajasini \% da toping. $(\alpha=100 \%)$\\
A) 25\\
B) 20\\
C) 30\\
D) 35
  \item $828 \mathrm{~g} 50 \%$ li $\mathrm{K}_{2} \mathrm{CO}_{3}$ eritmasining ma'lum qismi gidrolizga uchradi. Natijada eritma tarkibida $2,4 \mathrm{~mol} \mathrm{CO}_{3}{ }^{2-}$ ioni qolgan bo'lsa tuzning gidrolizlanish darajasini \% da toping. ( $\mathrm{a}=100 \%$ )\\
A) 25\\
B) 20\\
C) 30\\
D) 35
  \item $600 \mathrm{~g} 71 \% \mathrm{li} \mathrm{Al}\left(\mathrm{NO}_{3}\right)_{3}$ eritmasining ma’lum qismi gidrolizga uchradi. Natijada eritma tarkibida $1,9 \mathrm{~mol} \mathrm{Al}^{3+}$ ioni qolgan bo'lsa tuzning gidrolizlanish darajasini \% da toping. ( $\alpha=100 \%$ )\\
A) 4\\
B) 2\\
C) 3\\
D) 5
  \item $500 \mathrm{~g} 63,4 \%$ li $\mathrm{CrCl}_{3}$ eritmasining ma'lum qismi gidrolizga uchradi. Natijada eritma tarkibida $1,25 \mathrm{~mol} \mathrm{Cr}^{3+}$ ioni qolgan bo'lsa tuzning gidrolizlanish darajasini \% da toping. ( $\alpha=100 \%$ )\\
A) 37,5\\
B) 25\\
C) 30\\
D) 35,4
  \item $500 \mathrm{~g} 53,4 \%$ li AlBr 3 eritmasining ma'lum qismi gidrolizga uchradi. Natijada eritma tarkibida $0,95 \mathrm{~mol} \mathrm{Al}^{3+}$ ioni qolgan bo'lsa tuzning gidrolizlanish darajasini \% da toping. $\alpha=100 \%$ )\\
A) 4\\
B) 2\\
C) 3\\
D) 5
  \item $600 \mathrm{~g} 71 \% \mathrm{li} \mathrm{Al}\left(\mathrm{NO}_{3}\right)_{3}$ eritmasi tarkibidagi tuzning $10 \%$ miqdori to'liq gidrolizga uchragan bo'lsa, necha g $\mathrm{Al}(\mathrm{OH})_{3}$ cho'kmasi hosil bo'lgan? $(\alpha=100 \%)$\\
A) 78\\
B) 3,9\\
C) 7,8\\
D) 15,6
  \item $500 \mathrm{~g} 47,6 \% \mathrm{li} \mathrm{Cr}\left(\mathrm{NO}_{3}\right)_{3}$ eritmasi tarkibidagi tuzning 5\% miqdori to'liq gidrolizga uchragan bo'lsa, necha g $\mathrm{Cr}(\mathrm{OH})_{3}$ cho'kmasi hosil bo'lgan? ( $\alpha=100 \%$ )\\
A) 5,15\\
B) 10,3\\
C) 15,45\\
D) 20,6
  \item $400 \mathrm{~g} 61 \%$ li $\mathrm{Na}_{2} \mathrm{SiO}_{3}$ eritmasi tarkibidagi tuzning $15 \%$ miqdori to'liq gidrolizga uchragan bo'lsa, necha g $\mathrm{H}_{2} \mathrm{SiO}_{3}$ cho'kmasi hosil bo'lgan? ( $\alpha=100 \%$ )\\
A) 46,8\\
B) 11,7\\
C) 23,4\\
D) 35,1
  \item $500 \mathrm{~g} 65 \%$ li $\mathrm{FeCl}_{3}$ eritmasi tarkíbidagi tuzning $20 \%$ miqdori to'liq gidrolizga uchragan bo'lsa, necha g $\mathrm{Fe}(\mathrm{OH})_{3}$ cho'kmasi hosil bo'lgan? $(\alpha=100 \%)$\\
A) 32,1\\
B) 21,4\\
C) 42,8\\
D) 10,7
  \item $400 \mathrm{~g} 60 \%$ li $\mathrm{MgSO}_{4}$ eritmasi tarkibidagi tuzning $10 \%$ miqdori to'liq gidrolizga uchragan bo'lsa, necha g $\mathrm{Mg}(\mathrm{OH})_{2}$ cho'kmasi hosil bo'lgan? ( $a=100 \%$ )\\
A) 11,6\\
B) 23,2\\
C) 5,8\\
D) 17,4
  \item $400 \mathrm{~g} 60,5 \%$ li $\mathrm{Fe}\left(\mathrm{NO}_{3}\right)_{3}$ eritmasi tarkibidagi tuzning $10 \%$ miqdori to'liq gidrolizga uchragan bo'lsa, necha g $\mathrm{Fe}(\mathrm{OH})_{3}$ cho'kmasi hosil bo'lgan? ( $\alpha=100 \%$ )\\
A) 32,1\\
B) 21,4\\
C) 42,8\\
D) 10,7
  \item $500 \mathrm{~g} 53,4 \%$ li $\mathrm{AlCl}_{3}$ eritmasi tarkibidagi tuzning $20 \%$ miqdori to'liq gidrolizga uchragan bo'lsa, necha g $\mathrm{Al}(\mathrm{OH})_{3}$ cho'kmasi hosil bo'lgan? ( $\alpha=100 \%$ )\\
31,2 A)\\
B) 3,9\\
C) 7,8\\
D) 15,6
  \item $600 \mathrm{~g} 63 \% \mathrm{li} \mathrm{Zn}\left(\mathrm{NO}_{3}\right)_{2}$ eritmasi tarkibidagi tuzning $10 \%$ miqdori to'liq\\
gidrolizga uchragan bo'lsa, necha g $\mathrm{Zn}(\mathrm{OH})_{2}$ cho'kmasi hosil bo'lgan? $(\alpha=100 \%)$\\
A) 9,9\\
B) 4,95\\
C) 19,8\\
D) 39,6
  \item $600 \mathrm{~g} 30 \% \mathrm{li} \mathrm{Fe}\left(\mathrm{NO}_{3}\right)_{2}$ eritmasi tarkibidagi tuzning $10 \%$ miqdori to'liq gidrolizga uchragan bo'lsa, necha g $\mathrm{Fe}(\mathrm{OH})_{2}$ cho'kmasi hosil bo'lgan? ( $\mathrm{a}=100 \%$ )\\
A) 27\\
B) 9\\
C) 36\\
D) 18
  \item $500 \mathrm{~g} 38 \% \mathrm{li} \mathrm{MgCl}_{2}$ eritmasi tarkibidagi tuzning $10 \%$ miqdori to'liq gidrolizga uchragan bo'lsa, necha g $\mathrm{Mg}(\mathrm{OH})_{2}$ cho'kmasi hosil bo'lgan? ( $a=100 \%$ )\\
A) 11,6\\
B) 23,2\\
C) 5,8\\
D) 17,4
  \item $800 \mathrm{ml} 0,1 \mathrm{M} \mathrm{li} \mathrm{Al}{ }_{2}\left(\mathrm{SO}_{4}\right)_{3}$ eritmasida $0,416 \mathrm{~mol}$ ion bo'lsa, tuzning gidrolizlanish darajasini (\%) aniqlang. ( $\alpha=100 \%$ )\\
A) 4\\
B) 5\\
C) 8\\
D) 10
  \item $500 \mathrm{ml} 0,2 \mathrm{M} \mathrm{li} \mathrm{AlCl} 3$ eritmasida 0,42 mol ion bo'lsa, tuzning gidrolizlanish darajasini (\%) aniqlang. ( $\mathrm{a}=100 \%$ )\\
A) 4\\
B) 5\\
C) 8\\
D) 10
  \item $1000 \mathrm{ml} 0,5 \mathrm{M} \mathrm{li} \mathrm{MgSO}_{4}$ eritmasida $1,04 \mathrm{~mol}$ ion bo'lsa, tuzning gidrolizlanish darajasini (\%) aniqlang. ( $\alpha=100 \%$ )\\
A) 4\\
B) 5\\
C) 8\\
D) 10
  \item $800 \mathrm{ml} 0,4 \mathrm{M}$ li $\mathrm{Zn}\left(\mathrm{NO}_{3}\right)_{2}$ eritmasida $0,976 \mathrm{~mol}$ ion bo'lsa, tuzning gidrolizlanish darajasini (\%) aniqlang. ( $\mathrm{a}=100 \%$ )\\
A) 4\\
B) 5\\
C) 8\\
D) 10
  \item $600 \mathrm{ml} 0,2 \mathrm{M}$ li $\mathrm{Cr}\left(\mathrm{NO}_{3}\right)_{3}$ eritmasida $0,4896 \mathrm{~mol}$ ion bo'lsa, tuzning gidrolizlanish darajasini (\%) aniqlang. ( $\mathrm{a}=100 \%$ )\\
A) 4\\
B) 5\\
C) 8\\
D) 10
  \item $2000 \mathrm{ml} 0,3 \mathrm{M}$ li $\mathrm{Al}\left(\mathrm{NO}_{3}\right)_{3}$ eritmasida $2,496 \mathrm{~mol}$ ion bo'lsa, tuzning gidrolizlanish darajasini (\%) aniqlang. ( $\mathrm{a}=100 \%$ )\\
A) 4\\
B) 5\\
C) 8\\
D) 10
  \item $1200 \mathrm{ml} 0,2 \mathrm{M}$ li $\mathrm{Fe}\left(\mathrm{NO}_{3}\right)_{2}$ eritmasida $0,732 \mathrm{~mol}$ ion bo'lsa, tuzning gidrolizlanish darajasini (\%) aniqlang. ( $\mathrm{a}=100 \%$ )\\
A) 4\\
B) 5\\
C) 8\\
D) 10
  \item $900 \mathrm{ml} 0,3 \mathrm{M} \mathrm{li} \mathrm{FeCl}{ }_{3}$ eritmasida 1,134 mol ion bo'lsa, tuzning gidrolizlanish darajasini (\%) aniqlang. ( $\alpha=100 \%$ )\\
A) 4\\
B) 5\\
C) 8\\
D) 10
  \item $1000 \mathrm{ml} 0,5 \mathrm{M}$ li $\mathrm{Fe}\left(\mathrm{NO}_{3}\right)_{3}$ eritmasida 2.04 mol ion bo'lsa, tuzning gidrolizlanish darajasini (\%) aniqlang. ( $\alpha=100 \%$ )\\
A) 4\\
B) 5\\
C) 8\\
D) 10
  \item $1500 \mathrm{ml} 0,2 \mathrm{M}$ li $\mathrm{MgCl}_{2}$ eritmasida $0,924 \mathrm{~mol}$ ion bo'lsa, tuzning gidrolizlanish darajasini (\%) aniqlang. ( $\mathrm{a}=100 \%$ )\\
A) 4\\
B) 5\\
C) 8\\
D) 10